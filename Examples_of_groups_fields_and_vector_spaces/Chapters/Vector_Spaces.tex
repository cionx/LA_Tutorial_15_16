\chapter{Beispiele für Vektorräume}
Sofern nicht anders angegeben, bezeichnen $K$ und $L$ im folgenden Körper mit $K \subseteq L$.





\section{Einige kleinere Beispiele}
\begin{enumerate}[leftmargin=*]
 \item
  $K$ ist ein $K$-Vektorraum, indem man die Multiplikation $K \times K \to K$, $(x,y) \mapsto x \cdot y$ als Skalarmultiplikation wählt. Dass $\lambda \cdot (v+w) = \lambda v + \lambda w$ und $(\lambda+\mu) \cdot v = \lambda \cdot v + \mu \cdot v$ für alle $\lambda, \mu \in K$ und $v,w \in V$ folgt direkt aus der Distributivität der Multiplikation. Dass $\lambda \cdot (\mu \cdot v) = (\lambda \cdot \mu) \cdot v$ für alle $\lambda, \mu \in K$ folgt aus der Assoziativität der Multiplikation. Dass $1 \cdot v = v$ für alle $v \in V$ folgt aus der Definition des Einselementes $1$.
 \item
  Wie bereits in \ref{sec: product of groups} gesehen ist $K^n = \{(x_1, \dotsc, x_n) \mid x_1, \dotsc, x_n \in K\}$ mit der eintragsweisen Addition
  \[
   (x_1, \dotsc, x_n) + (y_1, \dotsc, y_n)
   = (x_1 + y_1, \dotsc, x_n + y_n)
  \]
  eine abelsche Gruppe. Mit der eintragsweise Skalarmultiplikation
  \[
   \lambda \cdot (x_1, \dotsc, x_n)
   = (\lambda x_1, \dotsc, \lambda x_n)
   \quad
   \text{für alle $\lambda \in K$ und $(x_1, \dotsc, x_n) \in K^n$}
  \]
  ist $K^n$ zu einem $K$-Vektorraum: Für alle $\lambda \in K$ und $(x_1, \dotsc, x_n), (y_1, \dotsc, y_n) \in K^n$ ist
  \begin{align*}
    &\, \lambda \cdot ((x_1, \dotsc, x_n) + (y_1, \dotsc, y_n))
   =    \lambda \cdot (x_1 + y_1, \dotsc, x_n + y_n) \\
   =&\, (\lambda (x_1 + y_1), \dotsc, \lambda (x_n + y_n))
   =    (\lambda x_1 + \lambda y_m, \dotsc, \lambda x_n + \lambda y_n) \\
   =&\, (\lambda x_1, \dotsc, \lambda x_n) + (\lambda y_1, \dotsc, \lambda y_n)
   =    \lambda \cdot (x_1, \dotsc, x_n) + \lambda \cdot (y_1, \dotsc, y_n).
  \end{align*}
  Für alle $\lambda, \mu \in K$ und $(x_1, \dotsc, x_n) \in K^n$ ist
  \begin{gather*}
   \begin{aligned}
     &\, (\lambda+\mu) \cdot (x_1, \dotsc, x_n)
    =    ((\lambda+\mu)x_1, \dotsc, (\lambda+\mu)x_n) \\
    =&\, (\lambda x_1 + \mu x_1, \dotsc, \lambda x_n + \mu x_n)
    =    (\lambda x_1, \dotsc, \lambda x_n) + (\mu x_1, \dotsc, \mu x_n) \\
    =&\, \lambda \cdot (x_1, \dotsc, x_n) + \mu \cdot (x_1, \dotsc, x_n)
   \end{aligned}
  \shortintertext{und}
   \begin{aligned}
    \lambda \cdot (\mu \cdot (x_1, \dotsc, x_n))
    &= \lambda \cdot (\mu x_1, \dotsc, \mu  x_n) \\
    &= (\lambda \mu x_1, \dotsc, \lambda \mu x_n)
    = (\lambda \mu) \cdot (x_1, \dotsc, x_n).
   \end{aligned}
  \end{gather*}
  Für alle $(x_1, \dotsc, x_n) \in K^n$ gilt schließlich
  \[
   1 \cdot (x_1, \dotsc, x_n)
   = (1 \cdot x_1, \dotsc, 1 \cdot x_n)
   = (x_1, \dotsc, x_n).
  \]
  Ingesamt zeigt dies, dass $K^n$ mit der eintragsweisen Addition und Skalarmultiplikation einen $K$-Vektorraum bildet.
 \item
  Ist $V$ ein $K$-Vektorraum, so ist $\{0\} \subseteq V$ ein Untervektorraum. Man bezeichnet diesen als den \emph{Nulluntervektorraum}.
 \item
  Die $L$-Vektorraum auf $L$ schränkt sich zu einer $K$-Vektorraumstruktur auf $L$ ein. $L$ ist also ein $K$-Vektorraum durch $\lambda \cdot v = \lambda v$ für alle $\lambda \in K$ und $v \in L$, wobei die punktlose Multiplikation auf der rechten Seite die Multiplikation des Körpers $L$ bezeichnet.
 \item
  Ist allgemein $V$ ein $L$-Vektorraum, so wird $V$ durch Einschränkung der Skalarmultiplikation zu einem $K$-Vektorraum. Dass die Axiome eines $K$-Vektorraums erfüllt sind, folgt direkt daraus, dass sie für Skalare aus $L$ erfüllt sind. Man bezeichnet diese Einschränkung von $L$-Vektorräumen zu $K$-Vektorräumen als \emph{Skalarrestriktion}.
 \item
  Ist $V$ ein $\Cbb$-Vektorraum, so ist $V$ auch durch die Skalarmultiplikation
  \[
   \lambda * v = \overline{\lambda} \cdot v
   \quad
   \text{für alle $\lambda \in Cbb$ und $v \in V$}
  \]
  ein $\Cbb$-Vektorraum: Für alle $\lambda \in \Cbb$ und $v,w \in V$ ist
  \[
   \lambda * (v+w)
   = \overline{\lambda} \cdot (v+w)
   = \overline{\lambda} \cdot v + \overline{\lambda} \cdot w
   = \lambda * v + \lambda * w.
  \]
  Für alle $\lambda, \mu \in \Cbb$ und $v \in V$ ist
  \begin{gather*}
   (\lambda + \mu) * v
   = \overline{\lambda + \mu} \cdot v
   = (\overline{\lambda} + \overline{\mu}) \cdot v
   = \overline{\lambda} \cdot v + \overline{\mu} \cdot v
   = \lambda * v + \mu * v.
  \shortintertext{und}
   \lambda * (\mu * v)
   = \overline{\lambda} \cdot (\overline{\mu} \cdot v)
   = (\overline{\lambda} \cdot \overline{\mu}) \cdot v
   = \overline{\lambda \cdot \mu} \cdot v
   = (\lambda \cdot \mu) * v.
  \end{gather*}
  Außerdem ist für alle $v \in V$
  \[
   1 * v = 1 \cdot v = v.
  \]
 \item
  Ist allgemeinen $\phi \colon K \to K'$ ein Körperhomomorphismus und $V$ ein $K'$-Vektorraum, so wird $V$ ein $K$-Vektorraum durch die Skalarmultiplikation
  \[
   \lambda * v = \phi(\lambda) \cdot v
   \quad
   \text{für alle $\lambda \in K$ und $v \in V$}.
  \]
  Für alle $\lambda, \mu \in K$ und $v, w \in V$ ist nämlich
  \begin{gather*}
    \lambda * (v+w)
    = \phi(\lambda) \cdot (v+w)
    = \phi(\lambda) \cdot v + \phi(\lambda) \cdot w
    = \lambda * v + \lambda * w
  \shortintertext{und}
   \begin{aligned}
    (\lambda + \mu) * v
    = \phi(\lambda + \mu) \cdot v
    &= (\phi(\lambda) + \phi(\mu)) \cdot v \\
    &= \phi(\lambda) \cdot v + \phi(\mu) \cdot v
    = \lambda * v + \mu * v,
   \end{aligned}
  \shortintertext{sowie}
   \begin{aligned}
    \lambda * (\mu * v)
    = \phi(\lambda) \cdot (\phi(\mu) \cdot v)
    &= (\phi(\lambda) \cdot \phi(\mu)) \cdot v \\
    &= \phi(\lambda \cdot \mu) \cdot v
    = (\lambda \cdot \mu) * v,
   \end{aligned}
  \end{gather*}
  und für alle $v \in V$ gilt
  \[
   1 * v = \phi(1) \cdot v = 1 \cdot v = v.
  \]
  
  Die vorherigen beiden Beispiele ergeben sich als Sonderfall hievon: Die Skalarrestriktion ergibt sich durch die Inklusion $K \hookrightarrow L$ und das vorherige Beispiel durch die Konjugation $\Cbb \to \Cbb, z \mapsto \overline{z}$.
\end{enumerate}


\section{Untervektorräume bezüglich linearer Abbildungen}
Es seien $V$ und $W$ $K$-Vektorräume und $f \colon V \to W$ sei eine $K$-lineare Abbildung.



\subsection{Kern und Bild einer linearen Abbildung}
Wir zeigen, dass der \emph{Kern} von $f$
\[
 \ker(f) \coloneqq \{v \in V \mid f(v) = 0\}
\]
ein Untervektorraum von $V$ ist: Da $f$ insbesondere ein Gruppenhomomorphismus der unterliegenden abelschen Gruppen ist, haben wir in \ref{sec: subgroups and group homomorphisms} bereits gesehen, dass $\ker(f)$ eine additive Untergruppe ist. Für alle $\lambda \in K$ und $v \in \ker(f)$ ist auch
\[
 f(\lambda v) = \lambda f(v) = \lambda \cdot 0 = 0,
\]
also $\lambda v \in \ker(f)$. Das zeigt, dass $\ker(f)$ ein Untervektorraum ist.

\begin{bem}
 Wir haben in \ref{sec: subgroups and group homomorphisms} auch schon gesehen, dass $f$ genau dann injektiv ist, wenn $\ker(f) = \{0\}$.
\end{bem}

Das \emph{Bild} von $f$
\[
 \im(f) = \{f(v) \mid v \in V\}
\]
ist ein Untervektorraum von $W$: Da $f$ insbesondere ein Gruppenhomomorphismus der unterliegenden abelschen Gruppen ist, haben wir in \ref{sec: subgroups and group homomorphisms} bereits gesehen, dass $\im(f)$ eine additive Untergruppe ist. Für $w \in \im(f)$ gibt es ein $v \in V$ mit $w = f(v)$, weshalb für alle $\lambda \in K$ auch
\[
 \lambda w
 = \lambda f(v)
 = f(\lambda w)
 \in \im(f).
\]
Das zeigt, dass $\im(f)$ ein Untervektorraum ist.



\subsection{Bilder und Urbilder von Untervektorräumen}
Ist $U \subseteq V$ ein Untervektorraum, so ist das Bild $f(U)$ ein Untervektorraum von $W$: Da $f$ inbesondere ein Gruppenhomomorphismen der unterliegenden abelschen Gruppen ist haben wir bereits in \ref{sec: subgroups and group homomorphisms} gesehen, dass $f(U)$ ein additive Untergruppe ist. Ist $w \in f(U)$, so gibt es $u \in U$ mit $w = f(u)$. Da $U$ ein Untervektorraum ist, ist für alle $\lambda \in K$ auch $\lambda u \in U$, weshalb für alle $\lambda \in K$ auch
\[
 \lambda w
 = \lambda f(u)
 = f(\lambda u)
 \in f(U).
\]
Also ist $f(U)$ ein Untervektorraum von $f$.


\begin{bem}
 Wie bereits in \ref{sec: subgroups and group homomorphisms} lassen sich Rechnungen auch hier umgehen: Da $f$ linear ist, ist es auch die Einschränkung $f|_{U} \colon U \to W$, $u \mapsto f(u)$. Wie bereits gesehen, ist deshalb $f(U) = \im(f|_{U})$ ein Untervektorraum.
\end{bem}


Ist $U \subseteq W$ ein Untervektorraum, so ist das Urbild $f^{-1}(U)$ ein Untervektorraum von $V$: Da $f$ inbesondere ein Gruppenhomomorphismus zwischen den unterliegenden abelschen Gruppen ist, haben wir bereits in \ref{sec: subgroups and group homomorphisms} gesehen, dass $f^{-1}(U)$ eine additive Untergruppe der unterliegenden abelschen Gruppe von $V$ ist. Für $v \in f^{-1}(U)$ ist $f(v) \in U$, also für alle $\lambda \in K$ auch
\[
 f(\lambda v) = \lambda f(v) \in U,
\]
da $U$ ein Untervektorraum ist. Also ist $f^{-1}(U)$ ein Untervektorraum.


\begin{bem}
 Ähnlich wie in \ref{sec: subgroups and group homomorphisms} folgt auch hieraus, dass
 \[
  \ker(f) = f^{-1}(\{0\})
 \]
 ein Untervektorraum ist.
\end{bem}





\section{Schnitte und Vereinigungen von Untervektorräumen}


\subsection{Schnitte von Untervektorräumen}
Es sei $V$ ein $K$-Vektorraum und $\{U_i\}_{i \in I}$ eine Kollektion von Untervektorräumen, d.h.\ für jedes $i \in I$ ist $U_i \subseteq V$ ein Untervektorraum. Wir zeigen, dass dann auch der Schnitt $U \coloneqq \bigcap_{i \in I} U_i$ ein Untervektorraum von $V$ ist:

Für alle $i \in I$ ist $U_i$ ein Untervektorraum, und somit $0 \in U_i$. Deshalb ist auch $0 \in U$. Sind $x,y \in U$, so ist $x,y \in U_i$ für alle $i \in I$. Für jedes $i \in I$ ist $U_i$ ein Untervektorraum und deshalb auch $x+y \in U_i$. Also ist auch $x+y \in U$. Ist außerdem $\lambda \in K$, so ist daher auch $\lambda x \in U_i$ für alle $i \in I$ und somit $\lambda x \in U$. Insgesamt zeigt dies, dass $U = \bigcap_{i \in I} U_i$ ein Untervektorraum von $V$ ist.


\subsection{Vereinigungen von Untervektorräumen}
Die beliebige Vereinigung von Untervektorräumen ist im Allgemeinen kein Untervektorraum. Für einen Körper $K$ sind beispielsweise $U_1, U_2 \subseteq K^2$ mit $U_1 = \{(x,0) \mid x \in K\}$ und $U_2 = \{(0,y) \mid y \in K\}$ zwei Untervektorräume, deren Vereinigung $U \coloneqq U_1 \cup U_2$ kein Untervektorraum ist, da $(1,0), (0,1) \in U$ aber $(1,0)+(0,1) = (1,1) \notin U$.

Ist $V$ ein beliebiger $K$-Vektorraum, und sind $U_1, U_2 \subset V$ zwei Untervektorräume, so ist die Vereinigung $U_1 \cup U_2$ genau dann ein Untervektorraum, wenn $U_1 \subseteq U_2$ oder $U_2 \subseteq U_1$: Es sind nämlich $U_1$ und $U_2$ Untergruppen der unterliegenden abelschen Gruppe von $V$, und ist $U_1 \cup U_2$ ebenfalls ein Untervektorraum von $V$, so ist dann auch $U_1 \cup U_2$ eine Untergruppe der unterliegenden abelschen Gruppe von $V$. Wie bereits in \ref{ssec: union of subgroups} gesehen, muss daher $U_1 \subseteq U_2$ oder $U_2 \subseteq U_1$.


\subsection{Aufsteigende Vereinigung von Untervektorräumen}
Es sei $V$ ein $K$-Vektorraum und $\{U_n\}_{n \in \Nbb}$ eine Kollektion von Untervektorräumen mit $U_n \subseteq U_m$ falls $n \leq m$, d.h.\ wir haben eine aufsteigende Kette
\[
 U_0 \subseteq U_1 \subseteq U_2 \subseteq U_3 \subseteq U_4 \subseteq \dotsb \subseteq V
\]
von Untervektorräumen. Dann ist auch die Vereinigung $U \coloneqq \bigcup_{n \in \Nbb} U_n$ ein Untervektorraum:

Es ist $0 \in U_0 \subseteq U$. Sind $x,y \in U$ so gibt es $n_x, n_y \in \Nbb$ mit $x \in U_{n_x}$ und $y \in U_{n_y}$. Für $m \coloneqq \max\{n_x, n_y\}$ ist $U_{n_x}, U_{n_y} \subseteq U_m$ und somit auch $x,y \in U_m$. Damit ist auch $x+y \in U_m \subseteq U$, da $U_m$ ein Untervektorraum ist. Für alle $\lambda \in K$ ist außerdem $\lambda x \in U_{n_x} \subseteq U$, da $U_{n_x}$ ein Untervektorraum ist. Ingesamt zeigt dies, dass $U$ ein Untervektorraum von $V$ ist.


\begin{bem}
 Wie bereits in Bemerkung \ref{bem: increasing union of subgroups} über die aufsteigende Vereinigung von Untergruppen lässt sich hier $\Nbb$ durch eine total geordnete Menge, oder sogar durch eine gerichtete Menge ersetzen.
\end{bem}





\section{Die lineare Hülle einer Teilmenge}
\begin{defi}
 Für einen $K$-Vektorraum $V$ und eine Teilmenge $S \subseteq V$ ist
 \[
  \Ell(S)
  \coloneqq
  \left\{
   \sum_{s \in S} \lambda_s s
   \,\middle|\,
   \text{$s \in S$, $\lambda_s \in K$ für alle $s \in S$ mit $\lambda_s = 0$ für fast alle $s \in S$ }
  \right\}
 \]
 die \emph{lineare Hülle} von $S$, bzw.\ den \emph{($K$-)Span} von $S$.
\end{defi}

\begin{bem}
 Die Formulierung „für fast alle“ bedeutet „alle bis auf endlich viele“. Das $\lambda_s = 0$ für fast alle $s \in S$ bedeutet also, dass die Menge $\{s \in S \mid \lambda_s \neq 0\}$ endlich ist.
\end{bem}

Es sei $V$ ein $K$-Vektorraum und $S \subseteq V$ eine Teilmenge. Wir zeigen, dass $\Ell(S)$ ein Untervektorraum von $V$ ist:

Setzt man $\lambda_s = 0$ für alle $s \in S$, so ist $\sum_{s \in S} \lambda_s s = \sum_{s \in S} 0 = 0$, also ist $0 \in \Ell(S)$.

Sind $x,y \in \Ell(S)$ so gibt es Koeffizienten $\lambda_s, \mu_s \in K$, $s \in S$ mit $\lambda_s = 0$ für fast alle $s \in S$ und $\mu_s = 0$ für fast alle $s \in S$, so dass $x = \sum_{s \in S} \lambda_s s$ und $y = \sum_{s \in S} \mu_s s$. Es ist damit auch $\lambda_s + \mu_s = 0$ für fast alle $s \in S$ und
\[
 x + y
 = \sum_{s \in S} \lambda_s s + \sum_{s \in S} \mu_s s
 = \sum_{s \in S} (\lambda_s + \mu_s) s.
\]
Also ist auch $x+y \in \Ell(S)$. Für $\mu \in K$ ist außerdem auch $\mu \cdot \lambda_s = 0$ für fast alle $s \in S$ und
\[
 \mu x
 = \mu \sum_{s \in S} \lambda s
 = \sum_{s \in S} (\mu \lambda_s) x.
\]
Also ist auch $x \in \Ell(S)$. Ingesamt zeigt dies, dass $\Ell(S)$ ein Untervektorraum von $V$ ist.

\begin{bem}\label{bem: span is the smallest vector space}
 $\Ell(S)$ ist der kleinste Untervektorraum, der $S$ enthält, d.h.\ ist $U \subseteq V$ ein Untervektorraum mit $S \subseteq U$, so ist auch $\Ell(S) \subseteq U$. Dies folgt direkt daraus, dass $U$ unter der Skalarmultiplikation und Addition von $V$ abgeschlossen ist.
 
 Für $\mc{U} \coloneqq \{U \subseteq V \mid \text{$U$ ist ein Untervektorraum mit $S \subseteq U$}\}$ ist auch $W \coloneqq \bigcap_{U \in \mc{U}} U$ ein Untervektorraum von $V$, da Schnitte von Untervektorräumen wieder Untervektorräume sind. Da $S \subseteq U$ für alle $U \in \mc{U}$ ist auch $S \subseteq \bigcap_{U \in \mc{U}} U = W$. Ist $U \subseteq V$ ein beliebiger Untervektorraum mit $S \subseteq U$ so ist $U \in \mc{U}$ und somit $W = \bigcap_{U' \in \mc{U}} U' = W$. Also ist auch $W$ der kleinste Untervektorraum von $V$, der $S$ enthält.
 
 Es ist daher $\Ell(S) = W = \bigcap_{U \in \mc{U}} U$. (Da $\Ell(S)$ der kleinste Untervektorraum von $V$ ist, der $S$ enthält, und $W$ ein Untervektorraum ist, der $S$ enthält, ist $\Ell(S) \subseteq W$. Da $W$ der kleinste Untervektorraum ist, der $S$ enthält, und auch $\Ell(S)$ ein Untervektorraum ist, der $S$ enthält, ist auch $W \subseteq \Ell(S)$. Also ist bereits $\Ell(S) = W$.)
\end{bem}





\section{Summen von Untervektorräumen}

\begin{defi}
 Es sei $V$ ein $K$-Vektorraum und $\{U_i\}_{i \in I}$ eine Kollektion von Untervektorräumen von $V$, d.h.\ für alle $i \in I$ ist $U_i \subseteq V$ ein Untervektorraum. Dann ist
 \[
  \sum_{i \in I} U_i
  \coloneqq
  \left\{
   \sum_{i \in I} u_i
   \,\middle|\,
   \text{$u_i \in U_i$ für alle $i \in I$, und $u_i = 0$ für fast alle $i \in I$}
  \right\}.
 \]
 die \emph{Summe} der Untervektorräume $U_i$, $i \in I$.
\end{defi}

Wir zeigen, dass $\sum_{i \in I} U_i$ ein Untervektorraum von $V$ ist: Indem man $u_i = 0$ für alle $i \in I$ wählt, ergibt sich, dass $0 = \sum_{i \in I} u_i \in \sum_{i \in I} U_i$. Sind $x,y \in \sum_{i \in I} U_i$, so gibt es $u_i, v_i \in U_i$, $i \in I$ mit $u_i = 0$ für fast alle $i \in I$ und $v_i = 0$ für fast alle $i \in I$, so dass $x = \sum_{i \in I} u_i$ und $y = \sum_{i \in I} v_i$. Es ist daher auch $u_i + v_i = 0$ für fast alle $i \in I$ mit
\[
 x + y
 = \sum_{i \in I} u_i + \sum_{i \in I} v_i
 = \sum_{i \in I} (u_i + v_i).
\]
Also ist auch $x+y \in \sum_{i \in I} U_i$. Ist außerdem $\mu \in K$, so ist auch $\mu u_i = 0$ für fast alle $i \in I$, und somit
\[
 \mu x
 = \mu \sum_{i \in I} u_i
 = \sum_{i \in I} (\mu u_i).
\]
Also ist auch $\mu x \in \sum_{i \in I} U_i$. Ingesamt zeigt dies, dass $\sum_{i \in I} U_i$ ein Untervektorraum von $V$ ist.


\begin{bem}
 \begin{enumerate}[leftmargin=*]
  \item
   Analog zu Bemerkung \ref{bem: span is the smallest vector space} ergibt sich, dass $\sum_{i \in I} U_i$ der kleinste Untervektorraum von $V$ ist, der $U_i$ für alle $i \in I$ enthält, und dass somit
   \[
    \sum_{i \in I} U_i
    =
    \bigcap \{W \subseteq V \mid \text{$W$ ist ein Untervektorraum mit $U_i \subseteq W$ für alle $i \in I$}\}.
   \]
  \item
   Dass $\sum_{i \in I} U_i$ der kleinte Untervektorraum ist, der $U_i$ für alle $i \in I$ enthält, ist äquivalent dazu, dass $\sum_{i \in I} U_i$ der kleinste Untervektorraum ist, der $\bigcup_{i \in I} U_i$ enthält. Deshalb ist $\sum_{i \in I} U_i = \Ell(\bigcup_{i \in I} U_i)$.
  \item
   Ist $S \subseteq V$ ist $\Ell(\{s\}) = \{\lambda s \mid \lambda \in K\}$. Daher ist
   \begin{align*}
    \Ell(S)
    &=
    \left\{
     \sum_{s \in S} \lambda_s s
     \,\middle|\,
     \text{$s \in S$, $\lambda_s \in K$ für alle $s \in S$ mit $\lambda_s = 0$ für fast alle $s \in S$ }
    \right\} \\
    &=
    \left\{
     \sum_{s \in S} u_s
     \,\middle|\,
     \text{$u_s \in \Ell(s)$ für alle $s \in S$, und $u_s = 0$ für fast alle $i \in I$}
    \right\}
    =
    \sum_{s \in S} \Ell(\{s\}).
   \end{align*}
  \item
   Für endlich viele Untervektorräume $U_1, \dotsc, U_n \subseteq V$ schreibt man auch $U_1 + \dotsb + U_n$ und $\sum_{i=1}^n U_i$ für $\sum_{i \in \{1, \dotsc, n\}} U_i$.
  \item
   Für je zwei endlichdimensional Untervektorräume $U, W \subseteq V$ gilt die Dimensionsformel
   \[
    \dim(U + W) = \dim(U) + \dim(W) - \dim(U \cap W).
   \]
 \end{enumerate}
\end{bem}





\section{Abbildungen in einen Vektorraum}\label{sec: maps into a vector space}
Es sei $X$ eine beliebige Menge und $V$ ein $K$-Vektorraum. Auf der Menge
\[
 \Abb(X,V) = \{f \colon f \colon X \to V\}
\]
Wir definieren für $f,g \in \Abb(X,V)$ und $\lambda \in K$ eine die punktweise Addition und Skalarmultiplikation durch
\begin{gather*}
 (f+g)(x) \coloneqq f(x) + g(x)
 \quad
 \text{für alle $x \in X$}
\shortintertext{und}
 (\lambda \cdot f)(x) \coloneqq \lambda \cdot f(x)
 \quad
 \text{für alle $x \in X$}.
\end{gather*}
Wir zeigen, dass $\Abb(X,V)$ zusammen mit dieser Addition und Skalarmultiplikation einen $K$-Vektorraum bildet:

Für alle $f,g,h \in \Abb(X,V)$ ist
\begin{align*}
 (f+(g+h))(x)
 &= f(x) + (g+h)(x)
 = f(x) + g(v) + h(x) \\
 &= (f+g)(x) + h(x)
 = ((f+g)+h)(x)
\end{align*}
für alle $x \in X$ und somit $f+(g+h) = (f+g)+h$. Also ist die Addition assoziativ. Für alle $f,g \in \Abb(X,V)$ ist
\[
 (f+g)(x) = f(x) + g(x) = g(x) + h(x) = (g+h)(x)
 \quad
 \text{für alle $x \in X$},
\]
also ist $f+g = g+f$, die Addition also kommutativ. Die \emph{Nullfunktion} ist definiert als
\[
 N \colon X \to V, x \mapsto 0.
\]
Für alle $f \in \Abb(X,V)$ ist
\[
 (f+N)(x) = f(x) + N(x) = f(x) + 0 = f(x)
 \quad
 \text{für alle $x \in X$},
\]
also ist $f+N = f$. Also ist $N$ neutral bezüglich der Addition. Für $f \in \Abb(X,V)$ sei
\[
 \tilde{f} \colon X \to V
 \quad\text{und}\quad
 \tilde{f}(x) \coloneqq -f(x)
 \quad
 \text{für alle $x \in X$}.
\]
Es ist dann
\[
 (f+\tilde{f})(x)
 = f(x) + \tilde{f}(x)
 = f(x) - f(x)
 = 0
 = N(x)
 \quad
 \text{für alle $x \in X$},
\]
also $f+\tilde{f} = N$. Somit ist $\tilde{f}$ additiv invers zu $f$. Ingesamt zeigt dies, dass $+$ tatsächlich eine Addition auf $\Abb(X,V)$ definiert.

Für alle $\lambda, \mu \in K$ und $f,g \in \Abb(X,V)$ ist
\begin{align*}
 (\lambda \cdot (f+g))(x)
 &= \lambda \cdot (f+g)(x)
 = \lambda \cdot (f(x)+g(x)) \\
 &= \lambda \cdot f(x) + \lambda \cdot g(x)
 = (\lambda \cdot f)(x) + (\lambda \cdot g)(x)
 = (\lambda \cdot f + \lambda \cdot g)(x)
\end{align*}
für alle $x \in X$ und somit $\lambda \cdot (f+g) = \lambda \cdot f + \lambda \cdot g$; es ist
\begin{align*}
 ((\lambda + \mu) \cdot f)(x)
 = (\lambda+\mu) \cdot f(x)
 &= \lambda \cdot f(x) + \mu \cdot f(x) \\
 &= (\lambda \cdot f)(x) + (\mu \cdot f)(x)
 = (\lambda \cdot f + \mu \cdot f)(x)
\end{align*}
für alle $x \in X$ und somit $(\lambda + \mu) \cdot f = \lambda \cdot f + \mu \cdot f$; es ist
\[
 (\lambda \cdot (\mu \cdot f))(x)
 = \lambda \cdot (\mu \cdot f)(x)
 = \lambda \cdot (\mu \cdot f(x))
 = (\lambda \cdot \mu) \cdot f(x)
 = ((\lambda \cdot \mu) \cdot f)(x)
\]
für alle $x \in X$ und somit $\lambda \cdot (\mu \cdot f) = (\lambda \cdot \mu) \cdot f$. Außerdem ist für jedes $f \in \Abb(X,V)$
\[
 (1 \cdot f)(x)
 = 1 \cdot f(x)
 = f(x)
 \quad
 \text{für alle $x \in X$},
\]
und somit $1 \cdot f = f$.

Ingesamt zeigt dies, dass $\Abb(X,V)$ zusammen mit der punktweisen Addition und Skalarmultiplikation einen $K$-Vektorraum bildet.





\section{\texorpdfstring{$\Hom$}{Hom}-Räume}
Es seien $V$ und $W$ zwei $K$-Vektorräume und
\[
 \Hom_K(V, W) \coloneqq \{f \colon V \to W \mid \text{$f$ ist linear}\}.
\]
Wir zeigen, dass $\Hom_K(V,W) \subseteq \Abb(V,W)$ einen Untervektorraum bildet, und somit ebenfalls einen Vektorraum bezüglich der punktweisen Addition und Skalarmultiplikation.

Die Nullfunktion $0 \in \Abb(V,W)$ ist linear, da $0(v+w) = 0 = 0(v)+0(w)$ sowie $0(\lambda v) = 0 = \lambda 0(v)$ für alle $v,w \in V$ und $\lambda \in K$. Also ist $0 \in \Hom_K(V,W)$.

Sind $f,g \in \Hom_K(V,W)$ und $\lambda \in K$, so ist für alle $v,w \in V$ und $\mu \in K$
\begin{gather*}
 \begin{aligned}
 (f+g)(v+w)
 &= f(v+w) + g(v+w)
 = f(v) + f(w) + g(v) + g(w) \\
 &= f(v)+g(v) + f(w)+g(w)
 = (f+g)(v) + (f+g)(w)
 \end{aligned}
\shortintertext{und}
 (\lambda f)(\mu v)
 = \lambda f(\mu v)
 = \lambda \mu f(v)
 = \mu \lambda f(v)
 = \mu (\lambda f)(v),
\end{gather*}
also auch $f+g \in \Hom_K(V,W)$ und $\lambda f \in \Hom_K(V,W)$. Dies zeigt, dass $\Hom_K(V,W)$ ein Untervektorraum von $\Abb(V,W)$ ist.


\begin{bem}
 \begin{enumerate}[leftmargin=*]
  \item
   Für einen $K$-Vektorraum $V$ ist $\End_K(V) \coloneqq \Hom_K(V,V)$.
  \item
   Ist $V$ ein $K$-Vektorraum, so heißt $V^* \coloneqq \Hom(V,K)$ der \emph{Dualraum} von $V$.
 \end{enumerate}
\end{bem}





\section{Matrizenräume}


\subsection{Die (\texorpdfstring{$m \times n$}{mxn})-Matrizen \texorpdfstring{$\Mat(m \times n, K)$}{Mat(mxn,K)}}
Die ($m \times n$)-Matrizen $\Mat(m \times n, K)$ bilden zusammen mit der eintragsweisen Addition
\[
 (a_{ij})_{\substack{i=1,\dotsc,m \\ j=1,\dotsc,n}} + (b_{ij})_{\substack{i=1,\dotsc,m \\ j=1,\dotsc,n}}
 = (a_{ij}+b_{ij})_{\substack{i=1,\dotsc,m \\ j=1,\dotsc,n}}
\]
eine abelsche Gruppe. Zusammen mit der eintragsweisen Skalarmultiplikation
\[
 \lambda \cdot (a_{ij})_{\substack{i=1,\dotsc,m \\ j=1,\dotsc,n}}
 = (\lambda a_{ij})_{\substack{i=1,\dotsc,m \\ j=1,\dotsc,n}}
\]
wird $\Mat(m \times n, K)$ zu einem $K$-Vektorraum: Für alle Skalare $\lambda, \mu \in K$ und Matrizen $(a_{ij})_{i,j}, (b_{ij})_{i,j} \in \Mat(m \times n, K)$ ist
\begin{gather*}
 \begin{aligned}
  \lambda \cdot ((a_{ij})_{i,j} + (b_{ij})_{i,j})
  &= \lambda \cdot (a_{ij} + b_{ij})_{i,j}
  = (\lambda (a_{ij}+b_{ij}))_{i,j} \\
  &= (\lambda a_{ij} + \lambda b_{ij})_{i,j}
  = (\lambda a_{ij})_{i,j} + (\lambda b_{ij})_{i,j}
  = \lambda (a_{ij})_{i,j} + \lambda (b_{ij})_{i,j},
 \end{aligned}
\shortintertext{und}
 \begin{aligned}
  (\lambda+\mu) \cdot (a_{ij})_{i,j}
  &= ((\lambda+\mu)a_{ij})_{i,j} \\
  &= (\lambda a_{ij} + \mu a_{ij})_{i,j}
  = \lambda \cdot (a_{ij})_{i,j} + \mu \cdot (a_{ij})_{i,j},
 \end{aligned}
\shortintertext{sowie}
 \lambda \cdot (\mu \cdot (a_{ij})_{i,j})
 = \lambda \cdot (\mu \cdot a_{ij})_{i,j}
 = (\lambda \cdot \mu a_{ij})_{i,j}
 = (\lambda \cdot \mu) \cdot (a_{ij})_{i,j},
\end{gather*}
und für alle $(a_{ij})_{i,j} \in \Mat(m \times n, K)$ gilt
\[
 1 \cdot (a_{ij})_{i,j}
 = (1 \cdot a_{ij})_{i,j}
 = (a_{ij})_{i,j}.
\]
Ingesamt zeigt dies, dass $\Mat(m \times n, K)$ zusammen mit der üblichen eintragsweisen Addition und Skalarmultiplikation einen $K$-Vektorraum bildet.



\subsection{Diagonalmatrizen und (echte) obere/untere Dreiecksmatrizen}
%TODO: Tippen



\subsection{Die (schief)symmetrischen Matrizen}


\begin{defi}
 Es sei $A \in \Mat(m \times n, K)$. Dann ist die \emph{transponierte} Matrix die ($n \times m$)-Matrix $A^T$ definiert durch
 \begin{gather*}
  (A^T)_{ij} = A_{ji}
  \quad
  \text{für alle $1 \leq i \leq n$ und $1 \leq j \leq m$},
 \shortintertext{also}
  A^T
  =
  \begin{pmatrix}
   A_{11} & A_{12} & \cdots & A_{1n} \\
   A_{21} & A_{22} & \cdots & A_{2n} \\
   \vdots & \vdots & \ddots & \vdots \\
   A_{m1} & A_{m2} & \cdots & A_{mn}
  \end{pmatrix}^T
  =
  \begin{pmatrix}
   A_{11} & A_{21} & \cdots & A_{m1} \\
   A_{12} & A_{22} & \cdots & A_{m2} \\
   \vdots & \vdots & \ddots & \vdots \\
   A_{1n} & A_{2n} & \cdots & A_{mn}
  \end{pmatrix}.
 \end{gather*}
\end{defi}


\begin{bem}\label{bem: properties of the transpose}
 \begin{enumerate}[leftmargin=*]
  \item
   Anschaulich gesehen entsteht $A^T$ aus $A$ durch Spiegelung der Matrix an der Diagonalen.
  \item
   Transponieren ist linear, d.h.\ die Abbildung
   \[
    \mc{T} \colon \Mat(m \times n, K) \to \Mat(n \times m, K), A \mapsto A^T
   \]
   ist linear: Für alle Skalare $\lambda \in K$ und Matrizen $A, B \in \Mat(m \times n, K)$ gilt für alle $1 \leq i \leq n$ und $1 \leq j \leq m$, dass
   \[
    ((A+B)^T)_{ij}
    = (A+B)_{ji}
    = A_{ji} + B_{ji}
    = (A^T)_{ij} + (B^T)_{ij}
    = (A^T + B^T)_{ij}
   \]
   und somit $(A+B)^T = A^T + B^T$, sowie
   \[
    ((\lambda A)^T)_{ij}
    = (\lambda A)_{ji}
    = \lambda A_{ji}
    = \lambda (A^T)_{ij}
    = (\lambda A^T)_{ij}
   \]
   und somit $(\lambda A)^T = \lambda A^T$.
 \end{enumerate}
\end{bem}


\begin{defi}
 Eine Matrix $A \in \Mat(n \times n, K)$ heißt \emph{symmetrisch}, falls \mbox{$A^T = A$}, und \emph{schiefsymmetrisch}, bzw.\ \emph{alternierend}, falls $A^T = -A$. Es ist
 \begin{align*}
  \Sym_n(K) &\coloneqq \{A \in \Mat(n \times n, K) \mid \text{$A$ ist symmetrisch}\}
 \shortintertext{und}
  \Alt_n(k) &\coloneqq \{A \in \Mat(n \times n, K) \mid \text{$A$ ist schiefsymmetrisch}\}.
 \end{align*}
\end{defi}


\begin{bem}
 \begin{enumerate}[leftmargin=*]
  \item
   Eine quadratische Matrix $A \in \Mat(n \times n, K)$ ist nach der Definition der Transponierten genau dann symmetrisch (bzw.\ schiefsymmetrisch), falls $A_{ij} = A_{ji}$ (bzw.\ $A_{ij} = -A_{ji}$) für alle $1 \leq i,j \leq n$.
  \item
   Ist $A \in \Mat(m \times n, K)$ mit $A^T = A$ oder $A^T = -A$, so ist notwendigerweise $m = n$. Es genügt daher, den Begriff einer symmetrischen, bzw. schiefsymmetrischen Matrix für quadratische Matrizen zu definieren.
 \end{enumerate}
\end{bem}


\begin{bsp}
 Die Einheitsmatrix $I_n \in \Mat(n \times n, K)$ ist symmetrisch. Die Matrizen
 \[
  \begin{pmatrix}
    4 & 7 & -6 \\
    7 & 9 &  2 \\
   -6 & 2 &  4
  \end{pmatrix}
  \in \Mat(3 \times 3, \Qbb)
  \quad\text{und}\quad
  \begin{pmatrix}
   1        & \sqrt{2} & 3 \\
   \sqrt{2} & 4        & 5 \\
   3        & 5        & 6
  \end{pmatrix}
  \in \Mat(3 \times 3, \Rbb)
 \]
 sind symmetrisch. Die beiden Matrizen
 \[
  \begin{pmatrix}
      0 & i          & 2         & -3-i   \\
     -i & 0          & \sqrt{2}i &  5    \\
     -2 & -\sqrt{2}i & 0         & -6+2i \\
    3+i & -5         & 6-2i      & 0
  \end{pmatrix}
  \in \Mat(3 \times 3, \Cbb),
  \begin{pmatrix}
    0 & 8 &  9 & 1 \\
    5 & 0 &  6 & 8 \\
    4 & 7 &  0 & 3 \\
   12 & 5 & 10 & 0
  \end{pmatrix}
  \in \Mat(4 \times 4, \Fbb_{13})
 \]
 sind schiefsymmetrisch. Die Matrizen
 \[
  \begin{pmatrix}
   5 & 1 & 3 \\
   2 & 2 & 1 \\
   3 & 2 & 5
  \end{pmatrix}
  \in \Mat(3 \times 3, \Qbb)
  \quad\text{und}\quad
  \begin{pmatrix}
   1 & 1 & 1 \\
   0 & 1 & 0 \\
   1 & 1 & 1
  \end{pmatrix}
  \in \Mat(3 \times 3, \Fbb_2)
 \]
 sind weder symmetrisch noch schiefsymmetrisch.
\end{bsp}


Wir zeigen, dass $\Sym_n(K)$ und $\Alt_n(K)$ Untervektorräume von $\Mat(n \times n, K)$ bilden: Da $0^T = 0 = -0$ ist $0 \in \Sym_n(K)$ und $0 \in \Alt_n(K)$.

Für $A, B \in \Sym_n(K)$ und $\lambda \in A$ ist
\[
 (A+B)^T = A^T + B^T = A+B
 \quad\text{und}\quad
 (\lambda A)^T = \lambda A^T = \lambda A,
\]
also auch $A+B \in \Sym_n(K)$ und $\lambda A \in \Sym_n(K)$. Das zeigt, dass $\Sym_n(K)$ ein Untervektorraum von $\Mat(n \times n, K)$ ist.

Für $A, B \in \Alt_n(K)$ und $\lambda \in K$ ist
\begin{gather*}
 (A+B)^T = A^T+B^T = -A-B = -(A+B)
\shortintertext{und}
 (\lambda A)^T = \lambda A^T = \lambda(-A) = -\lambda A,
\end{gather*}
also auch $A+B \in \Alt_n(K)$ und $\lambda A \in \Alt_n(K)$. Das zeigt, dass $\Alt_n(K)$ ein Untervektorraum von $\Mat(n \times n, K)$ ist

\begin{bsp}
 \begin{enumerate}[leftmargin=*]
  \item
   Es sind
   \[
    \Sym_2(\Rbb) =
    \left\{
     \begin{pmatrix}
      a & b \\
      b & c
     \end{pmatrix}
    \,\middle|\,
      a,b,c \in \Rbb
    \right\}
    \quad\text{und}\quad
    \Alt_2(\Rbb) =
    \left\{
     \begin{pmatrix}
       0 & a \\
      -a & 0
     \end{pmatrix}
    \,\middle|\,
      a \in \Rbb
    \right\}.
   \]
   (Man bemerke, dass die Diagonaleintrage einer schiefsymmetrischen Matrix $A \in \Alt_n(\Rbb)$ alle $0$ seien müssen, da $A_{ii} = -A_{ii}$ für alle $1 \leq i \leq n$ gilt.)
  \item
   Es ist ist
   \begin{gather*}
    \Sym_4(\Qbb) =
    \left\{
     \begin{pmatrix}
      a & b & c & d \\
      b & e & f & g \\
      c & f & h & i \\
      d & g & i & j
     \end{pmatrix}
    \,\middle|\,
      a,b,c,d,e,f,g,h,i,j \in \Qbb
    \right\}.
   \shortintertext{sowie}
    \Alt_5(\Qbb) =
    \left\{
     \begin{pmatrix}
       0 &  a &  b &  c & d \\
      -a &  0 &  e &  f & g \\
      -b & -e &  0 &  h & i \\
      -c & -f & -h &  0 & j \\
      -d & -g & -i & -j & 0
     \end{pmatrix}
    \,\middle|\,
      a,b,c,d,e,f,g,h,i,j \in \Qbb
    \right\}.
   \end{gather*}
  \item
   Ist $K$ ein Körper mit $\kchar(K) = 2$, so ist $x = -x$ für alle $x \in K$. Daher ist dann $\Alt_n(K) = \Sym_n(K)$ für alle $n \in \Nbb$.
 \end{enumerate}
\end{bsp}


\begin{bem}
 \begin{enumerate}[leftmargin=*]
  \item
   Analog lässt sich zeigen, dass
   \[
    E_\lambda \coloneqq \{A \in \Mat(n \times n, K) \mid A^T = \lambda A\}
   \]
   für alle $\lambda \in K$ einen Untervektorraum ist. Dabei ist insbesondere $\Sym_n(K)= E_1$ und $\Alt_n(K) = E_{-1}$.
  \item
   Ist $\kchar(K) = 2$, so ist $1 = -1$ und somit $\Sym_n(K) = \Alt_n(K)$.
  \item
   Ist $\kchar(K) \neq 2$, so lässt sich jede Matrix $M \in \Mat(n \times n, K)$ als Summe $M = S + A$ mit $S \in \Sym_n(K)$ und $A \in \Alt_n(K)$ schreiben, wobei $A$ und $S$ beide eindeutig sind: Existiert eine solche Zerlegung, so ist sie Eindeutig, da dann $S = (M + M^T)/2$ und $A = (M - M^T)/2$. Ist andererseits $M \in \Mat(n \times n, K)$ eine beliebige quadratische Matrix, so ist $S = (M+M^T)/2$ symmetrisch und $A = (M-M^T)/2$ schiefsymmetrisch (wegen der Linearität des Transponierens) und es gilt $M = S+A$.
   
   Für $A \in \Mat(4 \times 4, \Cbb)$ mit
   \begin{gather*}
    A =
    \begin{pmatrix}
      6+8i &  1+8i &  2+7i \\
     -1-i  &  9-8i &  5-7i \\
     -7    & -3-6i &  7-2i
    \end{pmatrix}
   \shortintertext{ist}
    \begin{aligned}
     A
     &= \frac{1}{2}(A+A^T) + \frac{1}{2}(A-A^T) \\
     &=
     \left(
      \renewcommand{\arraystretch}{2.1}
      \begin{array}{ccc}
       6+8i                        & \dfrac{7}{2}i      & -\dfrac{5}{2}+\dfrac{7}{2}i \\
       \dfrac{7}{2}i               & 9 - 8i             & 1-\dfrac{13}{2}i \\
       -\dfrac{5}{2}+\dfrac{7}{2}i & 1 - \dfrac{13}{2}i & 7-2i
      \end{array}
     \right)
     +
     \left(
      \renewcommand{\arraystretch}{2.1}
      \begin{array}{ccc}
        0                          & 1+\dfrac{9}{2}i  & \dfrac{9}{2}+\dfrac{7}{2}i \\
       -1-\dfrac{9}{2}i            & 0                & 4-\dfrac{1}{2}i           \\
       -\dfrac{9}{2}-\dfrac{7}{2}i & -4+\dfrac{1}{2}i & 0
      \end{array}
     \right).
    \end{aligned}
   \end{gather*}
 \end{enumerate} 
\end{bem}


\subsection{Die spurlosen Matrizen \texorpdfstring{$\sll_n(K)$}{sln(K)}}
%TODO: Tippen





\section{Funktionenräume}
Wie in \ref{sec: maps into a vector space} gesehen wird $\Abb(X,K)$ vermöge der punktweisen Addition und Skalarmultiplikation zu einem $K$-Vektorraum, d.h.\ für alle $f,g \in \Abb(X,K)$ und $\lambda \in K$ ist
\[
 (f+g)(x) = f(x)+g(x)
 \quad\text{und}\quad
 (\lambda \cdot f)(x) = \lambda \cdot f(x)
 \quad\text{für alle $x \in X$}.
\]
Wir geben in diesem Abschnitt verschiedene Beispiele für Vektorräume von reellwertigen Funktionen, wie sie etwa in der Analysis eine Rolle spielen. Wir werden diese jeweils als Untervektorräume von Vektorräumen der Form $\Abb(X,\Rbb)$ konstruieren.


\subsection{Beschränkte Funktionen}
Es sei $X$ eine beliebige Menge. Eine Funktion $f \colon X \to \Rbb$ heißt \emph{beschränkt}, falls es eine Konstante $C \geq 0$ gibt, so dass $|f(x)| \leq C$ für alle $x \in X$. Es sei
\[
 B(X, \Rbb)
 \coloneqq \{f \colon X \to \Rbb \mid \text{$f$ ist beschränkt}\}
 \subseteq \Abb(X,\Rbb).
\]
Wir zeigen, dass $B(X,\Rbb)$ bezüglich der punktweisen Addition und Skalarmultiplikation einen $\Rbb$-Vektorraum bildet. Hierfür zeigen wir, dass $B(X, \Rbb) \subseteq \Abb(X,\Rbb)$ ein Untervektorraum ist.

Die Nullfunktion $N \in \Abb(X,\Rbb)$ ist beschränkt, denn für $C \coloneqq 0$ ist
\[
 |N(x)| = |0| = 0 \leq C 
 \quad\text{für alle $x \in X$}.
\]
Also ist die Nullfunktion $N$ beschränkt und deshalb $N \in B(X, \Rbb)$. Sind $f, g \in B(X, \Rbb)$, so gibt es Konstanten $C_f, C_g \geq 0$ mit
\[
 |f(x)| \leq C_f
 \quad\text{und}\quad
 |g(x)| \leq C_g
 \quad
 \text{für alle $x \in X$}.
\]
Für $C_{f+g} \coloneqq C_f + C_g$ ist $C_{f+g} \geq 0$ und nach der Dreieckungleichung ist
\[
 |(f+g)(x)|
 = |f(x) + g(x)|
 \leq |f(x)| + |g(x)|
 \leq C_f + C_g
 = C_{f+g}
 \quad
 \text{für alle $x \in X$}.
\]
Also ist auch $f+g$ beschränkt und somit $f+g \in B(X, \Rbb)$. Für $\lambda \in \Rbb$ und $C_{\lambda f} \coloneqq |\lambda| C_f \geq 0$ ist nach der Homogenität des Betrags
\[
 |(\lambda f)(x)|
 = |\lambda f(x)|
 = |\lambda| |f(x)|
 \leq |\lambda| C_f
 = C_{\lambda f}
 \quad
 \text{für alle $x \in X$}.
\]
Also ist auch $\lambda f$ beschränkt und somit $\lambda f \in B(X,\Rbb)$. Ingesamt zeigt dies, dass $B(X,\Rbb)$ ein Untervektorraum von $\Abb(X,\Rbb)$ ist.


\begin{bem}
 Ersetzt man $\Rbb$ und $\Cbb$, so enthält man den $\Cbb$-Vektorraum $B(X,\Cbb)$ der komplexwertigen beschränkten Funktionen auf $X$.
\end{bem}


\subsection{Gerade und ungerade Funktionen}
Eine Funktion $f \colon \Rbb \to \Rbb$ heißt \emph{gerade}, falls $f(x) = f(-x)$ für alle $x \in \Rbb$ und \emph{ungerade}, falls $f(-x) = -f(x)$ für alle $x \in \Rbb$. Es seien
\begin{align*}
 G(\Rbb,\Rbb) &\coloneqq \{f \colon \Rbb \to \Rbb \mid \text{$f$ ist gerade}\}
\shortintertext{und}
 U(\Rbb,\Rbb) &\coloneqq \{f \colon \Rbb \to \Rbb \mid \text{$f$ ist ungerade}\}.
\end{align*}
Wir zeigen, dass $G$ und $U$ mit der punktweisen Addition und Skalarmultiplikation einen $\Rbb$-Vektorraum, indem wir zeigen, dass $G(\Rbb, \Rbb) \subseteq \Abb(\Rbb,\Rbb)$ und $U(\Rbb, \Rbb) \subseteq \Abb(\Rbb, \Rbb)$ Untervektorräume sind.

Für die Nullfunktion $N \colon \Rbb \to \Rbb$ gilt $N(x) = 0 = N(-x)$ und $N(-x) = 0 = -0 = -N(x)$ für alle $x \in \Rbb$, also ist $N$ gerade und ungerade, und somit $N \in G(\Rbb, \Rbb)$ und $N \in U(\Rbb, \Rbb)$.

Sind $f,g \in G(\Rbb, \Rbb)$ und ist $\lambda \in \Rbb$, so ist
\begin{gather*}
 (f+g)(-x)
 = f(-x) + g(-x)
 = f(x) + g(x)
 = (f+g)(x)
 \quad\text{für alle $x \in \Rbb$}
\shortintertext{und}
 (\lambda f)(-x) = \lambda f(-x) = \lambda f(x) = (\lambda f)(x)
 \quad\text{für alle $x \in \Rbb$}.
\end{gather*}
Also sind dann auch $f+g$ und $\lambda f$ gerade und somit $f+g, \lambda f \in G(\Rbb, \Rbb)$. Das zeigt, dass $G(\Rbb, \Rbb)$ ein Untervektorraum von $\Abb(\Rbb, \Rbb)$ ist.

Sind $f,g \in U(\Rbb, \Rbb)$ und ist $\lambda \in \Rbb$, so ist
\begin{gather*}
 (f+g)(-x)
 = f(-x) + g(-x)
 = -f(x) - g(x)
 = -(f(x) + g(x))
 = -(f+g)(x)
\shortintertext{und}
 (\lambda f)(-x)
 = \lambda f(-x)
 = -\lambda f(x)
 = -(\lambda f)(x)
\end{gather*}
für alle $x \in \Rbb$. Also sind dann auch $f+g$ und $\lambda f$ ungerade und somit $f+g, \lambda f \in U(\Rbb, \Rbb)$. Das zeigt, dass $U(\Rbb, \Rbb)$ ein Untervektorraum von $\Abb(\Rbb, \Rbb)$ ist.


\begin{bem}
 \begin{enumerate}[leftmargin=*]
  \item
   Ist $f \colon \Rbb \to \Rbb$  eine beliebige Funktion, so gibt es eine eindeutige gerade Funktion $g \colon \Rbb \to \Rbb$ und eindeutige ungerade Funktion $u \colon \Rbb \to \Rbb$ mit $f = g+u$. Die Eindeutigkeit folgt daraus, falls $f = g+u$ mit $g$ gerade und $u$ ungerade, sich $g$ und $u$ durch $g(x) = (f(x) + f(-x))/2$ und $u(x) = (f(x) - f(-x))/2$ für alle $x \in \Rbb$ ergeben.
   
   Ist andererseits $f \colon \Rbb \to \Rbb$ beliebig, so definiert $g(x) \coloneqq (f(x)+f(-x))/2$ eine gerade Funktion und $u(x) \coloneqq (f(x)-f(-x))/2$ eine ungerade Funktion, und es gilt $f = g+u$. Das zeigt die Existenz.
  \item
   Für einen beliebigen Körper $K$ und zwei $K$-Vektorräume $V$ und $W$ kann man eine Funktion $f \colon V \to W$ als gerade definieren, falls $f(v) = f(-v)$ für alle $v \in V$, und als ungerade, falls $f(-v) = -f(v)$ für alle $v \in V$. Dann sind
   \begin{align*}
    G(V,W) &\coloneqq \{f \colon V \to W \mid \text{$f$ ist gerade}\}
   \shortintertext{und}
    U(V,W) &\coloneqq \{f \colon V \to W \mid \text{$f$ ist ungerade}\}
   \end{align*}
   Untervektorräume des $K$-Vektorraums $\Abb(V,W)$, und somit selber $K$-Vektorräume bezüglich der punktweisen Addition und Skalarmultiplikation.
   
   Ist $f \colon V \to W$ linear, so ist $f(-v) = -f(v)$ für alle $v \in V$, also $f$ ungerade. Daher ist $\Hom_K(V,W) \subseteq U(V,W)$ ein Untervektorraum.
 \end{enumerate}
\end{bem}



\subsection{Radialsymmetrische reellwertige Funktionen auf \texorpdfstring{$\Rbb^n$}{Rn}}
\begin{defi}
 Eine Funktion $f \colon \Rbb^n \to \Rbb$ heißt \emph{radialsymmetrisch}, falls es eine Funktion $\rho_f \colon [0,\infty) \to \Rbb$ gibt, so dass $f(x) = \rho(|x|)$ für alle $x \in \Rbb^n$.
\end{defi}

Es sei
\[
 R \coloneqq \{f \colon \Rbb^n \to \Rbb \mid \text{$f$ ist radialsymmetrisch}\}.
\]
Wir zeigen, dass $R$ ein Vektorraum bezüglich der punktweisen Addition und Skalarmultiplikation ist, indem wir zeigen, dass $R \subseteq \Abb(\Rbb^n, \Rbb)$ ein Untervektorraum ist:

Die Nullfunktion $N \colon \Rbb^n \to \Rbb$ ist radialsymmetrisch, denn für $\rho_N \colon [0,\infty) \to \Rbb$, $r \mapsto 0$ ist
\[
 N(x) = 0 = \rho_N(|0|)
 \quad
 \text{für alle $x \in \Rbb^n$}.
\]
Also ist $N \in R$. Sind $f,g \in R$, so gibt es Funktionen $\rho_f, \rho_g \colon [0,\infty) \to \Rbb$ mit $f(x) = \rho_f(|x|)$ und $g(x) = \rho_g(|x|)$ für alle $x \in \Rbb^n$. Für $\rho_{f+g} = \rho_f + \rho_g$ ist daher
\[
 (f+g)(x)
 = f(x) + g(x)
 = \rho_f(|x|) + \rho_g(|x|)
 = (\rho_f + \rho_g)(|x|)
 = \rho_{f+g}(|x|)
\]
für alle $x \in \Rbb^n$. Also ist $f+g$ radialsymmetrisch und deshalb auch $f+g \in R$. Ist ferner $\lambda \in \Rbb$, so gilt für $\rho_{\lambda f} = \lambda \rho_f$, dass
\[
 (\lambda f)(x)
 = \lambda f(x)
 = \lambda \rho_f(|x|)
 = (\lambda \rho_f)(|x|)
 = \rho_{\lambda f}(|x|)
 \quad
 \text{für alle $x \in \Rbb^n$}.
\]
Also ist auch $\lambda f$ radialsymmetrisch und somit $\lambda f \in R$. Dies zeigt, dass \mbox{$R \subseteq \Abb(\Rbb^n, \Rbb)$} ein Untervektorraum ist.



\subsection{Stetige reellwertige Funktionen auf \texorpdfstring{$\Rbb$}{R}}
Wir werden für dieses Beispiel den Begriff konvergenter Folgen nutzen. An die entsprechenden Begriffe und Eigeschaften erinnern wir in \ref{sec: sequences and series}. Wir werden insbesondere die Eigenschaften aus Bemerkung \ref{bem: properties of convergent sequences} nutzen.

\begin{defi}
 Eine Funktion $f \colon \Rbb \to \Rbb$ heißt \emph{stetig}, falls für jede konvergente Folge reeller Zahlen $(x_n)_{n \in \Nbb}$ auch die Bildfolge $(f(x_n))_{n \in \Nbb}$ konvergiert und dann auch
 \[
  f\left( \lim_{n \to \infty} x_n \right)
  = \lim_{n \to \infty} f(x_n).
 \]
 Es sei
 \[
  C(\Rbb,\Rbb) \coloneqq \{f \colon \Rbb \to \Rbb \mid \text{$f$ ist stetig}\}.
 \]
 (Der Buchstabe C steht hier für \emph{continuous}, die englische Übersetzung von \emph{stetig}.)
\end{defi}

Wir zeigen, dass Summen und skalare Vielfache von stetigen Funktionen wieder stetig sind, dass also $C(\Rbb,\Rbb)$ zusammen mit der punktweisen Addition und Skalarmultiplikation einen Vektorraum bildet, indem wir zeigen, dass $C(\Rbb, \Rbb) \subseteq \Abb(\Rbb, \Rbb)$ ein Untervektorraum ist.

Für die Nullfunktion $N \colon \Rbb \to \Rbb$ ist $N(x) = 0$ für alle $x \in \Rbb$. Ist $(x_n)_{n \in \Nbb}$ eine konvergente Folge reeller Zahlen, so ist deshalb die Bildfolge $(N(x_n))_{n \in \Nbb}$ die konstante Nullfolge. Inbesondere ist die Bildfolge konvergent und
\[
 \lim_{n \to \infty} f(x_n)
 = \lim_{n \to \infty} 0
 = 0
 = f\left( \lim_{n \to \infty} x_n \right).
\]
Also ist die Nullfunktion stetig und somit $N \in C(\Rbb,\Rbb)$.

Sind $f,g \in C(\Rbb,\Rbb)$ und ist $(x_n)_{n \in \Nbb}$ eine konvergente Folge reeller Zahlen, so konvergieren auch die beiden Folgen $(f(x_n))_{n \in \Nbb}$ und $(g(x_n))_{n \in \Nbb}$ und
\[
 \lim_{n \to \infty} f(x_n) = f\left( \lim_{n \to \infty} x_n \right)
 \quad\text{und}\quad
 \lim_{n \to \infty} g(x_n) = g\left( \lim_{n \to \infty} x_n \right).
\]
Daher ist konvergiert auch die Folge $(f(x_n) + g(x_n))_{n \in \Nbb} = ((f+g)(x_n))_{n \in \Nbb}$ und es gilt
\begin{align*}
 \lim_{n \to \infty} (f+g)(x_n)
 &= \lim_{n \to \infty} (f(x_n) + g(x_n))
 = \left(\lim_{n \to \infty} f(x_n)\right) + \left(\lim_{n \to \infty} g(x_n)\right) \\
 &=f\left( \lim_{n \to \infty} x_n \right) + g\left( \lim_{n \to \infty} x_n \right)
 = (f+g)\left( \lim_{n \to \infty} x_n \right).
\end{align*}
Also ist auch $f+g$ stetig und somit $f+g \in C(\Rbb, \Rbb)$.

Ist $\lambda \in \Rbb$, $f \in C(\Rbb, \Rbb)$ und $(x_n)_{n \in \Nbb}$ eine konvergente Folge reeller Zahlen, so konvergiert auch die Folge $(f(x_n))_{n \in \Nbb}$ und es ist $\lim_{n \to \infty} f(x_n) = f(\lim_{n \to \infty} x_n)$. Daher konvergiert auch die Folge $(\lambda f(x_n))_{n \in \Nbb} = ((\lambda f)(x_n))_{n \in \Nbb}$ und
\[
 \lim_{n \to \infty} (\lambda f)(x_n)
 = \lim_{n \to \infty}( \lambda f(x_n) )
 = \lambda \lim_{n \to \infty} f(x_n)
 = \lambda f\left( \lim_{n \to \infty} x_n \right)
 = (\lambda f) \left( \lim_{n \to \infty} x_n \right).
\]
Also ist auch $\lambda f$ stetig und somit $\lambda f \in C(\Rbb, \Rbb)$.

Ingesamt zeigt dies, dass $C(\Rbb, \Rbb)$ ein Untervektorraum von $\Abb(\Rbb, \Rbb)$ ist.


\begin{bem}
 Statt reellwertige Funktionen lassen sich auch komplexwertige Funktionen betrachten, und statt $\Rbb$ lässt sich auch $\Cbb$ als Definitionsbereich nutzen. So erhält man die $\Rbb$-Vektorräume
 \[
  C(\Rbb,\Rbb) = \{f \colon \Rbb \to \Rbb \mid \text{$f$ ist stetig}\}
  \quad\text{und}\quad
  C(\Cbb,\Rbb) = \{f \colon \Cbb \to \Rbb \mid \text{$f$ ist stetig}\}
 \]
 und die $\Cbb$-Vektorräume
 \[
  C(\Rbb,\Cbb) = \{f \colon \Rbb \to \Cbb \mid \text{$f$ ist stetig}\}
  \quad\text{und}\quad
  C(\Cbb,\Cbb) = \{f \colon \Cbb \to \Cbb \mid \text{$f$ ist stetig}\}.
 \]
 Fasst man $C(\Rbb, \Cbb)$ und $C(\Cbb, \Cbb)$ als $\Rbb$-Vektorräume auf, so sind $C(\Rbb, \Rbb) \subseteq C(\Rbb, \Cbb)$ und $C(\Cbb, \Rbb) \subseteq C(\Cbb, \Cbb)$ Untervektorräume.
\end{bem}



\subsection{Hölder-stetige Funktionen auf \texorpdfstring{$\Rbb$}{R}}
Es sei $\alpha > 0$.

\begin{defi}
 Eine Abbildung $f \colon \Rbb \to \Rbb$ heißt \emph{Hölder-stetig mit Exponent $\alpha$} falls es eine Konstante $C_f \geq 0$ gibt, so dass $|f(x)-f(y)| \leq C_f |x-y|$ für alle $x,y \in \Rbb$. Es sei
 \[
  C^\alpha(\Rbb)
  \coloneqq \{f \colon \Rbb \to \Rbb \mid \text{$f$ ist Hölder-stetig mit Exponent $\alpha$}\}.
 \]
\end{defi}

Wir zeigen, dass die Hölder-stetigen Funktionen mit Exponent $\alpha$ zusammen mit der punktweisen Addition und Skalarmultiplikation einen $\Rbb$-Vektorraum bildet, indem wir zeiges, dass $C^\alpha(\Rbb) \subseteq \Abb(\Rbb,\Rbb)$ ein Untervektorraum ist.

Für die Nullfunktion $N \colon \Rbb \to \Rbb$ und $C_0 \coloneqq 0$ ist
\[
 |N(x) - N(y)|
 = |0 - 0|
 = 0
 = 0 \cdot |x-y|^\alpha
 \leq C_0 \cdot |x-y|^\alpha
 \quad
 \text{für alle $x,y \in \Rbb$}.
\]
Also ist $N$ Hölder-stetig mit Exponent $\alpha$ und somit $N \in C^\alpha(\Rbb)$.

Sind $f,g \in C^\alpha(\Rbb)$, so gibt es Konstanten $C_f, C_g \geq 0$ mit $|f(x)-f(y)| \leq C_f |x-y|$ und $|g(x)-g(y)| \leq C_g |x-y|$ für alle $x,y \in \Rbb$. Für $C_{f+g} \coloneqq C_f + C_g$ ergibt sich mithilfe der Dreiecksungleichung, dass
\begin{align*}
 |(f+g)(x) - (f+g)(y)|
 &= |f(x)+g(x)-f(y)-g(y)|
 = |f(x)-f(y) + g(x)-g(y)| \\
 &\leq |f(x)-f(y)| + |g(x)-g(y)|
 \leq C_f |x-y|^\alpha + C_g |x-y|^\alpha \\
 &= (C_f + C_g) |x-y|^\alpha
 = C_{f+g} |x-y|^\alpha
\end{align*}
für alle $x,y \in \Rbb$. Also ist auch $f+g$ Hölder-stetig mit Exponent $\alpha$ und somit $f+g \in C^\alpha(\Rbb)$.

Ist $f \in C^\alpha(\Rbb)$ und $\lambda \in \Rbb$, so gibt es eine Konstante $C_f \geq 0$ mit $|f(x)-f(y)| \leq C_f |x-y|$ für alle $x,y \in \Rbb$. Für $C_{\lambda f} \coloneqq |\lambda| C_f$ ist deshalb
\begin{align*}
 |(\lambda f)(x) - (\lambda f)(y)|
 &= |\lambda f(x) - \lambda f(y)|
 = |\lambda (f(x) - f(y))| \\
 &= |\lambda| |f(x)-f(y)|
 \leq |\lambda| C_f |x-y|^\alpha
 = C_{\lambda f} |x-y|^\alpha
\end{align*}
für alle $x,y \in \Rbb$. Also ist auch $\lambda f$ Hölder-stetig mit Exponent $\alpha$ und somit $\lambda f \in C^\alpha(\Rbb)$.

Ingesamt zeigt dies, dass $C^\alpha(\Rbb) \subseteq \Abb(\Rbb, \Rbb)$ ein Untervektorraum ist.

\begin{bem}
 \begin{enumerate}[leftmargin=*]
  \item
   Es lässt sich zeigen, dass Hölder-stetige Funktionen bereits stetig sind (wie der Name vermuten lässt). Für $\alpha > 1$ ist zudem $C^\alpha(\Rbb) = \{0\}$, d.h.\ die Nullfunktionen ist die einzige Hölder-stetige Funktion $\Rbb \to \Rbb$ mit Exponent $\alpha > 1$.
  \item
   Hölder-stetige Funktionen mit Exponenten $\alpha = 1$ heißen auch \emph{Lipschitz-stetig}
 \end{enumerate}
\end{bem}


\subsection{Funktionen mit kompakten Träger}

%TODO: Tippen






\section{Folgenräume}
Für einen Körper $K$ sei
\[
 \ell(K)
 \coloneqq \Abb(\Nbb, K)
 = \{a \colon \Nbb \to K\}
 = \{(a_n)_{n \in \Nbb} \mid \text{$a_n \in K$ für alle $n \in K$}\}
\]
der Raum der Folgen in $K$. Wie bereits in \ref{sec: maps into a vector space} gesehen, ist $\ell(K)$ ein $K$-Vektorraum bezüglich der eintragsweisen Addition
\[
 (a_n)_{n \in \Nbb} + (b_n)_{n \in \Nbb} = (a_n + b_n)_{n \in \Nbb}
 \quad
 \text{für alle $(a_n)_{n \in \Nbb}, (b_n)_{n \in \Nbb} \in \ell(K)$}.
\]
und eintragsweisen Skalarmultiplikation
\[
 \lambda \cdot (a_n)_{n \in \Nbb}
 = (\lambda \cdot a_n)_{n \in \Nbb}
 \quad
 \text{für alle $\lambda \in K$ und $(a_n)_{n \in \Nbb} \in \ell(K)$}.
\]


\subsection{Beschränkte Folgen}
%TODO: Tippen


\subsection{Konvergente Folgen}
Für $\Kbb \in \{\Qbb, \Rbb, \Cbb\}$  sei
\[
 \mc{K}(\Kbb)
 \coloneqq
 \{(a_n)_{n \in \Nbb} \in \ell(\Kbb) \mid \text{$(a_n)_{n \in \Nbb}$ konvergiert (mit Grenzwert in $\Kbb$)}\}.
\]
Wir zeigen, dass $\mc{K}(\Kbb)$ ein $\Kbb$-Vektorraum bezüglich der eintragsweisen Addition und Skalarmultiplikation ist, indem wir zeigen, dass $\mc{K}(\Kbb) \subseteq \ell(\Kbb)$ ein Untervektorraum ist:

Wie bereits in Bemerkung \ref{bem: properties of convergent sequences} konvergiert die konstante Nullfolge $(0)_{n \in \Nbb}$ gegen $0$. Also ist $(0)_{n \in \Nbb} \in \mc{K}(\Kbb)$.

Sind $(x_n)_{n \in \Nbb}, (y_n)_{n \in \Nbb} \in \mc{K}(\Kbb)$, so konvergiert auch die Folge $(x_n + y_n)_{n \in \Nbb}$, weshalb auch $(x_n)_{n \in \Nbb} + (y_n)_{n \in \Nbb} = (x_n + y_n)_{n \in \Nbb} \in \mc{K}(\Kbb)$.

Ist $(x_n)_{n \in \Nbb} \in \mc{K}(\Kbb)$ und $\lambda \in \Kbb$, so konvergiert auch die Folge $(\lambda x_n)_{n \in \Nbb}$, weshalb auch $\lambda (x_n)_{n \in \Nbb} = (\lambda x_n)_{n \in \Nbb} \in \mc{K}(\Kbb)$.

Insgesamt zeigt dies, dass $\mc{K}(\Kbb)$ ein Untervektorraum von $\ell(\Kbb)$ ist.


\subsection{Cauchy-Folgen}
Für $\Kbb \in \{\Qbb, \Rbb, \Cbb\}$ sei
\[
 \mc{C}(\Kbb)
 \coloneqq
 \{(a_n)_{n \in \Nbb} \mid \text{$(a_n)_{n \in \Nbb}$ ist eine Cauchy-Folge}\}.
\]


\subsection{\texorpdfstring{$\ell^p$}{lp}-Räume}
%TODO: Tippen


\subsection{}
%TODO: Fast überall 0


\subsection{Rekursiv definierte Folgen}
%TODO: Tippen





\section{Potenzmengen als \texorpdfstring{$\Fbb_2$}{F2}-Vektorräume}
%TODO: Tippen





\section{Eigenräume}
\begin{defi}
 Es sei $V$ ein $K$-Vektorraum und $f \colon V \to V$ ein Endomorphismus. Ein Vektor $v \in V$ mit $v \neq 0$ heißt \emph{Eigenvektor} von $f$, falls es ein $\lambda \in K$ gibt, so dass $f(v) = \lambda v$. Der Skalar $\lambda$ heißt dann \emph{Eigenwert} von $f$. Für alle $\lambda \in K$ ist
 \begin{align*}
  \Eig(f,\lambda)
  \coloneqq&\, \{v \in V \mid f(v) = \lambda v\} \\ 
  =&\, \{v \in V \mid \text{$v$ ist ein Eigenvektor von $f$ zum Eigenwert $\lambda$}\} \cup \{0\}.
 \end{align*}
 der \emph{Eigenraum} von $f$ zum Eigenwert $\lambda$.
\end{defi}

Es sei $V$ ein $K$-Vektorraum, $f \colon V \to V$ ein Endomorphismus und $\lambda \in K$. Wir zeigen, dass der Eigenraum $\Eig(f,\lambda)$ ein Untervektorraum von $V$ ist.

Da $f(0) = 0 = \lambda \cdot 0$ ist $0 \in \Eig(f,\lambda)$. Für $v,w \in \Eig(f,\lambda)$ ist $f(v) = \lambda v$ und $f(w) = \lambda w$ und somit
\[
 f(v+w)
 = f(v) + f(w)
 = \lambda v + \lambda w
 = \lambda (v+w),
\]
also auch $v+w \in \Eig(f,\lambda)$. Ist $v \in \Eig(f,\lambda)$ und $\mu \in K$, so ist wegen $f(v) = \lambda v$ auch
\[
 f(\mu v)
 = \mu f(v)
 = \mu \lambda f(v)
 = \lambda (\mu v),
\]
also auch $\mu v \in \Eig(f,\lambda)$. Ingesamt zeigt dies, dass $\Eig(f,\lambda)$ ein Untervektorraum von $V$ ist.

\begin{bem}
 Die obigen Rechnungen lassen sich auch umgehen: Für $v \in V$ ist
 \[
  f(v) = \lambda v
  \iff
  f(v) - \lambda v = 0
  \iff 
  (f - \lambda \id_V)(v) = 0.
 \]
 Wegen der Vektorraumstruktur von $\End_K(V)$ ist auch $f - \lambda \id_V \colon V \to V$ linear. Also ist
 \[
  \ker(f - \lambda \id_V) = \Eig(f,\lambda)
 \]
 ein Untervektorraum.
\end{bem}


\begin{bsp}
 Es sei $\mc{T} \colon \Mat(n \times n, K) \to \Mat(n \times n, K)$, $A \mapsto A^T$ das Transponieren. Wie bereits in Bemerkung \ref{bem: properties of the transpose} gesehen ist $\mc{T}$ linear. Es ist
 \begin{gather*}
  \Eig(\mc{T},1)
  = \{A \in \Mat(n \times n, K) \mid A^T = 1 \cdot A = A\}
  = \Sym_n(K)
 \shortintertext{und}
  \Eig(\mc{T},-1)
  = \{A \in \Mat(n \times n, K) \mid A^T = (-1) \cdot A = -A\}
  = \Alt_n(K).
 \end{gather*}
 %TODO: Weitere Beispiele hinzufügen.
\end{bsp}








\section{Ausblick: \texorpdfstring{$\Qbb$}{Q}-Vektorräume und teilbare abelsche Gruppen}
%TODO: Tippen



\section{Produkte und direkte Summen}


\subsection{Produkte}
%TODO: Tippen


\subsection{Direkte Summen}
%TODO: Direkte Summen





\section{Freie Vektorräume}
%TODO: Tippen








