\section{Beispiele für Körper}


\subsection{Einige kleine Beispiele}
\begin{enumerate}[leftmargin=*]
 \item
  Die rationalen Zahlen $\Qbb$ und die reellen Zahlen $\Rbb$ bilden zusammen mit der üblichen Addition und Multiplikation jeweils einen Körper. $\Qbb$ ist ein Unterkörper von $\Rbb$.
 \item
  Die ganzen Zahlen $\Zbb$ bilden zusammen mit der üblichen Addition und Multiplikation \emph{keinen} Körper: Ansonsten wäre $(\Zbb\setminus\{0\}, \cdot)$ ein Körper, aber $2 \in \Zbb\setminus\{0\}$ besitzt kein multiplikativ Inverses in $\Zbb$.
 \item
  Für einen beliebigen Körper $K$ und $n \in \Nbb$, $n \geq 2$ bilden die quadratischen Matrizen $\Mat(n \times n, K)$ zusammen mit der gewöhnlichen Addition und Multiplikation von Matrizen keinen Körper: Zum einen ist die Multiplikation in diesem Fall nicht kommutativ, da
  \[
   \begin{pmatrix}
    1 & 0 \\
    0 & 0
   \end{pmatrix}
   \begin{pmatrix}
    0 & 1 \\
    0 & 0
   \end{pmatrix}
   =
   \begin{pmatrix}
    0 & 1 \\
    0 & 0
   \end{pmatrix}
   \quad\text{aber}\quad
   \begin{pmatrix}
    0 & 1 \\
    0 & 0
   \end{pmatrix}
   \begin{pmatrix}
    1 & 0 \\
    0 & 0
   \end{pmatrix}
   =
   \begin{pmatrix}
    0 & 0 \\
    0 & 0
   \end{pmatrix}.
  \]
  Zum anderen ist nicht jede Matrix $A \in \Mat(n \times n, K) \setminus \{0\}$ bezüglich der Matrixmultiplikation invertierbar, da etwa
  \[
   \begin{pmatrix}
    1 & 0 \\
    0 & 0
   \end{pmatrix}
   \begin{pmatrix}
    a & b \\
    c & d
   \end{pmatrix}
   =
   \begin{pmatrix}
    a & b \\
    0 & 0
   \end{pmatrix}
   \neq
   \begin{pmatrix}
    1 & 0 \\
    0 & 1
   \end{pmatrix}
  \]
  für alle $a,b,c,d \in K$.
\end{enumerate}



\subsection{Bilder von Körperhomomorphismen}
Sind $K$ und $L$ Körper und ist $\phi \colon K \to L$ ein Körperhomomorphismus, so ist $\im(\phi) \subseteq L$ ein Unterkörper:

Es ist $0 = \phi(0) \in \im(\phi)$. Für $x,y \in \im(\phi)$ gibt es $x', y' \in K$ mit $\phi(x') = x$ und $\phi(y') = y$, weshalb auch
\[
 x+y
 = \phi(x')+\phi(y')
 = \phi(x'+y')
 \in \im(\phi).
\]
Für $x \in \im(\phi)$ gibt es $x' \in K$ mit $\phi(x') = x$, weshalb auch
\[
 -x
 = -\phi(x')
 = \phi(-x') \in \im(\phi).
\]
Also ist $\im(\phi)$ eine Untergruppe der additiven Gruppe von $L$.

\begin{bem}
 Kürzer und eleganter lässt sich die obige Rechnung so zusammenfassen: Da $\phi$ insbesondere ein Gruppenhomomorphismus von der additiven Gruppe von $K$ in die additive Gruppe von $L$ ist, ist $\im(\phi) \subseteq L$ eine abelsche Untergruppe, da Bilder von Gruppenhomomorphismen stets Untergruppen sind. 
\end{bem}

Es ist $1 = \phi(1) \in \im(\phi)$. Sind $x,y \in \im(\phi)$, so gibt es $x',y' \in K$ mit $x = \phi(x')$ und $y = \phi(x')$, weshalb dann auch
\[
 x \cdot y = \phi(x') \cdot \phi(y') = \phi(x' \cdot y') \im(\phi).
\]
Ist schließlich $x \in \im(\phi)$ mit $x \neq 0$, so gibt es $x' \in K$ mit $x = \phi(x')$. Da $x \neq 0$ ist auch $x \neq 0$. Da $\phi$ ein Körperhomomorphismus ist, ist somit $\phi(x')$ invertierbar mit $\phi(x')^{-1} = \phi((x')^{-1}) \in \im(\phi)$, also $x^{-1} = \phi(x')^{-1} \in \im(\phi)$.

Insgesamt zeigt dies, dass $\im(\phi)$ ein Unterkörper von $L$ ist.

\begin{bem}
 \begin{enumerate}[leftmargin=*]
  \item
   Da $\phi$ als Körperhomomorphismus injektiv ist, ist die Abbildung $K \to \im(\phi), x \mapsto \phi(x)$ ein bijektiv und ein Körperhomomorphismus, also ein Körperisomorphismus. Inbesondere ist also $\im(\phi)$ isomorph zu $K$.
  \item
   Ist $L$ ein Körper und $K \subseteq L$ ein Unterkörper, so ist die Inklusion $K \to L$, $x \mapsto x$ ein Körperhomomorphismus. Es ist also jeder Unterkörper von $L$ gegeben als Bild eines Körperhomomorphismus $\phi \colon K \to L$, und wie bereits zuvor bemerkt $\im(\phi)$ ist isomorph zu $K$.
   
   Man kann deshalb einen Unterkörper $K$ eines Körpers $L$ statt als eine Teilmenge, die unter entsprechenden Rechenoperationen abgeschlossen ist, und somit selber wieder einen Körper bildet, auch als einen Körperhomomorphismus $K \to L$ definieren.
 \end{enumerate}
\end{bem}





\subsection{Die komplexen Zahlen \texorpdfstring{$\Cbb$}{C}}
Die komplexen Zahlen $\Cbb$ sind ein Körper, der die reellen Zahlen $\Rbb$ enthält, und in dem es ein Element $i \in \Cbb$ gibt, für das $i^2 = -1$. Wir geben an dieser Stelle zwei mögliche Konstruktionen der komplexen Zahlen an:



\subsubsection{Die komplexen Zahlen als besserer \texorpdfstring{$\Rbb^2$}{R2}}
Wir beginnen mit der additiven Gruppe $\Cbb \coloneqq \Rbb^2 = \{(a,b) \mid a,b \in \Rbb\}$. Die Addition ist gegeben durch
\[
 (a_1, b_1) + (a_2, b_2) = (a_1 + a_2, b_1 + b_2)
 \quad\text{für alle $(a_1, b_1), (a_2, b_2) \in \Cbb$}.
\]
Wir definieren zusätzlich eine Multiplikation $\cdot$ auf $\Cbb$ durch
\[
 (a_1, b_1) \cdot (a_2, b_2)
 = (a_1 a_2 - b_1 b_2, a_1 b_2 + a_2 b_1)
 \quad\text{für alle $(a_1, b_1), (a_2, b_2) \in \Cbb$}.
\]

\begin{beh}
 $\Cbb$ ist zusammen mit der obigen Addition und Multiplikation ein Körper.
\end{beh}
\begin{proof}
 Wir wissen bereits, dass $\Cbb$ zusammen mit der Addition $+$ eine abelsche Gruppe bildet. Das additiv neutrale Element ist $(0,0)$ und das additiv Inverse zu $(a,b) \in \Cbb$ ist $(-a,-b)$ Für alle $(a_1, b_1), (a_2, b_2), (a_3, b_3) \in \Cbb$ ist
 \begin{align*}
   &\, (a_1, b_1) \cdot ((a_2, b_2) \cdot (a_3, b_3))
  = (a_1, b_1) \cdot (a_2 a_3 - b_2 b_3, a_2 b_3 + a_3 b_2) \\
  =&\, (a_1 a_2 a_3 - a_1 b_2 b_3 - a_2 b_1 b_3 - a_3 b_1 b_2, a_2 a_3 b_1 - b_1 b_2 b_3 + a_1 a_2 b_3 + a_1 a_3 b_2) \\
  =&\, (a_1 a_2 - b_1 b_2, a_1 b_2 + a_2 b_1) \cdot (a_3, b_3)
  = ((a_1, b_1) \cdot (a_2, b_2)) \cdot (a_3, b_3),
 \end{align*}
 also ist die Multiplikation assoziativ. Für alle $(a_1, b_1), (a_2, b_2) \in \Cbb$ ist
 \begin{align*}
  (a_1, b_1) \cdot (a_2, b_2)
  &= (a_1 a_2 - b_1 b_2, a_1 b_2 + a_2 b_1) \\
  &= (a_2 a_1 - b_2 b_1, a_2 b_1 + a_1 b_2)
  = (a_2, b_2) \cdot (a_1, b_1),
 \end{align*}
 also ist die Multiplikation kommutativ. Für alle $(a,b) \in \Cbb$ ist
 \[
  (1,0) \cdot (a,b) = (1 \cdot a - 0 \cdot b, 0 \cdot a + 1 \cdot b) = (a,b),
 \]
 also ist $(1,0)$ neutral bezüglich der Multiplikation. Ist $(a,b) \in \Cbb$ mit $(a,b) \neq (0,0)$, so ist $a \neq 0$ oder $b \neq 0$, also $a^2+b^2 \neq 0$. Daher ist
 \[
  (a,b) \cdot \left(\frac{a}{a^2+b^2}, -\frac{b}{a^2+b^2}\right)
  = \left( a \frac{a}{a^2+b^2} + b \frac{b}{a^2+b^2}, b \frac{a}{a^2+b^2} - a \frac{b}{a^2+b^2} \right)
  = (1,0).
 \]
 Also ist $(a/(a^2+b^2), -b/(a^2+b^2))$ multiplikativ invers zu $(a,b)$.
 
 Die Distributivität folgt daraus, dass für alle $(a_1, b_1), (a_2, b_2), (a_3, b_3) \in \Cbb$
 \begin{align*}
   &\, (a_1, b_1) \cdot ((a_2, b_2) + (a_3, b_3)) \\
  =&\,    (a_1, b_1) \cdot (a_2 + a_3, b_2 + b_3) \\
  =&\, (a_1 (a_2 + a_3) - b_1 (b_2 + b_3), a_1 (b_2 + b_3) + (a_2 + a_3) b_1) \\
  =&\, (a_1 a_2 + a_1 a_3 - b_1 b_2 - b_1 b_3, a_1 b_2 + a_1 b_3 + a_2 b_1 + a_3 b_1) \\
  =&\, (a_1 a_2 - b_1 b_2, a_1 b_2 + a_2 b_1) + (a_1 a_3 - b_1 b_3, a_1 b_3 + a_3 b_1) \\
  =&\, (a_1, b_1) \cdot (a_2, b_2) + (a_1, b_1) \cdot (a_3, b_3).
 \end{align*}
 Ingesamt zeigt dies, dass $\Cbb$ zusammen mit der angegebenen Addition und Multiplikation ein Körper ist.
\end{proof}

Wir wollen noch etwas Notation einführen, um den Umgang mit $\Cbb$ angenehmer zu gestalten: Wir schreiben $0 = (0,0)$ und $1 = (1,0)$. Wir bemerken zunächst, dass für alle $a, a' \in \Rbb$
\begin{gather*}
 (a,0) + (a',0) = (a + a', 0)
\shortintertext{und}
 (a,0) \cdot (a',0) = (a a' - 0 \cdot 0, a \cdot 0 + 0 \cdot a') = (a a', 0)
\end{gather*}
und $1 = (1, 0)$. Deshalb ist die Abbildung $\phi \Rbb \colon \Cbb$, $a \mapsto (a,0)$ ein Körperhomomorphismus. Daher ist $\im(\phi) = \{(a,0) \mid a \in \Rbb\} \subseteq \Cbb$ ein Unterkörper, den wir durch $\phi$ mit $\Rbb$ identifizieren. Wir unterscheiden also im Folgenden nicht mehr zwischen $a \in \Rbb$ und $(a,0) \in \Cbb$.

Wir schreiben $i \coloneqq (0,1) \in \Cbb$. Für jedes $b \in \Rbb$ gilt
\[
 b \cdot i = (b,0) \cdot (0,1) = (b \cdot 0 - 0 \cdot 1, b \cdot 1 + 0 \cdot 0) = (0,b).
\]
Ein beliebiges Element $z = (a,b) \in \Cbb$ lässt sich daher als
\[
 z = (a,b) = (a,0) + (0,b) = a + bi
\]
schreiben. Die beiden reellen Zahlen $a,b \in \Rbb$ mit $z = a + bi$ sind eindeutig, da $z = (a,b)$.

Bemerke, dass $i^2 = (0,1) \cdot (0,1) = (0 \cdot 0 - 1 \cdot 1, 0 \cdot 1 + 1 \cdot 0) = (-1, 0) = -1$. Die Addition lässt sich nun auch so darstellen, dass für alle $a_1, a_2, b_1, b_2 \in \Rbb$
\[
 (a_1 + b_1 i) + (a_2 + b_2 i)
 = a_1 + a_2 + b_1 i + b_2 i
 = (a_1 + a_2) + (b_1 + b_2) i,
\]
und die Multiplikation so, dass
\begin{align*}
 (a_1 + b_1 i)(a_2 + b_2 i)
 &= a_1 a_2 + a_1 b_2 i + a_2 b_1 i + b_1 b_2 i^2 \\
 &= a_1 a_2 - b_1 b_2 + a_1 b_2 i + a_2 b_1 i \\
 &= (a_1 a_2 - b_1 b_2) + (a_1 b_2 + a_2 b_1) i.
\end{align*}



\subsubsection{Die komplexen Zahlen als reelle (\texorpdfstring{$2 \times 2$}{2x2})-Matrizen}
Unabhängig von der ersten Methode geben wir noch eine weitere Konstruktion der komplexen Zahlen an: Es sei
\[
 C \coloneqq
 \left\{
  \begin{pmatrix}
   a &           -b \\
   b & \phantom{-}a
  \end{pmatrix}
  \,\middle|\,
  a,b \in \Rbb
 \right\}
 \subseteq \Mat(2 \times 2, \Rbb).
\]
Wir zeigen, dass $C$ zusammen mit der gewöhnlichen Matrixaddition und -multiplikation einen Körper bildet:

Für alle $a, b \in \Rbb$ schreiben wir abkürzend
\[
 Z(a,b)
 \coloneqq
 \begin{pmatrix}
  a &           -b \\
  b & \phantom{-}a
 \end{pmatrix}.
\]
Für alle $a_1, a_2, b_1, b_2 \in \Rbb$ ist
\begin{gather*}
 \begin{aligned}
  Z(a_1, b_1) + Z(a_2, b_2)
  &=
  \begin{pmatrix}
   a_1 &           -b_1 \\
   b_1 & \phantom{-}a_1
  \end{pmatrix}
  +
  \begin{pmatrix}
   a_2 &           -b_2 \\
   b_2 & \phantom{-}a_2
  \end{pmatrix} \\
  &=
  \begin{pmatrix}
   a_1+a_2 &           -b_1+b_2 \\
   b_1+b_2 & \phantom{-}a_1+a_2
  \end{pmatrix}
  =
  Z(a_1+a_2, b_1+b_2)
 \end{aligned}
\shortintertext{und außerdem}
 \begin{aligned}
  Z(a_1,b_1) \cdot Z(a_2,b_2)
  &=
  \begin{pmatrix}
   a_1 &           -b_1 \\
   b_1 & \phantom{-}a_1
  \end{pmatrix}
  \begin{pmatrix}
   a_2 &           -b_2 \\
   b_2 & \phantom{-}a_2
  \end{pmatrix} \\
  &=
  \begin{pmatrix}
   a_1 a_2 - b_1 b_2 &           -a_1 b_2 - a_2 b_1 \\
   a_1 b_2 + a_2 b_1 & \phantom{-}a_1 a_2 - b_1 b_2
  \end{pmatrix}
  = Z(a_1 a_2 - b_1 b_2, a_1 b_2 + a_2 b_1).
 \end{aligned}
\end{gather*}

Um zu zeigen, dass $C$ bezüglich der Addition eine abelsche Gruppe bildet genügt es zu zeigen, dass $C$ eine Untergruppe der additiven Gruppe von $\Mat(2 \times 2, \Rbb)$ bildet.

Es ist $0 = Z(0,0) \in C$ und für alle $a_1, a_2, b_1, b_2 \in \Rbb$ ist
\[
 Z(a_1, b_1) + Z(a_2, b_2) = Z(a_1 + a_2, b_1 + b_2) \in C.
\]
Für alle $a,b \in \Rbb$ ist außerdem
\[
 -Z(a,b)
 =
 -\begin{pmatrix}
  a &           -b \\
  b & \phantom{-}a
 \end{pmatrix}
 =
 \begin{pmatrix}
  -a & \phantom{-}b \\
  -b &           -a
 \end{pmatrix}
 = Z(-a,-b) \in C.
\]
Also ist $C$ eine abelsche Untergruppe der additiven Gruppe von $\Mat(2 \times 2, \Rbb)$.

Die Assoziativität der Multiplikation ist bekannt. Die Matrixmultiplikation ist auf $C$ kommutativ, da für alle $a_1, a_2, b_1, b_2 \in \Rbb$
\begin{align*}
 Z(a_1, b_1) \cdot Z(a_2, b_2)
 &= Z(a_1 a_2 - b_1 b_2, a_1 b_2 + a_2 b_1) \\
 &= Z(a_2 a_1 - b_2 b_1, a_2 b_1 + a_1 b_2)
 = Z(a_2, b_2) \cdot Z(a_1, b_1).
\end{align*}

Da die Einheitsmatrix $I_2 \in \Mat(2 \times 2, \Rbb)$ das Einselement bezüglich der Matrixmultiplikation ist und $I_2 \in C$ ist $I_2$ das Einselement in $C$. Für $a,b \in \Rbb$ mit $Z(a,b) \neq 0$ ist $a \neq 0$ oder $b \neq 0$ und somit $a^2+b^2 \neq 0$. Da
\begin{align*}
  &\, Z(a,b) \cdot Z\left(\frac{a}{a^2+b^2}, -\frac{b}{a^2+b^2}\right) \\
 =&\, Z\left( a \frac{a}{a^2+b^2} + b\frac{b}{a^2+b^2}, b\frac{a}{a^2+b^2} - a\frac{b}{a^2+b^2} \right) \\
 =&\, Z(1, 0)
 = I_2
\end{align*}
ist $Z(a/(a^2+b^2), -b/(a^2+b^2))$ das multiplikativ Inverse zu $Z(a,b)$ (da die Matrixmultiplikation auf $C$ kommutativ ist, zeigt die obige Rechnung, dass $Z(a/(a^2+b^2), -b/(a^2+b^2))$ bereits beidseitig Invers zu $Z(a,b)$ ist).

Insgesamt zeigt dies, dass $C$ mit der üblichen Matrixaddition und -multiplikation einen Körper bildet.


\subsubsection{Äquivalenz der beiden Konstruktionen}
Die beiden Konstruktionen der komplexen Zahlen sind insofern äquivalent, als dass die Abbildung
\[
 \phi \colon \Cbb \to C, a + ib \mapsto Z(a,b) \quad \text{für alle $a,b \in \Rbb$}
\]
ein Körperisomorphismus ist: Die Bijektivität von $\phi$ folgt daraus, dass die beiden Abbildungen $\psi_1 \colon \Rbb^2 \to \Cbb, (a,b) \mapsto a+ib$ und $\psi_2 \colon \Rbb^2 \to C, (a,b) \mapsto Z(a,b)$ bijektiv sind, und daher auch $\phi = \psi_2 \psi_1^{-1}$. Außerdem ist $\phi(1) = \phi(1)= Z(1,0) = I_2$. Für alle $a,a',b,b' \in \Rbb$ ist zudem
\begin{gather*}
 \begin{aligned}
  \phi((a+ib)+(a'+ib'))
  &= \phi((a+a')+i(b+b'))
  = Z(a+a',b+b') \\
  &= Z(a,b) + Z(a',b')
  = \phi(a+ib) + \phi(a'+ib')
 \end{aligned}
\shortintertext{und}
 \begin{aligned}
  \phi( (a+ib) \cdot (a'+ib') )
  &= \phi( (aa'-bb')+i(ab'+a'b) ) \\
  &= Z(aa'-bb', ab'+a'b)
  = Z(a,b) \cdot Z(a',b').
 \end{aligned}
\end{gather*}
Ingesamt zeigt dies, dass $\phi$ ein Körperisomorphismus ist.





\subsection{Die endlichen Körper \texorpdfstring{$\Fbb_p$}{Fp}}
Es sei $n \in \Nbb$, $n \geq 1$. In \ref{ss: ZnZ} haben wir die endlichen abelschen Gruppen $\Zbb/n\Zbb$ konstruiert, indem wir auf $\Zbb$ die Äquivalenzrelation $\sim$ mit
\[
 k \sim l \iff \exists s \in \Zbb : k = l + sn
\]
definieren und auf der Menge der Äquivalenzklassen $\Zbb/n\Zbb = \Zbb/{\sim}$ die Addition
\[
 [k] + [l] = [k+l]
 \quad
 \text{für alle $k,l \in \Zbb$}
\]
definieren.

Die Multiplikation von $\Zbb$ liefert auf $\Zbb/n\Zbb$ eine Multiplikation durch
\[
 [k] \cdot [l] = [k \cdot l]
 \quad
 \text{für alle $k,l \in \Zbb$}.
\]
Wir zeigen im Folgenden, dass diese Multiplikation wohldefiniert ist, und dass $\Zbb/n\Zbb$ zusammen mit der obigen Addition und Multiplikation genau dann einen Körper bildet, wenn $n$ eine Primzahl ist.

Wir zeigen zunächst, dass die Multiplikation wohldefiniert ist: Es seien $k, k', l, l' \in \Zbb$ mit $k \sim k'$ und $l \sim l'$. Dann gibt es $s,t \in \Zbb$ mit $k = k' + sn$ und $l = l' + tn$. Deshalb ist
\[
 k \cdot l
 = (k' + sn) \cdot (l' + tn)
 = k' \cdot l' + l'sn + k'tn + stn^2
 = k' \cdot l' + (l's + k't + stn)n,
\]
also auch $k \cdot l \sim k' \cdot l'$. Also ist die Multiplikation wohldefiniert.

Für alle $k_1, k_2, k_3 \in \Zbb$ ist
\[
 [k_1] \cdot ([k_2] \cdot [k_3])
 = [k_1] \cdot [k_2 \cdot k_3]
 = [k_1 \cdot k_2 \cdot k_3]
 = [k_1 \cdot k_2] \cdot [k_3]
 = ([k_1] \cdot [k_3]) \cdot [k_3],
\]
also ist die Multiplikation assoziativ. Für alle $k, l \in \Zbb$ ist
\[
 [k] \cdot [l]
 = [k \cdot l]
 = [l \cdot k]
 = [l] \cdot [k],
\]
also ist die Multiplikation kommutativ. Für alle $k \in \Zbb$ ist
\[
 [1] \cdot [k] = [1 \cdot k] = [k],
\]
also ist $[1]$ ein Einselement bezüglich der Multiplikation (dass auch $[k] \cdot [1] = [k]$ folgt sofort mithilfe der Kommutativität).

Es bleibt zu zeigen, dass $\Zbb/n\Zbb$ genau dann ein Körper ist, wenn $n$ prim ist.

Angenommen, $p$ ist eine Primzahl. Es sei $x \in \Zbb/p\Zbb$ mit $x \neq 0$, also $x \neq [0]$. Dann gibt es $k \in \Zbb$ mit $0 < k < p$, so dass $x = [k]$. Nach Euklid gibt es $s,t \in \Zbb$ mit $sk + tp = \ggT(k,p)$. Da $0 < k < p$ und $p$ prim ist, sind $k$ und $p$ teilerfremd. Also ist $\ggT(k,p) = 1$ und somit $sk + tp = 1$. Es folgt, dass
\[
 [s] \cdot x
 = [s] \cdot [k]
 = [sk]
 = [sk + tp]
 = [1],
\]
also $[s] \cdot x = 1$. Somit ist $x$ multiplikativ invertierbar.

Angenommen, $n \in \Nbb$, $n \geq 1$ ist keine Primzahl. Ist $n = 1$, so ist $\Zbb/n\Zbb = \Zbb/1\Zbb = \{[0]\}$. Insbesondere ist daher $[0] = [1]$, also $\Zbb/n\Zbb$ kein Körper. Wir betrachten als den Fall, dass zusätzlich $n \geq 2$. Da $n$ nach Annahme nicht prim ist, gibt es $1 < \leq k,l < n$ mit $n = k \cdot l$. Da $0 < k,l < n$ ist $[k] \neq [0]$ und $[l] \neq [0]$. Da $n = k \cdot l$ ist aber
\[
 [k] \cdot [l]
 = [k \cdot l]
 = [n]
 = [0].
\]
Wäre $\Zbb/n\Zbb$ ein Körper, so würde aus $[k] \cdot [l]$ aber folgen, dass $[k] = 0$ oder $[l] = 0$, was im Widerspruch zu $[k] \neq [0]$ oder $[l] \neq [0]$ stünde. Also ist $\Zbb/n\Zbb$ unter den obigen Annahmen kein Körper.

Für eine Primzahl $p$ bezeichnet man den Körper $(\Zbb/p\Zbb, +, \cdot)$ als $\Fbb_p$. Dabei identifiziert man die Elemente von $\Zbb/p\Zbb$ wie bereits in \ref{ss: ZnZ} erläutert mit den entsprechenden Repräsentanten aus $\{0, \dotsc, p-1\}$, hat also $\Fbb_p = \{0, \dotsc, p-1\}$. Bezeichnen $+$ und $\cdot$ die Addition und Multiplikation auf $\Zbb$ und $\tilde{+}$ und $\tilde{\cdot}$ die Multiplikation auf $\Fbb_p$, so gilt dann
\[
 x \mathbin{\tilde{+}} y = (x + y) \bmod p
 \quad\text{und}\quad
 x \mathbin{\tilde{\cdot}} y = (x \cdot y) \bmod p
 \quad
 \text{für alle $x,y \in \Fbb_p$}.
\]


\begin{bsp}
 \begin{enumerate}[leftmargin=*]
  \item
   Es ist $\Fbb_2 = \{0, 1\}$. Die Addition ist gegeben durch $0 + 0 = 1 + 1 = 0$ und $0 + 1 = 1 + 0 = 1$, und die Multiplikation durch $0 \cdot 0 = 0 \cdot 1 = 1 \cdot 0 = 0$ und $1 \cdot 1 = 1$.
  \item
   Es ist $\Fbb_3 = \{0, 1, 2\}$. Die Addition und Multiplikation sind wie in den folgenden Tabellen gegeben:
   \begin{center}
    \begin{tabular}{|c|c|c|c|}
     \hline
     $+$ & $0$ & $1$ & $2$ \\\hline
     $0$ & $0$ & $1$ & $2$ \\\hline
     $1$ & $1$ & $2$ & $0$ \\\hline
     $2$ & $2$ & $0$ & $1$ \\\hline
    \end{tabular}
    \quad und \quad
    \begin{tabular}{|c|c|c|c|}
     \hline
     $\cdot$ & $0$ & $1$ & $2$ \\\hline
     $0$     & $0$ & $0$ & $0$ \\\hline
     $1$     & $0$ & $1$ & $2$ \\\hline
     $2$     & $0$ & $2$ & $1$ \\\hline
    \end{tabular}
   \end{center}
  \item
   Es ist $\Fbb_5 = \{0, 1, 2, 3, 4\}$ und die Addition und Multiplikation sind wie in den folgenden Tabellen gegeben:
   \begin{center}
    \begin{tabular}{|c|c|c|c|c|c|}
     \hline
     $+$ & $0$ & $1$ & $2$ & $3$ & $4$ \\\hline
     $0$ & $0$ & $1$ & $2$ & $3$ & $4$ \\\hline
     $1$ & $1$ & $2$ & $3$ & $4$ & $0$ \\\hline
     $2$ & $2$ & $3$ & $4$ & $0$ & $1$ \\\hline
     $3$ & $3$ & $4$ & $0$ & $1$ & $2$ \\\hline
     $4$ & $4$ & $0$ & $1$ & $2$ & $3$ \\\hline
    \end{tabular}
    \quad und \quad
    \begin{tabular}{|c|c|c|c|c|c|}
     \hline
     $\cdot$ & $0$ & $1$ & $2$ & $3$ & $4$ \\\hline
     $0$     & $0$ & $0$ & $0$ & $0$ & $0$ \\\hline
     $1$     & $0$ & $1$ & $2$ & $3$ & $4$ \\\hline
     $2$     & $0$ & $2$ & $4$ & $1$ & $3$ \\\hline
     $3$     & $0$ & $3$ & $1$ & $4$ & $2$ \\\hline
     $4$     & $0$ & $4$ & $3$ & $2$ & $1$ \\\hline
    \end{tabular}
   \end{center}
  \item
   Für $\Fbb_7 = \{0, 1, 2, 3, 4, 5, 6\}$ sind die Tabellen der Addition und Multiplikation gegeben durch
   \begin{center}
    \begin{tabular}{|c|c|c|c|c|c|c|c|}
     \hline
     $+$ & $0$ & $1$ & $2$ & $3$ & $4$ & $5$ & $6$ \\\hline
     $0$ & $0$ & $1$ & $2$ & $3$ & $4$ & $5$ & $6$ \\\hline
     $1$ & $1$ & $2$ & $3$ & $4$ & $5$ & $6$ & $0$ \\\hline
     $2$ & $2$ & $3$ & $4$ & $5$ & $6$ & $0$ & $1$ \\\hline
     $3$ & $3$ & $4$ & $5$ & $6$ & $0$ & $1$ & $2$ \\\hline
     $4$ & $4$ & $5$ & $6$ & $0$ & $1$ & $2$ & $3$ \\\hline
     $5$ & $5$ & $6$ & $0$ & $1$ & $2$ & $3$ & $4$ \\\hline
     $6$ & $6$ & $0$ & $1$ & $2$ & $3$ & $4$ & $5$ \\\hline
    \end{tabular}
    \quad und \quad
    \begin{tabular}{|c|c|c|c|c|c|c|c|}
     \hline
     $\cdot$ & $0$ & $1$ & $2$ & $3$ & $4$ & $5$ & $6$ \\\hline
     $0$     & $0$ & $0$ & $0$ & $0$ & $0$ & $0$ & $0$ \\\hline
     $1$     & $0$ & $1$ & $2$ & $3$ & $4$ & $5$ & $6$ \\\hline
     $2$     & $0$ & $2$ & $4$ & $6$ & $1$ & $3$ & $5$ \\\hline
     $3$     & $0$ & $3$ & $6$ & $2$ & $5$ & $1$ & $4$ \\\hline
     $4$     & $0$ & $4$ & $1$ & $5$ & $2$ & $6$ & $3$ \\\hline
     $5$     & $0$ & $5$ & $3$ & $1$ & $6$ & $4$ & $2$ \\\hline
     $6$     & $0$ & $6$ & $5$ & $4$ & $3$ & $2$ & $1$ \\\hline
    \end{tabular}
   \end{center}
  \item
   Man beachte, dass die obigen Tabelle alle symmetrisch sind, da die Addition und Multiplikation kommutativ sind.
  \item
   In $\Fbb_{11}$ gilt etwa $1/9 = 5$ und $1/7 = 8$, sowie $10 + 6 + 8 - 5 = 8$.
  \item
   Wir wollen die Matrix $A \in \Mat(4 \times 5, \Fbb_{11})$ mit
   \[
    A =
    \begin{pmatrix}
     4 &  9 & 6 &  3 &  2 \\
     3 &  9 & 6 & 10 & 10 \\
     5 &  5 & 5 &  5 &  2 \\
     0 &  8 & 8 &  1 &  1
    \end{pmatrix}
   \]
   durch elementare Zeilenumformungen in Zeilenstufenform bringen. Da wir uns über $\Fbb_{11}$ bewegen, müssen wir hierfür nicht mit Brüchen hantieren.
   \begin{align*}
    \begin{pmatrix}
     4 & 9 & 6 &  3 &  2 \\
     3 & 9 & 6 & 10 & 10 \\
     5 & 5 & 5 &  5 &  2 \\
     0 & 8 & 8 &  1 &  1
    \end{pmatrix}
    &\xrightarrow[3\mathrm{I}]{}
    \begin{pmatrix}
     1 & 5 & 7 &  9 &  6 \\
     3 & 9 & 6 & 10 & 10 \\
     5 & 5 & 5 &  5 &  2 \\
     0 & 8 & 8 &  1 &  1
    \end{pmatrix}
    \xrightarrow[\substack{\mathrm{II}+8\mathrm{I} \\ \mathrm{III}+6\mathrm{I}}]{}
    \begin{pmatrix}
     1 & 5 & 7 & 9 & 6 \\
     0 & 5 & 7 & 5 & 3 \\
     0 & 2 & 3 & 4 & 5 \\
     0 & 8 & 8 & 1 & 1
    \end{pmatrix}
    \\
    &\xrightarrow[9\mathrm{II}]{}
    \begin{pmatrix}
     1 & 5 & 7 & 9 & 6 \\
     0 & 1 & 8 & 1 & 5 \\
     0 & 2 & 3 & 4 & 5 \\
     0 & 8 & 8 & 1 & 1
    \end{pmatrix}
    \xrightarrow[\substack{\mathrm{III}+9\mathrm{II} \\ \mathrm{IV}+3\mathrm{II}}]{}
    \begin{pmatrix}
     1 & 5 &  7 & 9 & 6 \\
     0 & 1 &  8 & 1 & 5 \\
     0 & 0 &  9 & 2 & 6 \\
     0 & 0 & 10 & 4 & 5
    \end{pmatrix}
    \\
    &\xrightarrow[5\mathrm{III}]{}
    \begin{pmatrix}
     1 & 5 &  7 &  9 & 6 \\
     0 & 1 &  8 &  1 & 5 \\
     0 & 0 &  1 & 10 & 8 \\
     0 & 0 & 10 &  4 & 5
    \end{pmatrix}
    \xrightarrow[\mathrm{IV}+\mathrm{III}]{}
    \begin{pmatrix}
     1 & 5 & 7 &  9 & 6 \\
     0 & 1 & 8 &  1 & 5 \\
     0 & 0 & 1 & 10 & 8 \\
     0 & 0 & 0 &  3 & 2
    \end{pmatrix}
    \\
    &\xrightarrow[4\mathrm{IV}]{}
    \begin{pmatrix}
     1 & 5 & 7 &  9 & 6 \\
     0 & 1 & 8 &  1 & 5 \\
     0 & 0 & 1 & 10 & 8 \\
     0 & 0 & 0 &  1 & 8
    \end{pmatrix}
    \xrightarrow[\substack{\mathrm{I}+2\mathrm{IV} \\ \mathrm{II}+10\mathrm{IV} \\ \mathrm{III}+\mathrm{IV}}]{}
    \begin{pmatrix}
     1 & 5 & 7 & 0 & 0 \\
     0 & 1 & 8 & 0 & 8 \\
     0 & 0 & 1 & 0 & 5 \\
     0 & 0 & 0 & 1 & 8
    \end{pmatrix}
    \\
    &\xrightarrow[\substack{\mathrm{I}+4\mathrm{III} \\ \mathrm{II}+3\mathrm{III}}]{}
    \begin{pmatrix}
     1 & 5 & 0 & 0 & 9 \\
     0 & 1 & 0 & 0 & 1 \\
     0 & 0 & 1 & 0 & 5 \\
     0 & 0 & 0 & 1 & 8
    \end{pmatrix}
    \xrightarrow[\mathrm{I}+6\mathrm{II}]{}
    \begin{pmatrix}
     1 & 0 & 0 & 0 & 4 \\
     0 & 1 & 0 & 0 & 1 \\
     0 & 0 & 1 & 0 & 5 \\
     0 & 0 & 0 & 1 & 8
    \end{pmatrix}
   \end{align*}  
 \end{enumerate}
\end{bsp}
 




\subsection{Quadratische Körpererweiterungen}
Die komplexen Zahlen $\Cbb$ entstehen aus den reellen Zahlen $\Rbb$ durch hinzufügen einer Quadratwurzel von $-1$, d.h.\ eines Elementes $i$ mit $i^2 = -1$. Wir wollen hier noch weiter Beispiele von Körpern angeben, die durch hinzufügen von Wurzeln aus bereits bekannten Körpern entstehen.


\subsubsection{Der Körper \texorpdfstring{$\Qbb[i]$}{Q[i]}}
Es sei
\[
 \Qbb[i]
 \coloneqq
 \{ q_1 + q_2 i \mid q_1, q_2 \in \Qbb \}
 \subseteq \Cbb
\]
Wir zeigen, dass $\Qbb[i]$ ein Unterkörper von $\Cbb$ ist. (Man bezeichnet $\Qbb[i]$ als „$\Qbb$ adjungiert $i$“.)

Es ist $0 = 0 + 0 \cdot i \in \Qbb[i]$. Für $z,w \in \Qbb[i]$ ist $z = q_1 + q_2 i$ und $w = p_1 + p_2 i$ mit $q_1, q_2, p_1, p_2 \in \Qbb$. Daher ist auch
\[
 z + w
 = (q_1 + q_2 i) + (p_1 + p_2 i)
 = (q_1 + p_1) + (q_2 + p_2) i
 \in \Qbb[i].
\]
Ist $z \in \Qbb$ so ist $z = q_1 + q_2 i$ mit $q_1, q_2 \in \Qbb$, weshalb auch
\[
 -z = -(q_1 + q_2 i) = (-q_1) + (-q_2) i \in \Qbb[i].
\]
Das zeigt, dass $\Qbb[i]$ eine Untergruppe der additiven Gruppe von $\Cbb$ ist.

Es ist auch $1 = 1 + 0 \cdot i \in \Qbb[i]$. Sind $z, w \in \Qbb[i]$ so gibt es $q_1, q_2, p_1, p_2 \in \Qbb$ mit $z = q_1 + i q_2$ und $w = p_1 + i p_2$. Es ist daher auch
\[
 z \cdot w
 = (q_1 + iq_2)(p_1 + ip_2)
 = (q_1 p_1 - q_2 p_2) + i(q_1 p_2 + q_2 p_1)
 \in \Qbb[i].
\]
Ist $z \in \Cbb[i]$ mit $z \neq 0$ so ist $z = q_1 + i q_2$ mit $q_1, q_2 \in \Qbb$ und $q_1 \neq 0$ oder $q_2 \neq 0$. Da $z \neq 0$ ist auch $q_1 - i q_2 = \overline{z} \neq 0$ Deshalb ist auch
\[
 \frac{1}{z}
 = \frac{1}{q_1 + i q_2}
 = \frac{q_1 - i q_2}{(q_1 + i q_2)(q_1 - i q_2)}
 = \frac{q_1 - i q_2}{q_1^2 + q_2^2}
 = \frac{q_1}{q_1^2 + q_2^2} + i \frac{-q_2}{q_1^2+q_2^2}
 \in \Qbb[i].
\]

Ingesamt zeigt dies, dass $\Qbb[i]$ ein Unterkörper von $\Cbb$ ist. $\Qbb[i]$ ist der kleinste Unterkörper von $\Cbb$, der $i$ enthält, d.h.\ ist $K \subseteq \Cbb$ ein Unterkörper mit $i \in K$, so ist bereits $\Qbb[i] \subseteq K$.



\subsubsection{Der Körper \texorpdfstring{$\Qbb\left[\sqrt{2}\right]$}{Q[sqrt(2)]}}
Nach gleichen Prinzip zeigen wir, dass
\[
 \Qbb\left[\sqrt{2}\right] \coloneqq \{ q_1 + q_2 \sqrt{2} \mid q_1, q_2 \in \Qbb \} \subseteq \Rbb
\]
ein Unterkörper von $\Rbb$ ist. (Man bezeichnet $\Qbb\left[\sqrt{2}\right]$ als „$\Qbb$ adjungiert $\sqrt{2}$“.)

Es ist $0 = 0 + 0 \cdot 0 \in \Qbb\left[\sqrt{2}\right]$. Sind $x,y \in \Qbb\left[\sqrt{2}\right]$, so gibt es $q_1, q_2, p_1, p_2 \in \Qbb$ mit $x = q_1 + q_2 \sqrt{2}$ und $y = p_1 + p_2 \sqrt{2}$. Es ist daher auch
\[
 x+y
 = (q_1 + q_2 \sqrt{2}) + (p_1 + p_2 \sqrt{2})
 = (q_1 + p_1) + (q_2 + p_2) \sqrt{2}
 \in \Qbb\left[\sqrt{2}\right].
\]
Ist $x \in \Qbb\left[\sqrt{2}\right]$ mit $x = q_1 + q_2 \sqrt{2}$ für $q_1, q_2 \in \Qbb$, so ist auch
\[
 -x = -(q_1 + q_2 \sqrt{2}) = (-q_1) + (-q_2) \sqrt{2} \in \Qbb\left[\sqrt{2}\right].
\]
Das zeigt, dass $\Qbb\left[\sqrt{2}\right]$ eine Untergruppe der additiven Gruppe von $\Rbb$ ist.

\begin{bem}
 Die obigen Rechnungen lassen sich auch großteils umgehen: Die Abbildung $\varphi \colon \Qbb^2 \to \Rbb$, $(q_1, q_2) \to \Qbb\left[\sqrt{2}\right]$ ist ein Gruppenhomomorphismus, da
 \begin{align*}
  \varphi( (q_1, q_2) + (p_1, p_2) )
  &= \varphi( (q_1 + p_1, q_2 + p_2) )
  = (q_1 + p_1) + (q_2 + p_2) \sqrt{2} \\
  &= (q_1 +  q_2 \sqrt{2}) + (p_1 + p_2 \sqrt{2})
  = \varphi((q_1, q_2)) + \varphi((p_1,p_2))
 \end{align*}
 für alle $(q_1, q_2), (p_1, p_2) \in \Qbb^2$. Daher ist
 \[
  \Qbb\left[\sqrt{2}\right]
  = \{q_1 + q_2 \sqrt{2} \mid q_1, q_2 \in \Qbb\}
  = \im(\varphi)
 \]
 eine Untergruppe der additiven Gruppe von $\Rbb$.
\end{bem}

Es ist $1 = 1 + 0 \cdot \sqrt{2} \in \Qbb\left[\sqrt{2}\right]$. Für $x, y \in \Qbb\left[\sqrt{2}\right]$ ist $x = q_1 + q_2 \sqrt{2}$ und $y = p_1 + p_2 \sqrt{2}$ mit $q_1, q_2, p_1, p_2 \in \Qbb$. Deshalb ist auch
\[
 x \cdot y
 = (q_1 + q_2 \sqrt{2}) (p_1 + p_2 \sqrt{2})
 = q_1 q_2 + 2 p_1 p_2 + (q_1 p_2 + q_2 p_1) \sqrt{2} \in \Qbb\left[\sqrt{2}\right].
\]

Es sei $x \in \Qbb\left[\sqrt{2}\right]$ mit $x \neq 0$. Dann ist $x = q_1 + q_2 \sqrt{2}$ mit $q_1, q_2 \in \Qbb$. Da $x \neq 0$ ist $q_1 \neq 0$ oder $q_2 \neq 0$. Es folgt, dass auch $q_1 - q_2 \sqrt{2} \neq 0$. Andernfalls wäre nämlich $\sqrt{2} = q_1/q_2$ (falls $q_2 \neq 0$) oder $1/\sqrt{2} = q_2/q_1$ und somit ebenfalls $\sqrt{2} = q_1/q_2$ (falls $q_1 \neq 0$). Dies stünde im Widerspruch zur Irrationalität von $\sqrt{2}$. Es folgt, dass auch
\begin{align*}
 \frac{1}{x}
 &= \frac{1}{q_1 + q_2 \sqrt{2}}
 = \frac{q_1 - q_2 \sqrt{2}}{(q_1 + q_2 \sqrt{2})(q_1 - q_2 \sqrt{2})} \\
 &= \frac{q_1 - q_2 \sqrt{2}}{q_1^2 - 2 q_2^2}
 = \frac{q_1}{q_1^2 - 2q_2^2} + \frac{-q_2}{q_1^2 - 2q_2^2} \sqrt{2}
 \in \Qbb\left[\sqrt{2}\right].
\end{align*}

Insgesamt zeigt dies, dass $\Qbb\left[\sqrt{2}\right]$ ein Unterkörper von $\Rbb$ ist. Man bezeichnet $\Qbb[\sqrt{2}]$ als „$\Qbb$ adjungiert $\sqrt{2}$“. $\Qbb\left[\sqrt{2}\right]$ ist der kleinste Unterkörper von $\Rbb$, der $\sqrt{2}$ enthält, d.h.\ ist $K \subseteq \Rbb$ ein Unterkörper mit $\sqrt{2} \in K$, so ist bereits $\Qbb\left[\sqrt{2}\right] \subseteq K$.


\subsubsection{\texorpdfstring{$K[\alpha]$}{K[α]} mit \texorpdfstring{$\alpha \notin K$}{α not in K} und \texorpdfstring{$\alpha^2 \in K$}{α² in K}}
Die Konstruktion von $\Qbb[i]$ und $\Qbb\left[\sqrt{2}\right]$ lässt sich wie folgt verallgemeimern: Es sei $L$ ein Körper und $K \subseteq L$ ein Unterkörper. Es gebe ein Element $d \in K$, das in $K$ keine Quadratwurzel besitzt, d.h.\ $x^2 \neq d$ für alle $x \in K$, das aber in $L$ eine Quadratwurzel $\omega$ besitzt, also $\omega \in L$ mit $\omega^2 = d$. Dann ist
\[
 K[\omega] \coloneqq \{x + y \omega \mid x,y \in K\}
\]
ein Unterkörper von $L$. $K[\omega]$ ist dann der kleinste Unterkörper von $L$, der $K$ und $\omega$ enthält. Wir zeigen, dass $K[\omega]$ ein Unterkörper von $L$ ist:

Es ist $0 = 0 + 0 \cdot \omega \in K[\omega]$. Für $z_1, z_2 \in K[\omega]$ gibt es $x_1, y_1, x_2, y_2 \in \Rbb$ mit $z_1 = x_1 + y_1 \omega$ und $z_2 = x_2 + y_2 \omega$, weshalb auch
\[
 z_1 + z_2
 = (x_1 + y_1 \omega) + (x_2 + y_2 \omega)
 = (x_1 + x_2) + (y_1 + y_2) \omega
 \in K[\omega].
\]
Für $z \in K[\omega]$ gibt es es $x,y \in K$ mit $z = x+y\omega$, weshalb auch
\[
 -z = -(x + y \omega) = (-x) + (-y) \omega \in K[\omega].
\]
Das zeigt, dass $K[\omega]$ eine Untergruppe der additiven Gruppe von $L$ ist.

\begin{bem}
 Auch hier lässt sich die Rechnung damit abkürzen, dass die Abbildung $\varphi \colon K^2 \to L$, $(x,y) \mapsto x+y\omega$ ein Gruppenhomomorphismus ist, da
 \begin{align*}
  \varphi((x,y) + (x',y'))
  &= \varphi(x+x', y+y') \\
  &= (x+x') + (y+y')\omega
  = (x + y\omega) + (x' + y'\omega)
 \end{align*}
 für alle $(x, y), (x', y') \in K^2$, und somit
 \[
  \im(\varphi)
  = \{\varphi((x,y)) \mid (x,y) \in K^2\}
  = \{x+y\omega \mid x,y \in K\}
  = K[\omega]
 \]
 eine Untergruppe.
\end{bem}

Es ist $1 = 1 + 0 \cdot \omega \in K[\omega]$. Für $z_1, z_2 \in K[\omega]$ gibt es $x_1, x_2, y_1, y_2 \in K$ mit $z_1 = x_1 + y_1 \omega$ und $z_2 = x_2 + y_2 \omega$. Da $\omega^2 = d \in K$ ist
\[
 z_1 \cdot z_2
 = (x_1 + y_1 \omega) (x_2 + y_2 \omega)
 = (x_1 x_2 + d y_1 y_2) + (x_1 y_2 + x_2 y_1) \omega
 \in K[\omega].
\]
Ist $z \in K[\omega]$ mit $z \neq 0$, so gibt es $x,y \in K$ mit $z = x+y\omega$ und $x \neq 0$ oder $y \neq 0$. Dann ist auch $x-y\omega \neq 0$. Ansonsten wäre nämlich $\omega = x/y \in K$ (falls $y \neq 0$) oder $\omega^{-1} = y/x \in K$ (falls $x \neq 0$) und somit auch dann $\omega = x/y \in K$. Dies wiederspräche aber der Annahme, dass $\omega \notin K$ (da $a^2 \neq d$ für alle $a \in K$). Deshalb ist auch
\[
 \frac{1}{z}
 = \frac{1}{x+y\omega}
 = \frac{x-y\omega}{(x+y\omega)(x-y\omega)}
 = \frac{x-y\omega}{x^2 + dy^2}
 = \frac{x}{x^2 + dy^2} + \frac{-y}{x^2 + dy^2}\omega \in K[\omega].
\]

Ingesamt zeigt dies, dass $K[\omega]$ ein Unterkörper von $L$ ist.


\subsubsection{Formales Hinzufügen von Wurzeln}
Das vorherige Beispiel zeigt: Ist $K$ ein Körper und besitzt $d \in K$ keine Quadratwurzel, und gibt es einen größeren Körper $L \supsetneq K$, in dem $d$ ein Quadratwurzel besitzt, so lässt sich $K$ zu einem neuem Körper $K'$ erweitern, in dem $d$ eine Quadratwurzel besitzt (nämlich eine aus $L$). Das „Problem“ hierbei ist allerdings, dass diese Konstruktion auf den größeren Körper $L$ angewiesen ist, in dem eine entsprechende Quadratwurzel existiert.

Wir wollen hier noch zeigen, wie man zu einem beliebigen Körper auf formale Art und Weise Quadratwurzeln hinzufügen kann. Die Konstruktion läuft analog dazu, wie sich $\Cbb$ als besserer $\Rbb^2$ konstruieren lässt.

Es sei also $K$ ein Körper und es sei $d \in K$ ein Element, dass keine Quadratwurzel in $K$ besitzt, d.h.\ $x^2 \neq d$ für alle $x \in K$. Wir definieren $L \coloneqq K^2$. $L$ ist eine abelsche Gruppe mit der Addition
\[
 (x,y) + (x', y') = (x+x', y+y')
 \quad
 \text{für alle $x,x',y,y' \in K$}.
\]
Wir definieren auf $L$ eine Multiplikation $\cdot$ durch
\[
 (x_1, y_1) \cdot (x_2, y_2) \coloneqq (x_1 x_2 + d y_1 y_2, x_1 y_2 + x_2 y_1).
\]
Wir zeigen, dass diese Multiplikation die abelsche Gruppe $L$ zu einem Körper erweitert: Die Multiplikation ist assoziativ, denn für alle $x_1, x_2, x_3, y_1, y_2, y_3 \in K$ ist
\begin{align*}
  &\, (x_1, y_1) \cdot ((x_2, y_2) \cdot (x_3, y_3)) \\
 =&\, (x_1, y_1) \cdot (x_2 x_3 + d y_2 y_3, x_2 y_3 + x_3 y_2) \\
 =&\, (x_1 (x_2 x_3 + d y_2 y_3) + d y_1 (x_2 y_3 + x_3 y_2), x_1 (x_2 y_3 + x_3 y_2) + (x_2 x_3 + d y_2 y_3) y_1) \\
 =&\, (x_1 x_2 x_3 + d x_1 y_2 y_3 + d x_2 y_1 y_3 + d x_3 y_1 y_2, x_1 x_2 y_3 + x_1 x_3 y_2 + x_2 x_3 y_1 + d y_1 y_2 y_3) \\
 =&\, ((x_1 x_2 + d y_1 y_2) x_3 + d (x_1 y_2 + x_2 y_1) y_3, (x_1 x_2 + d y_1 y_2) y_3 + x_3 (x_1 y_2 + x_2 y_1)) \\
 =&\, (x_1 x_2 + d y_1 y_2, x_1 y_2 + x_2 y_1) \cdot (x_3, y_3) \\
 =&\, ((x_1, y_1) \cdot (x_2, y_2)) \cdot (x_3, y_3).
\end{align*}
Die Multiplikation ist kommutativ, da für alle $x_1, x_2, y_1, y_2 \in K$
\begin{align*}
 (x_1, y_1) \cdot (x_2, y_2)
 &= (x_1 x_2 + d y_1 y_2, x_1 y_2 + x_2 y_1) \\
 &= (x_2 x_1 + d y_2 y_1, x_2 y_1 + x_1 y_2)
 = (x_2, y_2) \cdot (x_1, y_1).
\end{align*}
Für alle $x,y \in K$ ist
\[
 (1,0) \cdot (x,y)
 = (1 \cdot x + d \cdot 0 \cdot y, 1 \cdot y + 0 \cdot x)
 = (x,y),
\]
also ist $(1,0)$ neutral bezüglich der Multiplikation. Für $(x,y) \in L$ mit $(x,y) \neq 0$ ist $x \neq 0$ oder $y \neq 0$. Dann ist $x^2 - dy^2 \neq 0$: Ansonsten wäre $d = (x/y)^2$ (falls $y \neq 0$) oder $1/d = (y/x)^2$ (falls $x \neq 0$) und somit ebenfalls $d = (x/y)^2$. Dies würde im Widerspruch dazu stehen, dass $d$ keine Quadratwurzel in $K$ hat. Da also $x^2 - dy^2 \neq 0$ ist
\begin{align*}
  &\, (x,y) \cdot \left( \frac{x}{x^2-dy^2}, \frac{-y}{x^2-dy^2} \right) \\
 =&\, \left( x\frac{x}{x^2-dy^2} + dy\frac{-y}{x^2-dy^2}, x\frac{-y}{x^2-dy^2}+y\frac{x}{x^2-dy^2} \right)
 = (1,0),
\end{align*}
weshalb $(x/(x^2-dy^2), -y/(x^2-dy^2))$ multiplikativ invers zu $(x,y)$ ist.

Insgesamt zeigt dies, dass $L$ zusammen mit der angegebenen Addition und Multiplikation ein Körper ist.

Wie bereits bei der Konstruktion von $\Cbb$ führen wir noch Notation ein: Wir schreiben $1 = (1,0)$ und $0 = (0,0)$. Wir bemerken zunächst, dass für alle $x, x' \in K$
\begin{gather*}
 (x, 0) + (x', 0) = (x+x', 0)
\shortintertext{und}
 (x, 0) \cdot (x', 0)
 = (x \cdot x' + 0 \cdot 0, x \cdot 0 + 0 \cdot x')
 = (x \cdot x', 0).
\end{gather*}
Die Abbildung $\phi \colon K \to L$, $x \mapsto (x,0)$ ist also ein Körperhomomorphismus. Deshalb ist das Bild \mbox{$\im(\phi) = \{(x,0) \mid x \in K\} \subseteq L$} ein Unterkörper, den wir durch $\phi$ mit $K$ identifizieren. Wir unterscheiden also nicht zwischen dem Element $x \in K$ und dem entsprechenden Tupel $\phi(x) = (x,0) \in L$.

Wir schreiben außerdem $\alpha \coloneqq (0,1) \in L$. Man bemerke, dass
\[
 \alpha^2
 = (0,1) \cdot (0,1)
 = (0 \cdot 0 + d \cdot 1 \cdot 1, 0 \cdot 1 + 1 \cdot 0)
 = (d,0)
 = d.
\]
Also ist $\alpha$ eine Quadratwurzel von $d$ in $L$. Für alle $y \in K$ gilt
\[
 y\alpha
 = (y,0) \cdot (0,1)
 = (y \cdot 0 + d \cdot 0 \cdot 1, y \cdot 1 + 0 \cdot 0)
 = (0,y).
\]
Für alle $x,y \in L$ gilt deshalb
\[
 (x,y)
 = (x,0) + (0,y)
 = x + y \alpha.
\]
Mit den obigen Notationen erhalten wir also, dass
\[
 L = \{x + y\alpha \mid x,y \in K\}
\]
und $\alpha^2 = d$, und für jedes $z \in L$ ist die Zerlegung $z = x + y\alpha$ mit $x,y \in K$ eindeutig.



% \subsection{Quotientenkörper}




















% \subsubsection{Die Gaußschen Zahlen}
% Die \emph{Gaußschen Zahlen} sind definiert als
% \[
%  \Zbb[i] \coloneqq \{a + b i \mid a,b \in \Zbb\}.
% \]
% 
% Wie bereits für $\Qbb[i]$ ergibt sich auch für $\Zbb[i]$, dass es sich um eine additive Untergruppe von $\Cbb$ handelt: Es ist $0 \in \Zbb[i]$. Für $z, w \in \Zbb[i]$ ist $z = a + bi$ und $w = c + di$ mit $a,b,c,d \in \Zbb$, weshalb
% \[
%  z + w
%  = (a + bi) + (c + di)
%  = (a+c) + (b+d)i
% \]
% mit $a+c, b+d \in \Zbb$ und somit $z+w \in \Zbb[i]$. Für $z \in \Zbb[i]$ ist $z = a + bi$ mit $a,b \in \Zbb$, weshalb $-z = -(a+bi) = (-a) + (-b)i$ mit $-a, -b \in \Zbb$ und somit auch $-z \in \Zbb[i]$.
% 
% 
% \begin{bem}
%  Im Gegensatz zu $\Qbb[i]^\times = \Qbb[i] \cap \Cbb^\times$ ist $\Zbb[i] \cap \Cbb^{\times}$ \emph{keine} Untergruppe von $\Cbb^\times$, da $2 \in \Zbb[i] \cap \Cbb^\times$ aber $1/2 \notin \Zbb[i]$ und somit auch nicht $1/2 \in \Zbb[i] \cap \Cbb^\times$.
% \end{bem}














