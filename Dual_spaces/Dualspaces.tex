\documentclass[a4paper,10pt]{article}
%\documentclass[a4paper,10pt]{scrartcl}

\usepackage{../generalstyle}
\usepackage{specificstyle}

\setromanfont[Mapping=tex-text]{Linux Libertine O}
% \setsansfont[Mapping=tex-text]{DejaVu Sans}
% \setmonofont[Mapping=tex-text]{DejaVu Sans Mono}

\title{Dualräume}
\author{Jendrik Stelzner}
\date{\today}

\begin{document}
\maketitle

\tableofcontents










\section{Grundlegende Definitionen}

\begin{defi}
 Ist $V$ ein $K$-Vektorraum, so heißt
 \[
  V^* \coloneqq \{f \colon V \to K \mid \text{$f$ ist $K$-linear}\}
 \]
 der \emph{Dualraum} von $V$.
\end{defi}

Ist $V$ ein $K$-Vektorraum, so ist auch $V^*$ ein $K$-Vektorraum durch die punktweise Addition und Skalarmultiplikation, d.h.\ für alle $f,g \in V^*$ und $\lambda \in K$ ist
\[
 (f+g)(v) = f(v) + g(v)
 \quad\text{und}\quad
 (\lambda f)(v) = \lambda f(v)
 \quad \text{für alle $v \in V$}.
\]

\begin{defi}
 Sind $V$ und $W$ $K$-Vektorräume und ist $\varphi \colon V \to W$ $K$-linear, so ist
 \[
  \varphi^* \colon W^* \to V^*, \quad f \mapsto f \circ \varphi
 \]
 die \emph{duale Abbildung} zu $f$. (Man beachte, dass sich die „Richtung“ der Abbildung ändert.)
\end{defi}

$\varphi^*$ ist also die Prekomposition mit $\varphi$. Die Wohldefiniertheit der dualen Abbildung ergibt sich daraus, dass die Verknüpfung von linearen Abbildungen wieder linear ist: Ist nämlich $f \in W^*$, als $f \colon W \to K$ linear, und $\varphi \colon V \to W$ linear, so ist auch die Verknüpfung $f \circ \varphi \colon V \to K$ linear, also $f \circ \varphi \in V^*$.

Die duale Abbildung ist ebenfalls linear: Es sei $\varphi \colon V \to W$ linear und es seien $f,g \in W^*$ und $\lambda \in K$. Für alle $v \in V$ ist dann
\begin{align*}
 \varphi^*(f+g)(v)
 &= ((f+g) \circ \varphi)(v)
 = (f+g)(\varphi(v)) \\
 &= f(\varphi(v)) + g(\varphi(v))
 = (f \circ \varphi)(v) + (g \circ \varphi)(v) \\
 &= \varphi^*(f)(v) + \varphi^*(g)(v)
 = (\varphi^*(f)+\varphi^*(g))(v),
\end{align*}
und somit $\varphi^*(f+g) = \varphi^*(f) + \varphi^*(g)$, sowie
\begin{align*}
 \varphi^*(\lambda f)(v)
 &= ((\lambda f) \circ \varphi)(v)
 = (\lambda f)(\varphi(v))
 = \lambda f(\varphi(v)) \\
 &= \lambda \varphi^*(f)(v)
 = (\lambda \varphi^*(f))(v),
\end{align*}
und somit $\varphi^*(\lambda f) =  \lambda \varphi^*(f)$.

Wir fassen zusammen:

\begin{shaded}
 Für jeden $K$-Vektorraum ist auch $V^*$ durch die punktweise Addition und Skalarmultiplikation ein $K$-Vektorraum. Jede lineare Abbildung $\varphi \colon V \to W$ zwischen $K$-Vektorräumen $V$ und $W$ induziert durch Prekomposition eine $K$-lineare Abbildung $\varphi^* \colon W^* \to V^*$.
\end{shaded}










\section{Der Dualraum von $K^n$}\label{sec: dual space of Kn}
Wir fixieren in diesem Abschnitt eine Dimension $n \in \Nbb$. Bevor wir uns abstrakten Vektorräumen zuwenden, wollen zunächst den Dualraum $(K^n)^*$ verstehen.

\begin{bem}
 Wir unterscheiden im Folgenden nicht zwischen $K$ und $K^1$, also dem Skalar $\lambda \in K$ und dem entsprechenden Tupel $(\lambda) \in K^1$ mit einem Eintrag.
\end{bem}

Für jede Matrix $A \in \Mat(1 \times n, K)$ ist die Abbildung
\[
 m_A \colon K^n \to K, \quad x \mapsto A \cdot x
\]
$K$-linear, und für jede $K$-lineare Abbildung $f \colon K^n \to K = K^1$ (also $f \in (K^n)^*$) gibt es eine eindeutige ($1 \times n$)-Matrix $M_f \in \Mat(1 \times n, K)$ mit
\[
 f(x) = M_f \cdot x
 \quad \text{für alle $x \in K^n$},
\]
also $f = m_{M_f}$. Zusammengefasst sagt dies, dass die Abbildung
\[
 \Phi \colon \Mat(1 \times n, K) \xrightarrow{\sim} (K^n)^*, \quad A \mapsto m_A
\]
eine Bijektion ist, wobei $\Phi^{-1}(f) = M_f$ für alle $f \in (K^n)^*$. (Die Tilde $\sim$ über dem Pfeil symbolisiert, dass es sich um eine Bijektion handelt.)

$\Phi$ ist auch $K$-linear, denn für alle $A, B \in \Mat(1 \times n, K)$ und $\lambda \in K$ ist
\begin{align*}
 \Phi(A+B)(x)
 &= m_{A+B}(x)
 = (A+B)x
 = Ax + Bx \\
 &= m_A(x) + m_B(x)
 = (m_A+m_B)(x)
 = (\Phi(A)+\Phi(B))(x)
\end{align*}
für alle $x \in K^n$ und somit $\Phi(A+B) = \Phi(A) + \Phi(B)$, und da
\[
 \Phi(\lambda A)(x)
 = m_{\lambda A}(x)
 = (\lambda A)x
 = \lambda Ax
 = \lambda m_A(x)
 = (\lambda m_A)(x)
 = (\lambda \Phi(A))(x)
\]
für alle $x \in K^n$ ist auch $\Phi(\lambda A) = \lambda \Phi(A)$.

Damit ist also $\Phi$ ein Isomorphismus von $K$-Vektorräumen. Da wir $\Mat(1 \times n, K)$ und $K^n$ verstehen, können wir nun $\Phi$ nutzen, um auch $V^*$ zu verstehen.

Wir wissen, dass jedes Element $f \in (K^n)^*$ von der Form $f = m_A$ für eine eindeutige Matrix $A = \begin{pmatrix} a_1 & \cdots & a_n \end{pmatrix} \in \Mat(1 \times n, K)$ ist. Da
\[
 f(x)
 = m_A(x)
 = A \cdot \vect{x_1 \\ \vdots \\ x_n}
 = a_1 x_1 + \dotsb + a_n x_n
 \quad
 \text{für alle $x = \vect{x_1 \\ \vdots \\ x_n} \in K^n$}
\]
erhalten wir auch eine Beschreibung von $(K^n)^*$, die ohne Matrizen auskommt:


\begin{shaded}
 Jedes $f \in (K^n)^*$ ist von der Form
 \[
  f(x) = a_1 x_1 + \dotsb + a_n x_n
  \quad
  \text{für alle $x = \vect{x_1 \\ \vdots \\ x_n} \in K^n$}
 \]
 mit eindeutig bestimmten $a_1, \dotsc, a_n \in K$.
\end{shaded}

Damit erhalten wir aus direkt eine Basis von $(K^n)^*$: Für jedes $1 \leq i \leq n$ sei $f_i \colon K^n \to K$ definiert als
\[
 f_i \colon K^n \to K, \quad \vect{x_1 \\ \vdots \\ x_n} \mapsto x_i,
\]
d.h.\ $f_i(x)$ gibt die $i$-te Koordinate von $x$ an. Nach dem obigen Ergebnis sind die Abbildungen $f_1, \dotsc, f_n$ linear und bilden eine Basis von $(K^n)^*$, denn für alle Skalare $a_1, \dotsc, a_n \in K$ ist
\[
 (a_1 f_1 + \dotsb + a_n f_n)(x)
 = a_1 f_1(x) + \dotsb + a_n f_n(x)
 = a_1 x_1 + \dotsb + a_n x_n.
\]
Wir können unser bisheriges Ergebnis also auch umformulieren:

\begin{shaded}
 Die Koordinatenabbildungen $f_1, \dotsc, f_n \colon K^n \to K$ mit
 \[
  f_i \colon K^n \to K, \quad \vect{x_1 \\ \vdots \\ x_n} \mapsto x_i
 \]
 bilden eine Basis von $(K^n)^*$. Für jedes $f \in (K^n)^*$ gibt es daher eindeutige Skalare $a_1, \dotsc, a_n \in K$ mit $f = a_1 f_1 + \dotsb + a_n f_n$.
\end{shaded}

Man beachte, dass $f_i(e_j) = \delta_{ij}$ für alle $1 \leq i,j \leq n$; da $(e_1, \dotsc, e_n)$ eine Basis von $K^n$ bildet, sind $(f_1, \dotsc, f_n)$ dadurch eindeutig bestimmt\footnote{Ist nämlich $g \colon K^n \to K$ eine lineare Abbildung, so ist $g$ durch die Werte $g(e_1), \dotsc, g(e_n)$ eindeutig bestimmt, da $g(x) = \sum_{i=1}^n x_i g(e_i)$ für alle $x = (x_1, \dotsc, x_n)^T \in K^n$.}.  Man sagt hierzu, dass die Basen $(e_1, \dotsc, e_n)$ von $K^n$ und $(f_1, \dotsc, f_n)$ von $(K^n)^*$ \emph{dual} zueinander sind.

Diese Dualität der Basen $(e_1, \dotsc, e_n)$ und $(f_1, \dotsc, f_n)$ hat für uns die praktische Konsequenz, dass wir für ein Element $f \in (K^n)^*$ die Koeffizienten $a_1, \dotsc, a_n \in K$ mit $f = a_1 f_1 + \dotsb + a_n f_n$ durch
\begin{equation}\label{eq: coefficients and duality}
 f(e_i)
 = (a_1 f_1 + \dotsb + a_n f_n)(e_i)
 = a_1 \underbrace{f_1(e_i)}_{=\delta_{1i}} + \dotsb + a_n \underbrace{f_n(e_i)}_{=\delta_{ni}}
 = a_i
\end{equation}
bestimmen können. Es ist also $f = f(e_1) f_1 + \dotsb + f(e_n) f_n$ für alle $f \in (K^n)^*$.

Damit haben wir nun eine Beschreibung von $(K^n)^*$ gefunden:

\begin{shaded}
 Die Koordinatenfunktionen $(f_1, \dotsc, f_n)$ bilden eine Basis des Dualraums $(K^n)^*$, und ist $f \in (K^n)^*$ mit $f = \sum_{i=1}^n a_i f_i$, so sind die Koeffizienten $a_1, \dotsc, a_n \in K$ durch $a_i = f(e_i)$ für alle $1 \leq i \leq n$ gegeben.
\end{shaded}

Wir bemerken noch, dass $(K^n)^*$ inbesondere $n$-dimensional ist, also $K^n$ und $(K^n)^*$ gleichdimensional ist.






\section{Der Dualraum eines endlichendimensionalen Vektorraums}
Wir möchten die obigen Ergebnisse nun gerne auf beliebige endlichdimensionale Vektorräume verallgemeinern. Es sei hierfür $V$ ein $n$-dimensionaler $K$-Vektorraum. Ähnlich wie für $K^n$ würden wir gerne eine Basis von „Koordinatenfunktionen“ von $V$ angeben. Das Problem ist, dass es in $V$ a priori keinen geeigneten Begriff von Koordinaten gibt.

Die Lösung dieses Problems besteht darin, mithilfe einer Basis Koordinaten einzuführen. Es sei also $\mc{B} = (v_1, \dotsc, v_n)$ eine Basis von $V$. Jedes Element $v \in V$ lässt sich als eindeutige Linearkombination $v = \lambda_1 v_1 + \dotsb + \lambda_n v_n$ schreiben. Wir können nun die eindeutigen Koeffizienten $\lambda_1, \dotsc, \lambda_n$ als „Koordinaten“ von $v$ auffassen, und damit für alle $1 \leq i \leq n$ die Koordinatenfunktion
\[
 f^\mc{B}_i \colon V \to K, \quad \lambda_1 v_1 + \dotsb + \lambda_n v_n \mapsto \lambda_i
\]
zu definieren.

Als Verallgemeinerung der Ergebnisse für $(K^n)^*$ überlegen wir uns nun, dass $\mc{B}^* \coloneqq (f^\mc{B}_1, \dotsc, f^\mc{B}_n)$ eine Basis von $V^*$ bildet. Dass $f^\mc{B}_i$ für alle $1 \leq i \leq n$ linear ist, also $f^\mc{B}_i \in V^*$, folgt dabei direkt daraus, dass die Addition und Skalarmultiplikation in $V$ dadurch gegeben ist, dass für alle $\lambda_1, \dotsc, \lambda_n, \mu_1, \dotsc, \mu_n, \mu \in K$
\begin{gather*}
 \begin{gathered}
    (\lambda_1 v_1 + \dotsb + \lambda_n v_n) + (\mu_1 v_1 + \dotsb + \mu_n v_n) \\
  = (\lambda_1 + \mu_1) v_1 + \dotsb + (\lambda_n + \mu_n) v_n
 \end{gathered}
\shortintertext{und}
 \mu \cdot (\lambda_1 v_1 + \dotsb + \lambda_n v_n)
 = (\mu \lambda_1) v_1 + \dotsb + (\mu \lambda_n) v_n.
\end{gather*}

Entscheident ist nun, dass $f^{\mc{B}}_i(v_j) = \delta_{ij}$ für alle $1 \leq i \leq n$. Es verhält sich also $\mc{B}^* = (f^\mc{B}_1, \dotsc, f^\mc{B}_n)$ \emph{dual} zu der Basis $\mc{B} = (v_1, \dotsc, v_n)$ von $V$. Dies wollen wir nun ausnutzen.

Als erstes überlegen wir uns, dass $V^*$ von $\mc{B}^*$ erzeugt wird. Es sei also $f \in V^*$, und wir suchen $a_1, \dotsc, a_n \in K$ mit $f = a_1 f^\mc{B}_1 + \dotsb + a_n f^\mc{B}_n$. Analog zu \eqref{eq: coefficients and duality} erhalten wir, dass dann
\[
 f(v_i)
 = (a_1 f^\mc{B}_1 + \dotsb + a_n f^\mc{B}_n)(v_i)
 = a_1 \underbrace{f^\mc{B}_1(v_i)}_{=\delta_{1i}} + \dotsb + a_n \underbrace{f^\mc{B}_n(v_i)}_{=\delta_{ni}}
 = a_i
\]
gelten muss. Wir definieren daher $a_i \coloneqq f(v_i)$ für alle $1 \leq i \leq n$ und betrachten $f' \coloneqq a_1 f^\mc{B}_1 + \dotsb + a_n f^\mc{B}_n$. Indem wir in der obigen Rechnung $f$ durch $f'$ ersetzen, erhalten wir, dass $f'(v_i) = a_i = f(v_i)$ für alle $1 \leq i \leq n$. Wegen der Linearität von $f'$ und $f$ gilt daher bereits $f'(v) = f(v)$ für alle $v \in \Ell(v_1, \dotsc, v_n)$. Da $\mc{B}$ ein Basis von $V$ ist, ist bereits $\Ell(v_1, \dotsc, v_n) = V$, und somit $f'(v) = f(v)$ für alle $v \in V$. Also ist $f = f' = a_1 f^\mc{B}_1 + \dotsb + a_n f^\mc{B}_n$. Somit ist $\mc{B}^*$ ein Erzeugendensystem von $V^*$.

Die lineare Unabhängigkeit ergibt sich ähnlich: Für Koeffizienten $a_1, \dotsc, a_n \in K$ mit $0 = a_1 f_1 + \dotsb + a_n f_n$ ist analog zur obigen Rechnung
\[
 0 = f(v_i) = a_i
 \quad
 \text{für alle $1 \leq i \leq n$}.
\]
Also ist $\mc{B}^*$ linear unabhängig.

Damit haben wir eine Basis von $V^*$ gefunden.

\begin{shaded}
 Ist $\mc{B} = (v_1, \dotsc, v_n)$ ein Basis von $V$, so ergeben die Koordinatenfunktionen
 \[
  f^\mc{B}_i \colon V \to K, \quad \lambda_1 v_1 + \dotsb + \lambda_n v_n \mapsto \lambda_i
  \quad
  \text{für alle $1 \leq i \leq n$}
 \]
 eine Basis $\mc{B}^* = (f^\mc{B}_1, \dotsc, f^\mc{B}_n)$ von $V^*$, die dual zur Basis $\mc{B}$ ist (in dem Sinne, dass $f^\mc{B}_i(v_j) = \delta_{ij}$ für alle $1 \leq i,j \leq n$). Jedes $f \in V^*$ ist also von der Form
 \[
  f(\lambda_1 v_1 + \dotsb + \lambda_n v_n)
  = a_1 \lambda_1 + \dotsb + a_n \lambda_n
  \quad
  \text{für alle $\lambda_1, \dotsc, \lambda_n \in K$}
 \]
 für eindeutig bestimmte $a_1, \dotsc, a_n \in K$. Dabei gilt $a_i = f(v_i)$ für alle $1 \leq i \leq n$.
\end{shaded}

Zum einen ergibt sich hiermit, dass $\dim(V^*) = n$, also $V^*$ und $V$ gleichdimensional sind. Es lässt sich auch ein Isomorphismus $V \to V^*$ angeben: Zwischen der Basis $\mc{B} = (v_1, \dotsc, v_n)$ von $V$ und der dualen Basis $\mc{B}^* = (f^\mc{B}_1, \dotsc, f^\mc{B}_n)$ von $V^*$ haben wir eine naheliegende Bijektion $v_i \mapsto f^\mc{B}_i$; diese können wir zu einem Isomorphismus
\[
 \Phi_{\mc{B}} \colon V \to V^*,
 \quad
 \lambda_1 v_1 + \dotsb + \lambda_n v_n
 \mapsto
 \lambda_1 f^\mc{B}_1 + \dotsb + \lambda_n f^\mc{B}_n
\]
fortsetzen. Wir sagen auch, die Basis $\mc{B}$ \emph{induziert} den Isomorphismus $\Phi_\mc{B} \colon V \to V^*$.

Eine weitere interessante Beobachtung ist, dass man die Element aus $V$ durch die Abbildungen aus $V^*$ trennen, bzw.\ unterscheiden kann, d.h.\ es gibt für je zwei Elemente $v,w \in V$ mit $v \neq w$ ein Element $f \in V^*$ mit $f(v) \neq f(w)$:

Da $v \neq w$ ist nämlich $v-w \neq 0$. Wir können daher $v_1 \coloneqq v-w$ zu einer Basis $\mc{B} = (v_1, v_2, \dotsc, v_n)$ von $V$ ergänzen. Für die duale Basis $\mc{B}^* = (f^\mc{B}_1, \dotsc, f^\mc{B}_n)$ von $V^*$ gilt dann $1 = f^\mc{B}_1(v_1) = f^\mc{B}_1(v-w) = f^\mc{B}_1(v) - f^\mc{B}_1(w)$, also $f^\mc{B}_1(v) \neq f^\mc{B}_1(w)$.










\section{Konkrete Rechenbeispiele}
Wir betrachten nun verschieden konkrete Beispiele für duale Basen und stellen Element des Dualraums durch diese da.


\subsection{Die Spur von $2 \times 2$-Matrizen}
Die Spur einer ($2 \times 2$)-Matrix ist definiert als
\[
 \spur
 \begin{pmatrix}
  a_{11} & a_{12} \\
  a_{21} & a_{22}
 \end{pmatrix}
 = a_{11}+a_{22},
\]
d.h.\ $\spur A = A_{11} + A_{22}$ ist die Summe der Diagonaleinträge von $A$. Die Spur ist linear, denn für alle $A,B \in \Mat(2 \times 2, \Rbb)$ und $\lambda \in \Rbb$ ist
\begin{gather*}
 \begin{aligned}
  \spur(A+B)
  &= (A+B)_{11} + (A+B)_{22}
  = A_{11} + B_{11} + A_{22} + B_{22} \\
  &= A_{11} + A_{22} + B_{11} + B_{22}
  = \spur(A) + \spur(B),
 \end{aligned}
\intertext{sowie}
 \spur(\lambda A)
 = (\lambda A)_{11} + (\lambda A)_{22}
 = \lambda A_{11} + \lambda A_{22}
 = \lambda (A_{11} + A_{22})
 = \lambda \spur(A).
\end{gather*}
Also ist $\spur \in (\Mat(2 \times 2, \Rbb))^*$. Wir wollen verschiedene duale Basen betrachten, und $\spur$ jeweils bezüglich dieser darstellen.





\subsubsection{Naheliegende Basis}
Eine naheliegende Basis von $\Mat(2 \times 2, \Rbb)$ ist $\mc{B} = (E_1, E_2, E_3, E_4)$ mit
\[
 E_1 = \begin{pmatrix} 1 & 0 \\ 0 & 0 \end{pmatrix}, \quad
 E_2 = \begin{pmatrix} 0 & 1 \\ 0 & 0 \end{pmatrix}, \quad
 E_2 = \begin{pmatrix} 0 & 0 \\ 1 & 0 \end{pmatrix}, \quad
 E_3 = \begin{pmatrix} 0 & 0 \\ 0 & 1 \end{pmatrix}.
\]
Für die zugehörige duale Basis $\mc{B}^* = (f^{\mc{B}}_1, f^{\mc{B}}_2, f^{\mc{B}}_3, f^{\mc{B}}_4)$ von $\Mat(2 \times 2, \Rbb)$ ist dann etwa
\[
 f^{\mc{B}}_1\left( \begin{pmatrix} a & b \\ c & d \end{pmatrix} \right)
 = f^{\mc{B}}_1( a E_1 + b E_2 + c E_3 + d E_4 )
 = a.
\]
Analog lassen sich auch $f^{\mc{B}}_2$, $f^{\mc{B}}_3$ und $f^{\mc{B}}_4$ betrachten, und wir erhalten, dass
\[
 f^{\mc{B}}_1(A) = A_{11}, \quad
 f^{\mc{B}}_2(A) = A_{12}, \quad
 f^{\mc{B}}_3(A) = A_{21}, \quad
 f^{\mc{B}}_4(A) = A_{22}
\]
für alle $A \in \Mat(2 \times 2, \Rbb)$.

Wollen wir nun die Spur durch die Basis $\mc{B}^*$ ausdrücken, so ergibt sich etwa durch direktes Hinsehen, dass $\spur = f^{\mc{B}}_1 + f^{\mc{B}}_4$, da
\[
 (f^{\mc{B}}_1 + f^{\mc{B}}_4)(A)
 = f^{\mc{B}}_1(A) + f^{\mc{B}}_4(A)
 = A_{11} + A_{22}
 = \spur(A)
\]
für alle $A \in \Mat(2 \times 2, \Rbb)$.

Alternativ wissen wir, dass $\spur = a_1 f^{\mc{B}}_1 + a_2 f^{\mc{B}}_2 + a_3 f^{\mc{B}}_3 + a_4 f^{\mc{B}}_4$, wobei $a_i = \spur(E_i)$ für alle $1 \leq i \leq 4$. Durch direktes (und sehr einfaches) Ausrechnen ergibt sich, dass $\spur(E_1) = \spur(E_4) = 1$ und \mbox{$\spur(E_3) = \spur(E_4) = 0$}. Somit ergibt sich, dass $\spur = f^{\mc{B}}_1 + f^{\mc{B}}_4$.

Wir bemerken noch, dass der durch die Basis $\mc{B}$ induzierte Isomorphismus
\begin{gather*}
 \Phi_{\mc{B}} \colon \Mat(2 \times 2, \Rbb) \to \left( \Mat(2 \times 2, \Rbb) \right)^*
\intertext{für $A \in \Mat(2 \times 2, \Rbb)$ durch}
 \begin{aligned}
  \Phi_{\mc{B}}(A)
  &= \Phi_{\mc{B}}(A_{11} E_1 + A_{12} E_2 + A_{21} E_3 + A_{22} E_4) \\
  &= A_{11} f^{\mc{B}}_1 + A_{12} f^{\mc{B}}_2 + A_{21} f^{\mc{B}}_3 + A_{22} f^{\mc{B}}_4
 \end{aligned}
\end{gather*}
gegeben ist. Für jede weitere Matrix $B \in \Mat(2 \times 2, \Rbb)$ ist also
\begin{align*}
 (\Phi_{\mc{B}}(A))(B)
 &= A_{11} f^{\mc{B}}_1(B) + A_{12} f^{\mc{B}}_2(B) + A_{21} f^{\mc{B}}_3(B) + A_{22} f^{\mc{B}}_4(B) \\
 &= A_{11} B_{11} + A_{12} B_{12} + A_{21} B_{21} + A_{22} B_{22}.
\end{align*}





\subsubsection{Eine alternative Basis}
Als weiter Basis von $\Mat(2 \times 2, \Rbb)$ betrachten wir nun $\mc{C} = (F_1, F_2, F_3, F_4)$ mit
\[
 F_1 = \begin{pmatrix} 1 & 0 \\  0 &  1 \end{pmatrix}, \;
 F_2 = \begin{pmatrix} 1 & 0 \\  0 & -1 \end{pmatrix}, \;
 F_3 = \begin{pmatrix} 0 & 1 \\  1 &  0 \end{pmatrix}, \;
 F_4 = \begin{pmatrix} 0 & 1 \\ -1 &  0 \end{pmatrix}.
\]

Zunächst überlegen wir uns, dass $\mc{C}$ tatsächlich eine Basis von $\Mat(2 \times 2, \Rbb)$ ist: Da $\mc{B} = (E_1, E_2, E_3, E_4)$ eine Basis von $\Mat(2 \times 2, \Rbb)$ ist, und da
\begin{equation}\label{eqn: change of basis}
 \begin{aligned}
  E_1 &= \frac{1}{2}(F_1+F_2), &
  E_2 &= \frac{1}{2}(F_3+F_4), \\
  E_3 &= \frac{1}{2}(F_3-F_4), &
  E_4 &= \frac{1}{2}(F_1-F_2),
 \end{aligned}
\end{equation}
ist $\mc{C}$ ein Erzeugendensystem von $\Mat(2 \times 2, \Rbb)$; da $\Mat(2 \times 2, \Rbb)$ vierdimensional ist, und $\mc{C}$ aus vier Vektoren besteht, muss $\mc{C}$ bereits linear unabhängig, und somit eine Basis sein.

Wir wollen zunächst verstehen, wie die duale Basis $\mc{C}^* = (f^{\mc{C}}_1, f^{\mc{C}}_2, f^{\mc{C}}_3, f^{\mc{C}}_4)$ aussieht. Mithilfe von \eqref{eqn: change of basis} sehen wir, dass für alle $A \in \Mat(2 \times 2, \Rbb)$
\begin{align*}
 A
 &= A_{11} E_1 + A_{12} E_2 + A_{21} E_3 + A_{22} E_4 \\
 &= \frac{A_{11}}{2} (F_1 + F_2) + \frac{A_{12}}{2} (F_3 + F_4)
   + \frac{A_{21}}{2} (F_3 - F_4) + \frac{A_{22}}{2} (F_1 - F_2) \\
 &= \frac{A_{11}+A_{22}}{2} F_1 + \frac{A_{11}-A_{22}}{2} F_2 + \frac{A_{12}+A_{21}}{2} F_3 + \frac{A_{12}-A_{21}}{2} F_4.
\end{align*}
Für alle $A \in \Mat(2 \times 2, \Rbb)$ ist deshalb
\begin{equation*}
 \begin{aligned}
  f^{\mc{C}}_1(A) &= \frac{A_{11}+A_{22}}{2}, &
  f^{\mc{C}}_2(A) &= \frac{A_{11}-A_{22}}{2}, \\
  f^{\mc{C}}_3(A) &= \frac{A_{12}+A_{21}}{2}, &
  f^{\mc{C}}_4(A) &= \frac{A_{12}-A_{21}}{2}.
 \end{aligned}
\end{equation*}

Hierdurch sieht man bereits, dass für alle $A \in \Mat(2 \times 2, \Rbb)$
\[
 \spur(A)
 = A_{11} + A_{22}
 = 2 \ \frac{A_{11} + A_{22}}{2}
 = 2 f^{\mc{C}}_1(A).
\]
Also ist $\spur = 2f^{\mc{C}}_1$. 

Auch hier lässt sich wieder der allgemeine Ansatz nutzen: Wir wissen bereits, dass $\spur = a_1 f^{\mc{C}}_1 + a_2 f^{\mc{C}}_2 + a_3 f^{\mc{C}}_3 + a_4 f^{\mc{C}}_4$ mit $a_i = \spur(F_i)$ für alle $1 \leq i \leq 4$. Da $\spur(F_1) = 2$ und $\spur(F_2) = \spur(F_3) = \spur(F_4) = 0$ ergibt sich damit, dass $\spur = 2 f^{\mc{C}}_1$.

Der durch die Basis $\mc{C}$ induzierte Isomorphismus
\[
 \Phi_{\mc{C}} \colon \Mat(2 \times 2, \Rbb) \to \left( \Mat(2 \times 2, \Rbb) \right)^*
\]
ist dadurch gegeben, dass für jede Matrix $A \in \Mat(2 \times 2, \Rbb)$
\begin{align*}
  &\, \Phi_{\mc{C}}(A) \\
 =&\, \Phi_{\mc{C}}\left( \frac{A_{11}+A_{22}}{2} F_1 + \frac{A_{11}-A_{22}}{2} F_2 + \frac{A_{12}+A_{21}}{2} F_3 + \frac{A_{12}-A_{21}}{2} F_4 \right) \\
 =&\, \frac{A_{11}+A_{22}}{2} f^{\mc{C}}_1 + \frac{A_{11}-A_{22}}{2} f^{\mc{C}}_2 + \frac{A_{12}+A_{21}}{2} f^{\mc{C}}_3 + \frac{A_{12}-A_{21}}{2} f^{\mc{C}}_4.
\end{align*}
Für jede andere Matrix $B \in \Mat(2 \times 2, \Rbb)$ ist daher
\begin{align*}
  &\, (\Phi_{\mc{C}}(A))(B) \\
 =&\, \frac{A_{11}+A_{22}}{2} f^{\mc{C}}_1(B)
      + \frac{A_{11}-A_{22}}{2} f^{\mc{C}}_2(B) \\
  &\, + \frac{A_{12}+A_{21}}{2} f^{\mc{C}}_3(B)
      + \frac{A_{12}-A_{21}}{2} f^{\mc{C}}_4(B) \\
 =&\, \frac{A_{11}+A_{22}}{2} \frac{B_{11}+B_{22}}{2}
      + \frac{A_{11}-A_{22}}{2} \frac{B_{11}-B_{22}}{2} \\
  &\, + \frac{A_{12}+A_{21}}{2} \frac{B_{12}+B_{21}}{2}
      + \frac{A_{12}-A_{21}}{2} \frac{B_{12}-B_{21}}{2} \\
 =&\, \frac{1}{2}\left( A_{11} B_{11} + A_{12} B_{12} + A_{21} B_{21} + A_{22} B_{22} \right).
\end{align*}
Wir heben hier noch hervor, dass insbesondere $\Phi_{\mc{B}} \neq \Phi_{\mc{C}}$.





\subsubsection{Eine alternativere Basis}
Wir betrachten $\mc{D} = (G_1, G_2, G_3, G_4)$ mit
\[
 G_1 = \begin{pmatrix} 2 & 3 \\ 1 & 0 \end{pmatrix}, \;
 G_2 = \begin{pmatrix} 2 & 3 \\ 0 & 4 \end{pmatrix}, \;
 G_3 = \begin{pmatrix} 0 & 3 \\ 5 & 4 \end{pmatrix}, \;
 G_4 = \begin{pmatrix} 6 & 0 \\ 5 & 4 \end{pmatrix}.
\]
Wir zeigen zunächst, dass $\mc{D}$ eine Basis von $\Mat(2 \times 2, \Rbb)$ ist. Da $\mc{D}$ aus vier Element besteht und $\Mat(2 \times 2, \Rbb)$ vierdimensional ist, genügt es zu zeigen, dass $\mc{D}$ linear unabhängig ist. Es seien also $\lambda_1, \lambda_2, \lambda_3, \lambda_4 \in \Rbb$ mit
\[
 \lambda_1 G_1 + \lambda_2 G_2 + \lambda_3 G_3 + \lambda_4 G_4 = 0.
\]
Durch Betrachtung der einzelnen Einträge ergibt sich, dass dies äquivalent dazu ist, dass
\[
 \left\{
  \begin{matrix}
   2\lambda_1 & +2\lambda_2 &             & +6\lambda_4 & = 0, \\
   3\lambda_1 & +3\lambda_2 & +3\lambda_3 &             & = 0, \\
    \lambda_1 &             & +5\lambda_3 & +5\lambda_4 & = 0, \\
              & 4\lambda_2  & +4\lambda_3 & +4\lambda_4 & = 0.
  \end{matrix}
  \right.
\]
Wir formen das entsprechend homogene LGS um:
\begin{align*}
 \begin{pmatrix*}
  2 & 2 & 0 & 6 \\
  3 & 3 & 3 & 0 \\
  1 & 0 & 5 & 5 \\
  0 & 4 & 4 & 4
 \end{pmatrix*}
 &\xrightarrow[\substack{\text{I}/2 \\ \text{II}/3 \\ \text{IV}/4}]{}
 \begin{pmatrix*}
  1 & 1 & 0 & 3 \\
  1 & 1 & 1 & 0 \\
  1 & 0 & 5 & 5 \\
  0 & 1 & 1 & 1
 \end{pmatrix*}
 \xrightarrow[\substack{\text{II}-\text{I} \\ \text{III}-\text{I}}]{}
 \begin{pmatrix*}[r]
  1 &  1 & 0 &  3 \\
  0 &  0 & 1 & -3 \\
  0 & -1 & 5 &  2 \\
  0 &  1 & 1 &  1
 \end{pmatrix*}
 \\
 &\xrightarrow[\text{III}+\text{IV}]{}
 \begin{pmatrix*}[r]
  1 & 1 & 0 &  3 \\
  0 & 0 & 1 & -3 \\
  0 & 0 & 6 &  3 \\
  0 & 1 & 1 &  1
 \end{pmatrix*}
 \xrightarrow[\text{III}/3]{}
 \begin{pmatrix*}[r]
  1 & 1 & 0 &  3 \\
  0 & 0 & 1 & -3 \\
  0 & 0 & 2 &  1 \\
  0 & 1 & 1 &  1
 \end{pmatrix*}
 \\
 &\xrightarrow[\text{III}-2\text{II}]{}
 \begin{pmatrix*}[r]
  1 & 1 & 0 &  3 \\
  0 & 0 & 1 & -3 \\
  0 & 0 & 0 &  7 \\
  0 & 1 & 1 &  1
 \end{pmatrix*}
 \xrightarrow[\text{Reihenfolge}]{}
 \begin{pmatrix*}[r]
  1 & 1 & 0 &  3 \\
  0 & 1 & 1 &  1 \\
  0 & 0 & 1 & -3 \\
  0 & 0 & 0 &  7
 \end{pmatrix*}
\end{align*}
Wir erhalten also, dass $\lambda_4 = \lambda_3 = \lambda_2 = \lambda_1 = 0$. Also ist $\mc{D}$ linear unabhängig und somit eine Basis von $\Mat(2 \times 2, \Rbb)$.

Wir wollen nun die Spur durch die duale Basis $\mc{D}^* = (f^{\mc{D}}_1, f^{\mc{D}}_2, f^{\mc{D}}_3, f^{\mc{D}}_4)$ ausdrücken. Hierfür wissen wir, dass $\spur = a_1 f^{\mc{D}}_1 + a_2 f^{\mc{D}}_2 + a_3 f^{\mc{D}}_3 + a_4 f^{\mc{D}}_4$, wobei $a_i = \spur(G_i)$ für alle $1 \leq i \leq 4$. Durch direktes (und einfaches) ausrechnen ergibt sich, dass
\[
 \spur(G_1) = 2, \quad
 \spur(G_2) = 6, \quad
 \spur(G_3) = 4, \quad
 \spur(G_4) = 10.
\]
Also ist $\spur = 2 f^{\mc{D}}_1 + 6 f^{\mc{D}}_2 + 4 f^{\mc{D}}_3 + 10 f^{\mc{D}}_4$.

Ein genaueres Ausrechnen der dualen Basis $\mc{D}^*$ wollen wir uns hier ersparen; es ergibt sich aber, dass für alle $A \in \Mat(2 \times 2, \Rbb)$
\begin{align*}
 f^{\mc{D}}_1(A) &= \frac{5}{42} A_{11} + \frac{5}{21} A_{12} + \frac{1}{21} A_{21} -\frac{5}{21} A_{22}, \\
 f^{\mc{D}}_2(A) &= \frac{1}{42} A_{11} + \frac{1}{21} A_{12} - \frac{4}{21} A_{21} + \frac{17}{84} A_{22}, \\
 f^{\mc{D}}_3(A) &= -\frac{1}{7} A_{11} + \frac{1}{21} A_{12} + \frac{1}{7} A_{21} + \frac{1}{28} A_{22}, \\
 f^{\mc{D}}_4(A) &= \frac{5}{42} A_{11} - \frac{2}{21} A_{12} + \frac{1}{21} A_{21} + \frac{1}{84} A_{22}.
\end{align*}
Auch hiermit lässt sich dann ausrechnen, dass $\spur = 2 f^{\mc{D}}_1 + 6 f^{\mc{D}}_2 + 4 f^{\mc{D}}_3 + 10 f^{\mc{D}}_4$.





\subsection{Verschieden Basen von $\Rbb^2$}



\subsubsection{Die Standardbasis von $\Rbb^2$}
Betrachten wir $\Rbb^2$ mit der Standardbasis $\mc{B} = (e_1, e_2)$, so sind für die duale Basis $\mc{B}^* = (f^\mc{B}_1, f^\mc{B}_2)$ die entsprechenden Koordinatenfunktionen $f^\mc{B}_1, f^\mc{B}_2 \colon \Rbb^2 \to \Rbb$ dadurch gegeben, dass
\[
 f^\mc{B}_i\left( \vect{x_1 \\ x_2} \right)
 = f^\mc{B}_i(x_1 e_1 + x_2 e_2) = x_i.
\]
für alle $i \in \{1,2\}$. Also handelt es sich bei $\mc{B}^*$ hierbei um die Basis von $(\Rbb^2)^*$, die wir schon in Abschnitt \ref{sec: dual space of Kn} entdeckt haben.

Der durch die Basis $\mc{B}$ induzierte Isomorphismus $\Phi_\mc{B} \colon \Rbb^2 \to (\Rbb^2)^*$ ist dadurch gegeben, dass
\[
 \Phi_\mc{B}\left( \vect{x_1 \\ x_2} \right)
 = \Phi_\mc{B}(x_1 e_1 + x_2 e_2)
 = x_1 f^{\mc{B}}_1 + x_2 f^{\mc{B}}_2
 \quad
 \text{für alle $\vect{x_1 \\ x_2} \in \Rbb^2$}.
\]
Für alle $\vect{y_1,y_2}^T \in \Rbb^2$ ist daher
\[
 \left( \Phi_\mc{B}\left( \vect{x_1 \\ x_2} \right) \right)\left( \vect{y_1 \\ y_2} \right)
 = x_1 f^{\mc{B}}_1\left( \vect{y_1 \\ y_2} \right) + x_2 f^{\mc{B}}_2\left( \vect{y_1 \\ y_2} \right)
 = x_1 y_1 + x_2 y_2.
\]
(Den aufmerksamen Leser mag dies an das Skalarprodukt von $\Rbb^2$ erinnern. Dies ist kein Zufall.)




\subsubsection{Erste andere Basis}
Es sei nun $\mc{C} = (c_1, c_2)$ mit
\[
 c_1 = \vect{1 \\ 1}
 \quad\text{und}\quad
 c_2 = \vect{\phantom{-}1 \\ -1}.
\]
Da $e_1 = (c_1+c_2)/2$ und $e_2 = (c_1-c_2)/2$ ist $\mc{C}$ ein Erzeugendensystem von $\Rbb^2$. Da $\mc{C}$ aus zwei Elementen besteht und $\Rbb^2$ zweidimensional ist, ist $\mc{C}$ damit bereits eine Basis von $\Rbb^2$.

Für die duale Basis $\mc{C}^* = (f^{\mc{C}}_1, f^{\mc{C}}_2)$ ist nun
\begin{gather*}
 f^{\mc{C}}_1\left( \vect{x_1 \\ x_2} \right)
 = f^{\mc{C}}_1\left( \frac{x_1 + x_2}{2} c_1 + \frac{x_1 - x_2}{2} c_2 \right)
 = \frac{x_1 + x_2}{2}
\shortintertext{sowie}
 f^{\mc{C}}_2\left( \vect{x_1 \\ x_2} \right)
 = f^{\mc{C}}_2\left( \frac{x_1 + x_2}{2} c_1 + \frac{x_1 - x_2}{2} c_2 \right)
 = \frac{x_1 - x_2}{2}.
\end{gather*}

Wollen wir nun etwa das Element $f \in (\Rbb^2)^*$ mit
\[
 f\left( \vect{x_1 \\ x_2} \right)
 = 7x_1 - 3x_2
 \quad
 \text{für alle $\vect{x_1 \\ x_2} \in \Rbb^2$}
\]
als Linearkombination von $f^{\mc{C}}_1$ und $f^{\mc{C}}_2$ darstellen, so haben wir $f = a_1 f^{\mc{C}}_1 + a_2 f^{\mc{C}}_2$ und die Koeffizienten sind durch $a_1 = f(c_1) = 4$ und $a_2 = f(c_2) = 10$ gegeben. Also ist $f = 4 f^{\mc{C}}_1 + 10 f^{\mc{C}}_2$.


\subsubsection{Zweite andere Basis}\label{ssec: second different basis of R2}
Es sei nun $\mc{D} = (d_1, d_2)$ mit
\[
 d_1 = c_1 = \vect{1 \\ 0}
 \quad\text{und}\quad
 d_2 = \vect{\phantom{-}4 \\ -2}.
\]
Da $e_1 = d_1$ und $e_2 = 2d_1 - d_2/2$ ist $\mc{D}$ ein Erzeugendensystem von $\Rbb^2$. Da $\mc{D}$ aus zwei Elementen besteht, und $\Rbb^2$ zweidimensional ist, ist $\mc{D}$ damit bereits eine Basis von $\Rbb^2$.

Für die duale Basis $\mc{D}^* = (f^{\mc{D}}_1, f^{\mc{D}}_2)$ ist nun
\begin{gather*}
 \begin{aligned}
  f^{\mc{D}}_1\left( \vect{x_1 \\ x_2} \right)
  &= f^{\mc{D}}_1\left( x_1 e_1 + x_2 e_2 \right)
  = f^{\mc{D}}_1\left( x_1 d_1 + x_2 \left( 2d_1 - \frac{1}{2} d_2 \right) \right) \\
  &= f^{\mc{D}}_1\left( (x_1 + 2x_2) d_1 - \frac{x_2}{2} d_2 \right)
  = x_1 + 2 x_2,
 \end{aligned}
\shortintertext{sowie analog}
 f^{\mc{D}}_2\left( \vect{x_1 \\ x_2} \right)
 = f^{\mc{D}}_2\left( (x_1 - 2x_2) d_1 - \frac{x_2}{2} d_2 \right)
 = -\frac{1}{2} x_2.
\end{gather*}

Obwohl der erste Vektor der Standardbasis $\mc{B}$ und der Basis $\mc{D}$ gleich sind ($e_1 = d_1$), sind die entsprechenden Elemente $f^{\mc{B}}_1$ und $f^{\mc{D}}_1$ der dualen Basis $\mc{B}^*$ ud $\mc{D}^*$ verschieden. Hieraus wird deutlich, dass die Elemente der duale Basis $\mc{C}^*$ einer Basis $\mc{C}$ von der kompletten Basis $\mc{C}$ abhängen.










\section{Notation}
Ist $V$ ein endlichdimensionaler $K$-Vektorraum und $\mc{B} = (v_1, \dotsc, v_n)$ eine Basis von $V$, so bezeichnet man die entsprechende duale Basis $\mc{B}^*$ häufig mit $(v_1^*, \dotsc, v_n^*)$ statt mit $(f^\mc{B}_1, \dotsc, f^\mc{B}_n)$, d.h.\ es ist $v_i^* = f^\mc{B}_i$ für alle $1 \leq i \leq n$.

Diese Notation hat den Vorteil, dass sie automatisch zur Verfügung steht, ohne ein zusätzliches Symbol (nämlich $f$) einzuführen, und ohne die Elemente der dualen Basis $\mc{B}^*$ explizit bennen zu müssen.

Sie hat allerdings den Nachteil, dass aus der Notation $\mc{B}^* = (v_1^*, \dotsc, v_n^*)$ nicht klar wird, dass das Element $v_i^*$ von der gesamten Basis $\mc{B}$ abhängt, und nicht nur von dem einzelnen Basiselement $v_i$. So kann es durchaus passieren, dass wir zwei Basen $(v_1, \dotsc, v_n)$ und $(w_1, \dotsc, w_n)$ von $V$ haben, so dass zwar $v_1 = w_1$, aber $v_1^* \neq w_1^*$ (siehe etwa Abschnitt \ref{ssec: second different basis of R2}).

Wir halten also fest:

\begin{shaded}
 Ist $V$ ein endlichdimensionaler $K$-Vektorraum und $\mc{B} = (v_1, \dotsc, v_n)$ ein Basis von $V$, so hängt ein Element $v_i^*$ der dualen Basis $(v_1^*, \dotsc, v_n^*)$ im Allgemeinen von der kompletten Basis $\mc{B}$ ab, und nicht nur von dem einzelnen Basiselement $v_i$.
\end{shaded}










\section{Der Bidualraum}
Nachdem wir den Dualraum endlichdimensionaler Vektorräume verstanden haben, gehen wir jetzt noch einen Schritt weiter, und versuchen den Dualraum des Dualsraums zu verstehen.

\begin{defi}
 Ist $V$ ein $K$-Vektorraum, so heißt $V^{**} \coloneqq (V^*)^*$ der \emph{Bidualraum} von $V$.
\end{defi}

Elemente von $V^{**}$ sind also lineare Abbildungen $V^* \to K$. Nachdem wir den Dualraum mithilfe der dualen Basis verstanden haben, erscheint es nun naheliegend, so etwas wie eine „doppeltduale“ Basis einzuführen. Tatsächlich haben wir bereits eine solche Basis:

Ist $V$ ein endlichdimensionaler $K$-Vektorraum, $\mc{B} = (v_1, \dotsc, v_n)$ eine Basis von $V$, und $\mc{B}^* = (v_1^*, \dotsc, v_1^*)$ die duale Basis von $\mc{B}$, so können wir die duale Basis von $\mc{B}^*$ bilden, um eine Basis $(\mc{B}^*)^* = ((v_1^*)^*, \dotsc, (v_n^*)^*)$ von $(V^*)^* = V^{**}$ zu erhalten. Wir schreiben auch nur $\mc{B}^{**} \coloneqq (\mc{B}^*)^*$ und $v_i^{**} \coloneqq (v_i^*)^*$ für alle $1 \leq i \leq n$.

Der Leser mag beim Anblick dieser itertieren Dualisierung in Panik ausbrechen: Wenn der Dualraum mit seinen duale Basen schon seltsam und umständlich erscheinen, so dürfte der Bidualraum mit der doppeltdualen Basis als nahezu unverstehbar anmuten. Es stellt sich jedoch heraus, das der Bidualraum $V^{**}$ mit seiner Basis $\mc{B}^{**}$ deutlich einfacher zu verstehen ist, also $V^*$ und $\mc{B}^*$.

Für jedes $v \in V$ sei
\[
 \eval_v \colon V^* \to K, \quad f \mapsto f(v).
\]
Man bezeichnet $\eval_v$ als die \emph{Evaluation} (d.h.\ Auswertung) an $v$. Diese Evaluation ist linear, denn für alle $f,g \in V^*$ und $\lambda \in K$ ist
\begin{gather*}
 \eval_v(f+g)
 = (f+g)(v)
 = f(v) + g(v)
 = \eval_v(f) + \eval_v(g)
\shortintertext{sowie}
 \eval_v(\lambda f)
 = (\lambda f)(v)
 = \lambda f(v)
 = \lambda \eval_v(f).
\end{gather*}
Also ist $\eval_v \in V^{**}$. Die Abbildung $\eval \colon V \to V^{**}$, $v \mapsto \eval_v$ ist außerdem selber linear, denn für alle $v,w \in V$ und $\lambda \in K$ ist
\begin{gather*}
 \begin{aligned}
  \eval_{v+w}(f)
  &= f(v+w)
  = f(v) + f(w) \\
  &= \eval_v(f) + \eval_w(f)
  = (\eval_v + \eval_w)(f)
 \end{aligned}
\shortintertext{sowie}
 \eval_{\lambda v}(f)
 = f(\lambda v)
 = \lambda f(v)
 = \lambda \eval_v(f)
 = (\lambda \eval_v)(f)
\end{gather*}
für alle $f \in V^*$, und somit $\eval_{v+w} = \eval_v + \eval_w$ und $\eval_{\lambda v} = \lambda \eval_v$.

Ist nun $f \in V^*$ mit $f = a_1 v_1^* + \dotsb + a_n v_n^*$, so ist
\[
 v_i^{**}(f)
 = v_i^{**}(a_1 v_1^* + \dotsb + a_n v_n^*)
 = a_i
 \quad\text{für alle $1 \leq i \leq n$}.
\]
Wir wissen aber bereits, dass auch $a_i = f(v_i)$ für alle $1 \leq i \leq n$. Also ist
\[
 v_i^{**}(f) = f(v_i) = \eval_{v_i}(f)
 \quad \text{für alle $1 \leq i \leq n$}.
\]
Da dies für alle $f \in V^*$ gilt, ist damit schon $v_i^{**} = \eval_{v_i}$ für alle $1 \leq 1 \leq n$. Die Elemente $v_1^{**}, \dotsc, v_n^{**} \in V^{**}$ sind also nichts anderes als die Evaluationen an den Basiselementen $v_1, \dotsc, v_n$.

Außerdem zeigt dies, dass die lineare Abbildung $\eval \colon V \to V^{**}$, $v \mapsto \eval_v$ die Basis \mbox{$\mc{B} = (v_1, \dotsc, v_n)$} von $V$ auf die Basis $\mc{B}^{**} = (v_1^{**}, \dotsc, v_n^{**})$ von $V^{**}$ abbildet, und somit schon ein Isomorphismus von Vektorräumen ist. Wir erhalten also:

\begin{shaded}
 Ist $V$ ein endlichdimensionaler $K$-Vektorraum, so ist die lineare Abbildung \mbox{$\eval \colon V \to V^{**}$}, $v \mapsto \eval_v$ ein Isomorphismus, wobei $\eval_v$ die Evaluation an $V$ bezeichnet.
 
 Ist ferner $\mc{B} = (v_1, \dotsc, v_n)$ eine Basis von $V$, so ist $v_i^{**} = \eval_{v_i}$ für alle $1 \leq i \leq n$.
\end{shaded}

Hieraus ergibt sich auch eine interessante Beobachtung: Verknüpfen wir den Isomorphismus $\Phi_\mc{B} \colon V \to V^*$, der durch die Basis $\mc{B}$ induziert wird, und den Isomorphismus $\Phi_{\mc{B}^*} \colon V^* \to V^{**}$, der durch die duale Basis induziert wird, so erhalten wir den Isomorphismus $\Phi_{\mc{B}^*} \circ \Phi_{\mc{B}}$. Dieser ist dadurch gegeben, dass 
\[
 (\Phi_{\mc{B}^*} \circ \Phi_{\mc{B}})(v_i)
 = \Phi_{\mc{B}^*}(\Phi_{\mc{B}}(v_i))
 = \Phi_{\mc{B}^*}(v_i^*)
 = v_i^{**}
 = \eval(v_i)
 \quad
 \text{für alle $1 \leq i \leq n$}.
\]
Da $\mc{B}$ eine Basis von $V$ ist, ist damit bereits $\Phi_{\mc{B}^*} \circ \Phi_{\mc{B}} = \eval$. Obwohl also sowohl $\Phi_{\mc{B}^*}$ als auch $\Phi_{\mc{B}}$ von der Wahl unser ursprünglichen Basis $\mc{B}$ abhängen, ist der entstehende Isomorphismus $\Phi_{\mc{B}^*} \circ \Phi_{\mc{B}} = \eval \colon V \to V^**$ unabhängig von der Basis $\mc{B}$.




























\end{document}
a