\section{Beispiele für Körper}


\subsection{Einige kleine Beispiele}
\begin{enumerate}[leftmargin=*]
 \item
  Die rationalen Zahlen $\Qbb$ und die reellen Zahlen $\Rbb$ bilden zusammen mit der üblichen Addition und Multiplikation jeweils einen Körper. $\Qbb$ ist ein Unterkörper von $\Rbb$.
 \item
  Die ganzen Zahlen $\Zbb$ bilden zusammen mit der üblichen Addition und Multiplikation \emph{keinen} Körper: Ansonsten wäre $(\Zbb\setminus\{0\}, \cdot)$ ein Körper, aber $2 \in \Zbb\setminus\{0\}$ besitzt kein multiplikativ Inverses in $\Zbb$.
 \item
  Für einen beliebigen Körper $K$ und $n \in \Nbb$, $n \geq 2$ bilden die quadratischen Matrizen $\Mat(n \times n, K)$ zusammen mit der gewöhnlichen Addition und Multiplikation von Matrizen keinen Körper: Zum einen ist die Multiplikation in diesem Fall nicht kommutativ, da
  \[
   \begin{pmatrix}
    1 & 0 \\
    0 & 0
   \end{pmatrix}
   \begin{pmatrix}
    0 & 1 \\
    0 & 0
   \end{pmatrix}
   =
   \begin{pmatrix}
    0 & 1 \\
    0 & 0
   \end{pmatrix}
   \quad\text{aber}\quad
   \begin{pmatrix}
    0 & 1 \\
    0 & 0
   \end{pmatrix}
   \begin{pmatrix}
    1 & 0 \\
    0 & 0
   \end{pmatrix}
   =
   \begin{pmatrix}
    0 & 0 \\
    0 & 0
   \end{pmatrix}.
  \]
  Zum anderen ist nicht jede Matrix $A \in \Mat(n \times n, K) \setminus \{0\}$ bezüglich der Matrixmultiplikation invertierbar, da etwa
  \[
   \begin{pmatrix}
    1 & 0 \\
    0 & 0
   \end{pmatrix}
   \begin{pmatrix}
    a & b \\
    c & d
   \end{pmatrix}
   =
   \begin{pmatrix}
    a & b \\
    0 & 0
   \end{pmatrix}
   \neq
   \begin{pmatrix}
    1 & 0 \\
    0 & 1
   \end{pmatrix}
  \]
  für alle $a,b,c,d \in K$.
\end{enumerate}



\subsection{Bilder von Körperhomomorphismen}
Sind $K$ und $L$ Körper und ist $\phi \colon K \to L$ ein Körperhomomorphismus, so ist $\im(\phi) \subseteq L$ ein Unterkörper:

Es ist $0 = \phi(0) \in \im(\phi)$. Für $x,y \in \im(\phi)$ gibt es $x', y' \in K$ mit $\phi(x') = x$ und $\phi(y') = y$, weshalb auch
\[
 x+y
 = \phi(x')+\phi(y')
 = \phi(x'+y')
 \in \im(\phi).
\]
Für $x \in \im(\phi)$ gibt es $x' \in K$ mit $\phi(x') = x$, weshalb auch
\[
 -x
 = -\phi(x')
 = \phi(-x') \in \im(\phi).
\]
Also ist $\im(\phi)$ eine Untergruppe der additiven Gruppe von $L$.

\begin{bem}
 Kürzer und eleganter lässt sich die obige Rechnung so zusammenfassen: Da $\phi$ insbesondere ein Gruppenhomomorphismus von der additiven Gruppe von $K$ in die additive Gruppe von $L$ ist, ist $\im(\phi) \subseteq L$ eine abelsche Untergruppe, da Bilder von Gruppenhomomorphismen stets Untergruppen sind. 
\end{bem}

Es ist $1 = \phi(1) \in \im(\phi)$. Sind $x,y \in \im(\phi)$, so gibt es $x',y' \in K$ mit $x = \phi(x')$ und $y = \phi(x')$, weshalb dann auch
\[
 x \cdot y = \phi(x') \cdot \phi(y') = \phi(x' \cdot y') \im(\phi).
\]
Ist schließlich $x \in \im(\phi)$ mit $x \neq 0$, so gibt es $x' \in K$ mit $x = \phi(x')$. Da $x \neq 0$ ist auch $x \neq 0$. Da $\phi$ ein Körperhomomorphismus ist, ist somit $\phi(x')$ invertierbar mit $\phi(x')^{-1} = \phi((x')^{-1}) \in \im(\phi)$, also $x^{-1} = \phi(x')^{-1} \in \im(\phi)$.

Insgesamt zeigt dies, dass $\im(\phi)$ ein Unterkörper von $L$ ist.

\begin{bem}
 \begin{enumerate}[leftmargin=*]
  \item
   Da $\phi$ als Körperhomomorphismus injektiv ist, ist die Abbildung $K \to \im(\phi), x \mapsto \phi(x)$ ein bijektiv und ein Körperhomomorphismus, also ein Körperisomorphismus. Inbesondere ist also $\im(\phi)$ isomorph zu $K$.
  \item
   Ist $L$ ein Körper und $K \subseteq L$ ein Unterkörper, so ist die Inklusion $K \to L$, $x \mapsto x$ ein Körperhomomorphismus. Es ist also jeder Unterkörper von $L$ gegeben als Bild eines Körperhomomorphismus $\phi \colon K \to L$, und wie bereits zuvor bemerkt $\im(\phi)$ ist isomorph zu $K$.
   
   Man kann deshalb einen Unterkörper $K$ eines Körpers $L$ statt als eine Teilmenge, die unter entsprechenden Rechenoperationen abgeschlossen ist, und somit selber wieder einen Körper bildet, auch als einen Körperhomomorphismus $K \to L$ definieren.
 \end{enumerate}
\end{bem}





\subsection{Die komplexen Zahlen \texorpdfstring{$\Cbb$}{C}}
Die komplexen Zahlen $\Cbb$ sind ein Körper, der die reellen Zahlen $\Rbb$ enthält, und in dem es ein Element $i \in \Cbb$ gibt, für das $i^2 = -1$. Wir geben an dieser Stelle zwei mögliche Konstruktionen der komplexen Zahlen an:



\subsubsection{Die komplexen Zahlen als besserer \texorpdfstring{$\Rbb^2$}{R2}}
Wir beginnen mit der additiven Gruppe $\Cbb \coloneqq \Rbb^2 = \{(a,b) \mid a,b \in \Rbb\}$. Die Addition ist gegeben durch
\[
 (a_1, b_1) + (a_2, b_2) = (a_1 + a_2, b_1 + b_2)
 \quad\text{für alle $(a_1, b_1), (a_2, b_2) \in \Cbb$}.
\]
Wir definieren zusätzlich eine Multiplikation $\cdot$ auf $\Cbb$ durch
\[
 (a_1, b_1) \cdot (a_2, b_2)
 = (a_1 a_2 - b_1 b_2, a_1 b_2 + a_2 b_1)
 \quad\text{für alle $(a_1, b_1), (a_2, b_2) \in \Cbb$}.
\]

\begin{beh}
 $\Cbb$ ist zusammen mit der obigen Addition und Multiplikation ein Körper.
\end{beh}
\begin{proof}
 Wir wissen bereits, dass $\Cbb$ zusammen mit der Addition $+$ eine abelsche Gruppe bildet. Das additiv neutrale Element ist $(0,0)$ und das additiv Inverse zu $(a,b) \in \Cbb$ ist $(-a,-b)$ Für alle $(a_1, b_1), (a_2, b_2), (a_3, b_3) \in \Cbb$ ist
 \begin{align*}
   &\, (a_1, b_1) \cdot ((a_2, b_2) \cdot (a_3, b_3))
  = (a_1, b_1) \cdot (a_2 a_3 - b_2 b_3, a_2 b_3 + a_3 b_2) \\
  =&\, (a_1 a_2 a_3 - a_1 b_2 b_3 - a_2 b_1 b_3 - a_3 b_1 b_2, a_2 a_3 b_1 - b_1 b_2 b_3 + a_1 a_2 b_3 + a_1 a_3 b_2) \\
  =&\, (a_1 a_2 - b_1 b_2, a_1 b_2 + a_2 b_1) \cdot (a_3, b_3)
  = ((a_1, b_1) \cdot (a_2, b_2)) \cdot (a_3, b_3),
 \end{align*}
 also ist die Multiplikation assoziativ. Für alle $(a_1, b_1), (a_2, b_2) \in \Cbb$ ist
 \begin{align*}
  (a_1, b_1) \cdot (a_2, b_2)
  &= (a_1 a_2 - b_1 b_2, a_1 b_2 + a_2 b_1) \\
  &= (a_2 a_1 - b_2 b_1, a_2 b_1 + a_1 b_2)
  = (a_2, b_2) \cdot (a_1, b_1),
 \end{align*}
 also ist die Multiplikation kommutativ. Für alle $(a,b) \in \Cbb$ ist
 \[
  (1,0) \cdot (a,b) = (1 \cdot a - 0 \cdot b, 0 \cdot a + 1 \cdot b) = (a,b),
 \]
 also ist $(1,0)$ neutral bezüglich der Multiplikation. Ist $(a,b) \in \Cbb$ mit $(a,b) \neq (0,0)$, so ist $a \neq 0$ oder $b \neq 0$, also $a^2+b^2 \neq 0$. Daher ist
 \[
  (a,b) \cdot \left(\frac{a}{a^2+b^2}, -\frac{b}{a^2+b^2}\right)
  = \left( a \frac{a}{a^2+b^2} + b \frac{b}{a^2+b^2}, b \frac{a}{a^2+b^2} - a \frac{b}{a^2+b^2} \right)
  = (1,0).
 \]
 Also ist $(a/(a^2+b^2), -b/(a^2+b^2))$ multiplikativ invers zu $(a,b)$.
 
 Die Distributivität folgt daraus, dass für alle $(a_1, b_1), (a_2, b_2), (a_3, b_3) \in \Cbb$
 \begin{align*}
   &\, (a_1, b_1) \cdot ((a_2, b_2) + (a_3, b_3)) \\
  =&\,    (a_1, b_1) \cdot (a_2 + a_3, b_2 + b_3) \\
  =&\, (a_1 (a_2 + a_3) - b_1 (b_2 + b_3), a_1 (b_2 + b_3) + (a_2 + a_3) b_1) \\
  =&\, (a_1 a_2 + a_1 a_3 - b_1 b_2 - b_1 b_3, a_1 b_2 + a_1 b_3 + a_2 b_1 + a_3 b_1) \\
  =&\, (a_1 a_2 - b_1 b_2, a_1 b_2 + a_2 b_1) + (a_1 a_3 - b_1 b_3, a_1 b_3 + a_3 b_1) \\
  =&\, (a_1, b_1) \cdot (a_2, b_2) + (a_1, b_1) \cdot (a_3, b_3).
 \end{align*}
 Ingesamt zeigt dies, dass $\Cbb$ zusammen mit der angegebenen Addition und Multiplikation ein Körper ist.
\end{proof}

Wir wollen noch etwas Notation einführen, um das Rechnen mit $\Cbb$ einfacher zu machen: Wir schreiben $0 = (0,0)$ und $1 = (1,0)$. Wir bemerken zunächst, dass für alle $a, a' \in \Rbb$
\[
 (a,0) + (a',0) = (a + a', 0)
 \quad\text{und}\quad
 (a,0) \cdot (a',0) = (a a' - 0 \cdot 0, a \cdot 0 + 0 \cdot a') = (a a', 0)
\]
und dass $0, 1 \in \{(a,0) \mid a \in \Rbb\}$. Also ist $\{(a,0) \mid a \in \Rbb\} \subseteq \Cbb$ ein Unterkörper und für die Abbildung $\phi \colon \Rbb \to \{(a,0) \mid a \in \Rbb\}$, $x \mapsto (x,0)$ gilt:
\begin{enumerate}[label=\roman*)]
 \item
  $\phi$ ist bijektiv.
 \item
  Für alle $a,a' \in \Rbb$ ist $\phi(a+a') = \phi(a)+\phi(a')$ und $\phi(aa') = \phi(a)\phi(a')$.
 \item
  Es ist $\phi(1) = 1$.
\end{enumerate}
Dies erlaubt es uns, $\Rbb$ mit dem Unterkörper $\{(a,0) \mid a \in \Rbb\} \subseteq \Cbb$ zu identifizieren. Wir unterscheiden also im Folgenden nicht mehr zwischen $a \in \Rbb$ und $(a,0) \in \Cbb$.

Wir schreiben $i \coloneqq (0,1) \in \Cbb$. Für jedes $b \in \Rbb$ gilt
\[
 b \cdot i = (b,0) \cdot (0,1) = (b \cdot 0 - 0 \cdot 1, b \cdot 1 + 0 \cdot 0) = (0,b).
\]
Ein beliebiges Element $z = (a,b) \in \Cbb$ lässt sich daher als
\[
 z = (a,b) = (a,0) + (0,b) = a + bi
\]
schreiben. Die beiden reellen Zahlen $a,b \in \Rbb$ mit $z = a + bi$ sind eindeutig, da $z = (a,b)$.

Bemerke, dass $i^2 = (0,1) \cdot (0,1) = (0 \cdot 0 - 1 \cdot 1, 0 \cdot 1 + 1 \cdot 0) = (-1, 0) = -1$. Die Addition lässt sich nun auch so darstellen, dass für alle $a_1, a_2, b_1, b_2 \in \Rbb$
\[
 (a_1 + b_1 i) + (a_2 + b_2 i)
 = a_1 + a_2 + b_1 i + b_2 i
 = (a_1 + a_2) + (b_1 + b_2) i,
\]
und die Multiplikation so, dass
\begin{align*}
 (a_1 + b_1 i)(a_2 + b_2 i)
 &= a_1 a_2 + a_1 b_2 i + a_2 b_1 i + b_1 b_2 i^2 \\
 &= a_1 a_2 - b_1 b_2 + a_1 b_2 i + a_2 b_1 i \\
 &= (a_1 a_2 - b_1 b_2) + (a_1 b_2 + a_2 b_1) i.
\end{align*}



\subsubsection{Die komplexen Zahlen als reelle (\texorpdfstring{$2 \times 2$}{2x2})-Matrizen}
Unabhängig von der ersten Methode geben wir noch eine weitere Konstruktion der komplexen Zahlen an: Es sei
\[
 C \coloneqq
 \left\{
  \begin{pmatrix}
   a &           -b \\
   b & \phantom{-}a
  \end{pmatrix}
  \,\middle|\,
  a,b \in \Rbb
 \right\}
 \subseteq \Mat(2 \times 2, \Rbb).
\]
Wir zeigen, dass $C$ zusammen mit der gewöhnlichen Matrixaddition und -multiplikation einen Körper bildet:

Für alle $a, b \in \Rbb$ schreiben wir abkürzend
\[
 Z(a,b)
 \coloneqq
 \begin{pmatrix}
  a &           -b \\
  b & \phantom{-}a
 \end{pmatrix}.
\]
Für alle $a_1, a_2, b_1, b_2 \in \Rbb$ ist
\begin{gather*}
 \begin{aligned}
  Z(a_1, b_1) + Z(a_2, b_2)
  &=
  \begin{pmatrix}
   a_1 &           -b_1 \\
   b_1 & \phantom{-}a_1
  \end{pmatrix}
  +
  \begin{pmatrix}
   a_2 &           -b_2 \\
   b_2 & \phantom{-}a_2
  \end{pmatrix} \\
  &=
  \begin{pmatrix}
   a_1+a_2 &           -b_1+b_2 \\
   b_1+b_2 & \phantom{-}a_1+a_2
  \end{pmatrix}
  =
  Z(a_1+a_2, b_1+b_2)
 \end{aligned}
\shortintertext{und außerdem}
 \begin{aligned}
  Z(a_1,b_1) \cdot Z(a_2,b_2)
  &=
  \begin{pmatrix}
   a_1 &           -b_1 \\
   b_1 & \phantom{-}a_1
  \end{pmatrix}
  \begin{pmatrix}
   a_2 &           -b_2 \\
   b_2 & \phantom{-}a_2
  \end{pmatrix} \\
  &=
  \begin{pmatrix}
   a_1 a_2 - b_1 b_2 &           -a_1 b_2 - a_2 b_1 \\
   a_1 b_2 + a_2 b_1 & \phantom{-}a_1 a_2 - b_1 b_2
  \end{pmatrix}
  = Z(a_1 a_2 - b_1 b_2, a_1 b_2 + a_2 b_1).
 \end{aligned}
\end{gather*}

Um zu zeigen, dass $C$ bezüglich der Addition eine abelsche Gruppe bildet genügt es zu zeigen, dass $C$ eine Untergruppe der additiven Gruppe von $\Mat(2 \times 2, \Rbb)$ bildet.

Es ist $0 = Z(0,0) \in C$ und für alle $a_1, a_2, b_1, b_2 \in \Rbb$ ist
\[
 Z(a_1, b_1) + Z(a_2, b_2) = Z(a_1 + a_2, b_1 + b_2) \in C.
\]
Für alle $a,b \in \Rbb$ ist außerdem
\[
 -Z(a,b)
 =
 -\begin{pmatrix}
  a &           -b \\
  b & \phantom{-}a
 \end{pmatrix}
 =
 \begin{pmatrix}
  -a & \phantom{-}b \\
  -b &           -a
 \end{pmatrix}
 = Z(-a,-b) \in C.
\]
Also ist $C$ eine abelsche Untergruppe der additiven Gruppe von $\Mat(2 \times 2, \Rbb)$.

Die Assoziativität der Multiplikation ist bekannt. Die Matrixmultiplikation ist auf $C$ kommutativ, da für alle $a_1, a_2, b_1, b_2 \in \Rbb$
\begin{align*}
 Z(a_1, b_1) \cdot Z(a_2, b_2)
 &= Z(a_1 a_2 - b_1 b_2, a_1 b_2 + a_2 b_1) \\
 &= Z(a_2 a_1 - b_2 b_1, a_2 b_1 + a_1 b_2)
 = Z(a_2, b_2) \cdot Z(a_1, b_1).
\end{align*}

Da die Einheitsmatrix $I_2 \in \Mat(2 \times 2, \Rbb)$ das Einselement bezüglich der Matrixmultiplikation ist und $I_2 \in C$ ist $I_2$ das Einselement in $C$. Für $a,b \in \Rbb$ mit $Z(a,b) \neq 0$ ist $a \neq 0$ oder $b \neq 0$ und somit $a^2+b^2 \neq 0$. Da
\begin{align*}
  &\, Z(a,b) Z\left(\frac{a}{a^2+b^2}, -\frac{b}{a^2+b^2}\right) \\
 =&\, Z\left( a \frac{a}{a^2+b^2} + b\frac{b}{a^2+b^2}, b\frac{a}{a^2+b^2} - a\frac{b}{a^2+b^2} \right) \\
 =&\, Z(1, 0)
 = I_2
\end{align*}
ist $Z(a/(a^2+b^2), -b/(a^2+b^2))$ das multiplikativ Inverse zu $Z(a,b)$ (da die Matrixmultiplikation auf $C$ kommutativ ist, zeigt die obige Rechnung, dass $Z(a/(a^2+b^2), -b/(a^2+b^2))$ bereits beidseitig Invers zu $Z(a,b)$ ist).

Insgesamt zeigt dies, dass $C$ mit der üblichen Matrixaddition und -multiplikation einen Körper bildet.


\subsubsection{Äquivalenz der beiden Konstruktionen}
Die beiden Konstruktionen der komplexen Zahlen sind insofern äquivalent, als dass die Abbildung
\[
 \phi \colon \Cbb \to C, a + ib \mapsto Z(a,b) \quad \text{für alle $a,b \in \Rbb$}
\]
ein Körperisomorphismus ist: Die Bijektivität von $\phi$ folgt daraus, dass die beiden Abbildungen $\psi_1 \colon \Rbb^2 \to \Cbb, (a,b) \mapsto a+ib$ und $\psi_2 \colon \Rbb^2 \to C, (a,b) \mapsto Z(a,b)$ bijektiv sind, und daher auch $\phi = \psi_2 \psi_1^{-1}$. Außerdem ist $\phi(1) = \phi(1)= Z(1,0) = I_2$. Für alle $a,a',b,b' \in \Rbb$ ist zudem
\begin{gather*}
 \begin{aligned}
  \phi((a+ib)+(a'+ib'))
  &= \phi((a+a')+i(b+b'))
  = Z(a+a',b+b') \\
  &= Z(a,b) + Z(a',b')
  = \phi(a+ib) + \phi(a'+ib')
 \end{aligned}
\shortintertext{und}
 \begin{aligned}
  \phi( (a+ib) \cdot (a'+ib') )
  &= \phi( (aa'-bb')+i(ab'+a'b) ) \\
  &= Z(aa'-bb', ab'+a'b)
  = Z(a,b) \cdot Z(a',b').
 \end{aligned}
\end{gather*}
Ingesamt zeigt dies, dass $\phi$ ein Körperisomorphismus ist.





\subsection{Quadratische Körpererweiterungen von \texorpdfstring{$\Qbb$}{Q}}



\subsubsection{\texorpdfstring{$\Qbb[i]$}{Q[i]}}
Es sei
\[
 \Qbb[i]
 \coloneqq
 \{ q_1 + q_2 i \mid q_1, q_2 \in \Qbb \}
 \subseteq \Cbb
\]
Wir zeigen, dass $\Qbb[i]$ ein Unterkörper von $\Cbb$ ist.

Es ist $0 = 0 + 0 \cdot i \in \Qbb[i]$. Für $z,w \in \Qbb[i]$ ist $z = q_1 + q_2 i$ und $w = p_1 + p_2 i$ mit $q_1, q_2, p_1, p_2 \in \Qbb$. Daher ist auch
\[
 z + w
 = (q_1 + q_2 i) + (p_1 + p_2 i)
 = (q_1 + p_1) + (q_2 + p_2) i
 \in \Qbb[i].
\]
Ist $z \in \Qbb$ so ist $z = q_1 + q_2 i$ mit $q_1, q_2 \in \Qbb$, weshalb auch
\[
 -z = -(q_1 + q_2 i) = (-q_1) + (-q_2) i \in \Qbb[i].
\]
Das zeigt, dass $\Qbb[i]$ eine Untergruppe der additiven Gruppe von $\Cbb$ ist.

Es ist auch $1 = 1 + 0 \cdot i \in \Qbb[i]$. Sind $z, w \in \Qbb[i]$ so gibt es $q_1, q_2, p_1, p_2 \in \Qbb$ mit $z = q_1 + i q_2$ und $w = p_1 + i p_2$. Es ist daher auch
\[
 z \cdot w
 = (q_1 + iq_2)(p_1 + ip_2)
 = (q_1 p_1 - q_2 p_2) + i(q_1 p_2 + q_2 p_1)
 \in \Qbb[i].
\]
Ist $z \in \Cbb[i]$ mit $z \neq 0$ so ist $z = q_1 + i q_2$ mit $q_1, q_2 \in \Qbb$ und $q_1 \neq 0$ oder $q_2 \neq 0$. Da $z \neq 0$ ist auch $q_1 - i q_2 = \overline{z} \neq 0$ Deshalb ist auch
\[
 \frac{1}{z}
 = \frac{1}{q_1 + i q_2}
 = \frac{q_1 - i q_2}{(q_1 + i q_2)(q_1 - i q_2)}
 = \frac{q_1 - i q_2}{q_1^2 + q_2^2}
 = \frac{q_1}{q_1^2 + q_2^2} + i \frac{-q_2}{q_1^2+q_2^2}
 \in \Qbb[i].
\]

Ingesamt zeigt dies, dass $\Qbb[i]$ ein Unterkörper von $\Cbb$ ist. $\Qbb[i]$ ist der kleinste Unterkörper von $\Cbb$, der $i$ enthält, d.h.\ ist $K \subseteq \Cbb$ ein Unterkörper mit $i \in K$, so ist bereits $\Qbb[i] \subseteq K$.



\subsubsection{\texorpdfstring{$\Qbb\left[\sqrt{2}\right]$}{Q[sqrt(2)]}}
Wir zeigen zunächst, dass
\[
 \Qbb\left[\sqrt{2}\right] \coloneqq \{ q_1 + q_2 \sqrt{2} \mid q_1, q_2 \in \Qbb \} \subseteq \Rbb
\]
ein Unterkörper von $\Rbb$ ist:

Es ist $0 = 0 + 0 \cdot 0 \in \Qbb\left[\sqrt{2}\right]$. Sind $x,y \in \Qbb\left[\sqrt{2}\right]$, so gibt es $q_1, q_2, p_1, p_2 \in \Qbb$ mit $x = q_1 + q_2 \sqrt{2}$ und $y = p_1 + p_2 \sqrt{2}$. Es ist daher auch
\[
 x+y
 = (q_1 + q_2 \sqrt{2}) + (p_1 + p_2 \sqrt{2})
 = (q_1 + p_1) + (q_2 + p_2) \sqrt{2}
 \in \Qbb\left[\sqrt{2}\right].
\]
Ist $x \in \Qbb\left[\sqrt{2}\right]$ mit $x = q_1 + q_2 \sqrt{2}$ für $q_1, q_2 \in \Qbb$, so ist auch
\[
 -x = -(q_1 + q_2 \sqrt{2}) = (-q_1) + (-q_2) \sqrt{2} \in \Qbb\left[\sqrt{2}\right].
\]
Das zeigt, dass $\Qbb\left[\sqrt{2}\right]$ eine Untergruppe der additiven Gruppe von $\Rbb$ ist.

\begin{bem}
 Die obigen Rechnungen lassen sich auch großteils umgehen: Die Abbildung $\varphi \colon \Qbb^2 \to \Rbb$, $(q_1, q_2) \to \Qbb\left[\sqrt{2}\right]$ ist ein Gruppenhomomorphismus, da
 \begin{align*}
  \varphi( (q_1, q_2) + (p_1, p_2) )
  &= \varphi( (q_1 + p_1, q_2 + p_2) )
  = (q_1 + p_1) + (q_2 + p_2) \sqrt{2} \\
  &= (q_1 +  q_2 \sqrt{2}) + (p_1 + p_2 \sqrt{2})
  = \varphi((q_1, q_2)) + \varphi((p_1,p_2))
 \end{align*}
 für alle $(q_1, q_2), (p_1, p_2) \in \Qbb^2$. Daher ist
 \[
  \Qbb\left[\sqrt{2}\right]
  = \{q_1 + q_2 \sqrt{2} \mid q_1, q_2 \in \Qbb\}
  = \im(\varphi)
 \]
 eine Untergruppe der additiven Gruppe von $\Rbb$.
\end{bem}

Es ist $1 = 1 + 0 \cdot \sqrt{2} \in \Qbb\left[\sqrt{2}\right]$. Für $x, y \in \Qbb\left[\sqrt{2}\right]$ ist $x = q_1 + q_2 \sqrt{2}$ und $y = p_1 + p_2 \sqrt{2}$ mit $q_1, q_2, p_1, p_2 \in \Qbb$. Deshalb ist auch
\[
 x \cdot y
 = (q_1 + q_2 \sqrt{2}) (p_1 + p_2 \sqrt{2})
 = q_1 q_2 + 2 p_1 p_2 + (q_1 p_2 + q_2 p_1) \sqrt{2} \in \Qbb\left[\sqrt{2}\right].
\]

Es sei $x \in \Qbb\left[\sqrt{2}\right]$ mit $x \neq 0$. Dann ist $x = q_1 + q_2 \sqrt{2}$ mit $q_1, q_2 \in \Qbb$. Da $x \neq 0$ ist $q_1 \neq 0$ oder $q_2 \neq 0$. Es folgt, dass auch $q_1 - q_2 \sqrt{2} \neq 0$. Andernfalls wäre nämlich $\sqrt{2} = q_1/q_2$ (falls $q_2 \neq 0$) oder $1/\sqrt{2} = q_2/q_1$ und somit ebenfalls $\sqrt{2} = q_1/q_2$ (falls $q_1 \neq 0$). Dies stünde im Widerspruch zur Irrationalität von $\sqrt{2}$. Es folgt, dass auch
\begin{align*}
 \frac{1}{x}
 &= \frac{1}{q_1 + q_2 \sqrt{2}}
 = \frac{q_1 - q_2 \sqrt{2}}{(q_1 + q_2 \sqrt{2})(q_1 - q_2 \sqrt{2})} \\
 &= \frac{q_1 - q_2 \sqrt{2}}{q_1^2 - 2 q_2^2}
 = \frac{q_1}{q_1^2 - 2q_2^2} + \frac{-q_2}{q_1^2 - 2q_2^2} \sqrt{2}
 \in \Qbb\left[\sqrt{2}\right].
\end{align*}

Insgesamt zeigt dies, dass $\Qbb\left[\sqrt{2}\right]$ ein Unterkörper von $\Rbb$ ist. Man bezeichnet $\Qbb[\sqrt{2}]$ als „$\Qbb$ adjungiert $\sqrt{2}$“. $\Qbb\left[\sqrt{2}\right]$ ist der kleinste Unterkörper von $\Rbb$, der $\sqrt{2}$ enthält, d.h.\ ist $K' \subseteq \Rbb$ ein Unterkörper mit $\sqrt{2} \in K'$, so ist bereits $\Qbb\left[\sqrt{2}\right] \subseteq K'$.





\subsubsection{Die Gaußschen Zahlen}
Die \emph{Gaußschen Zahlen} sind definiert als
\[
 \Zbb[i] \coloneqq \{a + b i \mid a,b \in \Zbb\}.
\]

Wie bereits für $\Qbb[i]$ ergibt sich auch für $\Zbb[i]$, dass es sich um eine additive Untergruppe von $\Cbb$ handelt: Es ist $0 \in \Zbb[i]$. Für $z, w \in \Zbb[i]$ ist $z = a + bi$ und $w = c + di$ mit $a,b,c,d \in \Zbb$, weshalb
\[
 z + w
 = (a + bi) + (c + di)
 = (a+c) + (b+d)i
\]
mit $a+c, b+d \in \Zbb$ und somit $z+w \in \Zbb[i]$. Für $z \in \Zbb[i]$ ist $z = a + bi$ mit $a,b \in \Zbb$, weshalb $-z = -(a+bi) = (-a) + (-b)i$ mit $-a, -b \in \Zbb$ und somit auch $-z \in \Zbb[i]$.


\begin{bem}
 Im Gegensatz zu $\Qbb[i]^\times = \Qbb[i] \cap \Cbb^\times$ ist $\Zbb[i] \cap \Cbb^{\times}$ \emph{keine} Untergruppe von $\Cbb^\times$, da $2 \in \Zbb[i] \cap \Cbb^\times$ aber $1/2 \notin \Zbb[i]$ und somit auch nicht $1/2 \in \Zbb[i] \cap \Cbb^\times$.
\end{bem}














