\chapter{Beispiele für Gruppen}





\section{Verschiedene kleinere Beispiele}





\begin{enumerate}[leftmargin=*]
 \item
  Ist $G = \{x\}$ eine einelementige Menge, so gibt es auf $G$ eine eindeutige Gruppenstruktur, gegeben durch $x \cdot x = x$. Man bezeichnet diese als die \emph{triviale Gruppe}.
 \item
  Ist $G$ eine Gruppe, so ist $\{1\} \subseteq G$ eine Untergruppe. Man bezeichnet diese als die \emph{triviale Untergruppe}.
 \item
  Die reellen Zahlen $\Rbb$ bilden zusammen mit der üblichen Addition $+$ eine abelsche Gruppe: Die Assoziativität und Kommutativität der Addition sind bekannt. $0 \in \Rbb$ ist das neutrale Element bezüglich der Addition, da $x + 0 = 0 + x = x$ für alle $x \in \Rbb$. Für $x \in \Rbb$ ist $-x \in \Rbb$ das inverse Element, da $x + (-x) = (-x) + x = 0$.
 \item
  Nach analoger Argumentation bilden die ganzen Zahlen $\Zbb$, die rationalen Zahlen $\Qbb$ und die komplexen Zahlen $\Cbb$ zusammen mit der gewöhnlichen Addition $+$ jeweils eine abelsche Gruppe. Wir erhalten eine Kette von Untergruppen $\Zbb \subseteq \Qbb \subseteq \Rbb \subseteq \Cbb$.
 \item
  Die natürlichen Zahlen $\Nbb$ bilden zusammen mit der üblichen Addition keine Gruppe: Andernfalls wäre $0 \in \Nbb$ das neutrale Element, da $0+n = n+0 = n$ für alle $n \in \Nbb$. Da es aber kein $n \in \Nbb$ mit $n+1 = 0$ gibt, besäße $1 \in \Nbb$ dann kein inverses Element.
 \item
  Die Menge $\Rbb^\times = \Rbb \setminus \{0\}$ bildet zusammen mit der üblichen Multiplikation $\cdot$ eine abelsche Gruppe: Die Assoziativität und Kommutativität der Multiplikation sind bekannt. $1 \in \Rbb^\times$ ist das neutrale Element bezüglich der Multiplikation, da $1 \cdot x = x \cdot 1 = x$ für alle $x \in \Rbb^\times$. Für $x \in \Rbb^\times$ ist $1/x \in \Rbb^\times$ das multiplikativ Inverse, da $x \cdot (1/x) = (1/x) \cdot x = 1$.
 \item
  Analog ergibt sich, dass $\Qbb^\times = \Qbb \setminus \{0\}$ und $\Cbb^\times = \Cbb \setminus \{0\}$ zusammen mit der üblichen Multiplikation abelsche Gruppen bilden. Wir erhalten somit eine Kette von Untergruppen $\Qbb^\times \subseteq \Rbb^\times \subseteq \Cbb^\times$.
 \item
  Die ganzen Zahlen $\Zbb$ bilden zusammen mit der üblichen Multiplikation keine Gruppe: Da $1 \cdot n = n \cdot 1 = n$ für alle $n \in \Zbb$ wäre $1$ das neutrale Element bezüglich der Multiplikation. Da es aber kein $n \in \Zbb$ mit $n \cdot 2 = 1$ gibt, besäße $2 \in \Zbb$ kein multiplikativ inverses Element.
 \item
  Ist $K$ ein Körper, so ist $K$ bezüglich der Addition eine abelsche Gruppe und $K^\times = K \setminus \{0\}$ zusammen mit der Multiplikation eine abelsche Gruppe.
 \item
  Es sei $\Rbb_+ \coloneqq \{x \in \Rbb \mid x > 0\} \subseteq \Rbb^\times$ die Menge der positiven reellen Zahlen. Es ist $1 \in \Rbb_+$, für alle $x,y \in \Rbb_+$ ist auch $x \cdot y \in \Rbb_+$, und für jedes $x \in \Rbb_+$ ist auch $1/x \in \Rbb_+$. Also ist $\Rbb_+ \subseteq \Rbb^\times$ eine Untergruppe. Insbesondere ist also $\Rbb_+$ zusammen mit der üblichen Multiplikation eine abelsche Gruppe.
 \item
  Die Menge der negativen reellen Zahlen $\Rbb_- \coloneqq \{x \in \Rbb \mid x < 0\}$ ist zusammen mit der üblichen Multiplikation keine Gruppe, da zwa, $-1 \in \Rbb_-$, aber $(-1) \cdot (-1) = 1 \notin \Rbb_-$.
 \item
  Für $x,y \in \Rbb_-$ sei $x * y \coloneqq -xy$. Wir zeigen, dass $\Rbb_-$ zusammen mit $*$ eine abelsche Gruppe bildet:
  
  Für $x,y \in \Rbb_-$ ist $x,y < 0$, also $xy > 0$ und somit $-xy < 0$. Also ist die Verknüpfung $\Rbb_- \times \Rbb_- \to \Rbb_-$, $(x,y) \mapsto x*y$ wohldefiniert.
  
  Für alle $x,y,z \in \Rbb_-$ ist
  \begin{align*}
   (x * y) * z
   &= (-xy) * z
   = -((-xy)z) \\
   &= xyz
   = -(x(-yz))
   = x * (-yz)
   = x * (y * z),
  \end{align*}
  also ist $*$ assoziativ. Für alle $x,y \in \Rbb_-$ ist
  \[
   x * y = -xy = -yx = y * x
  \]
  also ist $*$ kommutativ. Für alle $x \in \Rbb_-$ ist $(-1) * x = -((-1)x) = x$, also ist $-1$ ein neutrales Element bezüglich $*$. Für $x \in \Rbb_-$ ist $x * (1/x) = -(x \cdot (1/x)) = -1$, also ist $1/x$ bezüglich $*$ invers zu $x$.
  
  Ingesamt zeigt dies, dass $(\Rbb_-, *)$ eine abelsche Gruppe ist.
 \item
  Für alle $n \in \Nbb$, $n \geq 1$ ist $\Rbb^n = \{(x_1, \dotsc, x_n) \mid x_1, \dotsc, x_n \in \Rbb\}$ zusammen mit der \emph{eintragsweisen Addition}
  \[
   (x_1, \dotsc, x_n) + (y_1, \dotsc, y_n)
   = (x_1 + y_1, \dotsc, x_n + y_n)
  \]
  eine abelsche Gruppe:
  
  Für alle $(x_1, \dotsc, x_n), (y_1, \dotsc, y_n), (z_1, \dotsc, z_n) \in \Rbb^n$ ist
  \begin{align*}
    &\, (x_1, \dotsc, x_n) + ( (y_1, \dotsc, y_n) + (z_1, \dotsc, z_n) ) \\
   =&\, (x_1, \dotsc, x_n) + (y_1 + z_1, \dotsc, y_n + z_n) \\
   =&\, (x_1 + y_1 + z_1, \dotsc, x_n + y_n + z_n) \\
   =&\, (x_1 + y_1, \dotsc, x_n + y_n) + (z_1, \dotsc, z_n) \\
   =&\, ( (x_1, \dotsc, x_n) + (y_1, \dotsc, y_n) ) + (z_1, \dotsc, z_n),
  \end{align*}
  also ist die Addition assoziativ.
  
  Für alle $(x_1, \dotsc, x_n), (y_1, \dotsc, y_n) \in \Rbb^n$ ist
  \begin{align*}
   (x_1, \dotsc, y_n) + (y_1, \dotsc, y_n)
   &= (x_1 + y_1, \dotsc, x_n + y_n) \\
   &= (y_1 + x_1, \dotsc, y_n + x_n)
   = (y_1, \dotsc, y_n) + (x_1, \dotsc, x_n),
  \end{align*}
  also ist die Addition tatsächlich kommutativ.
  
  Für jedes $(x_1, \dotsc, x_n) \in \Rbb^n$ ist
  \[
   (0, \dotsc, 0) + (x_1, \dotsc, x_n)
   = (0 + x_1, \dotsc, 0 + x_n)
   = (x_1, \dotsc, x_n),
  \]
  also ist $(0, \dotsc, 0) \in \Rbb^n$ neutral bezüglich der Addition.
  
  Für jedes $(x_1, \dotsc, x_n) \in \Rbb^n$ ist
  \[
   (-x_1, \dotsc, -x_n) + (x_1, \dotsc, x_n)
   = (-x_1 + x_1, \dotsc, -x_n + x_n)
   = (0, \dotsc, 0),
  \]
  also ist $(-x_1, \dotsc, -x_n)$ invers zu $(x_1, \dotsc, x_n)$ bezüglich der Addition.
  
  Ingesamt zeigt dies, dass $(\Rbb^n, +)$ eine abelsche Gruppe ist.
 \item
  Analog ergibt sich, dass $\Zbb^n$, $\Qbb^n$ und $\Cbb^n$ zusammen mit der eintragsweisen Addition jeweils Gruppen bilden. Es ergibt sich damit eine Kette von Untergruppen $\Zbb^n \subseteq \Qbb^n \subseteq \Rbb^n \subseteq \Cbb^n$.
 \item
  Ist allgemeiner $K$ ein Körper, so ist $K^n$ mit der eintragsweisen Addition eine abelsche Gruppe.
\end{enumerate}





\section{Untergruppen bezüglich Gruppenhomomorphismen}\label{sec: subgroups and group homomorphisms}
Es seien $G$ und $H$ Gruppen und es sei $\varphi \colon G \to H$ ein Gruppenhomomorphismus.



\subsection{Der Kern eines Gruppenhomomorphismus}
\begin{defi}
 Der \emph{Kern} von $\varphi$ ist
 \[
  \ker(\varphi) \coloneqq \{g \in G \mid \varphi(g) = 1\}.
 \]
\end{defi}

Wir zeigen, dass $\ker(\varphi)$ eine Untergruppe von $G$ ist: Da $\varphi(1) = 1$ ist $1 \in \ker(\varphi)$. Sind $x,y \in \ker(\varphi)$, so ist $\varphi(x) = \varphi(y) = 1$ und deshalb
\[
 \varphi(x \cdot y)
 = \varphi(x) \cdot \varphi(y)
 = 1 \cdot 1
 = 1,
\]
also auch $x \cdot y \in \ker(\varphi)$. Ist $x \in \ker(\varphi)$, so ist $\varphi(x) = 1$ und deshalb
\[
 \varphi(x^{-1})
 = \varphi(x)^{-1}
 = 1^{-1}
 = 1,
\]
also auch $x^{-1} \in \ker(\varphi)$. Ingesamt zeigt dies, dass $\ker(\varphi)$ eine Untergruppe von $G$ ist.


\begin{bem}
 Es gilt, dass $\varphi$ genau dann injektiv ist, falls $\ker(\varphi) = \{1\}$, falls also der Kern möglichst klein ist:
 
 Ist $\varphi$ injektiv und $x \in \ker(\varphi)$ so ist $\varphi(x) = 1 = \varphi(1)$, wegen der Injektivität von $\varphi$ also $x = 1$. Somit ist dann $\ker(\varphi) = \{1\}$.
 
 Ist $\ker(\varphi) = \{1\}$ und sind $x,y \in G$ mit $\varphi(x) = \varphi(y)$, so ist
 \[
  \varphi(x\cdot y^{-1})
  = \varphi(x)\cdot \varphi(y^{-1})
  = \varphi(x) \cdot \varphi(y)^{-1}
  = \varphi(x) \cdot \varphi(x)^{-1}
  = 1,
 \]
 und deshalb $x \cdot y^{-1} \in \ker(\varphi) = \{1\}$. Da $x \cdot y^{-1} = 1$ ergibt sich durch Multiplikation mit $y$ von rechts, dass $x = y$. Also ist $\varphi$ injektiv.
\end{bem}



\subsection{Das Bild eines Gruppenhomomorphismus}
Wir zeigen, dass das Bild
\[
 \im(\varphi) = \varphi(G) = \{\varphi(g) \mid g \in G\}
\]
eine Untergruppe von $H$ ist: Da $1 = \varphi(1)$ ist $1 \in \im(\varphi)$. Sind $x,y \in \im(\varphi)$, so gibt es $x', y' \in G$ mit $\varphi(x') = x$ und $\varphi(y') = y$. Damit ist
\[
 x \cdot y
 = \varphi(x') \cdot \varphi(y')
 = \varphi(x' \cdot y'),
\]
also auch $x \cdot y \in \im(\varphi)$. Ist $x \in \im(\varphi)$, so gibt es $x' \in G$ mit $\varphi(x') = x$. Da damit
\[
 x^{-1}
 = \varphi(x')^{-1}
 = \varphi((x')^{-1})
\]
ist auch $x^{-1} \in \im(\varphi)$. Ingesamt zeigt dies, dass $\im(\varphi)$ ein Untergruppe von $H$ ist.

Ist $G$ abelsch, so ist es auch $\im(\varphi)$: Sind nämlich $x, y \in \im(\varphi)$, so gibt es $x', y' \in G$ mit $x = \varphi(x')$ und $y = \varphi(y')$, weshalb
\[
 x \cdot y
 = \varphi(x') \cdot \varphi(y')
 = \varphi(x' \cdot y')
 = \varphi(y' \cdot x')
 = \varphi(y') \cdot \varphi(x')
 = y \cdot x.
\]



\subsection{Bilder von Untergruppen}
Es sei $K \subseteq G$ eine Untergruppe. Wir zeigen, dass das Bild von $K$ unter $\varphi$, also
\[
 \varphi(K) = \{\varphi(k) \mid k \in K\},
\]
eine Untergruppe von $H$ ist: Da $1 \in K$ ist $1 = \varphi(1) \in H$. Sind $x,y \in \varphi(K)$ so gibt es $x', y' \in K$ mit $\varphi(x') = x$ und $\varphi(y') = y$. Da $K$ ein Untergruppe ist, ist auch $x' \cdot y' \in K$. Da außerdem
\[
 x \cdot y
 = \varphi(x') \cdot \varphi(y')
 = \varphi(x' \cdot y')
\]
ist dann auch $x \cdot y \in \varphi(K)$. Ist $x \in \varphi(K)$, so gibt es $x' \in K$ mit $\varphi(x') = x$. Da $K$ eine Untergruppe ist, ist auch $(x')^{-1} \in K$. Da außerdem
\[
 x^{-1}
 = \varphi(x')^{-1}
 = \varphi((x')^{-1})
\]
ist dann auch $x^{-1} \in \varphi(K)$. Ingesamt zeigt dies, dass $\varphi(K)$ eine Untergruppe von $K$ ist.

Ist $K$ abelsch, so ist es auch $\varphi(K)$: Für $x,y \in \im(\varphi)$ gibt es nämlich $x', y' \in K$ mit $x = \varphi(x')$ und $y = \varphi(y')$, weshalb
\[
 x \cdot y
 = \varphi(x') \cdot \varphi(y')
 = \varphi(x' \cdot y')
 = \varphi(y' \cdot x')
 = \varphi(y') \cdot \varphi(x')
 = y \cdot x.
\]


\begin{bem}
 Es fällt auf, dass der Beweis, dass $\varphi(K)$ ein Untergruppe ist, analog zu dem Beweis verläuft, dass $\im(\varphi)$ eine Untegruppe ist. Dies ist kein Zufall: Da $\varphi$ ein Gruppenhomomorphismus ist, ist auch die Einschränkung $\varphi|_K \colon K \to H$ ein Gruppenhomomorphismus. Daraus folgt direkt, dass $\varphi(K) = \im(\varphi|_K)$ eine Untergruppe von $H$ ist.
\end{bem}



\subsection{Urbilder von Untergruppen}
Es sei $K \subseteq H$ eine Untergruppe. Wir zeigen, dass das Urbild von $K$ unter $\varphi$, also
\[
 \varphi^{-1}(K) = \{x \in G_1 \mid \varphi(x) \in K\},
\]
eine Untergruppe von $G$ ist: Da $K$ eine Untergruppe ist, ist $\varphi(1) = 1 \in K$. Also ist $1 \in \varphi^{-1}(K)$. Sind $x,y \in \varphi^{-1}(K)$, so ist $\varphi(x), \varphi(y) \in K$. Da $K$ eine Untergruppe ist, ist damit auch $\varphi(xy) = \varphi(x)\varphi(y) \in K$, also $xy \in \varphi^{-1}(K)$. Ist $x \in \varphi^{-1}(K)$, so ist $\varphi(x) \in K$. Da $K$ eine Untergruppe ist, ist damit auch $\varphi(x^{-1}) = \varphi(x)^{-1} \in K$, also $x^{-1} \in \varphi^{-1}(K)$. Insgesamt zeigt dies, dass $\varphi^{-1}(K)$ eine Untergruppe von $G$ ist.


\begin{bem}
 Indem man $K = \{1\}$ wählt, erhält man auch so einen neuen Beweis, dass
 \[
  \varphi^{-1}(K)
  = \varphi^{-1}(\{1\})
  = \{x \in G \mid \varphi(x) \in \{1\}\}
  = \{x \in G \mid \varphi(x) = 1\}
  = \ker(\varphi)
 \]
 eine Untergruppe von $G$ ist.
\end{bem}





\section{Schnitte und Vereinigungen von Untergruppen}
Es sei $G$ eine Gruppe.


\subsection{Schnitte von Untergruppen}
Es sei $I$ eine Indexmenge und für jedes $i \in I$ sei $H_i \subseteq G$ eine Untergruppe. Wir zeigen, dass dann auch der Schnitt $\bigcap_{i \in I} H_i \subseteq G$ eine Untergruppe ist:

Da $H_i$ für jedes $i \in I$ eine Untergruppe ist, ist $1 \in H_i$ für jedes $i \in I$, und somit auch $1 \in \bigcap_{i \in I} H_i$.

Sind $x,y \in \bigcap_{i \in I} H_i$, so ist $x,y \in H_i$ für jedes $i \in I$. Damit ist auch $x \cdot y \in H_i$ für alle $i \in I$, da $H_i$ für jedes $i \in I$ eine Untergruppe ist. Also ist auch $x \cdot y \in \bigcap_{i \in I} H_i$.

Ist $x \in \bigcap_{i \in I} H_i$, so ist $x \in H_i$ für alle $i \in I$. Da $H_i$ für alle $i \in I$ eine Untergruppe ist, ist damit auch $x^{-1} \in H_i$ für jedes $i \in I$. Also ist auch $x^{-1} \in \bigcap_{i \in I} H_i$.

Insgesamt zeigt dies, dass auch $\bigcap_{i \in I} H_i$ eine Untergruppe von $G$ ist.


\subsection{Vereinigung von Untergruppen}
Vereinigungen von Untergruppen sind nicht notwendigerweise wieder Untergruppen: Wir zeigen, dass für zwei Untergruppen $H_1, H_2 \subseteq G$ die Vereinigung $H_1 \cup H_2$ genau dann eine Untergruppe ist, wenn $H_1 \subseteq H_2$ oder $H_2 \subseteq H_1$:

Ist $H_1 \subseteq H_2$ oder $H_2 \subseteq H_1$, so ist $H_1 \cup H_2 = H_2$ oder $H_1 \cup H_2 = H_1$, also $H_1 \cup H_2$ ebenfalls eine Untergruppe.

Es sei andererseits $H_1 \cup H_2$ wieder eine Untergruppe. Angenommen, es ist weder $H_1 \subseteq H_2$ noch $H_2 \subseteq H_1$. Dann gibt es $h_1 \in H_1$ mit $h_1 \notin H_2$ und $h_2 \in H_2$ mit $h_2 \notin H_1$. Da $h_1, h_2 \in H_1 \cup H_2$, und $H_1 \cup H_2$ nach Annahme eine Untergruppe ist, ist auch $h_1 h_2 \in H_1 \cup H_2$. Also ist $h_1 h_2 \in H_1$ oder $h_1 h_2 \in H_2$.

Ist $h_1 h_2 \in H_1$, so ist auch $h_2 = h_1^{-1} h_1 h_2 \in H_1$, was der Annahme $h_2 \notin H_1$ wiederspricht. Ist andererseits $h_1 h_2 \in H_2$, so ist $h_1 = h_1 h_2 h_2^{-1} \in H_2$, was der Annahme $h_1 \notin H_1 \cup H_2$.

Dieser Widerspruch zeigt, dass bereits $H_1 \subseteq H_2$ oder $H_2 \subseteq H_1$ gelten muss, wenn die Vereinigung $H_1 \cup H_2$ ebenfalls eine Untergruppe ist.

\begin{bsp}
 Durch direktes Nachrechnen ergibt sich, dass sowohl $H_1 \coloneqq \{(x, 0) \mid x \in \Rbb\}$ als auch $H_2 \coloneqq \{(0,y) \mid y \in \Rbb\}$ je eine additive Untergruppe von $\Rbb^2$ ist. (Eine abstrakte Alternative zum Nachrechnen: Die beiden Abbildungen $\iota_1 \colon \Rbb \to \Rbb^2, x \mapsto (x,0)$ und $\iota_2 \colon \Rbb \to \Rbb^2, y \mapsto (0,y)$ sind Gruppenhomomorphismen. Also sind $H_1 = \im(\iota_1)$ und $H_2 = \im(\iota_2)$ Untergruppen.) Da aber weder $H_1 \subseteq H_2$ (da $(1,0) \in H_1$ aber $(1,0) \notin H_2$) noch $H_2 \subseteq H_1$ (da $(0,1) \in H_2$ aber $(0,1) \notin H_1$), ist $H_1 \cup H_2$ keine Untergruppe von $\Rbb^2$.
\end{bsp}


\subsection{Aufsteigende Vereinigungen von Untergruppen}
Für jedes $n \in \Nbb$ sei $H_n \subseteq G$ eine Untergruppe, so dass $H_n \subseteq H_m$ falls $n \leq m$, d.h.\ wir haben eine aufsteigende Kette
\[
 H_0 \subseteq H_1 \subseteq H_2 \subseteq H_3 \subseteq \dotsb \subseteq G
\]
von Untergruppen. Wir zeigen, dass dann die Vereinigung $\bigcup_{n \in \Nbb} H_n$ ebenfalls eine Untergruppe von $G$ ist:

Es ist $1 \in H_0 \subseteq \bigcup_{n \in \Nbb} H_n$, da $H_0$ eine Untergruppe ist. Ist $x \in \bigcup_{n \in \Nbb} H_n$, so gibt es ein $m \in \Nbb$ mit $x \in H_m$. Da $H_m$ eine Untergruppe ist, ist auch $x^{-1} \in H_m \subseteq \bigcup_{n \in \Nbb} H_n$.

Für $x,y \in \bigcup_{n \in \Nbb} H_n$ gibt es $m_1, m_2 \in \Nbb$ mit $x \in H_{m_1}$ und $y \in H_{m_2}$. Da $m_1 \leq m_2$ oder $m_2 \leq m_1$ ist $H_{m_1} \subseteq H_{m_2}$ oder $H_{m_2} \subseteq H_{m_1}$. Wir betrachten o.B.d.A. den Fall, dass $m_1 \leq m_2$, also $H_{m_1} \subseteq H_{m_2}$. Da $x \in H_{m_1} \subseteq H_{m_2}$ und $y \in H_{m_2}$, und $H_{m_2}$ eine Untergruppe ist, ist auch $x \cdot y \in H_{m_2} \subseteq \bigcup_{n \in \Nbb} H_n$.

Insgesamt zeigt dies, dass $\bigcup_{n \in \Nbb} H_n$ eine Untergruppe von $G$ ist.

\begin{bem}
 \begin{enumerate}[leftmargin=*]
  \item
   Allgmeiner gilt: Ist $(I, \subseteq)$ eine total, bzw.\ linear geordnete Menge, und $H_i \subseteq G$ für jedes $i \in I$ eine Untergruppe, so dass $H_i \subseteq H_j$ für alle $i,j \in I$ mit $i \leq j$, so ist die aufsteigende Vereinigung $\bigcup_{i \in I} H_i \subseteq G$ eine Untergruppe. Der Beweis verläuft analog obigen Sonderfall $(\Nbb, \leq)$.
  \item
   Noch allgemeiner: Es sei $(I, \subseteq)$ eine \emph{gerichtete Menge}, d.h.\ eine partiell geordnete Menge, so dass es für alle $i,j \in I$ ein $k \in I$ mit $i \leq k$ und $j \leq k$ gibt. Für jedes $i \in I$ sei $H_i \subseteq G$ eine Untergruppe, so dass $H_i \subseteq H_j$ für alle $i,j \in I$ mit $i \leq j$. Dann ist auch die Vereinigung $\bigcup_{i \in I} H_i$ wieder eine Untergruppe. Der Beweis hierfür verläuft ähnlich zu dem Beweis für total geordnete Mengen.
 \end{enumerate}
\end{bem}





\section{Untergruppen von \texorpdfstring{$\Cbb^\times$}{Cx}}



\subsection{Der Einheitskreis \texorpdfstring{$S^1$}{S1}}
Der Betrag einer komplexen Zahl $z = x+iy \in \Cbb$ ist definiert als
\[
 |z| \coloneqq \sqrt{x^2+y^2} = \sqrt{z \overline{z}}.
\]
Für alle $z_1, z_2 \in \Cbb$ ist
\[
 |z_1 z_2|
 = \sqrt{z_1 z_2 \overline{z_1 z_2}}
 = \sqrt{z_1 \overline{z_1} z_2 \overline{z_2}}
 = \sqrt{z_1 \overline{z_1}} \sqrt{z_2 \overline{z_2}}
 = |z_1| |z_2|
\]
und es ist $|1| = \sqrt{1 \cdot \overline{1}} = \sqrt{1 \cdot 1} = \sqrt{1} = 1$.

\begin{defi}
 Der \emph{Einheitskreis} ist definiert als
\[
 S^1 \coloneqq \{z \in \Cbb \mid |z| = 1\}.
\]
\end{defi}

%TODO: Bild von S1 hinzufügen.

Wir zeigen, dass $S^1$ eine Untergruppe von $\Cbb^\times$ ist: Ist $z \in \Cbb$, so ist $0 \neq 1 = |z| = \sqrt{z \overline{z}}$, also $z \neq 0$ und somit $z \in \Cbb^\times$. Also ist $S^1 \subseteq \Cbb^\times$ eine Teilmenge.

Da $|1| = 1$ ist $1 \in S^1$. Für $z_1, z_2 \in S^1$ ist $|z_1| = |z_2| = 1$, also auch
\[
 |z_1 z_2| = |z_1| |z_2| = 1 \cdot 1 = 1,
\]
und somit $z_1 z_2 \in S^1$. Für $z \in S^1$ ist $|z|= 1$. Daher ist auch
\[
 1 = |1| = |z z^{-1}| = |z| |z^{-1}| = 1 \cdot |z^{-1}| = |z^{-1}|
\]
und deshalb $z^{-1} \in S^1$.

Das zeigt, dass $S^1$ ein Untergruppe von $\Cbb^\times$ ist.


\begin{bem}
 Die Aussage lässt sich auch abstrakter zeigen: Da $|z_1 z_2| = |z_1| \cdot |z_2|$ für alle $z_1, z_2 \in \Cbb$, und $|z| > 0$ für alle $z \neq 0$, ist die Abbildung $b \colon \Cbb^\times \to \Rbb_+$, $z \mapsto |z|$ ein Gruppenhomomorphismus. Daher ist
 \[
  S^1
  = \{z \in \Cbb \mid |z| = 1\}
  = \{z \in \Cbb^\times \mid |z| = 1\}
  = \{z \in \Cbb^\times \mid b(z) = 1\}
  = \ker(b)
 \]
 eine Untergruppe.
\end{bem}



\subsection{\texorpdfstring{$n$}{n}-te Einheitswurzeln}
Es sei $K$ ein Körper und $n \in \Nbb$, $n \geq 1$. Ein Element $x \in K$ heißt \emph{$n$-te Einheitswurzel} falls $x^n = 1$. Wir zeigen, dass
\[
 W_n(K) \coloneqq \{x \in K \mid x^n = 1\}
\]
eine Untergruppe von $K^\times$ ist: Da $1^n = 1$ ist $1 \in W_n(K)$. Sind $x,y \in W_n(K)$ so ist $x^n = 1$ und $y^n = 1$, also auch $(xy)^n = x^n y^n = 1 \cdot 1 = 1$ und deshalb $xy \in W_n(K)$. Für $x \in W_n(K)$ ist $x^n = 1$, also
\[
 (x^{-1})^n = (x^n)^{-1} = 1^{-1} = 1
\]
und deshalb $x \in W_n(K)$.

Ingesamt zeigt dies, dass $W_n(K)$ eine Untergruppe von $\Cbb^\times$ ist.

\begin{bem}
 Da $x^n \neq 0$ für alle $x \neq 0$ und $(xy)^n = x^n y^n$ für alle $x,y \in K$ ist die Abbildung $p_n \colon \Cbb^\times \to \Cbb^\times$, $z \mapsto z^n$ ein Gruppenhomomorphismus. Daher folgt, dass
 \[
  W_n(K)
  = \{x \in K \mid x^n = 1\}
  = \{x \in K^\times \mid x^n = 1\}
  = \{x \in K^\times \mid p_n(x) = 1\}
  = \ker(p_n)
 \]
 eine Untergruppe ist.
\end{bem}


Für $K = \Rbb$ ist beispielsweise
\[
 W_n(\Rbb)
 =
 \begin{cases}
  \{-1,1\} & \text{falls $n$ ungerade ist},\\
     \{1\} & \text{falls $n$ ungerade ist}.
 \end{cases}
\]

Für $K = \Cbb$ und $z \in W_n(\Cbb)$ ist $|z| \in \Rbb$ mit $|z| \geq 0$ und $|z|^n = |z^n| = |1| = 1$. Es muss also $|z| = 1$ und somit $z \in S^1$. Also ist $W_n(\Cbb)$ bereits eine Untergruppe von $S^1$. Es stellt sich heraus, dass $W_n(\Cbb)$ aus $n$ Elementen besteht, die gleichmäßig auf dem Einheitskreis $S^1$ verteilt sind.

%TODO: Bild hinzufügen.





\section{Klassifikation der Untergruppen von \texorpdfstring{$\Zbb$}{Z}}
In diesem Abschnitt bestimmen wir die Untergruppen von $\Zbb$ (d.h.\ von $(\Zbb,+)$, wobei $+$ die gewöhnliche Addition bezeichnet).

Für jedes $n \in \Nbb$ sei
\[
 n \Zbb
 \coloneqq \{n k \mid k \in \Zbb\}
 = \{a \in \Zbb \mid \text{$a$ ist durch $n$ teilbar}.
\]
Wir zeigen zuänchst, dass $n \Zbb$ für jedes $n \in \Nbb$ eine Untergruppe definiert:

Es ist $0 = n \cdot 0 \in \Zbb$. Sind $x,y \in n\Zbb$, so gibt es $k,l \in \Zbb$ mit $x = nk$ und $y = nl$. Da damit $x+y = nk + nl = n(k+l)$ ist auch $x+y \in n\Zbb$. Ist $x \in n\Zbb$, so gibt es $k \in \Zbb$ mit $x = nk$. Da damit $-x = -nk = n(-k)$ ist auch $-x \in \Zbb$. Das zeigt, dass $n\Zbb$ eine Untergruppe ist.

\begin{bem}
 Ein weniger rechenlastiger Beweis besteht darin, dass $\varphi_n \colon \Zbb \to \Zbb$, $k \mapsto nk$ ein Gruppenhomomorphismus ist, da $\varphi_n(k+l) = n(k+l) = nk + nl = \varphi_n(k) + \varphi_n(l)$ für alle $k,l \in \Zbb$, und deshalb
 \[
  \im(\varphi_n) = \{\varphi_n(k) \mid k \in \Zbb\} = \{nk \mid k \in \Zbb\} = n\Zbb
 \]
 eine Untergruppe von $\Zbb$ ist.
\end{bem}

Wir zeigen nun, dass jede Untergruppe $H \subseteq \Zbb$ von der Form $H = n\Zbb$ für ein eindeutiges $n \in \Nbb$ ist: Ist $H = \{0\}$, so gilt die Aussage mit $n = 0$. Wir betrachten daher den Fall, dass $\{0\} \subsetneq H$. Dann gibt es ein $m \in H$ mit $m > 0$, also ist $\{m \in H \mid m > 0\} \neq 0$. Da jede nicht-leere Teilmenge von $\Nbb$ ein minimales Element besitzt, existiert $n \coloneqq \min \{m \in H \mid m > 0\}$.

Da $n \in H$ ist auch $n\Zbb \subseteq H$: Für alle $k \in \Nbb$, $k \geq 1$ ist $n k = \sum_{i=1}^k n \in H$, es ist $n \cdot 0 = 0 \in H$ und für alle $k \in \Nbb$ mit $k \leq -1$ ist wegen $n(-k) \in H$ auch $nk = -n(-k) \in H$.

Wir zeigen nun, dass bereits $H = n\Zbb$. Angenommen, es gibt $m \in H$ mit $m \notin n\Zbb$. Dann gibt es eindeutige $k,l \in \Zbb$ mit $m = nk + l$, wobei $0 \leq l < n$. Da $m \notin n\Zbb$ ist $l \neq 0$, also $0 < l < n$. Da $m \in H$ ist dann aber auch $l = m - nk \in H$. Da $0 < l < n$ steht dies im Widerspruch zur Definition von $n = \min \{m' \in H \mid m' > 0\}$.

Das zeigt, dass $H = n\Zbb$ mit $n \in \Nbb$. Die Eindeutigkeit von $n$ folgt daraus, dass sich $n$ aus $H$ bestimmen lässt durch $n = \min \{m \in H \mid m > 0\}$.

Ingesamt zeigt dies, dass es eine Bijektion
\[
 \Nbb \rightarrow \{H \subseteq \Zbb \mid \text{$H$ ist eine Untergruppe}\}, n \mapsto n\Zbb
\]
gibt.





\section{Die symmetrische Gruppen \texorpdfstring{$S(X)$}{S(X)} und \texorpdfstring{$S_n$}{Sn} und Untergruppen}



\subsection{Die symmetrische Gruppen \texorpdfstring{$S(X)$}{S(X)} und \texorpdfstring{$S_n$}{Sn}}

Es sei $X$ eine beliebige Menge und $S(X) \coloneqq \{\sigma \colon X \to X \mid \text{$\sigma$ ist bijektiv}\}$. Für beliebige $\sigma_1, \sigma_2 \in S(X)$ ist auch $\sigma_1 \circ \sigma_2 \colon X \to X$ bijektiv, also $\sigma_1 \circ \sigma_2 \in S(X)$.

Wir zeigen, dass $(S(X), \circ)$ eine Gruppe ist: Die Assoziativität von $\circ$ folgt direkt aus der allgemeinen Assoziativität der Funktionskomposition.

Für die Identitätsfunktion $\id_X \colon X \to X$, $x \mapsto x$ ist $\id_X \in S(X)$, da $\id_X$ bijektiv ist. Da $\id_X \circ \sigma = \sigma = \sigma \circ \id_X$ ist $\id_X$ ein neutrales Element bezüglich $\circ$.

Ist $\sigma \in S(X)$ so ist $\sigma$ bijektiv und somit invertierbar, d.h.\ es gibt eine (eindeutige) Abbildung $\sigma^{-1} \colon X \to X$ mit $\sigma \circ \sigma^{-1} = \sigma^{-1} \circ \sigma = \id_X$. Also sind $\sigma^{-1}$ und $\sigma$ invers zueinander bezüglich $\circ$.

Insgesamt zeigt dies, dass $(S(X), \circ)$ eine Gruppe ist.

\begin{defi}
 Ist $X$ eine Menge, so heißt die Gruppe $(S(X), \circ)$ die \emph{symmetrische Gruppe von $X$}. Ein Element $\sigma \in S(X)$ heißt \emph{Permutation}. Für $n \in \Nbb$ ist $S_n \coloneqq S(\{1, \dotsc, n\})$.
\end{defi}


\begin{bem}
 Statt $S(X)$ schreibt man auch $\Sigma(X)$ und für $S_n$ schreibt man auch $\Sigma_n$.
\end{bem}


%TODO: S_2 und S_3 explizit bestimmen, allgemein S_n



\subsection{Normalisatoren \texorpdfstring{$N(Y)$}{N(Y)} und Zentralisatoren \texorpdfstring{$Z(Y)$}{Z(Y)}}


\begin{defi}
 Es sei $X$ eine Menge und $Y \subseteq X$ eine Teilmenge. Dann heißt
 \[
  N(Y) \coloneqq \{\sigma \in S(X) \mid \sigma(Y) = Y\}
 \]
 der \emph{Normalisator von $Y$} und
 \[
  Z(Y)
  \coloneqq
  \{\sigma \in S(X) \mid \text{$\sigma(y) = y$ für alle $y \in Y$}\}
 \]
 der \emph{Zentralisator von $Y$}.
\end{defi}

Wir zeigen zunächst, dass $N(Y)$ eine Untergruppe von $S(X)$ ist: Da $\id_X(Y) = Y$ ist $\id_X \in N(Y)$. Sind $\sigma, \tau \in N(Y)$ so ist $\sigma(Y) = Y$ und $\tau(Y) = Y$. Daher ist
\begin{align*}
 (\sigma \circ \tau)(Y)
 &= \{(\sigma \circ \tau)(y) \mid y \in Y\}
 = \{\sigma(\tau(y)) \mid y \in Y\} \\
 &= \sigma(\{\tau(y) \mid y \in Y\})
 = \sigma(\tau(Y))
 = \sigma(Y)
 = Y
\end{align*}
und deshalb auch $\sigma \circ \tau \in N(Y)$. Ist $\sigma \in N(X)$, so ist $\sigma(Y) = Y$ und deshalb
\begin{align*}
 Y
 &= \id_X(Y)
 = (\sigma^{-1} \circ \sigma)(Y)
 = \{(\sigma^{-1} \circ \sigma)(y) \mid y \in Y\}
 = \{\sigma^{-1}(\sigma(y)) \mid y \in Y\} \\
 &= \sigma^{-1}(\{\sigma(y) \mid y \in Y\})
 = \sigma^{-1}(\sigma(Y))
 = \sigma^{-1}(Y),
\end{align*}
also auch $\sigma^{-1} \in N(Y)$.

Insgesamt zeigt dies, dass $N(Y) \subseteq S(X)$ eine Untergruppe ist. Inbesondere ist daher $N(Y)$ zusammen mit der Funktionskomposition $\circ$ wieder eine Gruppe.

Als nächstes zeigen wir, dass auch $Z(Y) \subseteq S(X)$ eine Untergruppe ist: Da $\id_X(x) = x$ für alle $x \in X$, und damit inbesondere $\id_X(y) = y$ für alle $y \in Y$, ist $\id_X \in Z(Y)$. Sind $\sigma, \tau \in Z(Y)$, so ist $\sigma(y) = y$ und $\tau(y) = y$ für alle $y \in Y$ und deshalb auch
\[
 (\sigma \circ \tau)(y)
 = \sigma(\tau(y))
 = \sigma(y)
 = y
 \quad
 \text{für alle $y \in Y$}.
\]
Also ist auch $\sigma \circ \tau \in Z(Y)$. Ist $\sigma \in Z(Y)$, so ist $\sigma(y) = y$ für jedes $y \in Y$ und daher
\[
 \sigma^{-1}(y) = \sigma^{-1}(\sigma(y)) = y
 \quad
 \text{für alle $y \in Y$}.
\]
Also ist auch $\sigma \in Z(Y)$.

Ingesamt zeigt dies, dass $Z(Y)$ eine Untergruppe von $S(X)$ ist. Deshalb ist $Z(Y)$ zusammen mit der Funktionskomposition $\circ$ wieder eine Gruppe.

Für jedes $\sigma \in Z(Y)$ ist auch
\[
 \sigma(Y)
 = \{\sigma(y) \mid y \in Y\}
 = \{y \mid y \in Y\}
 = Y,
\]
also $\sigma \in N(Y)$. Wir haben also eine Kette von Untergruppen $Z(Y) \subseteq N(Y) \subseteq S(X)$.





\section{Die allgemeine lineare Gruppe \texorpdfstring{$\GL_n(k)$}{GLnk} und Untergruppen}
Im Folgenden sei $K$ ein (beliebiger) Körper.

\subsection{Die allgemeine lineare Gruppe \texorpdfstring{$\GL_n(k)$}{GLnk}}

\begin{defi}
 Eine Matrix $S \in \Mat(n \times n, K)$ heißt \emph{invertierbar}, falls es eine Matrix $T \in \Mat(n \times n, K)$ gibt, so dass $ST = I_n = TS$. (Hier bezeichnet $I_n \in \Mat(n \times n, K)$ die Einheitsmatrix.)
 
 Es ist $\GL_n(K) \coloneqq \{S \in \Mat(n \times n, K) \mid \text{$S$ ist invertierbar}\}$.
\end{defi}

Wir zeigen, dass $\GL_n(K)$ zusammen mit der üblichen Matrixmultiplikation eine Gruppe bildet:

Es ist bekannt, dass Matrixmultiplikation assoziativ ist.

Sind $S_1, S_2 \in \GL_n(K)$ invertierbar, so gibt es $T_1, T_2 \in \Mat(n \times n, K)$ mit $S_1 T_1 = I_n$ und $S_2 T_2 = I_n$. Da
\[
 (S_1 S_2) (T_2 T_1)
 = S_1 S_2 T_2 T_1
 = S_1 I_n T_1
 = S_1 T_1
 = I_n
\]
ist auch $S_1 S_2$ invertierbar. Das zeigt, dass die Multiplikation $\GL_n(K) \times \GL_n(K) \to \GL_n(K)$, $(S_1, S_2) \mapsto S_1 S_2$ wohldefiniert ist.

Da $I_n I_n = I_n$ ist $I_n$ invertierbar, also $I_n \in \GL_n(K)$. Da $S I_n = I_n S = S$ für alle $S \in \Mat(n \times n, k)$, und somit insbesondere für alle $S \in \GL_n(K)$, ist $I_n$ neutral bezüglich der Matrixmultiplikation.

Für jedes $S \in \GL_n(K)$ gibt es per Definition von $\GL_n(K)$ eine Matrix $T \in \Mat(n \times n, K)$ mit $ST = TS = I_n$. Aus diesen Gleichungen folgt, dass auch $T$ invertierbar ist, also $T \in \GL_n(K)$, und dass $T$ bezüglich der Matrixmultiplikation invers zu $S$ ist.

Insgesamt zeigt dies, dass $\GL_n(K)$ zusammen mit der üblichen Matrixmultiplikation eine Gruppe bildet. Man bezeichnet diese als die \emph{allgemeine lineare Gruppe} (\textbf{G}eneral \textbf{L}inear group).



\subsection{Die Diagonalmatrizen \texorpdfstring{$\D_n(k)$}{Dn(k)}}

\begin{defi}
 Eine quadratische Matrix $D = (d_{ij})_{i,j = 1, \dotsc, n} \in \Mat(n \times n, K)$ heißt \emph{Diagonalmatrix}, falls $d_{ij} = 0$ für alle $1 \leq i \neq j \leq n$, d.h.\ wenn $D$ von der Form
 \[
  D =
  \begin{pmatrix}
   d_{11} & 0      & \cdots & 0      \\
   0      & \ddots & \ddots & \vdots \\
   \vdots & \ddots & \ddots & 0      \\
   0      & \cdots & 0      & d_{nn}
  \end{pmatrix}
 \]
 ist. Für all $\lambda_1, \dotsc, \lambda_n \in K$ sei abkürzend $\diag(\lambda_1, \dotsc, \lambda_n) \in \Mat(n \times n, K)$ die Diagonalmatrix mit Diagonaleinträgen $\lambda_1, \dotsc, \lambda_n$, d.h.\
 \begin{gather*}
  \diag(\lambda_1, \dotsc, \lambda_n)_{ij} = \delta_{ij} \lambda_i = \delta_{ij} \lambda_j
  \quad
  \text{für alle $1 \leq i,j \leq n$},
 \shortintertext{also}
  \diag(\lambda_1, \dotsc, \lambda_n) =
  \begin{pmatrix}
   \lambda_1 & 0      & \cdots & 0      \\
   0         & \ddots & \ddots & \vdots \\
   \vdots    & \ddots & \ddots & 0      \\
   0         & \cdots & 0      & \lambda_n.
  \end{pmatrix}
 \end{gather*}
 Es sei
\[
 D_n(k) \coloneqq \{D \in \GL_n(k) \mid \text{$D$ ist eine Diagonalmatrix}\}.
\]
\end{defi}

Wir zeigen, dass $\D_n(k)$ zusammen mit der üblichen Matrixmultiplikation eine Gruppe bildet, indem wir zeigen, dass $\D_n(K)$ eine Untergruppe von $\GL_n(k)$ ist.

Wir wollen zunächt verstehe, wie das Produkt von Diagonalmatrizen mit anderen Matrizen aussieht. Hierfür sei $D \in \Mat(n \times n, K)$ eine Diagonalmatrix, $A \in \Mat(m \times n, K)$ sowie $B \in \Mat(n \times m, K)$. Es seien $\lambda_1, \dotsc, \lambda_n \in K$ mit $D = \diag(\lambda_1, \dotsc, \lambda_n)$. Für alle $1 \leq i \leq m$ und $1 \leq j \leq n$ ist
\[
 (AD)_{ij}
 = \sum_{k=1}^n A_{ik} D_{kj}
 = \lambda_j A_{ij},
\]
und für alle $1 \leq i \leq n$ und $1 \leq j \leq m$ ist
\[
 (DB)_{ij}
 = \sum_{k=1}^n D_{ik} B_{ij}
 = \lambda_i B_{ij}.
\]
Also ist
\[
 AD =
 \begin{pmatrix}
  \lambda_1 A_{11} & \cdots & \lambda_n A_{1n} \\
  \lambda_1 A_{21} & \cdots & \lambda_n A_{2n} \\
  \vdots           & \ddots & \vdots           \\
  \lambda_1 A_{m1} & \cdots & \lambda_n A_{mn} 
 \end{pmatrix}
 \quad\text{und}\quad
 DB =
 \begin{pmatrix}
  \lambda_1 B_{11} & \lambda_1 B_{12} & \cdots & \lambda_1 B_{1m} \\
  \vdots           & \vdots           & \ddots & \vdots \\
  \lambda_n B_{n1} & \lambda_n B_{n2} & \cdots & \lambda_n B_{nm}
 \end{pmatrix}
\]
Ist $D' \in \Mat(n \times n, K)$ eine weiter Diagonalmatrix mit Diagonaleinträgen $\mu_1, \dotsc, \mu_n \in K$, also $D' = \diag(\mu_1, \dotsc, \mu_n)$, so ist insbesondere
\begin{align*}
  &\, \diag(\lambda_1, \dotsc, \lambda_n) \cdot \diag(\mu_1, \dotsc, \mu_n)
 = D \cdot D' \\
 =&\,
 \begin{pmatrix}
  \lambda_1 D'_{11} & \cdots & \lambda_1 D'_{1n} \\
  \vdots            & \ddots & \vdots           \\
  \lambda_n D'_{n1} & \cdots & \lambda_n D'_{nn}
 \end{pmatrix}
 =
 \begin{pmatrix}
  \lambda_1 \mu_1 & 0      & \cdots & 0      \\
  0               & \ddots & \ddots & \vdots \\
  \vdots          & \ddots & \ddots & 0      \\
  0               & \cdots & 0      & \lambda_n \mu_n
 \end{pmatrix} \\
 =&\, \diag(\lambda_1 \mu_1, \dotsc, \lambda_n \mu_n).
\end{align*}
Die Multiplikation von Diagonalmatrizen funktioniert als „eintragsweise“.

Dies zeigt zum einen, dass das Produkt von Diagonalmatrizen wieder eine Diagonalmatrix ist. Also ist $\D_n(k)$ unter Produkten abgeschlossen. Außerdem ist $I_n = \diag(1, \dotsc, 1) \in D_n(k)$. Es bleibt zu zeigen, dass für alle $D \in \D_n(k)$ auch $D^{-1} \in \D_n(k)$.

Hierfür betrachten wir zunächst den Fall, dass $\lambda_i \neq 0$ für alle $1 \leq i \leq n$. Dann ist auch $D' \coloneqq \diag(\lambda_1^{-1}, \dotsc, \lambda_n^{-1})$ ein Diagonalmatrix, und es gilt
\[
 D \cdot D'
 = \diag(\lambda_1 \lambda_1^{-1}, \dotsc, \lambda_n \lambda_n^{-1})
 = \diag(1, \dotsc, 1)
 = I_n.
\]
Also ist dann $D^{-1} = D'$ ein Diagonalmatrix.

Tatsächlich muss, da $D$ invertierbar ist, bereits $\lambda_i \neq 0$ für alle $1 \leq i \leq n$: Ansonsten gebe ein $1 \leq i \leq n$ mit $\lambda_i = 0$ und somit $D_{ij} = 0$ für alle $1 \leq j \leq n$ (für alle $1 \leq j \leq n$ mit $j \neq i$ gilt bereits $D_{ij} = 0$, da $D$ ein Diagonalmatrix ist). Für jede Matrix $S \in \GL_n(k)$ wäre damit
\[
 (DS)_{ii}
 = \sum_{j=1}^n \underbrace{D_{ij}}_{=0} S_{ji}
 = 0.
\]
Insbesondere wäre somit $DS \neq I_n$ für alle $S \in \GL_n(K)$, was im Widerspruch zu der Invertierbarkeit von $D$ stünde.

Das zeigt, dass eine Diagonalmatrix $D = \diag(\lambda_1, \dotsc, \lambda_n)$ genau dann invertierbar ist, wenn $\lambda_i \neq 0$ für alle $1 \leq i \leq n$, und dann $D^{-1} = \diag(\lambda_1^{-1}, \dotsc, \lambda_n^{-1})$ ebenfalls eine Diagonalmatrix ist. Also ist $\D_n(k)$ ein Untergruppe von $\GL_n(k)$.





\subsection{Die orthogonale Gruppe \texorpdfstring{$\Ort_n(K)$}{On(K)}}

\begin{defi}
 Für $A \in \Mat(m \times n, K)$ ist das \emph{Transponierte von $A$} definiert als die Matrix $A^T \in \Mat(n \times m, K)$ mit $(A^T)_{ij} = A_{ij}$ für alle $1 \leq i \leq m$ und $1 \leq j \leq n$. Es ist also
 \[
  A^T
  =
  \begin{pmatrix}
   A_{11} & A_{12} & \cdots & A_{1n} \\
   A_{21} & A_{22} & \cdots & A_{2n} \\
   \vdots & \vdots & \ddots & \vdots \\
   A_{m1} & A_{m2} & \vdots & A_{mn}
  \end{pmatrix}^T
  =
  \begin{pmatrix}
   A_{11} & A_{21} & \cdots & A_{m1} \\
   A_{12} & A_{22} & \cdots & A_{m2} \\
   \vdots & \vdots & \ddots & \vdots \\
   A_{1n} & A_{2n} & \cdots & A_{mn}
  \end{pmatrix}.
 \]
 $A^T$ entsteht also aus $A$ durch Spiegelung der Matrix an der Hauptdiagonalen.
\end{defi}

\begin{bem}
 \begin{enumerate}[leftmargin=*]
  \item
   Für $A \in \Mat(l \times m, K)$ und $B \in \Mat(m \times n, K)$ ist $(AB)^T = B^T A^T$, da für alle $1 \leq i \leq n$ und $1 \leq j \leq l$
   \[
    (B^T A^T)_{ij}
    = \sum_{p=1}^m (B^T)_{ip} (A^T)_{pj}
    = \sum_{p=1}^m B_{pi} A_{jp}
    = \sum_{p=1}^m A_{jp} B_{pi}
    = (AB)_{ji}
    = ((AB)^T)_{ij}.
   \]
  \item
   Für $S \in \GL_n(K)$ ist auch $S^T \in \GL_n(K)$, mit $(S^T)^{-1} = (S^{-1})^T$, da
   \[
    S^T (S^{-1})^T = (S^{-1} S)^T = I_n^T = I_n
    \quad\text{und}\quad
    (S^{-1})^T S^T = (S S^{-1})^T = I_n^T = I_n.
   \]
 \end{enumerate}
\end{bem}

Für $n \in \Nbb$, $n \geq 1$ sei
\[
 \Ort_n(k)
 \coloneqq \{S \in \GL_n(k) \mid S^T S = I_n\}
 = \{S \in \GL_n(k) \mid S^{-1} = S^T\}.
\]
Wir zeigen, dass $\Ort_n(k) \subseteq \GL_n(k)$ eine Untergruppe ist:

Da $I_n^T I_n = I_n^2 = I_n$ ist $I_n \in \Ort_n(k)$. Für $S_1, S_2 \in \Ort_n(k)$ ist $S_1^T S_1 = I_n$ und $S_2^T S_2 = I_n$, also auch
\[
 (S_1 S_2)^T (S_1 S_2)
 = S_2^T S_1^T S_1 S_2
 = S_2^T I_n S_2
 = S_2^T S_2
 = I_n
\]
und somit $S_1 S_2 \in \Ort_n(k)$. Ist $S \in \Ort_n(k)$, so ist $S^T S = I_n$. Durch Multiplikation mit $(S^T)^{-1}$ von links und $S^{-1}$ von rechts ergibt sich daraus, dass
\begin{align*}
 I_n
 &= I_n^2
 = (S^T)^{-1} S^T S S^{-1}
 = (S^T)^{-1} (S^T S) S^{-1} \\
 &= (S^T)^{-1} I_n S^{-1}
 = (S^T)^{-1} S^{-1}
 = (S^{-1})^T S^{-1}.
\end{align*}
Somit ist auch $S^{-1} \in \Ort_n(k)$.

Das zeigt, dass $\Ort_n(k) \subseteq \GL_n(k)$ eine Untergruppe ist; daher ist $\Ort_n(k)$ zusammen mit der üblichen Matrixmultiplikation eine Gruppe. Man bezeichnet diese als die \emph{orthogonale Gruppe über $k$}.

\begin{bem}
 \begin{enumerate}[leftmargin=*]
  \item
   Für $S \in \Ort_n(\Rbb)$ ist die Abbildung $D_S \colon \Rbb^n \to \Rbb^n$, $x \mapsto S \cdot x$ linear. Geometrisch gesehen sind die Abbildungen $D_S$ mit $S \in \Ort_n(\Rbb)$ genau die Drehspiegelungen (d.h.\ Drehungen, Spiegelungen und Kombinationen) des $n$-dimensionalen Raumes. Die Gruppe $\Ort_n(\Rbb)$ spielt daher eine wichtige Rolle. Man bezeichnet sie als die \emph{orthogonale Gruppe} und schreibt abkürzend $\Ort(n) \coloneqq \Ort_n(\Rbb)$.
  \item
   Analog zur obigen Rechnung kann man zeigen, dass für eine beliebige quadratische Matrix $B \in \Mat(n \times n, k)$ die Menge
   \[
    \Ort(B) \coloneqq \{S \in \GL_n(k) \mid S^T B S = B\}
   \]
   eine Untergruppe von $\GL_n(k)$ ist. Inbsondere ist $\Ort_n(k) = \Ort(I_n)$.
 \end{enumerate}
\end{bem}



\subsection{Die spezielle orthogonale Gruppe \texorpdfstring{$\SOrt(2)$}{SO2}}
Es sei
\[
 \SOrt(2) \coloneqq
 \left\{
  \begin{pmatrix}
   \cos(\varphi) &           -\sin(\varphi) \\
   \sin(\varphi) & \phantom{-}\cos(\varphi)
  \end{pmatrix}
  \,\middle|\,
  \varphi \in \Rbb
 \right\}.
\]
Wir zeigen, dass $\SOrt(2)$ mit der üblichen Matrixmultiplikation eine abelsche Gruppe bildet; hierfür nutzen wir die Additionstheoreme des Sinus und Kosinus: Für alle $\varphi, \psi \in \Rbb$ gilt
\begin{align*}
 \cos(\varphi+\psi)
 = \cos(\varphi)\cos(\psi) - \sin(\varphi)\sin(\psi)
\shortintertext{und}
 \sin(\varphi+\psi)
 = \sin(\varphi)\cos(\psi)+\sin(\psi)\cos(\varphi).
\end{align*}

Wir führen zunächst Notation ein: Für alle $\varphi \in \Rbb$ sei
\[
 D_\varphi \coloneqq
 \begin{pmatrix}
  \cos(\varphi) &           -\sin(\varphi) \\
  \sin(\varphi) & \phantom{-}\cos(\varphi)
 \end{pmatrix}.
\]
Es ist also $\SOrt(2) = \{D_\varphi \mid \varphi \in \Rbb\}$. Es ist $I_2 = D_0$ und für alle $\varphi, \psi \in \Rbb$ ist
\begin{align*}
 D_\varphi D_\psi
 &=
 \begin{pmatrix}
  \cos(\varphi) &           -\sin(\varphi) \\
  \sin(\varphi) & \phantom{-}\cos(\varphi)
 \end{pmatrix}
 \begin{pmatrix}
  \cos(\psi) & -\sin(\psi) \\
  \sin(\psi) & \phantom{-}\cos(\psi)
 \end{pmatrix} \\
 &=
 \begin{pmatrix}
  \cos(\varphi)\cos(\psi)-\sin(\varphi)\sin(\psi) &
  -\sin(\varphi)\cos(\psi)-\sin(\psi)\cos(\varphi) \\
  \sin(\varphi)\cos(\psi)+\sin(\psi)\cos(\varphi) &
  \cos(\varphi)\cos(\psi)-\sin(\varphi)\sin(\psi)
 \end{pmatrix} \\
 &=
 \begin{pmatrix}
  \cos(\varphi+\psi) &           -\sin(\varphi+\psi) \\
  \sin(\varphi+\psi) & \phantom{-}\cos(\varphi+\psi)
 \end{pmatrix}
 =
 D_{\varphi+\psi}.
\end{align*}
Für alle $\varphi \in \Rbb$ ist daher
\[
 D_{\varphi} D_{-\varphi} = D_0 = I_2
 \quad\text{und}\quad
 D_{-\varphi} D_{\varphi} = D_0 = I_2
\]
und deshalb, also $D_{\varphi}$ invertierbar und $D_{\varphi}^{-1} = D_{-\varphi}$, also
\[
 \begin{pmatrix}
  \cos(\varphi) &           -\sin(\varphi) \\
  \sin(\varphi) & \phantom{-}\cos(\varphi)
 \end{pmatrix}^{-1}
 =
 \begin{pmatrix}
  \cos(-\varphi) &           -\sin(-\varphi) \\
  \sin(-\varphi) & \phantom{-}\cos(-\varphi)
 \end{pmatrix}
 =
 \begin{pmatrix}
  \phantom{-}\cos(\varphi) & \sin(\varphi) \\
            -\sin(\varphi) & \cos(\varphi)
 \end{pmatrix}
\]
(denn $\sin(-\varphi) = -\sin(\varphi)$ und $\cos(-\varphi) = \cos(\varphi)$).


Wir zeigen zunächst, dass $\SOrt(2) \subseteq \GL_2(\Rbb)$ eine Untergruppe ist: Es ist \mbox{$I_2 = D_0 \in \SOrt(2)$}. Für $S, T \in \SOrt(2)$ gibt es $\varphi, \psi \in \Rbb$ mit $S = D_{\varphi}$ und $T = D_{\psi}$. Deshalb ist damit auch $S T = D_{\varphi} D_{\psi} = D_{\varphi+\psi} \in \SOrt(2)$. Für $S \in \SOrt$ gibt es $\varphi \in \Rbb$ mit $S = D_{\varphi}$, weshalb auch $S^{-1} = D_{\varphi}^{-1} = D_{-\varphi} \in \SOrt(2)$.

Das zeigt, dass $\SOrt(2) \subseteq \GL_2(\Rbb)$ eine Untergruppe ist. Deshhalb ist $\SOrt(2)$ zusammen mit der üblichen Matrixmultiplikation eine Gruppe. Es gilt noch zu zeigen, dass $\SOrt(2)$ abelsch ist: Sind $S, T \in \SOrt(2)$ so gibt es $\varphi, \psi \in \Rbb$ mit $S = D_\varphi$ und $T = D_\psi$. Es ist daher
\[
 S T
 = D_{\varphi} D_{\psi}
 = D_{\varphi+\psi}
 = D_{\psi+\varphi}
 = D_{\psi} D_{\varphi}
 = T S.
\]

Insgesamt zeigt dies, dass $\SOrt(2)$ zusammen mit der üblichen Matrixmultiplikation eine abelsche Gruppe bildet. Man bezeichnet diese als die \emph{spezielle orthogonale Gruppe (von Rang $2$)}.

\begin{bem}
 Die obige Rechnung lässt sich auch abkürzen: Da $D_{\varphi} \cdot D_{\psi} = D_{\varphi + \psi}$ für alle $\varphi, \psi \in \Rbb$ und $D_{\varphi}$ für alle $\varphi \in \Rbb$ invertierbar ist, also $D_{\varphi} \in \GL_2(\Rbb)$ für alle $\varphi \in \Rbb$, ist die Abbildung
 \[
  D \colon \Rbb \to \GL_2(\Rbb), \varphi \colon D_{\varphi}
 \]
 ein wohldefinierter Gruppenhomomorphismus. Folglich ist $\SOrt(2) = \im(D)$ eine Untergruppe. Da $\Rbb$ abelsch ist, ist es auch $\im(D)$
\end{bem}


\begin{bem}
 \begin{enumerate}[leftmargin=*]
  \item
   Wie der Name bereits vermuten lässt, ist die spezielle orthogonale Gruppe eine Untergruppe der orthogonalen Gruppe, d.h.\ $\SOrt(2) \subseteq \Ort(2)$. Dies folgt daraus, dass für alle $\varphi \in \Rbb$
   \begin{align*}
    D_\varphi^T D_\varphi
    &=
    \begin{pmatrix}
     \cos(\varphi) &           -\sin(\varphi) \\
     \sin(\varphi) & \phantom{-}\cos(\varphi)
    \end{pmatrix}^T
    \begin{pmatrix}
     \cos(\varphi) &           -\sin(\varphi) \\
     \sin(\varphi) & \phantom{-}\cos(\varphi)
    \end{pmatrix} \\
    &=
    \begin{pmatrix}
     \phantom{-}\cos(\varphi) & \sin(\varphi) \\
               -\sin(\varphi) & \cos(\varphi)
    \end{pmatrix}
    \begin{pmatrix}
     \cos(\varphi) & -\sin(\varphi) \\
     \sin(\varphi) & \phantom{-}\cos(\varphi)
    \end{pmatrix} \\
    &=
    \begin{pmatrix}
     \sin(\varphi)^2 + \cos(\varphi)^2 & 0 \\
     0 & \sin(\varphi)^2 + \cos(\varphi)^2
    \end{pmatrix}
    =
    \begin{pmatrix}
     1 & 0 \\
     0 & 1
    \end{pmatrix}
    = I_2.
   \end{align*}
  \item
   Für jede Matrix $S \in \GL_n(k)$ ist die Abbildung $M_S \colon \Rbb^2 \to \Rbb^2, x \mapsto S \cdot x$ linear. Ist $S = D_\varphi$, so ist $M_S = M_{D_\varphi}$ die Drehung der Ebene $\Rbb^2$ um den Winkel $\varphi$.
 \end{enumerate}
\end{bem}





\section{Die abelschen Gruppen \texorpdfstring{$\Zbb/n\Zbb$}{Z/nZ}}\label{ss: ZnZ}
Wir fixieren ein $n \in \Nbb$ mit $n \geq 1$.



\subsection{Konstruktion}

\begin{defi}
 Eine ganze Zahl $m \in \Zbb$ ist durch $n$ teilbar, falls es ein $s \in \Zbb$ mit $m = sn$ gibt. Man schreibt dann $n \mid m$.
\end{defi}

Wir definieren auf $\Zbb$ eine Äquivalenzrelation durch
\[
 k \sim l
 \iff n \mid (k-l)
 \iff \exists s \in \Zbb : k-l = sn
 \iff \exists s \in \Zbb : k = l + sn
\]
Es muss gezeigt werden, dass dies tatsächlich eine Äquivalenzrelation definiert.

\begin{beh}
 $\sim$ definiert eine Äquivalenzrelation auf $\Zbb$.
\end{beh}
\begin{proof}
 Für jedes $k \in \Zbb$ ist $k = k + 0 \cdot n$, also $k \sim k$. Das zeigt, dass $\sim$ reflexiv ist.
 
 Sind $k,l \in \Zbb$ with $k \sim l$, so gibt es ein $s \in \Zbb$ mit $k = l + sn$. Dann ist $l = k - sn = k + (-s)n$ mit $-s \in \Zbb$, also auch $l \sim k$. Das zeigt, dass $\sim$ symmetrisch ist.
 
 Sind $k_1, k_2, k_3 \in \Zbb$ mit $k_1 \sim k_2$ und $k_2 \sim k_3$, so gibt es $s, t \in \Zbb$ mit $k_1 = k_2 + sn$ und $k_2 = k_3 + tn$. Dann ist
 \[
  k_1
  = k_2 + sn
  = k_3 + tn + sn
  = k_3 + (t+s)n
 \]
 mit $t+s \in \Zbb$, also $k_1 \sim k_3$. Das zeigt, dass $\sim$ transitiv ist.
 
 Ingesamt zeigt dies, dass $\sim$ eine Äquivalenzrelation auf $\Zbb$ ist.
\end{proof}

Für $k \in \Zbb$ bezeichne im Folgenden
\[
 [k] = \{l \in \Zbb \mid k \sim l\}
\]
die Äquivalenzklasse von $\Zbb$. Außerdem sei
\[
 \Zbb/n\Zbb \coloneqq \Zbb/\sim = \{[k] \mid k \in \Zbb\}
\]
die Menge der Äquivalenzklassen. Wir definieren auf $\Zbb/n\Zbb$ eine binäre Verknüpfung durch
\begin{equation}\label{eqn: addition on residue classes}
 [k] + [l] = [k+l]
 \quad
 \text{für alle $k,l \in \Zbb$}.
\end{equation}
Wir müssen zeigen, dass diese Verknüpfung wohldefiniert ist, d.h.\ nicht von den Repräsentanten $k$ und $l$ der Äquivalenzklassen $[k]$ und $[l]$ abhängt.

\begin{beh}
 Die Verknüpfung in \eqref{eqn: addition on residue classes} ist wohldefiniert.
\end{beh}
\begin{proof}
 Es seien $k_1, k_2, l_1, l_2 \in \Zbb$ with $k_1 \sim k_2$ und $l_1 \sim l_2$. Es muss gezeigt werden, dass $[k_1+l_1] = [k_2+l_2]$, also $k_1+l_1 \sim k_2+l_2$
 
 Da $k_1 \sim k_2$ und $l_1 \sim l_2$ gibt es $s,t \in \Zbb$ with $k_1 = k_2 + sn$ und $l_1 = l_2 + tn$. Damit ist
 \[
  k_1 + l_1
  = k_2 + sn + l_2 + tn
  = k_2 + l_2 + (s+t)n,
 \]
 also auch $k_1+l_1 \sim k_2+l_2$.
\end{proof}

Wir zeigen nun, dass $\Zbb/n\Zbb$ zusammen mit der in \eqref{eqn: addition on residue classes} definierten binären Verknüpfung eine abelsche Gruppe bildet:

Für alle $k_1, k_2, k_3 \in \Zbb$ ist
\[
 ([k_1] + [k_2]) + [k_3]
 = [k_1 + k_2] + [k_3]
 = [k_1 + k_2 + k_3]
 = [k_1] + [k_2 + k_3]
 = [k_1] + ([k_2] + [k_3]),
\]
also ist die Verknüpfung assoziativ. Für alle $k,l \in \Zbb$ ist
\[
 [k] + [l] = [k+l] = [l+k] = [l] + [k],
\]
also ist die Verknüpfung komutativ. Für jedes $k \in \Zbb$ ist
\[
 [0] + [k] = [0 + k] = [k],
\]
also ist $[0]$ das neutrale Element bezüglich $+$. Für jedes $k \in \Zbb$ ist
\[
 [k] + [-k] = [k+(-k)] = [k-k] = [0],
\]
also ist $[-k]$ invers zu $[k]$ bezüglich $+$. Ingesamt zeigt dies, dass $\Zbb/n\Zbb$ zusammen mit der Addition $+$ aus \eqref{eqn: addition on residue classes} eine abelsche Gruppe bildet.



\subsection{Erklärung}
Wir wollen noch eine bessere Erklärung dafür geben, wie $\Zbb/n\Zbb$ aussieht und sich die Addition verhält.

Zunächst bemerken wir, dass für jedes $k \in \Zbb$
\begin{align*}
 [k]
 &= \{l \in \Zbb \mid k \sim l\}
 = \{l \in \Zbb \mid \exists s \in \Zbb : l = k + ns\} \\
 &= \{k + ns \mid s \in \Zbb\}
 = k + \{ns \mid s \in \Zbb\}
 = k + n\Zbb.
\end{align*}

\begin{bsp}
 \begin{enumerate}[label=\alph*),leftmargin=*]
  \item
   Es sei $n = 2$. Dann ist $[0] = 2\Zbb = \{\dotso, -4, -2, 0, 2, 4, \dotso\}$ die Menge der ganzen Zahlen und $[1] = \{\dotso, -5, -3, -1, 1, 3, 5, \dotso\}$ die Menge der ungeraden Zahlen. Dies sind die einzigen beiden Äquivalenzklassen (da jede ganze Zahl in genau einer der beiden Äquivalenzklassen enthalten ist). Also ist
   \[
    \Zbb/2\Zbb = \{[0], [1]\}
   \]
   und die Addition ist gegeben durch $[0] + [0] = [0]$, $[0] + [1] = [1] + [0] = [1]$ sowie $[1] + [1] = [2] = [0]$.
   
  \item
   Es sei $n = 3$. Dann ist
   \begin{align*}
    [0] &= 0+3\Zbb = \{\dotso, -6, -3, 0, 3, 6, \dotso\} \\
    [1] &= 1+3\Zbb = \{\dotso, -5, -2, 1, 4, 7, \dotso\} \\
    [2] &= 2+3\Zbb = \{\dotso, -4, -1, 2, 5, 8, \dotso\}.
   \end{align*}
   Da jede ganze Zahl in genau einer dieser Äquivalenzklassen enthalten ist, ist
   \[
    \Zbb/3\Zbb = \{[0], [1], [2]\}.
   \]
   Die Addition ist gegeben durch $[0] + [k] = [k] + [0] = [k]$ für alle $k \in \{0,1,2\}$, sowie $[1]+[1] = [2]$, $[1]+[2] = [2]+[1] = [3] = [0]$ und $[2]+[2] = [4] = [1]$.
 \end{enumerate}
\end{bsp}

Im Allgemeinen gilt, dass
\begin{align*}
 [0] &= 0+n\Zbb = \{\dotso, -2n, -n, 0, n, 2n, \dotso\}, \\
 [1] &= 1+n\Zbb = \{\dotso, -2n+1, -n+1, 1, n+1, 2n+1, \dotso\}, \\
 [2] &= 2+n\Zbb = \{\dotso, -2n+2, -n+2, 2, n+2, 2n+2, \dotso\}, \\
     &\;\;\vdots \\
 [n-1] &= (n-1)+n\Zbb = \{\dotso, -n-1, -1, n-1, 2n-1, 3n-1, \dotso\}.
\end{align*}
Da jede ganze Zahl in genau einer dieser Äquivalenzklassen vorkommt, ist
\[
 \Zbb/n\Zbb = \{[0], [1], \dotsc, [n-1]\}
\]
mit $[k] \neq [l]$ für alle $0 \leq k \neq l \leq n-1$, und die Addition ist gegeben durch
\[
 [k] + [l] = [(k+l) \bmod n]
 \quad
 \text{für alle $0 \leq k,l \leq n-1$}.
\]

Wir identifiziert daher für gewöhnlich die Repräsentanten $0, \dotsc, n-1$ mit den entsprechenden Äquivalenzklassen $[0], \dotsc, [n-1]$. Es ist also $\Zbb/n\Zbb = \{0, \dotsc, n-1\}$, und bezeichnet $+$ die Addition auf $\Zbb$ und $\tilde{+}$ die Addition auf $\Zbb/n\Zbb$, so ist $k \mathbin{\tilde{+}} l = (k+l) \bmod n$.





\section{Produkte von Gruppen}\label{sec: product of groups}

Es sei $I$ ein Indexmenge und für jedes $i \in I$ sei $G_i$ eine Gruppe. Wir definieren auf dem Produkt $\prod_{i \in I} G_i$ eine Multiplikation durch
\[
 (g_i)_{i \in I} \cdot (h_i)_{i \in I}
 = (g_i \cdot h_i)_{i \in I}
 \quad
 \text{für alle $(g_i)_{i \in I}, (h_i)_{i \in I} \in \prod_{i \in I} G_i$}.
\]
Wir zeigen, dass $\prod_{i \in I} G_i$ zusammen mit dieser Multiplikation $\cdot$ eine Gruppe ist:

Für alle $(g_i)_{i \in I}, (h_i)_{i \in I}, (k_i)_{i \in I} \in \prod_{i \in I} G_i$ ist
\begin{align*}
 ( (g_i)_{i \in I} \cdot (h_i)_{i \in I} ) \cdot (k_i)_{i \in I}
 &= (g_i h_i)_{i \in I} \cdot (k_i)_{i \in I}
 = (g_i h_i k_i)_{i \in I} \\r
 &= (g_i)_{i \in I} \cdot (h_i k_i)_{i \in I}
 = (g_i)_{i \in I} \cdot ( (h_i)_{i \in I} \cdot (k_i)_{i \in I} ),
\end{align*}
also ist $\cdot$ assoziativ.

Für alle $(g_i)_{i \in I} \in \prod_{i \in I}$ ist
\begin{align*}
 (1)_{i \in I} \cdot (g_i)_{i \in I}
 &= (1 \cdot g_i)_{i \in I}
 = (g_i)_{i \in I}
 \text{ und} \\
 (g_i)_{i \in I} \cdot (1)_{i \in I}
 &= (g_i \cdot 1)_{i \in I}
 = (g_i)_{i \in I},
\end{align*}
also ist $(1)_{i \in I}$ neutral bezüglich $\cdot$.

Für jedes $(g_1, \dotsc, g_n) \in G$ ist
\begin{align*}
 (g_i)_{i \in I} \cdot (g_i^{-1})_{i \in I}
 &= (g_i g_i^{-1})_{i \in I}
 = (1)_{i \in I} \text{ und} \\
 (g_i^{-1})_{i \in I} \cdot (g_i)_{i \in I}
 &= (g_i^{-1} \cdot g_i)_{i \in I}
 = (1)_{i \in I}
\end{align*}
Also ist $(g_i^{-1})_{i \in I}$ invers zu $(g_i)$ bezüglich $\cdot$.

Insgesamt zeigt dies, dass $\prod_{i \in I} G_i$ bezüglich $\cdot$ eine Gruppe ist.

Insbesondere ist für beliebige Gruppen $G_1, \dotsc, G_n$ auch $G_1 \times \dotsb \times G_n$ mit der eintragsweisen Multiplikation
\[
 (g_1, \dotsc, g_n) \cdot (h_1, \dotsc, h_n) = (g_1 h_1, \dotsc, g_n h_n)
\]
eine Gruppe. Für eine beliebige Gruppe $G$ ist daher auch $G^n$ mit der eintragsweisen Multiplikation eine Gruppe.


\begin{bem}
 \begin{enumerate}[leftmargin=*]
  \item
   Das Produkt $\prod_{i \in I} G_i$ ist genau dann abelsch, wenn für alle $i \in I$ die Gruppe $G_i$ abelsch ist. Inbesondere ist auch das Produkt $G_1 \times \dotsb \times G_n$ genau dann abelsch, wenn $G_1, \dotsc, G_n$ alle abelsch sind, und das Produkt $G^n$ ist genau dann abelsch, wenn $G$ abelsch ist.
  \item
   Die hier angegebene Gruppenstruktur auf $\prod_{i \in I} G_i$ ist die einzige Gruppenstruktur auf $\prod_{i \in I} G_i$, so dass für jedes $j \in I$ die Projektionsabbildung
   \[
    \pi_j \colon \prod_{i \in I} G_i \to G_j, (g_i)_{i \in I} \mapsto g_j
   \]
   ein Gruppenhomomorphismus ist.
  \item
   Das Produkt $\prod_{i \in I} G_i$ hat die folgende Eigenschaft: Ist $G$ eine beliebige Gruppe und für jedes $i \in I$ ein Gruppenhomomorphismus $\varphi_i \colon G \to G_i$ gegeben, so gibt es einen eindeutigen Gruppenhomomorphismus $\varphi \colon G \to \prod_{i \in I} G_i$ mit $\pi_i \circ \varphi = \varphi_i$ für jedes $i \in I$. $\varphi$ ist dann durch
   \[
    \varphi(g) = (\varphi_i(g))_{i \in I}
    \quad
    \text{für alle $g \in G$}
   \]
   gegeben.
 \end{enumerate}
\end{bem}





\section{Die Potenzmenge als abelsche Gruppe}
Es sei $X$ eine beliebige Menge und $\pwr(X) = \{S \mid S \subseteq X\}$ die Potenzmenge. Für je zwei Teilmengen $A, B \subseteq X$ ist auch $A \symm B = (A \cup B) \setminus (A \cap B) \subseteq X$. Also ist
\[
 \symm \colon \pwr(X) \times \pwr(X) \to \pwr(X),
 (A,B) \mapsto A \symm B
\]
eine binäre Verknüpfung. Wir zeigen, dass $\pwr(X)$ zusammen mit $\symm$ ein abelsche Gruppe bildet:

Die Kommutativität ergibt sich daraus, dass für alle $A, B \subseteq X$
\[
 A \symm B
 = (A \cup B) \setminus (A \cap B)
 = (B \cup A) \setminus (B \cap A)
 = B \symm A.
\]
Für jede Teilmenge $A \subseteq X$ ist
\[
 A \symm \emptyset
 = (A \cup \emptyset) \setminus (A \cap \emptyset)
 = A \setminus \emptyset
 = A,
\]
also ist $\emptyset$ neutral bezüglich $\symm$. Für jede Teilmeng $A \subseteq X$ ist
\[
 A \symm A
 = (A \cup A) \setminus (A \cap A)
 = A \setminus A
 = \emptyset,
\]
also ist $A$ selbstinvers bezöglich $\symm$.

Es muss nur noch die Assoziativität gezeigt werden: Hierfür bemerken wir zunächst, dass für alle Teilmengen $A, B \subseteq X$
\begin{align*}
 A \symm B
 &= (A \cup B) \setminus (A \cap B)
 = (A \cup B) \cap (A \cap B)^C
 = (A \cup B) \cap (A^C \cup B^C) \\
 &= (A \cap A^C) \cup (A \cap B^C) \cup (B \cap A^C) \cup (B \cap B^C)
 = (A \cap B^C) \cup (A^C \cap B).
\end{align*}
Für alle Teilmengen $A, B, C \subseteq X$ ist daher
\begin{align*}
  &\, A \symm (B \symm C) \\
 =&\, (A \cap (B \symm C)^C) \cup (A^C \cap (B \symm C)) \\
 =&\, \left( A \cap ((B \cap C^C) \cup (B^C \cup C))^C \right)
      \cup \left( A^C \cap ((B \cap C^C) \cup (B^C \cap C)) \right) \\
 =&\, \left( A \cap ((B \cap C^C)^C \cap (B^C \cup C)^C) \right)
      \cup \left( A^C \cap ((B \cap C^C) \cup (B^C \cap C)) \right) \\
 =&\, \left( A \cap (B^C \cup C) \cap (B \cap C^C) \right)
      \cup \left( A^C \cap ((B \cap C^C) \cup (B^C \cap C)) \right) \\
 =&\, (A \cap B^C \cap B) \cup (A \cap B^C \cap C^C) \cup (A \cap C \cap B) \cup (A \cap C \cap C^C) \\
  &\, \cup (A^C \cap B \cap C^C) \cup (A^C \cap B^C \cap C) \\
 =&\, (A \cap B \cap C) \cup (A \cap B^C \cap C^C) \cup (B \cap A^C \cap C^C) \cup (C \cap A^C \cap B^C).
\end{align*}
Da der rechte Ausdruck invariant unter Permutation von $A$, $B$ und $C$ ist, ist
\begin{align*}
  &\, (A \symm B) \symm C
 =    C \symm (A \symm B) \\
 =&\, (C \cap A \cap B) \cup (C \cap A^C \cap B^C) \cup (A \cap C^C \cap B^C) \cup (B \cap C^C \cap A^C) \\
 =&\, (A \cap B \cap C) \cup (A \cap B^C \cap C^C) \cup (B \cap A^C \cap C^C) \cup (C \cap A^C \cap B^C) \\
 =&\, A \symm (B \symm C).
\end{align*}
Dies zeigt die Assoziativität von $\symm$ auf $\pwr(X)$.

Ingesamt zeigt dies, dass $\pwr(X)$ bezüglich $\symm$ eine abelsche Gruppe bildet.


\begin{bem}
 Dieses Beispiel dient in erster Linie der Belustigung des Lesers und des Schreibers.
\end{bem}






% \section{Weitere kleine Beispiele}
% \begin{enumerate}[leftmargin=*]
%  \item
%   Es sei $k$ ein Körper und
%   \[
%    A \coloneqq
%    \left\{
%     \begin{pmatrix}
%      1 & a \\
%      0 & 1
%     \end{pmatrix}
%     \,\middle|\,
%     a \in k
%    \right\}.
%   \]
%   Wir zeigen, dass $A$ eine abelsche Untergruppe von $\GL_2(k)$ ist: Hierfür schreiben wir
%   \[
%    M_a \coloneqq
%    \begin{pmatrix}
%     1 & a \\
%     0 & 1
%    \end{pmatrix}
%    \quad
%    \text{für alle $a \in k$}
%   \]
%   und bemerken, dass $I_2 = M_0$ und für alle $a,b \in k$
%   \[
%    M_a M_b
%    =
%    \begin{pmatrix}
%     1 & a \\
%     0 & 1
%    \end{pmatrix}
%    \begin{pmatrix}
%     1 & b \\
%     0 & 1
%    \end{pmatrix}
%    =
%    \begin{pmatrix}
%     1 & a+b \\
%     0 & 1
%    \end{pmatrix}
%    = M_{a+b}.
%   \]
%   Für alle $a \in k$ ist daher insbesondere $M_a M_{-a} = M_0 = I_2$ und $M_{-a} M_a = M_0 = I_2$ und deshalb $M_a$ invertierbar mit $M_a^{-1} = M_{-a}$, also
%   \[
%    \begin{pmatrix}
%     1 & a \\
%     0 & 1
%    \end{pmatrix}^{-1}
%    =
%    \begin{pmatrix}
%     1 &           -a \\
%     0 & \phantom{-}1
%    \end{pmatrix}.
%   \]
%   
%   Es ist $I_2 = M_0 \in A$. Für alle $a,b \in k$ ist auch $M_a M_b = M_{a+b} \in A$. Für alle $a \in k$ ist $M_a$ in invertierbar mit $M_a^{-1} = M_{-a} \in A$. Das zeigt, dass $A$ eine Untergruppe von $\GL_2(k)$ ist. Da für alle $a,b \in k$
%   \[
%    M_a M_b = M_{a+b} = M_{b+a} = M_b M_a
%   \]
%   ist $A$ abelsch.
% \end{enumerate}





\section{Das semidirekte Produkt \texorpdfstring{$\Zbb \rtimes \Zbb$}{ZxZ}}
Wir definieren auf der Menge $\Zbb \rtimes \Zbb \coloneqq \{(n,m) \mid n,m \in \Zbb\}$ eine Verknüpfung $*$ durch
\[
 (n_1, m_1) * (n_2, m_2)
 = \left( n_1 + (-1)^{m_1} n_2, m_1 + m_2 \right)
 \quad
 \text{für alle $n_1, n_2, m_1, m_2 \in \Zbb$}.
\]
Wir zeigen, dass $\Zbb \rtimes \Zbb$ zusammen mit der Verknüpfung $*$ eine nicht-abelsche Gruppe bildet:

Für alle $(n_1, m_1), (n_2, m_2), (n_3, m_3) \in \Zbb \rtimes \Zbb$ ist
\begin{align*}
 (n_1, m_1) * ( (n_2, m_2) * (n_3, m_3) )
 &= (n_1, m_1) * (n_2 + (-1)^{m_2} n_3, m_2 + m_3) \\
 &= (n_1 + (-1)^{m_1} (n_2 + (-1)^{m_2} n_3), m_1 + m_2 + m_3) \\
 &= (n_1 + (-1)^{m_1} n_2 + (-1)^{m_1 + m_2} n_3, m_1 + m_2 + m_3) \\
 &= (n_1 + (-1)^{m_1} n_2, m_1 + m_2) * (n_3, m_3) \\
 &=( (n_1, m_1) * (n_2, m_2) ) * (n_3, m_3),
\end{align*}
was die Assoziativität zeigt. Für alle $(n,m) \in \Zbb \rtimes \Zbb$ ist
\begin{align*}
 (0,0) * (n,m) &= (0 + (-1)^0 n, 0 + m) = (n,m) \text{ und}\\
 (n,m) * (0,0) &= (n + (-1)^m \cdot 0, m+0) = (n,m),
\end{align*}
also ist $(0,0)$ neutral bezüglich $*$. Für $(n,m) \in \Zbb \rtimes \Zbb$ ist
\begin{gather*}
 \begin{aligned}
  (n,m) * \left( (-1)^{m+1} n, -m \right)
  &= \left( n + (-1)^m (-1)^{m+1} n, m + (-m) \right) \\
  &= (n-n, m-m)
  = (0,0)
 \end{aligned}
\intertext{und}
 \left( (-1)^{m+1} n, -m \right) * (n,m)
 = \left( (-1)^{m+1} n + (-1)^{-m} n, -m + m \right)
 = (0,0),
\end{gather*}
also $((-1)^{m+1} n, -m)$ invers zu $(n,m)$ bezüglich $*$. Das zeigt, dass $(\Zbb \rtimes \Zbb, *)$ eine Gruppe ist.

Diese Gruppe ist nicht abelsch, da etwa
\begin{align*}
 (1,0) * (0,1) &= (1 + (-1)^0 \cdot 0, 0 + 1) = (1, 1) \text{ und} \\
 (0,1) * (1,0) &= (0 + (-1)^1 \cdot 1, 1 + 0 = (-1, 1).
\end{align*}

Man bezeichnet die Gruppe $\Zbb \rtimes \Zbb$ als das \emph{semidirekte Produkt} von $\Zbb$ und $\Zbb$.










\section{Ausblick: Quotienten von abelschen Gruppen}
Als ein weiteres (abstraktes) Beispiel geben wir eine Verallgemeinerung der Konstruktion von $\Zbb/n\Zbb$ an, die für eine beliebige abelsche Gruppe $A$ und Untergruppe $H \subseteq A$ eine \emph{Quotientengruppe} $A/H$ liefert.

Wir definieren auf $A$ eine Äquivalenzrelation durch
\[
 x \sim y \iff x-y \in H.
\]
Es muss zunächst gezeigt werden, dass dies tatsächlich eine Äquivalenzrelation definiert:

Für alle $x \in A$ ist $x-x = 0 \in H$, da $H$ eine Untergruppe ist, also $x \sim x$, was die Reflexivität zeigt.

Für $x,y \in A$ mit $x \sim y$ ist $x-y \in H$. Da $H$ eine Untergruppe ist, ist damit auch $y-x = -(x-y) \in H$, also $y \sim x$. Das zeigt die Symmetrie von $\sim$.

Sind $x,y,z \in A$ mit $x \sim y$ und $y \sim z$, so ist $x-y \in H$ und $y-z \in H$. Da $H$ eine Untergruppe ist, ist damit auch $x-z = (x-y)+(y-z) \in H$, also $x \sim z$. Das zeigt die Transitivität von $\sim$.

Ingesamt zeigt dies, dass $\sim$ eine Äquivalenzrelation auf $A$ definiert. Wir schreiben
\[
 A/H
 \coloneqq \{[x] \mid x \in A\}
\]
für die Menge der Äquivalenzklassen.

Wir definieren auf $A/H$ eine Verknüpfung durch
\[
 [x]+[y] = [x+y].
\]
Wir müssen überprüfen, dass diese Verknüpfung wohldefiniert ist: Es seien $x, x', y, y' \in A$ mit $x \sim x'$ und $y \sim y'$. Dann ist $x-x' \in H$ und $y-y' \in H$. Da $H$ eine Untergruppe ist, ist auch
\[
 (x+y)-(x'+y') = (x-x') + (y-y') \in H,
\]
also $x+y \sim x'+y'$. Das zeigt die Wohldefiniertheit.

Wir zeigen, dass $A/H$ zusammen mit $+$ eine abelsche Gruppe bildet: Die Assoziativität folgt daraus, dass für alle $x,y,z \in A$
\[
 [x] + ([y] + [z])
 = [x] + [y+z]
 = [x+y+z]
 = [x+y] + [z]
 = ([x] + [y]) + [z].
\]
Die Kommutativität ergibt sich daraus, dass für alle $x,y \in A$
\[
 [x] + [y] = [x+y] = [y+x] = [y] + [x].
\]
Für die Äquivalenzklasse von $0$ gilt für alle $x \in A$, dass
\[
 [0] + [x] = [0+x] = [x] = [x+0] = [x] + [0], 
\]
weshalb $[0]$ neutral bezüglich $+$ ist. Für jedes $x \in A$ ist
\[
 [x] + [-x] = [x-x] = [0] = [x-x] = [-x] + [x],
\]
weshalb $[-x]$ invers zu $[x]$ bezüglich $+$ ist. Ingesamt zeigt dies, dass $(A/H, +)$ eine abelsche Gruppe ist.


\begin{bem}
 \begin{enumerate}[leftmargin=*]
  \item
   Für $A = \Zbb$ und $H = n\Zbb$ ergibt sich die bereits bekannte Konstruktion von $\Zbb/n\Zbb$.
  \item
   Die Äquivalenzklasse von $x \in A$ ist gegeben durch
   \begin{align*}
    [x]
    = \{y \in A \mid y-x \in H\}
    &= \{y \in A \mid \text{$y = x + h$ ein $h \in H$}\} \\
    &= \{x+h \mid h \in H\}
    = x + \{h \mid h \in H\}
    = x + H.
   \end{align*}
  \item
   Die Addition lässt sich alternativ auch wie folgt definieren: Für beliebige Teilmengen $R, S \subseteq A$ sei
   \[
    R + S = \{r+s \mid r \in R, s \in S\}.
   \]
   Für alle $x,y \in A$ ist dann
   \begin{align*}
    [x] + [y]
    &= (x+H) + (y+H)
    = \{x+h \mid h \in H\} + \{y+h' \mid h' \in H\} \\
    &= \{x+h+y+h' \mid h, h' \in H\}
    = \{(x+y)+(h+h') \mid h,h' \in H\} \\
    &= \{(x+y)+h'' \mid h'' \in H\}
    = (x+y)+H
    = [x+y].
   \end{align*}
   Diese Definition der Addition hat den Vorteil, dass es keine Wohldefiniertheit zu überprüfen gilt.
 \end{enumerate}
\end{bem}

























