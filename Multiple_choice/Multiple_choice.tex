\documentclass[a4paper,10pt]{exam}
%\documentclass[a4paper,10pt]{scrartcl}

\usepackage{../generalstyle}
\usepackage{specificstyle}

\setromanfont[Mapping=tex-text]{Linux Libertine O}
% \setsansfont[Mapping=tex-text]{DejaVu Sans}
% \setmonofont[Mapping=tex-text]{DejaVu Sans Mono}

\title{Multiple Choice}
\author{Jendrik Stelzner}
\date{\today}

\begin{document}
\begin{questions}



\question Im Folgenden sei $K$ ein Körper, es seien $n,m \in \Nbb$ mit $n,m \geq 1$, $V$ und $W$ seien $K$-Vektorräume und $f, g \colon V \to W$ seien $K$-linear.
\begin{checkboxes}
 \choice
  Sind $T_1, T_2, T_3 \subseteq V$, so dass $T_i \cup T_j$ für alle $1 \leq i \neq j \leq 3$ linear unabhängig ist, so ist auch $T_1 \cup T_2 \cup T_3$ linear unabhängig.
 \choice
  Ist $f$ ein Isomorphismus, so ist auch die duale Abbildung $f^*$ ein Isomorphismus.
 \choice
  Ist $A \in \GL_n(K)$, so sind die Potenzen $(A, A^2, A^3, \dotsc, A^{n^2})$ eine Basis von $\Mat(n \times n, K)$.
 \choice
  Ist $\ker f^* = \ker g^*$, so ist $f = g$.
 \choice
  Sind $U_1, U_2 \subseteq V$ Untervektorräume, so ist auch
  \[
   U_1 + U_2 = \{u_1 + u_2 \mid u_1 \in U_1, u_2 \in U_2\}
  \]
  ein Untervektorraum von $V$.
 \choice
  Es gilt $(f-g)^* = f^* - g^*$.
 \choice
  Ist $A \in \GL_n(K)$ symmetrisch (d.h.\ $A^T = A$), so ist auch $A^{-1}$ symmetrisch.
 \choice
  Sind $A \in \Mat(n \times m, K)$ und $B \in \Mat(m \times n, K)$ mit $AB = \mathbbm{1}_n$, so ist auch $BA = \mathbbm{1}_m$.
 \choice
  Ist $\lambda \in K$ ein Eigenwert von $A \in \Mat(n \times n, K)$, so ist $\lambda^2 + 2$ ein Eigenwert von $A^2 + 2\mathbbm{1}_n$.
 \choice
  Ist $U \subseteq V$ ein Untervektorraum, so ist $f(U)$ ein Untervektorraum von $W$ mit $f^*(f(U)^\perp) \subseteq U^\perp$.
\end{checkboxes}



\question Im Folgenden sei $K$ ein Körper, es sei $n \in \Nbb$ mit $n \geq 1$, $V$ sei ein $K$-Vektorraum und $f \colon V \to V$ sei linear.
\begin{checkboxes}
 \choice
  Ist $A \in \Mat(n \times n, \Rbb)$ und $\lambda > 0$ für jeden Eigenwert $\lambda$ von $A$, so ist $A$ invertierbar.
 \choice
  Ist $T \subseteq V$, so dass jede endliche Teilmenge von $T$ linear unabhängig ist, so ist auch $T$ linear unabhängig.
 \choice
  Es ist $\ker(f)^\perp = \ker(f^*)$.
 \choice
  Sind $A, B \in \Mat(n \times n, K)$, so dass es ein $S \in \GL_n(K)$ mit $B = S A S^{-1}$ gibt, so ist $\ker A = \ker B$.
 \choice
  Ist $K = \Cbb$ und $\dim_\Cbb V < \infty$, so ist $\dim_\Rbb V > \dim_\Cbb V$.
 \choice
  Ist $A \in \Mat(n \times n, \Rbb)$ mit $\det(A) = 1$ und $A_{ij} \in \Zbb$ für alle $1 \leq i,j \leq n$, so ist $A$ invertierbar mit $(A^{-1})_{ij} \in \Zbb$ für alle $1 \leq i,j \leq n$.
 \choice
  Sind $U_1, U_2 \subseteq V$ zwei Untervektorräume mit $U_1 \subseteq U_2$, so ist $U_1^\perp \subseteq U_2^\perp$.
 \choice
  Es seien $A, B \in \Mat(n \times n, K)$. Ist $\lambda \in K$ ein Eigenwert von $A$ und $\mu \in K$ eine Eigenwert von $B$, so ist $\lambda + \mu$ ein Eigenwert von $A + B$.
 \choice
  Für jedes $v \in V \setminus \{0\}$ ist die Menge $\{f \in \Ell(V,V) \mid \text{$v$ ist ein Eigenvektor von $f$}\}$ ein Untervektorraum von $\Ell(V,V)$.
\end{checkboxes}



\question Es sei $K$ ein Körper, $n \in \Nbb$ mit $n \geq 1$, $V$ ein endlichdimensionaler $K$-Vektorraum und $f \colon V \to V$ linear.
\begin{checkboxes}
 \choice
  Es ist $\rang(f) = \rang(f^*)$.
 \choice
  Ist $(U_n)_{n \in \Nbb}$ eine Kollektion von Untervektorräumen $U_n \subseteq V$, so dass $U_n \subseteq U_{n+1}$ für alle $n \in \Nbb$, so ist $\bigcup_{n \in \Nbb} U_n$ ein Untervektorraum von $V$.
 \choice
  Sind $v_1, v_2 \in V$, so dass $(v_1, v_2)$ linear unabhängig ist, so ist auch $(v_1 + v_2, v_1 - v_2)$ linear unabhängig.
 \choice
  Es ist $\chi_f(T) = \chi_{f^*}(T)$.
 \choice
  Die Menge $\{v \in V \mid \text{es gibt $\lambda \in K$ mit $f(v) = \lambda v$}\}$ ist ein Untervektorraum von $V$.
 \choice
  Ist $A \in \GL_n(K)$ mit $A^{-1} = A^T$, so ist $\det(A) = 1$.
 \choice
  Die Determinante $\det \colon \Mat(n \times n, K) \to K$ ist nicht linear.
 \choice
  Es seien $\mc{B} = (b_1, \dotsc, b_n)$ und $\mc{C} = (c_1, \dotsc, c_n)$ zwei Basen von $V$, und $\mc{B}^* = (b_1^*, \dotsc, b_n^*)$ und $\mc{C}^* = (c_1^*, \dotsc, c_n^*)$ die entsprechenden dualen Basen von $V^*$. Ist $\mc{B}^* = \mc{C}^*$, also $b_i^* = c_i^*$ für alle $1 \leq i \leq n$, so ist $\mc{B} = \mc{C}$.
 \choice
  Ist $A \in \GL_n(\Cbb)$ mit $A \in \Mat(n \times n, \Rbb)$, so ist auch $A^{-1} \in \Mat(n \times n, \Rbb)$.
\end{checkboxes}








\end{questions}
\end{document}
