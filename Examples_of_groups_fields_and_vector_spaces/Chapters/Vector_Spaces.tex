\chapter{Beispiele für Vektorräume}
Sofern nicht anders angegeben, bezeichnen $K$ und $L$ im folgenden Körper mit $K \subseteq L$.





\section{Einige kleinere Beispiele}
\begin{enumerate}[leftmargin=*]
 \item
  $K$ ist ein $K$-Vektorraum, indem man die Multiplikation $K \times K \to K$, $(x,y) \mapsto x \cdot y$ als Skalarmultiplikation wählt. Dass $\lambda \cdot (v+w) = \lambda v + \lambda w$ und $(\lambda+\mu) \cdot v = \lambda \cdot v + \mu \cdot v$ für alle $\lambda, \mu \in K$ und $v,w \in V$ folgt direkt aus der Distributivität der Multiplikation. Dass $\lambda \cdot (\mu \cdot v) = (\lambda \cdot \mu) \cdot v$ für alle $\lambda, \mu \in K$ folgt aus der Assoziativität der Multiplikation. Dass $1 \cdot v = v$ für alle $v \in V$ folgt aus der Definition des Einselementes $1$.
 \item
  Wie bereits in \ref{sec: product of groups} gesehen ist $K^n = \{(x_1, \dotsc, x_n) \mid x_1, \dotsc, x_n \in K\}$ mit der eintragsweisen Addition
  \[
   (x_1, \dotsc, x_n) + (y_1, \dotsc, y_n)
   = (x_1 + y_1, \dotsc, x_n + y_n)
  \]
  eine abelsche Gruppe. Mit der eintragsweise Skalarmultiplikation
  \[
   \lambda \cdot (x_1, \dotsc, x_n)
   = (\lambda x_1, \dotsc, \lambda x_n)
   \quad
   \text{für alle $\lambda \in K$ und $(x_1, \dotsc, x_n) \in K^n$}
  \]
  ist $K^n$ zu einem $K$-Vektorraum: Für alle $\lambda \in K$ und $(x_1, \dotsc, x_n), (y_1, \dotsc, y_n) \in K^n$ ist
  \begin{align*}
    &\, \lambda \cdot ((x_1, \dotsc, x_n) + (y_1, \dotsc, y_n))
   =    \lambda \cdot (x_1 + y_1, \dotsc, x_n + y_n) \\
   =&\, (\lambda (x_1 + y_1), \dotsc, \lambda (x_n + y_n))
   =    (\lambda x_1 + \lambda y_m, \dotsc, \lambda x_n + \lambda y_n) \\
   =&\, (\lambda x_1, \dotsc, \lambda x_n) + (\lambda y_1, \dotsc, \lambda y_n)
   =    \lambda \cdot (x_1, \dotsc, x_n) + \lambda \cdot (y_1, \dotsc, y_n).
  \end{align*}
  Für alle $\lambda, \mu \in K$ und $(x_1, \dotsc, x_n) \in K^n$ ist
  \begin{gather*}
   \begin{aligned}
     &\, (\lambda+\mu) \cdot (x_1, \dotsc, x_n)
    =    ((\lambda+\mu)x_1, \dotsc, (\lambda+\mu)x_n) \\
    =&\, (\lambda x_1 + \mu x_1, \dotsc, \lambda x_n + \mu x_n)
    =    (\lambda x_1, \dotsc, \lambda x_n) + (\mu x_1, \dotsc, \mu x_n) \\
    =&\, \lambda \cdot (x_1, \dotsc, x_n) + \mu \cdot (x_1, \dotsc, x_n)
   \end{aligned}
  \shortintertext{und}
   \begin{aligned}
    \lambda \cdot (\mu \cdot (x_1, \dotsc, x_n))
    &= \lambda \cdot (\mu x_1, \dotsc, \mu  x_n) \\
    &= (\lambda \mu x_1, \dotsc, \lambda \mu x_n)
    = (\lambda \mu) \cdot (x_1, \dotsc, x_n).
   \end{aligned}
  \end{gather*}
  Für alle $(x_1, \dotsc, x_n) \in K^n$ gilt schließlich
  \[
   1 \cdot (x_1, \dotsc, x_n)
   = (1 \cdot x_1, \dotsc, 1 \cdot x_n)
   = (x_1, \dotsc, x_n).
  \]
  Ingesamt zeigt dies, dass $K^n$ mit der eintragsweisen Addition und Skalarmultiplikation einen $K$-Vektorraum bildet.
 \item
  Ist $V$ ein $K$-Vektorraum, so ist $\{0\} \subseteq V$ ein Untervektorraum. Man bezeichnet diesen als den \emph{Nulluntervektorraum}.
 \item
  Die $L$-Vektorraum auf $L$ schränkt sich zu einer $K$-Vektorraumstruktur auf $L$ ein. $L$ ist also ein $K$-Vektorraum durch $\lambda \cdot v = \lambda v$ für alle $\lambda \in K$ und $v \in L$, wobei die punktlose Multiplikation auf der rechten Seite die Multiplikation des Körpers $L$ bezeichnet.
 \item
  Ist allgemein $V$ ein $L$-Vektorraum, so wird $V$ durch Einschränkung der Skalarmultiplikation zu einem $K$-Vektorraum. Dass die Axiome eines $K$-Vektorraums erfüllt sind, folgt direkt daraus, dass sie für Skalare aus $L$ erfüllt sind. Man bezeichnet diese Einschränkung von $L$-Vektorräumen zu $K$-Vektorräumen als \emph{Skalarrestriktion}.
 \item
  Ist $V$ ein $\Cbb$-Vektorraum, so ist $V$ auch durch die Skalarmultiplikation
  \[
   \lambda * v = \overline{\lambda} \cdot v
   \quad
   \text{für alle $\lambda \in Cbb$ und $v \in V$}
  \]
  ein $\Cbb$-Vektorraum: Für alle $\lambda \in \Cbb$ und $v,w \in V$ ist
  \[
   \lambda * (v+w)
   = \overline{\lambda} \cdot (v+w)
   = \overline{\lambda} \cdot v + \overline{\lambda} \cdot w
   = \lambda * v + \lambda * w.
  \]
  Für alle $\lambda, \mu \in \Cbb$ und $v \in V$ ist
  \begin{gather*}
   (\lambda + \mu) * v
   = \overline{\lambda + \mu} \cdot v
   = (\overline{\lambda} + \overline{\mu}) \cdot v
   = \overline{\lambda} \cdot v + \overline{\mu} \cdot v
   = \lambda * v + \mu * v.
  \shortintertext{und}
   \lambda * (\mu * v)
   = \overline{\lambda} \cdot (\overline{\mu} \cdot v)
   = (\overline{\lambda} \cdot \overline{\mu}) \cdot v
   = \overline{\lambda \cdot \mu} \cdot v
   = (\lambda \cdot \mu) * v.
  \end{gather*}
  Außerdem ist für alle $v \in V$
  \[
   1 * v = 1 \cdot v = v.
  \]
 \item
  Ist allgemeinen $\phi \colon K \to K'$ ein Körperhomomorphismus und $V$ ein $K'$-Vektorraum, so wird $V$ ein $K$-Vektorraum durch die Skalarmultiplikation
  \[
   \lambda * v = \phi(\lambda) \cdot v
   \quad
   \text{für alle $\lambda \in K$ und $v \in V$}.
  \]
  Für alle $\lambda, \mu \in K$ und $v, w \in V$ ist nämlich
  \begin{gather*}
    \lambda * (v+w)
    = \phi(\lambda) \cdot (v+w)
    = \phi(\lambda) \cdot v + \phi(\lambda) \cdot w
    = \lambda * v + \lambda * w
  \shortintertext{und}
   \begin{aligned}
    (\lambda + \mu) * v
    = \phi(\lambda + \mu) \cdot v
    &= (\phi(\lambda) + \phi(\mu)) \cdot v \\
    &= \phi(\lambda) \cdot v + \phi(\mu) \cdot v
    = \lambda * v + \mu * v,
   \end{aligned}
  \shortintertext{sowie}
   \begin{aligned}
    \lambda * (\mu * v)
    = \phi(\lambda) \cdot (\phi(\mu) \cdot v)
    &= (\phi(\lambda) \cdot \phi(\mu)) \cdot v \\
    &= \phi(\lambda \cdot \mu) \cdot v
    = (\lambda \cdot \mu) * v,
   \end{aligned}
  \end{gather*}
  und für alle $v \in V$ gilt
  \[
   1 * v = \phi(1) \cdot v = 1 \cdot v = v.
  \]
  
  Die vorherigen beiden Beispiele ergeben sich als Sonderfall hievon: Die Skalarrestriktion ergibt sich durch die Inklusion $K \hookrightarrow L$ und das vorherige Beispiel durch die Konjugation $\Cbb \to \Cbb, z \mapsto \overline{z}$.
\end{enumerate}


\section{Untervektorräume bezüglich linearer Abbildungen}
Es seien $V$ und $W$ $K$-Vektorräume und $f \colon V \to W$ sei eine $K$-lineare Abbildung.



\subsection{Kern und Bild einer linearen Abbildung}
Wir zeigen, dass der \emph{Kern} von $f$
\[
 \ker(f) \coloneqq \{v \in V \mid f(v) = 0\}
\]
ein Untervektorraum von $V$ ist: Da $f$ insbesondere ein Gruppenhomomorphismus der unterliegenden abelschen Gruppen ist, haben wir in \ref{sec: subgroups and group homomorphisms} bereits gesehen, dass $\ker(f)$ eine additive Untergruppe ist. Für alle $\lambda \in K$ und $v \in \ker(f)$ ist auch
\[
 f(\lambda v) = \lambda f(v) = \lambda \cdot 0 = 0,
\]
also $\lambda v \in \ker(f)$. Das zeigt, dass $\ker(f)$ ein Untervektorraum ist.

\begin{bem}
 Wir haben in \ref{sec: subgroups and group homomorphisms} auch schon gesehen, dass $f$ genau dann injektiv ist, wenn $\ker(f) = \{0\}$.
\end{bem}

Das \emph{Bild} von $f$
\[
 \im(f) = \{f(v) \mid v \in V\}
\]
ist ein Untervektorraum von $W$: Da $f$ insbesondere ein Gruppenhomomorphismus der unterliegenden abelschen Gruppen ist, haben wir in \ref{sec: subgroups and group homomorphisms} bereits gesehen, dass $\im(f)$ eine additive Untergruppe ist. Für $w \in \im(f)$ gibt es ein $v \in V$ mit $w = f(v)$, weshalb für alle $\lambda \in K$ auch
\[
 \lambda w
 = \lambda f(v)
 = f(\lambda w)
 \in \im(f).
\]
Das zeigt, dass $\im(f)$ ein Untervektorraum ist.



\subsection{Bilder und Urbilder von Untervektorräumen}
Ist $U \subseteq V$ ein Untervektorraum, so ist das Bild $f(U)$ ein Untervektorraum von $W$: Da $f$ inbesondere ein Gruppenhomomorphismen der unterliegenden abelschen Gruppen ist haben wir bereits in \ref{sec: subgroups and group homomorphisms} gesehen, dass $f(U)$ ein additive Untergruppe ist. Ist $w \in f(U)$, so gibt es $u \in U$ mit $w = f(u)$. Da $U$ ein Untervektorraum ist, ist für alle $\lambda \in K$ auch $\lambda u \in U$, weshalb für alle $\lambda \in K$ auch
\[
 \lambda w
 = \lambda f(u)
 = f(\lambda u)
 \in f(U).
\]
Also ist $f(U)$ ein Untervektorraum von $f$.


\begin{bem}
 Wie bereits in \ref{sec: subgroups and group homomorphisms} lassen sich Rechnungen auch hier umgehen: Da $f$ linear ist, ist es auch die Einschränkung $f|_{U} \colon U \to W$, $u \mapsto f(u)$. Wie bereits gesehen, ist deshalb $f(U) = \im(f|_{U})$ ein Untervektorraum.
\end{bem}


Ist $U \subseteq W$ ein Untervektorraum, so ist das Urbild $f^{-1}(U)$ ein Untervektorraum von $V$: Da $f$ inbesondere ein Gruppenhomomorphismus zwischen den unterliegenden abelschen Gruppen ist, haben wir bereits in \ref{sec: subgroups and group homomorphisms} gesehen, dass $f^{-1}(U)$ eine additive Untergruppe der unterliegenden abelschen Gruppe von $V$ ist. Für $v \in f^{-1}(U)$ ist $f(v) \in U$, also für alle $\lambda \in K$ auch
\[
 f(\lambda v) = \lambda f(v) \in U,
\]
da $U$ ein Untervektorraum ist. Also ist $f^{-1}(U)$ ein Untervektorraum.


\begin{bem}
 Ähnlich wie in \ref{sec: subgroups and group homomorphisms} folgt auch hieraus, dass
 \[
  \ker(f) = f^{-1}(\{0\})
 \]
 ein Untervektorraum ist.
\end{bem}


\section{Abbildungen in einen Vektorraum}
Es sei $X$ eine beliebige Menge und $V$ ein $K$-Vektorraum. Auf der Menge
\[
 \Abb(X,V) = \{f \colon X \to V\}
\]
Wir definieren für $f,g \in \Abb(X,V)$ und $\lambda \in K$ eine die punktweise Addition und Skalarmultiplikation durch
\begin{gather*}
 (f+g)(x) \coloneqq f(x) + g(x)
 \quad
 \text{für alle $x \in X$}
\shortintertext{und}
 (\lambda \cdot f)(x) \coloneqq \lambda \cdot f(x)
 \quad
 \text{für alle $x \in X$}.
\end{gather*}
Wir zeigen, dass $\Abb(X,V)$ zusammen mit dieser Addition und Skalarmultiplikation einen $K$-Vektorraum bildet:

Für alle $f,g,h \in \Abb(X,V)$ ist
\begin{align*}
 (f+(g+h))(x)
 &= f(x) + (g+h)(x)
 = f(x) + g(v) + h(x) \\
 &= (f+g)(x) + h(x)
 = ((f+g)+h)(x)
\end{align*}
für alle $x \in X$ und somit $f+(g+h) = (f+g)+h$. Also ist die Addition assoziativ. Für alle $f,g \in \Abb(X,V)$ ist
\[
 (f+g)(x) = f(x) + g(x) = g(x) + h(x) = (g+h)(x)
 \quad
 \text{für alle $x \in X$},
\]
also ist $f+g = g+f$, die Addition also kommutativ. Die \emph{Nullfunktion} ist definiert als
\[
 N \colon X \to V, x \mapsto 0.
\]
Für alle $f \in \Abb(X,V)$ ist
\[
 (f+N)(x) = f(x) + N(x) = f(x) + 0 = f(x)
 \quad
 \text{für alle $x \in X$},
\]
also ist $f+N = f$. Also ist $N$ neutral bezüglich der Addition. Für $f \in \Abb(X,V)$ sei
\[
 \tilde{f} \colon X \to V
 \quad\text{und}\quad
 \tilde{f}(x) \coloneqq -f(x)
 \quad
 \text{für alle $x \in X$}.
\]
Es ist dann
\[
 (f+\tilde{f})(x)
 = f(x) + \tilde{f}(x)
 = f(x) - f(x)
 = 0
 = N(x)
 \quad
 \text{für alle $x \in X$},
\]
also $f+\tilde{f} = N$. Somit ist $\tilde{f}$ additiv invers zu $f$. Ingesamt zeigt dies, dass $+$ tatsächlich eine Addition auf $\Abb(X,V)$ definiert.

Für alle $\lambda, \mu \in K$ und $f,g \in \Abb(X,V)$ ist
\begin{align*}
 (\lambda \cdot (f+g))(x)
 &= \lambda \cdot (f+g)(x)
 = \lambda \cdot (f(x)+g(x)) \\
 &= \lambda \cdot f(x) + \lambda \cdot g(x)
 = (\lambda \cdot f)(x) + (\lambda \cdot g)(x)
 = (\lambda \cdot f + \lambda \cdot g)(x)
\end{align*}
für alle $x \in X$ und somit $\lambda \cdot (f+g) = \lambda \cdot f + \lambda \cdot g$; es ist
\begin{align*}
 ((\lambda + \mu) \cdot f)(x)
 = (\lambda+\mu) \cdot f(x)
 &= \lambda \cdot f(x) + \mu \cdot f(x) \\
 &= (\lambda \cdot f)(x) + (\mu \cdot f)(x)
 = (\lambda \cdot f + \mu \cdot f)(x)
\end{align*}
für alle $x \in X$ und somit $(\lambda + \mu) \cdot f = \lambda \cdot f + \mu \cdot f$; es ist
\[
 (\lambda \cdot (\mu \cdot f))(x)
 = \lambda \cdot (\mu \cdot f)(x)
 = \lambda \cdot (\mu \cdot f(x))
 = (\lambda \cdot \mu) \cdot f(x)
 = ((\lambda \cdot \mu) \cdot f)(x)
\]
für alle $x \in X$ und somit $\lambda \cdot (\mu \cdot f) = (\lambda \cdot \mu) \cdot f$. Außerdem ist für jedes $f \in \Abb(X,V)$
\[
 (1 \cdot f)(x)
 = 1 \cdot f(x)
 = f(x)
 \quad
 \text{für alle $x \in X$},
\]
und somit $1 \cdot f = f$.

Ingesamt zeigt dies, dass $\Abb(X,V)$ zusammen mit der punktweisen Addition und Skalarmultiplikation einen $K$-Vektorraum bildet.





\section{\texorpdfstring{$\Hom$}{Hom}-Räume}
Es seien $V$ und $W$ zwei $K$-Vektorräume und
\[
 \Hom(V, W) \coloneqq \{f \colon V \to W \mid \text{$f$ ist linear}\}.
\]
Wir zeigen, dass $\Hom(V,W) \subseteq \Abb(V,W)$ einen Untervektorraum bildet, und somit ebenfalls einen Vektorraum bezüglich der punktweisen Addition und Skalarmultiplikation.

Die Nullfunktion $0 \in \Abb(V,W)$ ist linear, da $0(v+w) = 0 = 0(v)+0(w)$ sowie $0(\lambda v) = 0 = \lambda 0(v)$ für alle $v,w \in V$ und $\lambda \in K$. Also ist $0 \in \Hom(V,W)$.

Sind $f,g \in \Hom(V,W)$ und $\lambda \in K$, so ist für alle $v,w \in V$ und $\mu \in K$
\begin{gather*}
 \begin{aligned}
 (f+g)(v+w)
 &= f(v+w) + g(v+w)
 = f(v) + f(w) + g(v) + g(w) \\
 &= f(v)+g(v) + f(w)+g(w)
 = (f+g)(v) + (f+g)(w)
 \end{aligned}
\shortintertext{und}
 (\lambda f)(\mu v)
 = \lambda f(\mu v)
 = \lambda \mu f(v)
 = \mu \lambda f(v)
 = \mu (\lambda f)(v),
\end{gather*}
also auch $f+g \in \Hom(V,W)$ und $\lambda f \in \Hom(V,W)$. Dies zeigt, dass $\Hom(V,W)$ ein Untervektorraum von $\Abb(V,W)$ ist.


%TODO: \section{Eigenräume}


\section{Matrizenräume}


\subsection{Die (\texorpdfstring{$m \times n$}{mxn})-Matrizen \texorpdfstring{$\Mat(m \times n, K)$}{Mat(mxn,K)}}
Die ($m \times n$)-Matrizen $\Mat(m \times n, K)$ bilden zusammen mit der eintragsweisen Addition
\[
 (a_{ij})_{\substack{i=1,\dotsc,m \\ j=1,\dotsc,n}} + (b_{ij})_{\substack{i=1,\dotsc,m \\ j=1,\dotsc,n}}
 = (a_{ij}+b_{ij})_{\substack{i=1,\dotsc,m \\ j=1,\dotsc,n}}
\]
eine abelsche Gruppe. Zusammen mit der eintragsweisen Skalarmultiplikation
\[
 \lambda \cdot (a_{ij})_{\substack{i=1,\dotsc,m \\ j=1,\dotsc,n}}
 = (\lambda a_{ij})_{\substack{i=1,\dotsc,m \\ j=1,\dotsc,n}}
\]
wird $\Mat(m \times n, K)$ zu einem $K$-Vektorraum: Für alle Skalare $\lambda, \mu \in K$ und Matrizen $(a_{ij})_{i,j}, (b_{ij})_{i,j} \in \Mat(m \times n, K)$ ist
\begin{gather*}
 \begin{aligned}
  \lambda \cdot ((a_{ij})_{i,j} + (b_{ij})_{i,j})
  &= \lambda \cdot (a_{ij} + b_{ij})_{i,j}
  = (\lambda (a_{ij}+b_{ij}))_{i,j} \\
  &= (\lambda a_{ij} + \lambda b_{ij})_{i,j}
  = (\lambda a_{ij})_{i,j} + (\lambda b_{ij})_{i,j}
  = \lambda (a_{ij})_{i,j} + \lambda (b_{ij})_{i,j},
 \end{aligned}
\shortintertext{und}
 \begin{aligned}
  (\lambda+\mu) \cdot (a_{ij})_{i,j}
  &= ((\lambda+\mu)a_{ij})_{i,j} \\
  &= (\lambda a_{ij} + \mu a_{ij})_{i,j}
  = \lambda \cdot (a_{ij})_{i,j} + \mu \cdot (a_{ij})_{i,j},
 \end{aligned}
\shortintertext{sowie}
 \lambda \cdot (\mu \cdot (a_{ij})_{i,j})
 = \lambda \cdot (\mu \cdot a_{ij})_{i,j}
 = (\lambda \cdot \mu a_{ij})_{i,j}
 = (\lambda \cdot \mu) \cdot (a_{ij})_{i,j},
\end{gather*}
und für alle $(a_{ij})_{i,j} \in \Mat(m \times n, K)$ gilt
\[
 1 \cdot (a_{ij})_{i,j}
 = (1 \cdot a_{ij})_{i,j}
 = (a_{ij})_{i,j}.
\]
Ingesamt zeigt dies, dass $\Mat(m \times n, K)$ zusammen mit der üblichen eintragsweisen Addition und Skalarmultiplikation einen $K$-Vektorraum bildet.


\subsection{Die (schief)symmetrischen Matrizen}


\begin{defi}
 Es sei $A \in \Mat(m \times n, K)$. Dann ist die \emph{transponierte} Matrix die ($n \times m$)-Matrix $A^T$ definiert durch
 \begin{gather*}
  (A^T)_{ij} = A_{ji}
  \quad
  \text{für alle $1 \leq i \leq n$ und $1 \leq j \leq m$},
 \shortintertext{also}
  A^T
  =
  \begin{pmatrix}
   A_{11} & A_{12} & \cdots & A_{1n} \\
   A_{21} & A_{22} & \cdots & A_{2n} \\
   \vdots & \vdots & \ddots & \vdots \\
   A_{m1} & A_{m2} & \cdots & A_{mn}
  \end{pmatrix}^T
  =
  \begin{pmatrix}
   A_{11} & A_{21} & \cdots & A_{m1} \\
   A_{12} & A_{22} & \cdots & A_{m2} \\
   \vdots & \vdots & \ddots & \vdots \\
   A_{1n} & A_{2n} & \cdots & A_{mn}
  \end{pmatrix}.
 \end{gather*}
\end{defi}


\begin{bem}
 \begin{enumerate}[leftmargin=*]
  \item
   Anschaulich gesehen entsteht $A^T$ aus $A$ durch Spiegelung der Matrix an der Diagonalen.
  \item
   Transponieren ist linear, d.h.\ die Abbildung
   \[
    \mc{T} \colon \Mat(m \times n, K) \to \Mat(n \times m, K), A \mapsto A^T
   \]
   ist linear: Für alle Skalare $\lambda \in K$ und Matrizen $A, B \in \Mat(m \times n, K)$ gilt für alle $1 \leq i \leq n$ und $1 \leq j \leq m$, dass
   \[
    ((A+B)^T)_{ij}
    = (A+B)_{ji}
    = A_{ji} + B_{ji}
    = (A^T)_{ij} + (B^T)_{ij}
    = (A^T + B^T)_{ij}
   \]
   und somit $(A+B)^T = A^T + B^T$, sowie
   \[
    ((\lambda A)^T)_{ij}
    = (\lambda A)_{ji}
    = \lambda A_{ji}
    = \lambda (A^T)_{ij}
    = (\lambda A^T)_{ij}
   \]
   und somit $(\lambda A)^T = \lambda A^T$.
 \end{enumerate}
\end{bem}


\begin{defi}
 Eine Matrix $A \in \Mat(n \times n, K)$ heißt \emph{symmetrisch}, falls \mbox{$A^T = A$}, und \emph{schiefsymmetrisch}, bzw.\ \emph{alternierend}, falls $A^T = -A$. Es ist
 \begin{align*}
  \Sym_n(K) &\coloneqq \{A \in \Mat(n \times n, K) \mid \text{$A$ ist symmetrisch}\}
 \shortintertext{und}
  \Alt_n(k) &\coloneqq \{A \in \Mat(n \times n, K) \mid \text{$A$ ist schiefsymmetrisch}\}.
 \end{align*}
\end{defi}


\begin{bem}
 Ist $A \in \Mat(m \times n, K)$ mit $A^T = A$ oder $A^T = -A$, so ist notwendigerweise $m = n$. Es genügt daher, den Begriff einer symmetrischen, bzw. schiefsymmetrischen Matrix für quadratische Matrizen zu definieren.
\end{bem}


Wir zeigen, dass $\Sym_n(K)$ und $\Alt_n(K)$ Untervektorräume von $\Mat(n \times n, K)$ bilden: Da $0^T = 0 = -0$ ist $0 \in \Sym_n(K)$ und $0 \in \Alt_n(K)$.

Für $A, B \in \Sym_n(K)$ und $\lambda \in A$ ist
\[
 (A+B)^T = A^T + B^T = A+B
 \quad\text{und}\quad
 (\lambda A)^T = \lambda A^T = \lambda A,
\]
also auch $A+B \in \Sym_n(K)$ und $\lambda A \in \Sym_n(K)$. Das zeigt, dass $\Sym_n(K)$ ein Untervektorraum von $\Mat(n \times n, K)$ ist.

Für $A, B \in \Alt_n(K)$ und $\lambda \in K$ ist
\begin{gather*}
 (A+B)^T = A^T+B^T = -A-B = -(A+B)
\shortintertext{und}
 (\lambda A)^T = \lambda A^T = \lambda(-A) = -\lambda A,
\end{gather*}
also auch $A+B \in \Alt_n(K)$ und $\lambda A \in \Alt_n(K)$. Das zeigt, dass $\Alt_n(K)$ ein Untervektorraum von $\Mat(n \times n, K)$ ist


\begin{bem}
 \begin{enumerate}[leftmargin=*]
  \item
   Analog lässt sich zeigen, dass
   \[
    E_\lambda \coloneqq \{A \in \Mat(n \times n, K) \mid A^T = \lambda A\}
   \]
   für alle $\lambda \in K$ einen Untervektorraum ist. Dabei ist insbesondere $\Sym_n(K)= E_1$ und $\Alt_n(K) = E_{-1}$.
  \item
   Ist $\kchar(K) = 2$, so ist $1 = -1$ und somit $\Sym_n(K) = \Alt_n(K)$.
  \item
   Ist $\kchar(K) \neq 2$, so lässt sich jede Matrix $M \in \Mat(n \times n, K)$ als Summe $M = S + A$ mit $S \in \Sym_n(K)$ und $A \in \Alt_n(K)$ schreiben, wobei $A$ und $S$ beide eindeutig sind: Existiert eine solche Zerlegung, so ist sie Eindeutig, da dann $S = (M + M^T)/2$ und $A = (M - M^T)/2$. Ist andererseits $M \in \Mat(n \times n, K)$ eine beliebige quadratische Matrix, so ist $S = (M+M^T)/2$ symmetrisch und $A = (M-M^T)/2$ schiefsymmetrisch (wegen der Linearität des Transponierens) und es gilt $M = S+A$.
 \end{enumerate} 
\end{bem}








