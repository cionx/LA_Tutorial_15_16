%\documentclass[a4paper,10pt]{article}
\documentclass[a4paper,10pt]{scrartcl}

\usepackage{../generalstyle}
\usepackage{specificstyle}

\setromanfont[Mapping=tex-text]{Linux Libertine O}
% \setsansfont[Mapping=tex-text]{DejaVu Sans}
% \setmonofont[Mapping=tex-text]{DejaVu Sans Mono}

\title{Lösungen zu Übungszettel 7}
\author{Jendrik Stelzner}
\date{\today}

\begin{document}
\maketitle









\addtocounter{section}{1}










\section{}
Es sei $\mc{S} \subseteq K$ eine $L$-Basis von $K$ und $\mc{B} \subseteq V$ eine $K$-Basis von $V$. Wir zeigen, dass
\[
 \mc{C} \coloneqq \{\mu b \mid (\mu, b) \in \mc{S} \times \mc{B}\}
\]
eine $L$-Basis von $V$ ist, wobei die Elemente $\mu b$ mit $(\mu, b) \in \mc{S} \times \mc{B}$ paarweise verschieden sind.

Wir zeigen zunächst, dass $\mc{C}$ ein $L$-Erzeugendensystem von $V$ ist; hierfür fixieren wir zunächst ein $v \in V$. Da $\mc{B}$ ist ein $K$-Erzeugendensystem von $V$ ist, gibt es $\lambda_b \in K$ mit $b \in \mc{B}$, so dass $\lambda_b = 0$ für fast alle $b \in B$ und $v = \sum_{b \in \mc{B}} \lambda_b b$.

Da $\mc{S}$ ein $L$-Erzeugendensystem von $K$ ist, gibt es nun für jedes $b \in \mc{B}$ Koeffizienten $c^b_\mu \in L$ mit $\mu \in \mc{S}$, so dass $c^b_\mu = 0$ für fast alle $\mu \in L$ und $\lambda_b = \sum_{\mu \in \mc{S}} c^b_\mu \mu$.

Ist dabei $b \in \mc{B}$ mit $\lambda_b = 0$, so ist dabei $c^b_\mu = 0$ für alle $\mu \in L$, da $\mc{S}$ auch linear unabhängig über $L$ ist. Somit ist $c^b_\mu = 0$ für fast alle $(\mu,b) \in  \mc{S} \times \mc{B}$. Da
\[
 v
 = \sum_{b \in \mc{B}} \lambda_b b
 = \sum_{b \in \mc{B}} \sum_{\mu \in \mc{S}} c^b_\mu \mu b
 = \sum_{(\mu,b) \in \mc{S} \times \mc{B}} c^b_\mu \, \mu b
\]
ist somit $v \in \Ell_L(\mc{C})$. Also ist $\mc{C}$ ein $L$-Erzeugendensystem von $V$.

Sind andererseits $c^b_\mu \in L$ mit $(\mu,b) \in \mc{S} \times \mc{B}$, so dass $c^b_\mu = 0$ für fast alle $(\mu,b) \in \mc{S} \times \mc{B}$ und $0 = \sum_{(\mu,b) \in \mc{S} \times \mc{B}} c^b_\mu \, \mu b$, so ist
\[
 0
 = \sum_{(\mu,b) \in \mc{S} \times \mc{B}} c^b_\mu \, \mu b
 = \sum_{b \in \mc{B}} \sum_{\mu \in \mc{S}} c^b_\mu \, \mu b
 = \sum_{b \in \mc{B}} \left( \sum_{\mu \in \mc{S}} c^b_\mu \mu \right) b.
\]
Da $\mc{B}$ linear unabhängig über $K$ ist, folgt hieraus, dass $\sum_{\mu \in \mc{S}} c^b_\mu \mu = 0$ für alle $b \in \mc{B}$. Da $\mc{S}$ linear unabhängig über $L$ ist, ist bereits $c^b_\mu = 0$ für alle $(\mu,b) \in \mc{S} \times \mc{B}$. Also ist $\mc{C}$ auch linear unabhängig über $L$ und die Elemente $\mu b$ mit $(\mu,b) \in \mc{S} \times \mc{B}$ sind paarweise verschieden.

Somit ist $\mc{C}$ ein $L$-Basis von $V$ und
\[
 \# \mc{C} = \#(\mc{S} \times \mc{B}) = (\# \mc{S}) \cdot (\# \mc{B}).
\]
Da $\dim_L(K) = \# \mc{S}$, $\dim_K(V) = \# \mc{B}$ und $\dim_L(V) = \# \mc{C}$ folgt daraus, dass
\[
 \dim_L(V) = \dim_L(K) \cdot \dim_K(V).
\]
Man bemerke, dass $\dim_K(V) > 0$ da $V \neq 0$, und dass $\dim_L(K) > 0$ da $K \neq 0$. Daher $\dim_L(V) < \infty$ genau dann, wenn $\dim_L(K) < \infty$ und $\dim_K(V) < \infty$.













\section{}
Für alle $w_1, w_2 \in W$ ist
\begin{align*}
 f^{-1}(w_1 + w_2)
 &= f^{-1}(f(f^{-1}(w_1)) + f(f^{-1}(w_2))) \\
 &= f^{-1}(f(f^{-1}(w_1) + f^{-1}(w_2)))
 = f^{-1}(w_1) + f^{-1}(w_2),
\end{align*}
und für alle $\lambda \in K$ und $w \in W$ ist
\[
 f^{-1}(\lambda w)
 = f^{-1}(\lambda f(f^{-1}(w)))
 = f^{-1}(f(\lambda f^{-1}(w)))
 = \lambda f^{-1}(w).
\]
Also ist auch $f^{-1}$ linear.








\section{}
Für eine lineare Abbildung $f \colon V \to W$ zeigen wir die Äquivalenz der folgenden Aussagen:

\begin{enumerate}[label=(\roman*)]
 \item\label{enum: image of generating set}
  Für jedes Erzeugendensystem $S$ von $V$ ist $f(S)$ ein Erzeugendensystem von $W$.
 \item\label{enum: image of a subset}
  Es gibt eine Teilmenge $S \subseteq V$, so dass $f(S)$ ein Erzeugendensystem von $W$ ist.
 \item\label{enum: surjective}
  $f$ ist surjektiv.
\end{enumerate}


(\ref{enum: image of generating set} $\implies$ \ref{enum: image of a subset}) Dies ergibt sich direkt daraus, dass man das Erzeugendensystem $S = V$ betrachtet.

(\ref{enum: image of a subset} $\implies$ \ref{enum: surjective}) Da $f(S)$ ein Erzeugendensystem von $W$ ist, gilt $\Ell(f(S)) = W$. Also ist
\[
 W = \Ell(f(S)) =  f(\Ell(S)) \subseteq f(V) \subseteq W,
\]
also bereits $f(V) = W$. Somit ist $f$ surjektiv.

(\ref{enum: surjective} $\implies$ \ref{enum: image of generating set}) Da $f$ surjektiv ist, ist $f(V) = W$. Ist $S \subseteq V$ ein Erzeugendensystem, so gilt deshalb
\[
 \Ell(f(S)) = f(\Ell(S)) = f(V) = W.
\]
Also ist $f(S)$ ein Erzeugendensystem von $W$.

Das zeigt die Äquivalenz der Aussagen. Gilt nun eine (und damit alle) dieser Aussagen und ist $\mc{B} \subseteq V$ eine Basis, so ist $f(\mc{B}) \subseteq W$ nach \ref{enum: image of generating set} ein Erzeugendensystem von $W$. Also ist dann
\[
 \dim V = |\mc{B}| \geq |f(\mc{B})| \geq \dim W.
\]











\section{}
Es sei
\[
 A =
 \begin{pmatrix}
  a_{11} & \cdots & a_{1n} \\
  \vdots & \ddots & \vdots \\
  a_{m1} & \cdots & a_{mn}
 \end{pmatrix},
\]
und für alle $1 \leq j \leq n$ sei
\[
 a_j = \vect{a_{1j} \\ \vdots \\ a_{mj}}
\]
der $j$-te Spaltenvektor von $A$.





\subsection{}
Die Matrix $\tilde{A}$ entstehe aus $A$ durch elementare Zeilenumformungen und sei in Zeilenstufenform.

Dass die Zeilen von $A$ linear unabhängig sind, ist äquivalent dazu, dass die $\tilde{A}$ keine Nullzeilen enthält.

Da $\tilde{A}$ in Zeilenstufenform ist, enthält $\tilde{A}$ genau dann keine Nullzeilen, wenn das lineare Gleichungssystem $\tilde{A}x = y$ für alle $y \in K^m$ eine Lösung hat

Da $\tilde{A}$ durch $A$ aus elementaren Zeilenumformungen hervorgeht, hat das lineare Gleichungssystem $\tilde{A}x = y$ genau dann für jedes $y \in K^m$ eine Lösung, wenn das lineare Gleichungssystem $Ax = y$ für alle $y \in K^m$ ein Lösung hat (diese Lösungen müssen allerdings nicht notwendigerweise gleich sein).

Dass das lineare Gleichungssystem $Ax = y$ für jedes $y \in K^m$ eine Lösung hat, bedeutet aber nichts anderes, als dass die Abbildung $A \cdot -$ surjektiv ist.

Also sind die Zeilen von $A$ genau dann linear unabhängig, falls $A \cdot -$ surjektiv ist.



\subsection{}
Für alle $\lambda_1, \dotsc, \lambda_n \in K$ ist
\begin{align*}
 A \cdot \vect{\lambda_1 \\ \vdots \\ \lambda_n}
 &=
 \begin{pmatrix}
  a_{11} & \cdots & a_{1n} \\
  \vdots & \ddots & \vdots \\
  a_{m1} & \cdots & a_{mn}
 \end{pmatrix}
 \cdot
 \vect{\lambda_1 \\ \vdots \\ \lambda_n}
 =
 \vect{\lambda_1 a_{11} + \dotsb + \lambda_n a_{1n} \\ \vdots \\ \lambda_1 a_{m1} + \dotsb + \lambda_n a_{mn}} \\
 &= \lambda_1 \vect{a_{11} \\ \vdots \\ a_{m1}} + \dotsb + \lambda_n \vect{a_{1n} \\ \vdots \\ a_{mn}}
 = \lambda_1 a_1 + \dotsb + \lambda_n a_n.
\end{align*}
Dass $\lambda_1 a_1 + \dotsb + \lambda_n a_n = 0$ eine Linearkombination der Spaltenvektoren $a_1, \dotsc, a_n$ ist, ist also äquivalent dazu, dass
\[
 A \cdot \vect{\lambda_1 \\ \vdots \\ \lambda_n} = 0.
\]
Also sind die Spalten von $A$ genau dann linear unabhängig, wenn $\ker(A) = 0$, also $A$ injektiv ist.























\end{document}
