\chapter{Äquivalenz- und Ordnungsrelationen}

\begin{defi}
 Eine \emph{Relation} auf einer Mengen $X$ ist eine Teilmenge $R \subseteq X \times X$. Für $x,y \in X$ sagen wir, dass $x$ in Relation zu $y$ steht, wenn $(x,y) \in R$.
 
 Die Relation $R$ heißt
 \begin{enumerate}[label=\roman*)]
  \item
   \emph{reflexiv}, falls $(x,x) \in R$ für alle $x \in X$,
  \item
   \emph{symmetrisch}, falls für alle $x,y \in X$ mit $(x,y) \in R$ auch $(y,x) \in R$,
  \item
   \emph{anti-symmetrisch}, falls für alle $x,y \in X$ mit $(x,y) \in R$ und $(y,x) \in R$ bereits $x=y$.
  \item
   \emph{transitiv}, falls für alle $x,y,z \in X$ mit $(x,y) \in R$ und $(y,z) \in R$ auch $(x,y) \in R$,
 \end{enumerate}
\end{defi}

\begin{bem}
 Manche Autoren nutzen für $(x,y) \in R$ auch die Notation $xRy$.
\end{bem}

\begin{bsp}
 \begin{enumerate}[leftmargin=*]
  
  \item
   Auf der Menge $\Rbb$ der reellen Zahlen definieren wir die Relation $R$ durch
   \[
    R_1 \coloneqq \{(x,y) \in \Rbb \times \Rbb \mid x \leq y\}.
   \]
   $R$ ist reflexiv, denn für alle $x \in \Rbb$ ist $x \leq x$, also $(x,x) \in R$. $R$ ist \emph{nicht} symmetrisch, da etwa $(1,2) \in R$ aber $(2,1) \notin R$. $R$ ist allerdings schief-symmetrisch, denn für $x,y \in \Rbb$ mit $(x,y), (y,x) \in R$ ist $x \leq y$ und $y \leq x$, also $x = y$. $R$ ist auch transitiv, denn für $x,y,z \in \Rbb$ mit $(x,y), (y,z) \in R$ ist $x \leq y$ und $y \leq z$, also auch $x \leq z$ und somit $(x,z) \in \Rbb$.
  
  \item
   Wir definieren auf der Menge $\Qbb$ der rationalen Zahlen eine Relation $R$ durch
   \[
    R_1 = \{(x,y) \in \Qbb \times \Qbb \mid x < y\}.
   \]
   $R$ ist weder reflexiv, noch symmetrisch oder anti-symmetrisch. $R$ ist allerdings transitiv.
  
  \item
   Es sei $X$ eine beliebige Menge und
   \[
    R = \{(x,x) \mid x \in X\}.
   \]
   (Es steht also jedes Element nur zu sich selbst in Relation.) Es folgt direkt, dass $R$ reflexiv ist. $R$ ist symmetrisch, denn für $(x,y) \in R$ ist $x = y$, also $(y,x) = (x,y) = (x,x) \in R$. $R$ ist auch anti-symmetrisch, denn für $x,y \in X$ mit $(x,y), (y,x) \in R$ ist $x = y$. $R$ ist transitiv, den für $x,y,z \in X$ mit $(x,y), (y,z) \in R$ ist $x = y$ und $y = z$, also auch $x = z$ und somit $(x,z) \in R$.
  
  \item
   Auf der Menge $\Zbb$ der ganzen Zahlen definieren wir die Relation
   \[
    R = \{(n,n+1) \mid n \in \Zbb\}.
   \]
   (Es steht also jedes ganze Zahl mit genau ihrem Nachfolger in Relation.) $R$ ist nicht reflexiv, da etwa $(1,1) \notin R$. $R$ ist auch nicht symmetrisch, denn für $n,m \in \Zbb$ mit $(n,m) \in R$ ist $m = n+1$, also $(m,n) = (n+1,n) \notin R$. $R$ ist auch nicht transitiv, denn für $l,m,n \in \Zbb$ mit $(l,m),(m,n) \in R$ ist $m = l+1$ und $n = m+1$, also $n = l+2$ und somit $(l,n) = (l,l+2) \notin R$.
   
   $R$ ist allerdings anti-symmetrisch: Sind $n,m \in \Zbb$ mit $(n,m), (m,n) \in R$, so ist $m = n+1$ ud $n = m+1$, also $n = n+2$, was nicht möglich ist. Es gibt also keine $n,m \in \Zbb$ mit $(n,m) \in R$; daher handelt es sich bei der Anti-Symmetrie von $R$ um eine leere Aussage, die daher gilt.
  
  \item
   Auf $\Rbb$ definieren wir die Reation
   \[
    R
    = \{(x,y) \in \Rbb \times \Rbb \mid x = \pm y\}.
   \]
   (Zwei Zahlen sind also in Relation, wenn sie bis auf Vorzeichen gleich sind.) Da $x \in R$ für alle $x \in \Rbb$ ist $\Rbb$ reflexiv. $R$ ist symmmetrisch, denn sind $x,y \in \Rbb$ mit $(x,y) \in R$, so ist $x = \pm y$, also auch $y = \pm x$ und somit $(y,x) \in R$. $R$ ist allerdings nicht anti-symmetrisch, da etwa $(1,-1), (-1,1) \in R$ aber $1 \neq -1$. $R$ ist allerdings transitiv, denn für $x,y,z \in \Rbb$ mit $(x,y), (y,z) \in R$ ist $x = \pm y$ und $y = \pm z$, also auch $x = \pm z$, und somit $(x,z) \in R$.
   
  \item
   Wir definieren auf der Menge der reellen Zahlen $\Rbb$ eine Relation
   \[
    R = \{(x,y) \in \Rbb \times \Rbb \mid y = x^2\}.
   \]
   ($x$ steht also in Relation zu $y$, falls $x$ ein Quadratwurzel von $y$ ist.) $R$ ist nicht reflexiv, da etwa $(2,2) \notin R$. $R$ ist auch nicht symmetrisch, da etwa $(2,4) \in R$ aber $(4,2) \notin R$. $R$ ist auch nicht transitiv, da etwa $(2,4) \in R$ und $(4,16) \in R$, aber $(2,16) \notin R$.
   
   $R$ ist allerdings schief-symmetrisch: Sind $x, y \in \Rbb$ mit $(x,y), (y,x) \in R$, so ist $y = x^2$ und $x = y^2$. Inbesondere sind daher $x,y \geq 0$ mit $x = y^2 = (x^2)^2 = x^4$ und analog $y = y^4$. Es ist also $x \in \{0,1\}$ und $y \in \{0,1\}$. Ist $x = 0$, so ist $y = x^2 = 0$ und $x = 0 = y$. Ist $x = 1$, so ist $y = x^2 = 1$ und somit $x = 1 = y$. Es ist also in beiden Fällen $x = y$.
 \end{enumerate}
\end{bsp}





\section{Äquivalenzrelationen}
\begin{defi}
 Eine Relation $\sim$ auf einer Menge $X$ heißt \emph{Äquivalenzrelation}, falls $\sim$ reflexiv, symmetrisch und transitiv ist. Für alle $x,y \in X$ schreibt man $x \sim y$ statt $(x,y) \in \sim$.
\end{defi}

\begin{bsp}
 \begin{enumerate}[leftmargin=*]
  \item
   Es sei $X$ eine Menge. Wie bereits gesehen ist Gleichheit $=$ eine Äquivalenzrelation auf $X$, d.h.\
   \[
    R = \{(x,y) \mid x = y\} \subseteq X \times X
   \]
   ist eine Äquivalenzrelation.
  \item
   Für $n,m \in \Zbb$ sei $n \sim m$ genau dann, wenn $n-m$ gerade ist, d.h.\ falls es $k \in \Zbb$ mit $n-m = 2k$ gibt. Wir zeigen, dass $\sim$ eine Äquivalenzrelation auf $\Zbb$ definiert:
   
   $\sim$ ist reflexiv, denn alle $n \in \Zbb$ ist $n-n = 0$ gerade, also $n \sim n$.
   
   Sind $n,m \in \Zbb$ mit $n \sim m$, so ist $n-m$ gerade, also $n-m = 2k$ für ein $k \in \Zbb$. Dann ist auch $m-n = 2(-k)$ gerade, also $m \sim n$. Also ist $\sim$ symmetrisch.
   
   Sind $p,q,r \in \Zbb$ mit $p \sim q$ und $q \sim r$, so sind $p-q$ und $q-r$ ungerade, d.h.\ es gibt $k,l \in \Zbb$ mit $p-q = 2k$ und $q-r = 2l$. Daher ist auch $p-r = (p-q)+(q-r) = 2k + 2l = 2(k+l)$ gerade, also $p \sim r$. Das zeigt, dass $\sim$ auch transitiv ist.
  \item
   Es sei $X$ ein Menge und $f \colon X \to Y$ eine Funktion. Für $x_1, x_2 \in X$ sei $x_1 \sim x_2$ genau dann, wenn $f(x_1) = f(x_2)$. Dann ist $\sim$ ein Äquivalenzrelation:
   
   Dass $\sim$ reflexiv ist, folgt direkt daraus, dass $f(x) = f(x)$ für alle $x \in X$.
   
   Sind $x_1, x_2 \in X$ mit $x_1 \sim x_2$, also $f(x_1) = f(x_2)$, so ist auch $f(x_2) = f(x_1)$, also $x_2 \sim x_2$. Das zeigt, dass $\sim$ symmetrisch ist.
   
   Sind $x_1, x_2, x_3 \in X$ mit $x_1 \sim x_2$ und $x_2 \sim x_3$, also $f(x_1) = f(x_2)$ und $f(x_2) = f(x_3)$, so ist deshalb $f(x_1) = f(x_3)$, also $x_1 \sim x_3$. Somit ist $\sim$ auch transitiv.
  \item
   Auf $x,y \in \Rbb$ sei $x \sim y$ genau dann, wenn $x-y \in \Zbb$. Hierdurch wird eine Äquivalenzrelation auf $\Rbb$ definiert:
   
   Dass $\sim$ reflexiv ist, also $x \sim x$ für alle $x \in\ \Rbb$, folgt direkt daraus, dass $x-x = 0 \in \Zbb$ für alle $x \in \Rbb$.
   
   Sind $x,y \in \Rbb$ mit $x \sim y$, also $x-y \in \Zbb$, so ist auch $y-x \in \Zbb$, und somit $y \sim x$. Deshalb ist $\sim$ symmetrisch.
   
   Sind $x,y,z \in \Rbb$ mit $x \sim y$ und $y \sim z$, so ist $x-y, y-z \in \Zbb$. Deshalb ist auch $x-z = (x-y)+(y-z) \in \Zbb$, also $x \sim z$. Das zeigt die Transitivität von $\sim$.
 \end{enumerate}
\end{bsp}

\begin{defi}
 Es sei $\sim$ eine Äquivalenzrelation auf einer Menge $X$. Für $x \in X$ ist
 \[
  [x]_{\sim} \coloneqq \{y \in X \mid x \sim y\}
 \]
 die \emph{Äquivalenzklasse} von $x$, und $X/{\sim} \coloneqq \{[x] \mid x \in X\}$ ist die Menge der Äquivalenzklassen von $X$.
\end{defi}

\begin{bem}
 \begin{enumerate}[leftmargin=*]
  \item
   Ist klar, um welche Äquivalenzrelation es sich handelt, so schreibt man auch nur $[x]$ statt $[x]_\sim$.
  \item
   Ist $\sim$ eine Äquivalenzrelation auf einer Menge $X$, so gilt für alle $x,y \in X$, dass entweder $[x] \cap [y] = \emptyset$ oder $[x] = [y]$.
   
   Ist nämlich $x \sim y$, so ist für jedes $z \in X$ genau dann $z \sim x$, falls $z \sim y$, da
   \begin{gather*}
    x \sim z
    \implies y \sim x, x \sim z
    \implies y \sim z
   \shortintertext{und analog}
    y \sim z
    \implies x \sim y, y \sim z
    \implies x \sim z.
   \end{gather*}
   Also ist in diesem Fall
   \[
    [x] = \{z \in X \mid x \sim z\} = \{z \in X \mid y \sim z\} = [y].
   \]
   
   Ist andererseits $x \nsim y$, und gebe es ein $z \in [x] \cap [y]$, so wäre $x \sim z$ und $z \sim y$, wegen der Transitivität von $\sim$ also auch $x \sim y$, ein Widerspruch.
   
   Es ist also $x \sim y \implies [x] = [y]$ und $x \nsim y \implies [x] \cap [y] = \emptyset$. Da entweder $x \sim y$ oder $x \nsim y$ handelt es sich schon jeweils um Äquivalenzen.
 \end{enumerate}
\end{bem}

\begin{bsp}
 \begin{enumerate}[leftmargin=*]
  \item
   Für die Gleichheit $=$ auf einer Menge $X$ ist für jedes $x \in X$ die Äquivalenzklasse gegeben durch
   \[
    [x] = \{y \in X \mid y = x\} = \{x\}.
   \]
   Also ist die Abbildung
   \[
    X \to X/{\sim}, \quad x \mapsto [x]
   \]
   eine Bijektion.
  \item
   Wir betrachten auf $\Zbb$ die Äquivalenzrelation $\sim$ mit $n \sim m$ genau dann wenn $n-m$ gerade ist, d.h.\ wenn $n-m = 2k$ für ein $k \in \Zbb$. Für $n \in \Zbb$ ist dann
   \[
    [n]
    = \{m \in \Zbb \mid \text{$n-m = 2k$ für ein $k \in \Zbb$}\}
    = \{n + 2k \mid k \in \Zbb\}
    = n+2\Zbb.
   \]
   Inbesondere ist deshalb $[0] = \{2k \mid k \in \Zbb\} = 2\Zbb$ die Menge der geraden Zahlen und $[1] = \{1+2k \mid k \in \Zbb\} = 1+2\Zbb$ die Menge der ungeraden Zahlen. Ist $n \in \Zbb$ gerade, so ist $n \in [0]$, also $[n] \cap [0] \neq \emptyset$ und somit $[n] = [0]$ die Menge der geraden Zahlen. Ist andererseits $n$ ungerade, so ist $[n] \in [1]$, also $n \cap [1] \neq \emptyset$ und somit $[n] = [1]$ die Menge der ganzen Zahlen.
   
   Es gibt in diesem Fall also nur zwei verschieden Äquivalenzklassen, nämlich die geraden Zahlen $[0]$ und die ungeraden Zahlen $[1]$. Also ist die Abbildung
   \[
    \{0,1\} \mapsto \Zbb/{\sim}, \quad n \mapsto [n]
   \]
   eine Bijektion.
  \item
   Wir betrachten auf $\Rbb$ die Äquivalenzrelation $\sim$ mit $x \sim y$ genau dann, wenn $x-y$ ganzzahlig ist, d.h.\ wenn $x-y \in \Zbb$. Für jedes $x \in \Rbb$ ist dann
   \begin{align*}
    [x]
    &= \{y \in \Rbb \mid x-y \in \Zbb\}
    = \{y \in \Rbb \mid \text{es gibt $k \in \Zbb$ mit $x-y = k$}\} \\
    &= \{x+k \mid k \in \Zbb\}
    = x + \Zbb.
   \end{align*}
   Für jede ganze Zahl $n \in \Zbb$ ist etwa $[n] = \Zbb$, und es sind
   \begin{align*}
    \left[ \frac{1}{2} \right]
    &= \left\{ \dotso, -\frac{3}{2}, -\frac{1}{2}, \frac{1}{2}, \frac{3}{2}, \frac{5}{2}, \dotso \right\}
   \shortintertext{und}
    [\pi] &= \{ \dotso, \pi-2, \pi-1, \pi, \pi+1, \pi+2, \dotso \}.
   \end{align*}
   Es ist nun einfach zu sehen, dass es für jedes $x \in \Rbb$ genau ein $x' \in [0,1)$ mit $x \sim x'$ gibt. Daher ist die Abbildung
   \[
    [0,1) \to \Rbb/{\sim}, \quad x \mapsto [x]
   \]
   eine Bijektion.
  \item
   Durch den Betracht $b \colon \Cbb \to \Rbb_{\geq 0}$, $z \mapsto |z|$ ergibt sich eine Äquivalenzrelation auf $\Cbb$ durch
   \[
    z \sim w
    \iff b(z) = b(w)
    \iff |z| = |w|.
   \]
   Für jedes $r \in \Rbb$ mit $r \geq 0$ ist 
   \[
    [r]
    = \{z \in \Cbb \mid z \sim r\}
    = \{z \in \Cbb \mid |z| = |r|\}
    = \{z \in \Cbb \mid |z| = r\}.
   \]
   Also ist $[0] = \{0\}$ und für alle $r > 0$ ist $[r] = \{z \in \Cbb \mid |z| = r\}$ der Kreis mit Radius $r$ um den Mittelpunkt $0 \in \Cbb$. Da $[z] = [|z|]$ für alle $z \in \Cbb$ und $[r_1] \cap [r_2] = \emptyset$ für alle $0 \leq r_1 < r_2$ ist die Abbildung
   \[
    [0,\infty) \to \Cbb/{\sim}, \quad z \mapsto [z]
   \]
   eine Bjiektion.
 \end{enumerate}
\end{bsp}

\begin{defi}
 Es sei $\sim$ ein Äquivalenzrelation auf einer Menge $X$. Ein Element $x \in A$ einer Äquivalenzklasse $A \in X/{\sim}$ bezeichnet man als \emph{Repräsentantensystem} der Äquivalenzklasse $A$. Eine Teilmenge $R \subseteq X$ heißt \emph{Repräsentantensystem}, falls es für jedes $x \in X$ genau ein $r \in R$ gibt, so dass $x \sim r$, d.h.\ die Abbildung
 \[
  R \to X/{\sim}, \quad r \mapsto [r]
 \]
 ist eine Bijektion.
\end{defi}

\begin{bem}
 Allgemein erhält man für eine Äquivalenzrelation $\sim$ auf einer Menge $X$ ein Repräsentantensystem, indem man aus jeder Äquivalenzklasse $A \in X/{\sim}$ einen Repräsentanten $r_A \in A$ wählt, und diese Repräsentanten anschließend zu $R = \{r_A \mid A \in X/{\sim}\}$ zusammenfasst.
\end{bem}


\begin{bsp}
 \begin{enumerate}[leftmargin=*]
  \item
   Ist $X$ ein Menge, so ist für die Äquivalenzrelation der Gleichheit $=$ die Menge $X$ selbst das einzige Repräsentantensystem.
  \item
   Bezüglich der Äquivalenzrelation $\sim$ auf $\Zbb$ mit $n \sim m$ genau dann, wenn $n-m$ gerade ist, ist ein Repräsentantensystem gegeben durch $\{0,1\}$.
   
   Da $\Zbb/{\sim} = \{2\Zbb, 1+2\Zbb\}$, die Äquivalenzklassen von $\sim$ also die geraden und ungeraden Zahlen sind, sind die Repräsentantensysteme von $\sim$ genau die Teilmengen $R \subseteq \Zbb$ von der Form $R = \{n,m\}$ mit $n$ gerade und $m$ ungerade.
  \item
   Bezüglich der Äquivalenzrelation $\sim$ auf $\Rbb$ mit $x \sim y$ genau dann, wenn $x-y \in \Zbb$, ist ein mögliches Repräsentantensystem durch $[0,1)$ gegeben. Allgemeiner ist für jedes $y \in \Rbb$ eine Repräsentantensystem durch das halboffene Intervall $[y,y+1)$ gegeben.
  \item
   Für die Äquivalenzrelation $\sim$ auf $\Cbb$ mit $z \sim w$ genau dann, wenn $|z| = |w|$, ist ein mögliches Repräsentantensystem durch das halboffene reelle Intervall $[0,\infty) \subseteq \Rbb$ gegeben.
 \end{enumerate}
\end{bsp}









\section{Ordnungsrelationen}
\begin{defi}
 Eine Relation $R$ auf einer Menge $X$ heißt \emph{Ordnungsrelation}, falls $R$ reflexiv, anti-symmetrisch und transitiv ist.
\end{defi}

\begin{bsp}
 \begin{enumerate}[leftmargin=*]

  \item
   Wie bereits gesehen, ist
   \[
    R = \{(x,y) \in \Rbb \times \Rbb \mid x \leq y\}
   \]
   eine Ordnungsrelation auf $\Rbb$.
   
  \item
   Für eine Menge $X$ sei
   \[
    R = \{(A,B) \in \pwr(X) \times \pwr(X) \mid A \subseteq B\},
   \]
   wobei $\pwr(X) = \{A \mid A \subseteq X\}$ die Potenmenge von $X$ bezeichnet. $R$ ist reflexiv, den für jede Teilmenge $A \subseteq X$ ist $A \subseteq A$. $R$ ist anti-symmetrisch, denn sind $A, B \subseteq X$ zwei Teilmengen mit $(A,B), (B,A) \in R$, also $A \subseteq B$ und $B \subseteq A$, so ist bereits $A = B$. $R$ ist auch transitiv, denn für Teilmengen $A,B,C \subseteq X$ mit $(A,B), (B,C) \in R$ ist $A \subseteq B$ und $B \subseteq C$, also auch $A \subseteq C$, und somit $(A,C) \in R$.
   
  \item
   Für zwei ganze Zahlen $n,m \in \Zbb$ sagen wir, \emph{$n$ teilt $m$}, wenn es ein $k \in \Zbb$ mit $m = kn$ gibt; wir schreiben dann $n \mid m$.
   \begin{enumerate}[label=\alph*)]
    \item
     Wir definieren auf $\Nbb$ die Relation
     \[
      R = \{(n,m) \in \Nbb \mid n \mid m\}.
     \]
     $R$ ist reflexiv, denn für alle $n \in \Nbb$ ist $n = 1 \cdot n$, also $n \mid n$.
     
     $R$ ist anti-symmetrisch, denn sind $n,m \in \Nbb$ mit $(n,m), (m,n) \in R$, so ist $n \mid m$ und $m \mid n$, d.h.\ es gibt $k,l \in \Zbb$ mit $m = kn$ und $n = lm$. Es ist dann $n = km = kln$. Ist $n = 0$, so ist auch $m = kn = 0$, also $n = m$. Ist $n \neq 0$, so folgt aus $n = kln$ dass $kl = 1$; da $k,l \in \Zbb$ ist bereits $k = l = \pm 1$. Da $m = kn$ muss $k = l = 1$. Somit ist $n = k m = m$. In beiden Fällen ist also $n = m$.
     
     $R$ ist auch transitiv, denn für $p,q,r \in \Nbb$ mit $(p,q), (q,r) \in R$ ist $p \mid q$ und $q \mid r$, d.h.\ es gibt $k,l \in \Zbb$ mit $q = kp$ und $r = lq$. Es ist daher $r = lq = lkp$, also auch $p \mid r$, und somit $(p,r) \in R$.
     
     Also ist $R$ eine Ordnungsrelation auf $\Nbb$.
    \item
     Wir definieren auf $\Zbb$ die Relation
     \[
      R = \{(n,m) \in \Zbb \mid n \mid m\}.
     \]
     Wie bereits zuvor ergibt sich, dass $R$ reflexiv und transitiv ist. $R$ ist allerdings nicht anti-symmetrisch. So ist etwa $(-1) \mid 1$ und $1 \mid (-1)$, da $1 = (-1) \cdot (-1)$ und $-1 = (-1) \cdot 1$, aber $1 \neq -1$. $R$ definiert also keine Ordnungsrelation auf $R$.
   \end{enumerate}
 \end{enumerate}
\end{bsp}

Die obigen Beispiele motivieren die folgende Definition.

\begin{defi}
 Eine \emph{geordnete Menge} ist ein Paar $(X,\leq)$ bestehend aus einer Menge $X$ und einer Ordnungsrelation $\leq$ auf $X$. Für $x,y \in X$ schreibt man $x \leq y$ statt $(x,y) \in \leq$. Außerdem führt man für $x,y \in X$ die folgenden Notationen ein:
 \begin{enumerate}[label=\roman*)]
  \item
   Man schreibt $y \geq x$ falls $x \leq y$.
  \item
   Man schreibt $x < y$ falls $x \leq y$ und $x \neq y$.
  \item
   Man schreibt $y > x$ falls $y \geq x$ und $x \neq y$.
 \end{enumerate}
\end{defi}

\begin{bem}
 \begin{enumerate}[leftmargin=*]
  \item
   Häufig nennt man die Ordnungsrelation nicht explizit, d.h.\ statt von einer geordneten Menge $(X,\leq)$ spricht man von einer geordneten Menge $X$.
  \item
   Ist $(X,\leq)$ eine geordnete Menge und $Y \subseteq X$ eine Teilmenge, so ist auch die Einschränkung $\leq_Y = \leq \cap (Y \times Y) \subseteq Y \times Y$ eine Ordnungsrelation auf $Y$, d.h.\
   \[
    y_1 \leq_Y y_2 \iff y_1 \leq y_2
    \quad
    \text{für alle $y_1, y_2 \in Y$}.
   \]
   Wegen dieser Äquivalenz schreibt man auch einfach nur $\leq$ statt $\leq_Y$, benutzt also kein zusätzliches Symbol für die eingeschränkte Ordnungsrelation.
 \end{enumerate}
\end{bem}

\begin{defi}
 Eine geordnete Menge $X$ heißt \emph{linear geordnet}, bzw.\ \emph{total geordnet}, falls für alle $x,y \in X$ gilt, dass $x \leq y$ oder $y \leq x$.
\end{defi}

\begin{bsp}
 \begin{enumerate}[leftmargin=*]
  \item
   Die reellen Zahlen sind bezüglich der üblichen Ordnungsrelation $\leq$ linear geordnet.
  \item
   Ist $X$ eine Menge, so ist $\pwr(X)$ bezüglich der Teilmengenrelation $\subseteq$ im Allgemeinen nicht linear geordnet: Besitzt $X$ mindestens zwei verschiedene Elemente $x,y \in X$, so sind $\{x\}, \{y\} \subseteq X$ zwei Teilmengen mit $\{x\} \nsubseteq \{y\}$ und $\{y\} \nsubseteq \{y\}$.
  \item
   Die natürlichen Zahlen $\Nbb$ sind bezüglich der Teilbarkeitsrelation $|$ nicht linear geordnet, da etwa $2 \nmid 3$ und $3 \nmid 2$.
  \item
   Ist $(X,\leq)$ eine total geordnete Menge und $Y \subseteq X$ eine Teilmenge, so ist auch $Y$ linear geordnet bezüglich $\leq$.
 \end{enumerate}
\end{bsp}

\begin{defi}
 Es sei $X$ eine geordnete Menge und $Y \subseteq X$ eine Teilmenge. Ein Element $x \in X$ heißt
 \begin{enumerate}[label=\roman*)]
  \item
   \emph{obere Schranke} von $Y$, falls $y \leq x$ für all $y \in Y$,
  \item
   \emph{untere Schranke} von $Y$, falls $x \leq y$ für alle $y \in Y$,
  \item
   \emph{Supremum} von $Y$, falls $x$ eine kleineste obere Schranke von $Y$ ist, d.h.\ $x$ ist eine obere Schranke von $Y$, und für jede obere Schranke $s$ von $Y$ ist $x \leq s$,
  \item
   \emph{Infimum} von $Y$, falls $x$ eine größte untere Schranke von $Y$ ist, d.h.\ $x$ ist eine untere Schranke von $Y$, und für jede untere Schranke $t$ von $Y$ ist $t \leq x$.
 \end{enumerate}
\end{defi}


























\chapter{Konvergenz und Summierbarkeit von Folgen}\label{sec: sequences and series}
Wir erinnern hier an grundlegende Definitionen und Aussagen über konvergente Folgen und Reihen. In diesem Abschnitt sei $\Kbb \in \{\Qbb, \Rbb, \Cbb\}$.


\begin{defi}
 Es sei $(x_n)_{n \in \Nbb} \in \ell(\Kbb)$, also $(x_n)_{n \in \Nbb}$ eine Folge mit Werten in $\Kbb$, und $x \in \Kbb$. Wir sagen, dass die Folge $(x_n)_{n \in \Nbb}$ gegen $x$ \emph{konvergiert}, falls es für jedes $\varepsilon > 0$ ein $N \in \Nbb$ gibt, so dass $|x_n - x| < \varepsilon$ für alle $n \geq N$. Die Folge $(x_n)_{n \in \Nbb}$ heißt dann \emph{konvergent} und $x$ heißt \emph{Grenzwert} der Folge $(x_n)_{n \in \Nbb}$. Wir schreiben abkürzend, dass $x_n \to x$ für $n \to \infty$ falls die Folge $(x_n)_{n \in \Nbb}$ gegen $x$ konvergiert.
\end{defi}


\begin{bem}
 Ist $(x_n)_{n \in \Nbb}$ eine konvergente $\Kbb$-wertige Folge, so ist der Grenzwert eindeutig: Angenommen, es gibt $x,y \in \Kbb$ mit $x_n \to x$ für $n \to \infty$ und $x_n \to y$ für $n \to \infty$, aber $x \neq y$. Dann ist $|x-y| > 0$ und somit auch $\varepsilon \coloneqq |x-y|/3 > 0$. Da $x_n \to x$ für $n \to \infty$ gibt es $N_x \in \Nbb$ mit $|x-x_n| < \varepsilon$ für alle $n \geq N_x$, und da $x_n \to y$ für $n \to \infty$ gibt es $N_y \in \Nbb$ mit $|y-x_n| < \varepsilon$ für alle $n \geq N_y$. Für $n \coloneqq \max \{N_1, N_2\}$ ist daher
 \[
  |x-y|
  = |x-x_n+x_n-y|
  \leq |x - x_n| + |y - x_n|
  < \varepsilon + \varepsilon
  = \frac{2}{3}|x-y|,
 \]
 was $|x-y| > 0$ widerspricht. Also muss bereits $x = y$.
 
 Konvergiert eine $\Kbb$-wertige Folge $(x_n)_{n \in \Nbb}$ gegen ein Element $x \in \Kbb$, so schreiben wir auch $\lim_{n \to \infty} x_n \coloneqq x$.
\end{bem}


\begin{bem}\label{bem: properties of convergent sequences}.
 \begin{enumerate}[leftmargin=*]
  \item
   Ist $(x_n)_{n \in \Nbb} \in \ell(\Kbb)$ eine konstante Folge, d.h.\ es gibt $c \in \Kbb$ mit $x_n = c$ für alle $n \in \Nbb$, so konvergiert $(x_n)_{n \in \Nbb}$ und $\lim_{n \to \infty} x_n = c$. Für beliebige $\varepsilon > 0$ ist nämlich $|c-x_n| = 0 < \varepsilon$ für alle $n \geq 0$.
  \item
   Sind $(x_n)_{n \in \Nbb}, (y_n)_{n \in \Nbb} \in \ell(\Kbb)$ konvergente Folgen, so konvergiert auch die Folge $(x_n + y_n)_{n \in \Nbb}$ und es gilt
   \[
    \lim_{n \to \infty} (x_n + y_n)
    = \left( \lim_{n \to \infty} x_n \right) + \left( \lim_{n \to \infty} y_n \right).
   \]
   Ist nämlich $x \coloneqq \lim_{n \to \infty} x_n$, $y \coloneqq \lim_{n \to \infty} y_n$ und $\varepsilon > 0$, so gibt es $N_x, N_y \in \Nbb$ mit $|x-x_n| < \varepsilon/2$ für alle $n \geq N_x$ und $|y-y_n| < \varepsilon/2$ für alle $n \geq N_y$, weshalb für alle $n \geq N \coloneqq \max\{N_x, N_y\}$ auch
   \[
    |(x+y)-(x_n+y_n)|
    = |(x-x_n) + (y-y_n)|
    \leq |x - x_n| + |y - y_n|
    < \frac{\varepsilon}{2} + \frac{\varepsilon}{2}
    = \varepsilon.
   \]
  \item
   Ist $(x_n)_{n \in \Nbb} \in \ell(\Kbb)$ konvergent und $\lambda \in \Kbb$, so ist auch die Folge $(\lambda x_n)_{n \in \Nbb}$ konvergent und $\lim_{n \to \infty} (\lambda x_n) = \lambda \lim_{n \to \infty} x_n$: Es sei $x \coloneqq \lim_{n \to \infty} x_n$. Ist $\lambda = 0$, so ist $(\lambda x_n)_{n \in \Nbb}$ die konstante Nullfolge und somit
   \[
    \lim_{n \to \infty} \lambda x_n
    = \lim_{n \to \infty} 0
    = 0
    = 0 \cdot \lim_{n \to \infty} x_n.
   \]
   Ist $\lambda \neq 0$ und $\varepsilon > 0$, so gibt es $N \in \Nbb$ mit $|x-x_n| < \varepsilon/|\lambda|$ für alle $n \geq N$. Für alle $n \geq N$ ist daher auch
   \[
    |\lambda x - \lambda x_n|
    = |\lambda| |x-x_n|
    < |\lambda| \frac{\varepsilon}{|\lambda|}
    = \varepsilon.
   \]
 \end{enumerate}
\end{bem}


\begin{defi}
 Eine Folge $(x_n)_{n \in \Nbb} \in \ell(\Kbb)$ heißt \emph{Cauchy-Folge}, falls es für jedes $\varepsilon > 0$ ein $N \in \Nbb$ gibt, so dass $|x_n - x_m| < \varepsilon$ für alle $n,m \geq N$.
\end{defi}


\begin{bem}
 \begin{enumerate}[leftmargin=*]
  \item
   Ist $(x_n)_{n \in \Nbb} \in \ell(\Kbb)$ konvergent, so ist $(x_n)_{n \in \Nbb}$ auch eine Cauchy-Folge: Ist nämlich $\varepsilon > 0$ und $x \coloneqq \lim_{n \to \infty} x_n$, so gibt es $N \in \Nbb$ mit $|x - x_n| < \varepsilon/2$ für alle $n \geq N$, weshalb für alle $n,m \geq N$ auch
   \[
    |x_n - x_m|
    = |x_n - x + x - x_m|
    \leq |x_n - x| + |x - x_m|
    \leq \frac{\varepsilon}{2} + \frac{\varepsilon}{2}
    = \varepsilon.
   \]
  \item
   Für die reellen Zahlen $\Rbb$ und komplexen Zahlen $\Cbb$ gilt auch die Umkehrung: Ist $(x_n)_{n \in \Nbb}$ eine Cauchy-Folge reeller oder komplexer Zahlen, also $(x_n)_{n \in \Nbb} \in \ell(\Rbb)$ oder $(x_n)_{n \in \Nbb} \in \ell(\Cbb)$, so ist $(x_n)_{n \in \Nbb}$ auch konvergent. Man sagt, dass $\Rbb$ und $\Cbb$ \emph{vollständig} sind. (Wir werden dies hier nicht zeigen. Für die reellen Zahlen ist der Beweis abhängig davon, wie diese konstruiert werden. Die Vollständigkeit der komplexen Zahlen lässt sich dann aus der Vollständigkeit der reellen Zahlen folgern.)
  \item
   Für die rationalen Zahlen $\Qbb$ gilt diese Aussage nicht: Es gibt Cauchy-Folgen von rationalen Zahlen, die in $\Qbb$ keinen Grenzwert besitzen. $\Qbb$ ist also nicht vollständig. (Auch diese Aussage werden wir hier nicht zeigen.)
 \end{enumerate}
\end{bem}


\begin{defi}
 Für eine Folge $(a_n)_{n \in \Nbb} \in \ell(\Kbb)$ und $m \in \Nbb$ ist die $m$-te \emph{Partialsumme} der Folge $(a_n)_{n \in \Nbb}$ als $\sum_{n=0}^m a_n$ definiert. Die Folge $(a_n)_{n \in \Nbb}$ heißt \emph{summierbar}, falls die Folge der Partialsummen $(\sum_{n=0}^m a_n)_{m \in \Nbb}$ konvergiert. Es ist dann $\sum_{n=0}^\infty a_n \coloneqq \lim_{m \to \infty} \sum_{n=0}^m a_n$.
\end{defi}


\begin{bem}\label{bem: properties of convergent series}
 \begin{enumerate}[leftmargin=*]
  \item
   Die konstante Nullfolge $(0)_{n \in \Nbb}$ ist summierbar mit $\sum_{n=0}^\infty a_n = 0$. Für alle $m \in \Nbb$ ist nämlich $\sum_{n=0}^m 0 = 0$, also $(\sum_{n=0}^m a_n)_{m \in \Nbb} = 0$, die Folge der Partialsummen also ebenfalls die Nullfolge. Daher konvergiert die Folge der Partialsummen gegen $0$, also $\lim_{m \to \infty} \sum_{n=0}^m 0 = 0$. Dies bedeutet gerade, dass die Nullfolge $(0)_{n \in \Nbb}$ summierbar ist, und dass $\sum_{n=0}^\infty 0 = 0$.
  \item
   Sind $(a_n)_{n \in \Nbb}, (b_n)_{n \in \Nbb} \in \ell(\Kbb)$ summierbare Folgen, so ist auch die Folge $(a_n + b_n)_{n \in \Nbb}$ summierbar und $\sum_{n=0}^\infty (a_n+b_n) = (\sum_{n=0}^\infty a_n) + (\sum_{n=0}^\infty b_n)$.
   
   Ist nämlich $S_m \coloneqq \sum_{n=0}^m a_n$ und $T_n \coloneqq \sum_{n=0}^m b_n$ für alle $m \in \Nbb$, so bedeutet die Summierbarkeit von $(a_n)_{n \in \Nbb}$ und $(b_n)_{n \in \Nbb}$, dass die Folgen $(S_m)_{m \in \Nbb}$ und $(T_m)_{m \in \Nbb}$ konvergieren, und dass $S \coloneqq \lim_{m \to \infty} S_m = \sum_{n=0}^\infty a_n$ und $T \coloneqq \lim_{m \to \infty} T_m = \sum_{n=0}^\infty b_n$. Da $(S_m)_{m \in \Nbb}$ und $(T_m)_{m \in \Nbb}$ konvergieren, konvergiert auch die Folge $(S_m + T_m)_{m \in \Nbb}$ und $\lim_{m \to \infty} (S_m + T_m) = S + T$.
   
   Für alle $m \in \Nbb$ ist nun
   \begin{gather*}
    S_m + T_m
    = \left( \sum_{n=0}^m a_n \right) + \left( \sum_{n=0}^m b_n \right)
    = \sum_{n=0}^m (a_n + b_n)
   \shortintertext{und es ist}
    \left( \sum_{n=0}^\infty a_n \right) + \left( \sum_{n=0}^\infty b_n \right)
    = S + T
    = \lim_{m \to \infty} (S_m + T_m)
    = \lim_{m \to \infty} \sum_{n=0}^m (a_n + b_n).
   \end{gather*}
   Das zeigt, dass die Folge der Partialsummen $(\sum_{n=0}^m (a_n+b_n))_{m \in \Nbb}$ konvergiert, und dass $\lim_{m \to \infty} \sum_{n=0}^m (a_n+b_n) = (\sum_{n=0}^\infty a_n) + (\sum_{n=0}^\infty b_n)$. Dies bedeutet gerade, dass die Folge $(a_n + b_n)_{n \in \Nbb}$ summierbar ist, und dass $\sum_{n=0}^\infty (a_n+b_n) = (\sum_{n=0}^\infty a_n) + (\sum_{n=0}^\infty b_n)$.
  \item
   Ist $(a_n)_{n \in \Nbb} \in \ell(\Kbb)$ summierbar und $\lambda \in \Kbb$, so ist auch die Folge $(\lambda a_n)_{n \in \Nbb}$ summierbar und $\sum_{n=0}^\infty (\lambda a_n) = \lambda \sum_{n=0}^\infty a_n$.
   
   Ist $S_m \coloneqq \sum_{n=0}^m a_n$ die $m$-te Partialsumme der Folge $(a_n)_{n \in \Nbb}$, so bedeutet die Summierbarkeit der Folge $(a_n)_{n \in \Nbb}$, dass die Folge der Partialsummen $(S_m)_{m \in \Nbb}$ konvergiert, und dass $S \coloneqq \lim_{m \to \infty} S_m = \sum_{n=0}^\infty a_n$. Da die Folge $(S_m)_{m \in \Nbb}$ konvergiert, konvergiert auch die Folge $(\lambda S_m)_{m \in \Nbb}$ und es gilt $\lim_{m \to \infty} \lambda S_m = \lambda S$.
   
   Für alle $m \in \Nbb$ ist nun
   \begin{gather*}
    \lambda S_m
    = \lambda \sum_{n=0}^m a_n
    = \sum_{n=0}^m (\lambda a_n),
   \shortintertext{und es gilt}
    \lambda \sum_{n=0}^\infty a_n
    = \lambda \lim_{m \to \infty} S_m
    = \lim_{m \to \infty} \lambda S_m
    = \lim_{m \to \infty} \lambda \sum_{n=0}^m a_n
    = \lim_{m \to \infty} \sum_{n=0}^m (\lambda a_n).
   \end{gather*}
   Die Konvergenz der Folge $(\lambda S_m)_{m \in \Nbb}$ bedeutet also genau die Summierbarkeit der Folge $(\lambda a_n)_{n \in \Nbb}$, und es ist $\sum_{n=0}^\infty (\lambda a_n) = \lim_{m \to \infty} \sum_{n=0}^m (\lambda a_n) = \lambda \sum_{n=0}^\infty a_n$.
 \end{enumerate}
\end{bem}



\begin{defi}
 Eine Folge $(a_n)_{n \in \Nbb} \in \ell(\Kbb)$ heißt \emph{absolut summierbar}, falls die Folge $(|a_n|)_{n \in \Nbb}$ summierbar ist.
\end{defi}


\begin{bem}
 Ist $(a_n)_{n \in \Nbb}$ eine absolut summierbare Folge reeller oder komplexer Zahlen, so ist $(a_n)_{n \in \Nbb}$ auch summierbar: Dass die Folge $(a_n)_{n \in \Nbb}$ absolut summierbar ist, bedeutet, dass die Folge $(|a_n|)_{n \in \Nbb}$ summierbar ist. Das bedeutet, dass die Folge der Partialsummen $(\sum_{n=0}^m |a_n|)_{m \in \Nbb}$ konvergiert. Also ist die Folge der Partialsummen $(\sum_{n=0}^m |a_n|)_{m \in \Nbb}$ eine Cauchy-Folge.
 
 Es sei $\varepsilon > 0$. Da die Folge der Partialsummen $(\sum_{n=0}^m |a_n|)_{m \in \Nbb}$ eine Cauchy-Folge ist, gibt es $N \in \Nbb$, so dass für alle $l \geq m \geq N$
 \[
  \sum_{n=m+1}^l |a_n|
  = \left| \sum_{n=0}^l |a_n| - \sum_{n=0}^m |a_n| \right|
  < \varepsilon.
 \]
 Für alle $l \geq m \leq N$ ist deshalb auch
 \[
  \left| \sum_{n=0}^l a_n - \sum_{n=0}^m a_n \right|
  = \left| \sum_{n=m+1}^l a_n \right|
  \leq \sum_{n=m+1}^l |a_n|
  < \varepsilon.
 \]
 Deshalb ist die Folge der Partialsummen $(\sum_{n=0}^m a_n)_{m \in \Nbb}$ eine Cauchy-Folge. Da $\Rbb$ und $\Cbb$ vollständig sind, ist die Folge der Partialsummen $(\sum_{n=0}^m a_n)_{m \in \Nbb}$ deshalb bereits konvergent. Dies bedeutet, dass die Folge $(a_n)_{n \in \Nbb}$ summierbar ist.
\end{bem}


\begin{bem}
 Ist die Folge $(a_n)_{n \in \Nbb} \in \ell(\Kbb)$ summierbar, so ist $(a_n)_{n \in \Nbb}$ eine Nullfolge, d.h.\ $(a_n)_{n \in \Nbb}$ ist konvergent und $\lim_{n \to \infty} a_n = 0$: Für alle $m \in \Nbb$ sei $S_m \coloneqq \sum_{n=0}^m a_n$ die $m$-te Partialsumme von $(a_n)_{n \in \Nbb}$. Dass $(a_n)_{n \in \Nbb}$ summierbar ist, bedeutet, dass die Folge $(S_m)_{m \in \Nbb}$ konvergiert. Es konvergiert daher die Folge $(S_{m+1})_{m \in \Nbb}$ mit $\lim_{m \to \infty} S_{m+1} = \lim_{m \to \infty} S_m$. Daher konvergiert auch die Folge $(S_{m+1} - S_m)_{m \in \Nbb}$ mit
 \[
  \lim_{m \to \infty} (S_{m+1} - S_m)
  = \lim_{m \to \infty} S_{m+1} - \lim_{m \to \infty} S_m
  = \lim_{m \to \infty} S_m - \lim_{m \to \infty} S_m
  = 0.
 \]
 Für alle $m \in \Nbb$ ist jedoch
 \[
  S_{m+1} - S_m
  = \sum_{n=0}^{m+1} a_n - \sum_{n=0}^m a_n
  = a_{m+1},
 \]
 also ist $0 = \lim_{m \to \infty} (S_{m+1} - S_m) = \lim_{m \to \infty} a_{m+1} = \lim_{m \to \infty} a_m$.
\end{bem}