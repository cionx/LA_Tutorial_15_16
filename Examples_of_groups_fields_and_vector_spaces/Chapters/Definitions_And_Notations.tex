\chapter{Definitionen und Notationen}





\section{Gruppen}


\begin{defi}
 Eine \emph{Gruppe} ist ein Paar $(G,\cdot)$ bestehend aus einer Menge $G$ und einer binären Verknüpfung $\cdot \colon G \times G \to G$, $(g, h) \mapsto g \cdot h$, die die folgenden Bedingungen erfüllt:
 \begin{enumerate}[label=\roman*)]
  \item
   Die Verknüpfung $\cdot$ ist assoziativ, d.h.\ für alle $g_1, g_2, g_3 \in G$ ist $(g_1 \cdot g_2) \cdot g_3 = g_1 \cdot (g_2 \cdot g_3)$.
  \item
   Es gibt ein \emph{neutrales Element} $e \in G$, d.h.\ $e \cdot g = g \cdot e = g$ für alle $g \in G$.
  \item
   Für jedes Element $g \in G$ gibt es ein \emph{inverses Element} $h \in G$, d.h.\ $g \cdot h = h \cdot g = e$.
 \end{enumerate}
 Die Gruppe $(G, \cdot)$ heißt \emph{abelsch}\footnote{Benannt nach dem norwegischen Mathematiker Niels Henrik Abel.}, bzw. \emph{kommutativ}, falls $g \cdot h = h \cdot g$ für alle $g,h \in G$.
\end{defi}


\begin{bem} Es sei $(G, \cdot)$ eine Gruppe.
 \begin{enumerate}[leftmargin=*]
  \item
   Wegen der Assoziativität von $\cdot$ ist das Produkt $g_1 \dotsm g_n$ für alle $n \geq 1$ und $g_1, \dotsc, g_n \in G$ wohldefiniert.
  \item
   Je zwei neutrale Element $e, e' \in G$ sind gleich, da $e = e \cdot e' = e'$. Man spricht daher von \emph{dem} neutralen Element von $G$.
  \item
   Für jedes $g \in G$ sind je zwei inverse Elemente $h, h'$ von $g$ gleich, da
   \[
    h = h \cdot e = h \cdot g \cdot h' = e \cdot h' = h'.
   \]
   Man spricht daher von \emph{dem} inversen Element von $g$ und schreibt $g^{-1}$ für dieses.
  \item
   Es genügt zu fordern, dass es ein linksneutrales Element $e \in G$ gibt, d.h.\ dass $e \cdot g = g$ für alle $g \in G$, und dass jedes $g \in G$ ein linksinverses Element $h \in G$ gibt, d.h.\ dass $h \cdot g = e$. Es folgt dann, dann $e$ auch rechtsneutral ist (d.h.\ $g \cdot e = g$ für alle $g \in G$) und ein linksinverses Element $h \in G$ von $g$ auch schon rechtsinvers ist (d.h.\ $g \cdot h = e$).
  \item
   Für jedes $g \in G$ ist $g^{-1} \cdot g = g \cdot g^{-1} = e$ also $g$ das Inverse zu $g^{-1}$, und somit $g = (g^{-1})^{-1}$.
  \item
   Für alle $g, h \in G$ ist $(h^{-1} \cdot g^{-1}) \cdot (g \cdot h) = h^{-1} \cdot g^{-1} \cdot g \cdot h = h^{-1} \cdot e \cdot h = h^{-1} \cdot h = e$, also $h^{-1} \cdot g^{-1}$ das (Links)inverse zu $g \cdot h$ und somit $h^{-1} \cdot g^{-1} = (g \cdot h)^{-1}$.
  \item
   Ist $(G, \cdot)$ abelsch, so schreibt man die Verknüpfung häufig nicht \emph{multiplikativ} (d.h.\ als $\cdot$), sondern \emph{additiv} (d.h.\ als $+$).
  \item
   Schreibt man eine Gruppe $(G,\cdot)$ multiplikativ, so bezeichnet man das neutrale Element häufig mit $1$ statt $e$. Man schreibt auch $gh$ statt $g \cdot h$. Für alle $g \in G$ und $n \in \Zbb$ schreibt man abkürzend
   \[
    g^n \coloneqq
    \begin{cases}
     \underbrace{g \dotsm g}_{\text{$n$ viele}} & \text{falls $n \geq 1$} \\
     1                                          & \text{falls $n = 0$} \\
     (g^-1)^{-n}                                & \text{falls $n \leq -1$}.
    \end{cases}
   \]
   Es gilt dann $g^n \cdot g^m = g^{n+m}$ für alle $g \in G$ und $n,m \in \Zbb$, sowie $g^0 = 1$ und auch  $(g^n)^{-1} = (g^{-1})^n = g^{-n}$ für alle $g \in G$ und $n \in \Zbb$.
  \item
   Schreibt man eine abelsche Gruppe $(A,+)$ additiv, so bezeichnet man das neutrale Element mit $0$ und das neutrale Element von $a \in A$ als $-a$. Für alle $a \in A$ und $n \in \Zbb$ schreibt man dann abkürzend
   \[
    n a \coloneqq
    n \cdot a \coloneqq
    \begin{cases}
     \underbrace{a + \dotsb + a}_{\text{$n$ viele}} & \text{falls $n \geq 1$} \\
     0                                              & \text{falls $n = 0$} \\
     (-n) \cdot (-a)                                & \text{falls $n \leq -1$}.
    \end{cases}
   \]
   Es gilt dann $na + ma = (n+m)a$ für alle $a \in A$ und $n,m \in \Zbb$, sowie $0 \cdot a = 0$ und $-(n \cdot a) = n \cdot (-a) = (-n) \cdot a$ für alle $a \in A$ und $n \in \Zbb$.
  \item
   Obwohl eine Gruppe, per Definition, ein Paar $(G, \cdot)$ ist, nennt man die Verknüpfung $\cdot$ häufig nicht explizit. Statt von einer Gruppe $(G, \cdot)$, bzw.\ $(G, +)$ spricht man also nur von einer Gruppe $G$.
 \end{enumerate}
\end{bem}


\begin{defi}
 Es sei $G$ eine Gruppe. Eine Teilmenge $H \subseteq G$ heißt \emph{Untergruppe} falls die folgenden Bedingungen erfüllt sind:
 \begin{enumerate}[label=\roman*)]
  \item
   $e \in H$ für das neutrale Element $e \in G$.
  \item
   Für alle $h_1, h_2 \in H$ ist auch $h_1 h_2 \in H$.
  \item
   Für alle $h \in H$ ist auch $h^{-1} \in H$.
 \end{enumerate}
\end{defi}


\begin{bem}
 Es sei $G$ eine Gruppe.
 \begin{enumerate}[leftmargin=*]
  \item
   Statt zu fordern, dass $e \in H$, genügt es zu fordern, dass $H \neq 0$, dass es also irgendein $h \in G$ gibt mit $h \in H$. Dann ist nämlich auch $h^{-1} \in H$ und deshalb auch $e = h \cdot h^{-1} \in H$.
  \item
   Ist $H \subseteq G$ eine Untergruppe, so ist auch $(H, \cdot)$ eine Gruppe. Das neutrale Element in $H$ stimmt mit dem neutralen Element in $G$ überein und für jedes $h \in H$ entspricht das inverse Element in $H$ dem inversen Element in $G$.
  \item
   Ist $G$ abelsch, so ist auch $H$ abelsch.
  \item
   Ist $H \subseteq G$ eine Untergruppe und $K \subseteq H$ eine Untergruppe (der Gruppe $(H,\cdot)$), so ist auch $K \subseteq G$ eine Untergruppe.
  \item
   Eine Teilmenge $H \subseteq G$ ist genau dann eine Untergruppe, wenn $H \neq \emptyset$ (bzw.\ äquivalent $e \in H$) und $h_1 h_2^{-1} \in H$ für alle $h_1, h_2 \in H$.
 \end{enumerate}
\end{bem}


\begin{defi}
 Es seien $G$ und $H$ zwei Gruppen. Eine Abbildung $\varphi \colon G \to H$ heißt \emph{Gruppenhomomorphismus}, falls 
 \[
  \varphi(xy) = \varphi(x)\varphi(y) \quad \text{für alle $x,y \in G$}.
 \]
 Ist $\varphi$ injektiv, so heißt $\varphi$ \emph{Gruppenmonomorphismus}; ist $\varphi$ surjektiv, so heißt $\varphi$ \emph{Gruppenepimorphismus}; und ist $\varphi$ bijektiv, so heißt $\varphi$ \emph{Gruppenisomorphismus}. Ist $G = H$, so heißt $\varphi$ \emph{Gruppenendomorphismus}; ist $\varphi$ zusätzlich bijektiv, so heißt $\varphi$ Gruppenautomorphismus.
\end{defi}


\begin{center}
 \begin{tabular}{l|l|l}
             & allgemein & $G = H$ \\
   \hline
   allgemein & homo      & endo    \\
   injektiv  & mono      &         \\
   surjektiv & epi       &         \\
   bijektiv  & iso       & auto
 \end{tabular}
\end{center}


\begin{bem}
 \begin{enumerate}[leftmargin=*]
  \item
   Für jede Gruppe $G$ ist $\id_G \colon G \to G$ ein ein Gruppenautomorphismus.
  \item
   Sind $G_1$, $G_2$ und $G_3$ Gruppen und $\varphi \colon G_1 \to G_2$ und $\psi \colon G_2 \to G_3$ Gruppenhomomorphismen, so ist auch die Verknüpfung $\psi \circ \varphi$ ein Gruppenhomomorphismus, denn für alle $x,y \in G_1$ ist
   \begin{align*}
    (\psi \circ \varphi)(x \dot y)
    = \psi(\varphi(x \cdot y))
    &= \psi(\varphi(x) \cdot \varphi(y)) \\
    &= \psi(\varphi(x)) \cdot \psi(\varphi(y))
    = (\psi \circ \varphi)(x) \cdot (\psi \circ \varphi)(y).
   \end{align*}
  \item
   Sind $G$ und $H$ Gruppen und $\varphi \colon G \to H$ ein Gruppenhomomorphismus, so ist $\varphi(1) = 1$, denn
   \[
    \varphi(1)
    = \varphi(1 \cdot 1)
    = \varphi(1) \cdot \varphi(1)
   \]
   und durch Multiplikation mit $\varphi(1)$ ergibt sich, dass $1 = \varphi(1)$.
  \item
   Sind $G$ und $H$ Gruppen und $\varphi \colon G \to H$ ein Gruppenhomomorphismus, so ist $\varphi(g^{-1}) = \varphi(g)^{-1}$ für alle $g \in G$, da
   \[
    \varphi(g^{-1}) \cdot \varphi(g)
    = \varphi(g^{-1} \cdot g)
    = \varphi(1)
    = 1
   \]
   und $\varphi(g^{-1})$ somit das inverse Element zu $\varphi(g)$ ist.
  \item
   Sind $G$ und $H$ Gruppen und ist $\varphi \colon G \to H$ ein Gruppenisomorphismus (also ein Gruppenhomomorphismus und bijektiv), so ist auch $\varphi^{-1} \colon H \to G$ ein Gruppenisomorphismus, da für alle $x,y \in H$
   \begin{align*}
    \varphi^{-1}(x \cdot y)
    &= \varphi^{-1}( \varphi(\varphi^{-1}(x)) \cdot \varphi(\varphi^{-1}(y)) ) \\
    &= \varphi^{-1}( \varphi( \varphi^{-1}(x) \cdot \varphi^{-1}(y) ) )
    = \varphi^{-1}(x) \cdot \varphi^{-1}(y).
   \end{align*}
  \item
   Sind $G_1$ und $G_2$ zwei Gruppen, $\varphi \colon G_1 \to G_2$ ein Gruppenhomomorphismus und $H \subseteq G_1$ eine Untergruppe, so ist auch die Einschränkung
   \[
    \varphi|_{H} \colon H \to G_2, h \mapsto \varphi(h)
   \]
   ein Gruppenhomomorphismus, denn ist $\varphi(xy) = \varphi(x)\varphi(y)$ für alle $x,y \in G_1$, so gilt dies insbesondere für alle $x,y \in H$.
 \end{enumerate}
\end{bem}





\section{Körper}


\begin{defi}
 Ein Körper ist ein Tupel $(K, +, \cdot)$ bestehend aus einer Menge $K$ und zwei binären Verknüpfungen, einer \emph{Addition} $+ \colon K \times K \to K$, $(x,y) \mapsto x + y$ und einer \emph{Multiplikation} $\cdot \colon K \to K$, $(x,y) \mapsto xy$, so dass die folgenden Bedingungen erfüllt sind:
 \begin{enumerate}[label=\roman*)]
  \item
   $(K,+)$ ist eine abelsche Gruppe.
  \item
   Die Multiplikation $\cdot$ erfüllt die folgenden Bedingungen:
   \begin{enumerate}[label=\alph*)]
    \item
     $\cdot$ ist assoziativ.
    \item
     $\cdot$ ist kommutativ.
    \item
     Es gibt ein Einselement $1 \in K$, d.h.\ $1 \cdot x = x$ für alle $x \in K$, und es gilt $1 \neq 0$.
    \item
     Für jedes Element $x \in K$ mit $x \neq 0$ gibt es ein multiplikativ Inverses Element $y \in K$, d.h.\ $x \cdot y = 1$.
   \end{enumerate}
  \item
   Es gilt die Distributivität von $+$ und $\cdot$, d.h.\ für alle $x,y,z \in K$ ist
   \[
    x \cdot (y + z)
    = (x \cdot y) + (x \cdot z)
   \]
 \end{enumerate}
\end{defi}


\begin{bem}
 \begin{enumerate}[leftmargin=*]
  \item
   Die zweite Bedingung in der Definition lässt sich dadurch ersetzen, dass $(K\setminus\{0\},\cdot)$ eine abelsche Gruppe ist. Dann muss bei der Distributivität allerdings zusätzlich gefordert werden, dass auch $(x+y) \cdot z = (x \cdot z) + (y \cdot z)$ für alle $x,y,z \in K$.
  \item
   Die hier angegebene Definition hat den Vorteil, dass sie kürzer zu überprüfen ist.
  \item
   In der hier angegebenen Definition fordert man, dass $1 \neq 0$, da die Menge $\{0\}$ zusammen mit der Addition $0 + 0 = 0$ und $0 \cdot 0 = 0$ sonst einen Körper bilden würde (was man nicht möchte).
 \end{enumerate}
\end{bem}


\begin{bem}
 \begin{enumerate}[leftmargin=*]
  \item
   Man bezeichnet einen Körper meistens nur mit $K$ statt mit $(K,+,\cdot)$, d.h.\ man nennt die Addition und Multiplikation nicht explizit. Das neutrale Element bezüglich der Addition wird mit $0$ bezeichnet, das neutrale Element der Multiplikation mit $1$.
  \item
   Man rechnet „Punkt vor Strich“, d.h.\ für alle $x,y,z \in K$ ist
   \[
    x + y \cdot z \coloneqq x + (y \cdot z).
   \]
  \item
   In einem Körper $K$ gelten die gewöhnlichen Rechenregeln, beispielsweise die folgenden:
   \begin{enumerate}[label=\alph*)]
    \item
     Für alle $x \in K$ ist $0 \cdot x = 0$, denn
     \[
      0 \cdot x = (0+0) \cdot x = (0 \cdot x) + (0 \cdot x),
     \]
     woraus sich die Gleichung durch Addition mit $-(0 \cdot x)$ ergibt. Da $1 \neq 0$ ist $0$ insbesondere nicht multiplikativ invertierbar.
    \item
     Für alle $x \in K$ ist $-x = (-1) \cdot x$, da
     \[
      x + (-1) \cdot x
      = 1 \cdot x + (-1) \cdot x
      = (1 + (-1)) \cdot x
      = 0 \cdot x
      = 0.
     \]
    \item
     Für alle $x,y \neq 0$ ist auch $x \cdot y \neq 0$ mit $(x \cdot y)^{-1} = x^{-1} \cdot y^{-1}$. Es ist nämlich
     \[
      (x \cdot y) \cdot (x^{-1} \cdot y^{-1})
      = x \cdot x^{-1} \cdot y \cdot y^{-1}
      = 1 \cdot 1
      = 1,
     \]
     und $0$ ist nicht multiplikativ invertierbar.
    \item
     Für alle $x \neq 0$ ist $(-x)^{-1} = -(x^{-1})$. Da $(-1) \cdot (-1) = -(-1) = 1$ ist nämlich $-1 = (-1)^{-1}$ und somit
     \begin{align*}
      x^{-1} + (-x)^{-1})
      &= x^{-1} + ((-1) \cdot x)^{-1}
      = x^{-1} + (-1)^{-1} \cdot x^{-1} \\
      &= x^{-1} + (-1) \cdot x^{-1}
      = x^{-1} - x^{-1}
      = 0.
     \end{align*}
   \end{enumerate}
  \item
   Man bezeichnet die Gruppe $(K,+)$ als die \emph{additive Gruppe} von $K$. Man schreibt abkürzend $K^\times \coloneqq K\setminus\{0\}$; aus den obigen Rechenregeln folgt, dass $K^\times$ zusammen mit der Multiplikation des Körpers eine Gruppe bildet. Diese Gruppe $(K^\times,\cdot)$ wird als die \emph{multiplikative Gruppe} von $K$ bezeichnet.
  \item
   In der Definition eines Körpers lässt sich der Begriff einer Gruppe umgehen, indem man die darin versteckten Bedingungen alle explizit angibt: Ein Körper ist dann definiert als ein Tupel $(K,+,\cdot)$ bestehend aus einer Menge $K$ und zwei binären Verknüpfungen $+$ und $\cdot$ auf $K$, so dass die folgenden Bedingungen erfüllt sind:
   \begin{enumerate}[label=\roman*)]
    \item
     $+$ ist assoziativ, d.h.\ $x+(y+z) = (x+y)+z$ für alle $x,y,z \in K$.
    \item
     $+$ ist kommutativ, d.h.\ $x+y = y+x$ für alle $x,y \in K$.
    \item
     Es gibt ein additiv neutrales Element $0 \in K$, d.h.\ $x + 0 = x$ für alle $x \in K$.
    \item
     Es gibt für jedes $x \in K$ ein additiv inverses Element $y \in K$, d.h.\ $x+y = 0$.
    \item
     $\cdot$ ist assoziativ, d.h.\ $x \cdot (y \cdot z) = (x \cdot y) \cdot z$ für alle $x,y,z \in K$.
    \item
     $\cdot$ ist kommutativ, d.h.\ für alle $x,y \in K$ ist $x \cdot y = y \cdot x$.
    \item
     Es gibt ein multiplikatives neutrales Element $1 \in K\setminus\{0\}$, d.h.\ $x \cdot 1 = x$ für alle $x \in K$, und $1 \neq 0$.
    \item
     Für jedes $x \in K\setminus\{0\}$ gibt es ein multiplikativ inverses Element $y \in K$, d.h.\ $x \cdot y = 1$.
    \item
     $+$ ist distributiv bezüglich $\cdot$, d.h.\ $x \cdot (y+z) = (x \cdot y) + (x \cdot z)$ für alle $x,y,z \in K$.
   \end{enumerate}
 \end{enumerate}
\end{bem}


\begin{defi}
 Es sei $L$ ein Körper. Eine Teilmenge $K \subseteq L$ heißt \emph{Unterkörper}, falls die folgenden Bedingungen erfüllt sind:
 \begin{enumerate}[label=\roman*)]
  \item
   $K$ ist eine Untergruppe der additiven Gruppe von $L$.
  \item
   Es ist $1 \in K$, für alle $x,y \in K$ ist $x \cdot y \in K$, und für jedes $x \in K$ mit $x \neq 0$ ist auch $x^{-1} \in K$.
 \end{enumerate}
\end{defi}


\begin{bem}
 \begin{enumerate}[leftmargin=*]
  \item
   Ist $L$ ein Körper und $K \subseteq L$ ein Unterkörper, so ist $(K,+,\cdot)$ ebenfalls ein Körper. Es gilt $0_K = 0_L$ und $1_K = 1_L$ und für jedes $x \in K^\times$ stimmen die inversen Element in $K$ und $L$ überein.
  \item
   Ist $L$ ein Körper und $K \subseteq L$ ein Unterkörper, so ist $K \subseteq L$ eine additive Untergruppe und $K^\times \subseteq L^\times$ eine multiplikative Untergruppe.
  \item
   Ist $L$ ein Körper, $K \subseteq L$ ein Unterkörper und $K' \subseteq K$ ein Unterkörper (von $K$), so ist $K' \subseteq L$ ein Unterkörper.
 \end{enumerate}
\end{bem}


\begin{defi}
 Es seien $K$ und $L$ zwei Körper. Eine Abbildung $\phi \colon K \to L$ heißt \emph{Körperhomomorphismus}, falls
 \begin{enumerate}[label=\roman*)]
  \item
   $\phi(x+y) = \phi(x)+\phi(y)$ für alle $x,y \in K$
  \item
   $\phi(xy) = \phi(x)\phi(y)$ für alle $x,y \in K$, und
  \item
   $\phi(1) = 1$.
 \end{enumerate}
 Ist $\phi$ bijektiv, so ist $\phi$ ein \emph{Körperisomorphismus}. Ist $K = L$ und $\phi$ bijektiv, so ist $\phi$ ein \emph{Körperautomorphismus}.
\end{defi}


\begin{bem}
 \begin{enumerate}[leftmargin=*]
  \item
   Für jeden Körper $K$ ist die Identität $\id_K \colon K \to K$ ein Körperautomorphismus.
  \item
   Sind $K_1$, $K_2$ und $K_3$ Körper und $\phi \colon K_1 \to K_2$ und $\psi \colon K_2 \to K_3$ Körperhomomorphismen, so ist auch $\psi \circ \phi \colon K_1 \to K_3$ ein Körperhomomorphismus: Für alle $x,y \in K_1$ ist
   \begin{gather*}
    \begin{aligned}
     (\psi \circ \phi)(x+y)
     = \psi(\phi(x+y))
     &= \psi(\phi(x)+\phi(y)) \\
     &= \psi(\phi(x)) + \psi(\phi(y))
     = (\psi \circ \phi)(x) + (\psi \circ \phi)(y)
    \end{aligned}
   \shortintertext{und}
    \begin{aligned}
     (\psi \circ \phi)(x \cdot y)
     = \psi(\phi(x \cdot y))
     &= \psi(\phi(x) \cdot \phi(y)) \\
     &= \psi(\phi(x)) \cdot \psi(\phi(y))
     = (\psi \circ \phi)(x) \cdot (\psi \circ \phi)(y),
    \end{aligned}
   \shortintertext{und es gilt}
    (\psi \circ \phi)(1)
    = \psi(\phi(1))
    = \psi(1)
    = 1.
   \end{gather*}
  \item
   Sind $K$ und $L$ Körper und ist $\phi \colon K \to L$ ein Körperisomorphismus, so ist auch die Umkehrabbildung $\phi^{-1} \colon L \to K$ ein Körperisomorphismus: $\phi$ ist bijektiv und für alle $x,y \in L$ ist
   \begin{gather*}
    \begin{aligned}
     \phi^{-1}(x+y)
     &= \phi^{-1}(\phi(\phi^{-1}(x)) + \phi(\phi^{-1}(y))) \\
     &= \phi^{-1}(\phi( \phi^{-1}(x) + \phi^{-1}(y) )
     = \phi^{-1}(x) + \phi^{-1}(y)
    \end{aligned}
   \shortintertext{und}
    \begin{aligned}
     \phi^{-1}(x \cdot y)
     &= \phi^{-1}(\phi(\phi^{-1}(x)) \cdot \phi(\phi^{-1}(y))) \\
     &= \phi^{-1}(\phi( \phi^{-1}(x) \cdot \phi^{-1}(y) ))
     = \phi^{-1}(x) \cdot \phi^{-1}(y),
    \end{aligned}
   \shortintertext{sowie auch}
    \phi^{-1}(1) = \phi^{-1}(\phi(1)) = 1.
   \end{gather*}
  \item
   Sind $K$ und $L$ Körper und ist $\phi \colon K \to L$ ein Körperhomomorphismus, so ist $\phi$ ein Gruppenhomomorphismus von $(K,+)$ nach $(L,+)$, da $\phi(x+y) = \phi(x)+\phi(y)$ für alle $x,y \in K$.
   
   Insbesondere ist daher $\phi(0) = 0$ und $\phi(-x) = -\phi(x)$ für alle $x \in K$.
  \item
   Sind $K$ und $L$ Körper und ist $\phi \colon K \to L$ ein Körperhomomorphismus, so ist $\phi(x) \neq 0$ für alle $x \in K \setminus \{0\}$ sowie $\phi(x^{-1}) = \phi(x)^{-1}$ für alle $x \in K \setminus \{0\}$: Da
   \[
    1 = \phi(1) = \phi(x x^{-1}) = \phi(x) \phi(x^{-1})
   \]
   ist $\phi(x)$ invertierbar, also $\phi(x) \neq 0$, und $\phi(x^{-1})$ das Inverse von $\phi(x)$, also $\phi(x^{-1}) = \phi(x)^{-1}$.
   
   Da $\phi(x) \neq 0$ für alle $x \neq 0$ ist $\phi(K^\times) \subseteq L^\times$. Da außerdem $\phi(x \cdot y) = \phi(x) \cdot \phi(y)$ für alle $x,y \in K$, und somit insbesondere für alle $x,y \in K^\times$, induziert $\phi$ durch Einschränkung einen Gruppenhomomorphismus $\phi|_{K^\times} \colon K^\times \to L^\times$, $x \mapsto \phi(x)$.
  \item
   Körperhomomorphismen sind injektiv, d.h.\ sind $K$ und $L$ Körper und ist $\phi \colon K \to L$ ein Körperhomomorphismus, so ist $\phi$ injektiv: Sind $x, y \in K$ mit $x \neq y$, so ist $x-y \neq 0$ und somit $\phi(x-y) \neq 0$. Da $\phi(x)-\phi(y) = \phi(x-y) \neq 0$ ist $\phi(x) \neq \phi(y)$.
   
   Da Körperhomomorphismen stets injektiv sind, spricht man nicht von Körperepimorphismen oder Körpermonomorphismen: Jeder Körperhomomorphismus wäre ein Körpermonomorphismus und jeder Körperepimorphismus wäre bereits ein Körperisomorphismus.
 \end{enumerate}
\end{bem}





\section{Vektorräume}


\begin{defi}
 Ein $K$-Vektorraum ist ein Tripel $(K,V,\cdot)$ bestehend aus einem Körper $K$, einer abelschen Gruppe $V$ und einer Verknüpfung $\cdot K \times V \to V$, $(\lambda, v) \mapsto \lambda \cdot v$, die die folgenden Bedingungen erfüllt:
 \begin{enumerate}[label=\roman*)]
  \item
   $\lambda \cdot (v+w) = (\lambda \cdot v) + (\lambda \cdot w)$ für alle $\lambda \in K$ und $v,w \in V$.
  \item
   $(\lambda + \mu) \cdot v = (\lambda \cdot v) + (\mu \cdot v)$ für alle $\lambda, \mu \in K$ und $v \in V$.
  \item
   $\lambda \cdot (\mu \cdot v) = (\lambda \cdot \mu) \cdot v$ für alle $\lambda, \mu \in K$ und $v \in V$.
  \item
   $1 \cdot v = v$ für alle $v \in V$.
 \end{enumerate}
 Die Elemente $v \in V$ nennt man \emph{Vektoren} und die Elemente $\lambda \in K$ nennt man \emph{Skalare}. Die Verknüpfung $\cdot$ heißt \emph{Skalarmultiplikation}.
\end{defi}


\begin{bem}
 \begin{enumerate}[leftmargin=*]
  \item
   Man schreibt die Skalarmultiplikation auch also $\lambda v$ statt $\lambda \cdot v$.
  \item
   Man rechnet „Punkt vor Strich“, d.h.\ es ist
   \[
    \lambda \cdot v + w \coloneqq (\lambda \cdot v) + w
    \quad\text{für alle $\lambda \in K$ und $v,w \in V$}.
   \]
  \item
   Es ist wichtig, zu verstehen, dass man zwar Elemente des Körpers miteinander multiplizieren kann und auch Elemente aus dem Körper mit Elementen aus dem Vektorraum, dass man aber nicht Elemente aus dem Vektorraum miteinander multiplizieren kann. Produkte der Form $v \cdot w$ mit $v,w \in V$ ergeben für einen $K$-Vektorraum $V$ im Allgemeinen also keinen Sinn.
  \item
   Aus den gegebenen Axiomen eines $K$-Vektorraums $V$ folgen noch weiter Rechenregeln, wie etwa die folgenden:
   \begin{enumerate}[label=\alph*)]
    \item
     $0 \cdot v = 0$ für alle $v \in V$ (die $0$ auf der linken Seite ist die des Körpers, die $0$ auf der rechten Seite die des Vektorraums). Es ist nämlich
     \[
      0 \cdot v = (0+0) \cdot v = 0 \cdot v + 0 \cdot v,
     \]
     woraus sich die Aussage durch Subtraktion von $0 \cdot x$ von beiden Seiten der Gleichungskette ergibt.
    \item
     Für alle $\lambda \in K$ ist $\lambda \cdot 0 = 0$. Es ist nämlich
     \[
      \lambda \cdot 0
      = \lambda \cdot (0+0)
      = \lambda \cdot 0 + \lambda \cdot 0,
     \]
     woraus sich die Aussage durch Subtraktion von $\lambda \cdot 0$ von beiden Seiten der Gleichungkette ergibt.
    \item
     Sind andererseits $\lambda \in K$ und $v \in V$ mit $\lambda \neq 0$ und $v \neq 0$, so ist auch $\lambda \cdot v \neq 0$. Andernfalls wäre nämlich
     \[
      v
      = 1 \cdot v
      = \lambda^{-1} \cdot \lambda \cdot v = 
      = \lambda^{-1} \cdot 0
      = 0.
     \]
    \item
     $(-1) \cdot v = -v$ für alle $v \in V$. Da nämlich
     \[
      v + (-1) \cdot v = 1 \cdot v + (-1) \cdot v = (1 + (-1)) \cdot v = 0 \cdot v = 0
     \]
     ist $(-1) \cdot v$ das additiv Inverse zu $v$.
    \item
     Allgemeiner gilt $(-\lambda) \cdot v = -(\lambda \cdot v)$ für alle $\lambda \in K$ und $v \in V$. Der Beweis verläuft analog zum Fall $\lambda = 1$: Es ist
     \[
      \lambda \cdot v + (-\lambda) \cdot v
      = (\lambda + (-\lambda)) \cdot v
      = 0 \cdot v
      = 0,
     \]
     also ist $(-\lambda) \cdot v$ das additiv Inverse zu $\lambda \cdot v$.
   \end{enumerate}
 \end{enumerate}
\end{bem}



\begin{defi}
 Eine Teilmenge $U \subseteq V$ eines $K$-Vektorraums $V$ heißt \emph{Untervektorraum} falls
 \begin{enumerate}[label=\roman*)]
  \item
   $0 \in U$,
  \item
   für alle $u,v \in U$ ist auch $u+v \in V$,
  \item
   für alle $u \in U$ und $\lambda \in K$ ist auch $\lambda \cdot v \in U$.
 \end{enumerate}

\end{defi}


\begin{bem}
 \begin{enumerate}[leftmargin=*]
  \item
   Ist $U \subseteq V$ ein Untervektorraum, so ist $U$ insbesondere eine abelsche Untergruppe, denn für alle $u \in U$ ist auch $-u = (-1) \cdot u \in U$, und dass $0 \in U$ und $U$ unter der Addition abgeschlossen ist, ist gegeben.
  \item
   Für einen $K$-Vektorraum $V$ und Untervektorraum $U \subseteq V$ ist $(U,K,\cdot)$ ebenfalls ein $K$-Vektorraum. Da $U$ eine additive Untergruppe von $V$ ist, stimmt das Nullelement in $U$ mit dem in $V$ überein, und für jedes $u \in U$ stimmt das additiv Inverse Element in $U$ mit dem in $V$ überein.
  \item
   Ist $W$ ein $K$-Vektorraum, $V \subseteq W$ ein Untervektorraum und $U \subseteq V$ ein Untervektorraum (also $U$ ein Untervektorraum von $V$), so ist auch $U \subseteq W$ ein Untervektorraum.
 \end{enumerate}
\end{bem}


\begin{defi}
 Sind $V$ und $W$ zwei $K$-Vektorräume, so ist eine Abbildung $f \colon V \to W$ ein \emph{Vektorraumhomomorphismus}, bzw.\ \emph{linear}, falls
 \begin{enumerate}[label=\roman*)]
  \item
   $f(v_1 + v_2) = f(v_1) + f(v_2)$ für alle $v_1, v_2 \in V$ und
  \item
   $f(\lambda v) = \lambda f(v)$ für alle $\lambda \in K$ und $v \in V$.
 \end{enumerate}
 Ist $f$ injektiv, so ist $f$ ein \emph{Vektorraummonomorphismus}. Ist $f$ surjektiv, so eist $f$ ein \emph{Vektorraumepimorphismus}. Ist $f$ bijektiv, so ist $f$ ein \emph{Vektorraumisomorphismus}. Ist $V = W$, so ist $f$ ein \emph{Vektorraumendomorphismus}. Ist $V = W$ und $f$ bijektiv, so ist $f$ ein \emph{Vektorraumautomorphismus}.
\end{defi}


\begin{bem}
 \begin{enumerate}[leftmargin=*]
  \item
   Man spricht auch abkürzend nur von Homomorphismen, Monomorphismen, Epimorphismen, Isomorphismen, Endomorphismen und Automorphismen.
  \item
   Für jeden Vektorraum $V$ ist die Identität $\id_V \colon V \to V$ ein Automorphismus.
  \item
   Sind $V_1$, $V_2$ und $V_3$ $K$-Vektorräume und $f \colon V_1 \to V_2$ und $g \colon V_2 \to V_3$ linear, so ist auch $g \circ f \colon V_1 \to V_3$ linear, da für alle $x,y \in V_1$
   \begin{align*}
    (g \circ f)(x+y)
    = g(f(x+y))
    &= g(f(x)+g(y)) \\
    &= g(f(x)) + g(f(y))
    = (g \circ f)(x) + (g \circ f)(y),
   \end{align*}
   und für alle $\lambda \in K$ und $v \in V$
   \[
    (g \circ f)(\lambda v)
    = g(f(\lambda v))
    = g(\lambda f(v))
    = \lambda g(f(v))
    = \lambda (g \circ f)(v).
   \]
  \item
   Sind $V$ und $W$ $K$-Vektorräume und ist $f \colon V \to W$ ein Isomorphismus, so ist auch $f^{-1} \colon W \to V$ ein Isomorphismus: $f^{-1}$ ist bijektiv, für alle $x,y \in W$ ist
   \begin{align*}
    f^{-1}(x+y)
    &= f^{-1}( f(f^{-1}(x)) + f(f^{-1}(y)) ) \\
    &= f^{-1}(f( f^{-1}(x) + f^{-1}(y) )
    = f^{-1}(x) + f^{-1}(y)
   \end{align*}
   und für alle $\lambda \in K$ und $x \in W$ ist
   \[
    f^{-1}(\lambda x)
    = f^{-1}(\lambda f(f^{-1}(x)) )
    = f^{-1}(f( \lambda f^{-1}(x)))
    = \lambda f^{-1}(x).
   \]
 \end{enumerate}
\end{bem}





\section{Mengentheoretische Grundbegriffe und Notationen}



\subsection{Allgemeine Notationen und Begriffe}


\begin{defi}
 Es sei $I$ eine Indexmenge und für jedes $i \in I$ sei $A_i$ eine Menge. Dann ist die \emph{Vereinigung} $\bigcup_{i \in I} A_i$ definiert als
 \[
  \bigcup_{i \in I} A_i
  = \{x \mid \text{es gibt ein $i \in I$ mit $x \in A_i$}\}.
 \]
 Der \emph{Schnitt} $\bigcap_{i \in I} A_i$ ist definiert als
 \[
  \bigcap_{i \in I} A_i
  = \{x \mid \text{für alle $i \in I$ ist $x \in A_i$}\}.
 \]
 
 Für Mengen $A_1, \dotsc, A_n$ ist
 \[
  A_1 \cup \dotsb \cup A_n \coloneqq \bigcup_{i \in \{1, \dotsb, n\}} A_i.
  \quad\text{und}\quad
  A_1 \cap \dotsb \cap A_n \coloneqq \bigcap_{i \in \{1, \dotsb, n\}} A_i.
 \]
\end{defi}


\begin{defi}
 Für Mengen $A$ und $B$ ist die Differenz $A \setminus B$ als
 \[
  A \setminus B \{a \in A \mid a \notin B\}
 \]
 definiert. Die symmetrische Differenz $A \symm B$ ist als
 \[
  A \symm B = (A \cup B) \setminus (A \cap B)
 \]
 definiert.
\end{defi}


\begin{defi}
 Ist $A$ eine fixierte (!) Menge und $S \subseteq A$ eine Teilmenge, so ist das \emph{Komplement} $S^C$ als
 \[
  S^C
  \coloneqq A \setminus S
  = \{a \in A \mid a \notin S\}
 \]
 definiert.
\end{defi}


\begin{bem}
 \begin{enumerate}[leftmargin=*]
  \item
   Die Notation des Komplementes ergibt nur für Teilmengen $S \subseteq A$ einer fest fixierten Grundmenge  $A$ Sinn, und hängt $A$ ab!
  \item
   Ist $S \subseteq A$ eine Teilmenge, so ist $(S^C)^C = S$ und $S \cap S^C = \emptyset$. ($S^C$ ist die größte Teilmenge von $A$, die $S$ trivial schneidet, d.h.\ ist $T \subseteq A$ eine Teilmenge mit $T \cap S = \emptyset$, so ist $T \subseteq S^C$.)
  \item
   Sind $S, T \subseteq A$ Teilmengen, so ist
   \[
    S \setminus T
    = \{s \in S \mid s \notin T\}
    = \{s \in S \mid s \in T^C\}
    = S \cap T^C.
   \]
 \end{enumerate}
\end{bem}


\begin{bem}
 Ist $A$ eine Menge und $A_i$, $i \in I$ eine Kollektion von Teilmengen $A_i \subseteq A$, so gelten die \emph{De Morgansche Regeln}
 \[
  \left( \bigcup_{i \in I} A_i \right)^{\mathclap{C}} = \bigcap_{i \in I} A_i^C
  \quad\text{und}\quad
  \left( \bigcap_{i \in I} A_i \right)^{\mathclap{C}} = \bigcup_{i \in I} A_i^C.
 \]
 Für jedes $a \in A$ gilt nämlich
 \begin{align*}
      &\, a \in \left( \bigcup_{i \in I} A_i \right)^{\mathclap{C}}
  \iff    a \notin \bigcup_{i \in I} A_i
  \iff    \text{es gibt kein $i \in I$ mit $a \in A_i$} \\
  \iff&\, \text{für jedes $i \in I$ ist $a \notin A_i$}
  \iff    \text{für jedes $i \in I$ ist $a \in A_i^C$}
  \iff    a \in \bigcap_{i \in I} A_i^C,
 \end{align*}
 was die erste Regel zeigt. Die zweite Regel ergibt sich aus der ersten, da
 \[
  \left( \bigcap_{i \in I} A_i \right)^C
  = \left( \bigcap_{i \in I} (A_i^C)^C \right)^C
  = \left( \left( \bigcup_{i \in I} A_i^C \right)^C \right)^C
  = \bigcup_{i \in I} A_i^C.
 \]
\end{bem}


\begin{bem}
 Sind $A_i$, $i \in I$ und $B_i$, $i \in I$ zwei Kollektionen von Mengen, so ist
 \[
  \left( \bigcup_{i \in I} A_i \right) \cap \left( \bigcup_{j \in I} B_j \right)
  = \bigcup_{i,j \in I} (A_i \cap B_j),
 \]
 denn es ist
 \begin{align*}
      &\, x \in \left( \bigcup_{i \in I} A_i \right) \cap \left( \bigcup_{j \in I} B_j \right)
  \iff    \text{$x \bigcup_{i \in I} A_i$ und $\bigcup_{j \in I} B_j$} \\
  \iff&\, \text{es gibt $i,j \in I$ mit $x \in A_i$ und $x \in B_j$} 
  \iff    \text{es gibt $i,j \in I$ mit $x \in A_i \cap B_j$} \\
  \iff&\, x \in \bigcup_{i,j \in I} (A_i \cap B_j).
 \end{align*}
 Es ist daher auch
 \[
  \left( \bigcap_{i \in I} A_i \right) \cup \left( \bigcap_{j \in I} B_j \right)
  = \bigcap_{i,j \in I} (A_i \cup B_j),
 \]
 denn es ist
 \[
  \left( \bigcup_{i \in I} A_i^C \right) \cap \left( \bigcup_{j \in I} B_j^C \right)
  = \bigcup_{i,j \in I} (A_i^C \cap B_j^C),
 \]
 wodurch sich durch Anwendung des Komplements auf beiden Seiten und den De Morganschen Regeln die behauptete Gleichung ergibt.
\end{bem}




\subsection{Produkte und Potenzen}
Für Mengen $A_1, \dotsc, A_n$ wird das Produkt $A_1 \times \dotsb \times A_n$ für gewöhnlich als
\[
 A_1 \times \dotsb \times A_n
 = \{(a_1, \dotsc a_n) \mid a_1 \in A_1, \dotsc a_n, \in A_n\}
\]
definiert, wobei $(a_1, \dotsb, a_n)$ ein geordnetes Tupel ist. Für eine Menge $A$ und $n \in \Nbb$, $n \geq 1$ wird die Potenz $A^n$ dann als
\[
 A^n
 \coloneqq \underbrace{A \times \dotsb \times A}_{\text{$n$ viele}}
 = \{(a_1, \dotsc, a_n) \mid a_1, \dotsc, a_n \in A\}
\]
definiert. Wir geben hier eine formale Konstruktion von beliebig großen (bzw. mengen-großen) Produkten und Potenzen an.


\begin{defi}
 Es sei $I$ ein Indexmenge und für jedes $i \in I$ sei $A_i$ eine Menge. Dann ist das \emph{Produkt} $\prod_{i \in I} A_i$ definiert als
 \[
  \prod_{i \in I} A_i
  = \left\{ f \colon I \to \bigcup_{i \in I} A_i \,\middle|\, \text{$f(i) \in A_i$ für alle $i \in I$} \right\}
 \]
 und ein Element $f \in \prod_{i \in I} A_i$ wird als \emph{Tupel} $(f(i))_{i \in I}$ geschrieben. Für eine Menge $A$ und Indexmenge $I$ ist die \emph{Potenz} $A^I$ definiert als
 \[
  A^I
  \coloneqq \prod_{i \in I} A
  = \{f \colon I \to A\}.
 \]

\end{defi}


\begin{bem}
 \begin{enumerate}[leftmargin=*]
  \item
   Mithilfe der Tupelschreibweise ist
   \begin{gather*}
    \prod_{i \in I} A_i = \{(f(i))_{i \in I} \mid \text{$f(i) \in A_i$ für alle $i \in I$}\}
   \shortintertext{sowie}
    A^I = \{(f(i))_{i \in I} \mid \text{$f(i) \in A$ für alle $i \in I$}\}.
   \end{gather*}
   Schreibt man zusätzlich $f_i$ statt $f(i)$ (wie man es etwa von Folgen gewöhnt ist), so ist
   \begin{gather*}
    \prod_{i \in I} A_i = \{(f_i)_{i \in I} \mid \text{$f_i \in A_i$ für alle $i \in I$}\}
   \shortintertext{und}
    A^I = \{(f_i)_{i \in I} \mid \text{$f_i \in A$ für alle $i \in I$}\}.
   \end{gather*}
  \item
   Schreiben wir im Folgenden $\prod_{i \in I} A_i$ oder $A^I$, so verstehen wir dies in erster Linie. Die obige Definition von Tupeln als Funktionen dient für uns nur als mengentheoretische Implementation des Ganzen.

   Möchten wir tatsächlich die Menge der Funktionen zwischen zwei Mengen $X$ und $Y$ betrachten, so schreiben wir hierfür im Folgenden
   \[
    \Abb(X,Y) \coloneqq \{f \colon X \to Y\}.
   \]
   
   Sofern es uns nützlich erscheint, werden wir Gleichheit $\Abb(I,X)$ und $X^I$ aber nutzen, um uns doppelte Rechnungen zu ersparen. So werden wir Beispiele angeben, die über Produkte zustande kommen (etwa Produkte von Gruppen und Produkte von Vektorräumen), und aus diese ohne weiteren Rechenaufwand auch Beispiele von Abbildungsräumen (Abbildungen in eine Gruppe und Abbildungen in einen Vektorraum) erhalten.
 \end{enumerate}
\end{bem}





\section{Folgen und Matrizen}
\begin{defi}
 Es sei $X$ eine Menge. Eine \emph{Folge} auf $X$ ist eine Funktion $a \colon \Nbb \to X$. Die Folge $a$ wird als Tupel $a = (a_n)_{n \in \Nbb}$ geschrieben, wobei $a_n = a(n)$ für alle $n \in \Nbb$. Die Element $a_n$ mit $n \in \Nbb$ werden \emph{Einträge} von $(a_n)_{n \in \Nbb} \in K$ genannt. Es sei
 \begin{align*}
  \ell(X)
  &\coloneqq \Abb(\Nbb, X)
  = \{a \colon \Nbb \to X\} \\
  &= \{ (a_n)_{n \in \Nbb} \mid \text{$(a_n)_{n \in \Nbb}$ ist eine Folge auf $X$} \} \\
  &= \{ (a_n)_{n \in \Nbb} \mid \text{$a_n \in X$ für alle $n \in \Nbb$} \}.
 \end{align*}
\end{defi}


\begin{defi}
 Es seien $m, n \in \Nbb$ und $X$ ein Menge. Eine \emph{$(m \times n)$-Matrix} mit \emph{Einträgen} in $X$ ist eine Abbildung $A \colon \{1, \dotsc, m\} \times \{1, \dotsc, n\} \to X$. Die Matrix $A$ wird als Tupel
 \[
  (A_{ij})_{(i,j) \in \{1, \dotsc, m\} \times \{1, \dotsc, n\}}
  =
  (A_{ij})_{\substack{i = 1, \dotsc, m \\ j = 1, \dotsc, n}}
 \]
 geschrieben, wobei $A_{ij} = A((i,j)) \in X$ für alle $(i,j) \in \{1, \dotsc, m\} \times \{1, \dotsc, n\}$ also alle $1 \leq i \leq m$ und $1 \leq j \leq n$. $A$ wird auch als ein rechteckiges Schema
 \[
  A =
  \begin{pmatrix}
   A_{11} & \cdots & A_{1n} \\
   \vdots & \ddots & \vdots \\
   A_{m1} & \cdots & A_{mn}
  \end{pmatrix}
 \]
 geschrieben. Die Menge der ($m \times n$)-Matrizen mit Einträgen in $X$ wird mit
 \[
  \Mat(m \times n, X)
  =
  \left\{
   \begin{pmatrix}
    A_{11} & \cdots & A_{1n} \\
    \vdots & \ddots & \vdots \\
    A_{m1} & \cdots & A_{mn}
   \end{pmatrix}
   \,\middle|\,
    \text{$A_{ij} \in X$ für alle $1 \leq i \leq m$ und $1 \leq j \leq n$}
  \right\}
 \]
 bezeichnet.
\end{defi}


\begin{defi}
 Es sei $K$ ein Körper. Es seien $l,m,n \in \Nbb$, sowie $A \in \Mat(l \times m, K)$ und $B \in \Mat(m \times n, K)$. Das Produkt $A \cdot B \in \Mat(l \times n, K)$ ist als
 \[
  (A \cdot B)_{ik} = \sum_{j=1}^m A_{ij} B_{jk}
  \quad
  \text{für alle $1 \leq i \leq l$ und $1 \leq k \leq m$}
 \]
 definiert.
\end{defi}


\begin{bem}
 Das Matrixprodukt ist assoziativ, d.h.\ für alle $p,q,r,s \in \Nbb$ und Matrizen $A \in \Mat(p \times q, K)$, $B \in \Mat(q \times r, K)$ und $C \in \Mat(r \times s, K)$ ist $A \cdot (B \cdot C) = (A \cdot B) \cdot C$, denn für alle $1 \leq i \leq p$ und $1 \leq l \leq s$ ist
 \begin{align*}
  (A \cdot (B \cdot C))_{il}
  &= \sum_{j=1}^q A_{ij} (B \cdot C)_{jl}
  = \sum_{j=1}^q A_{ij} \sum_{k=1}^r B_{jk} C_{kl}
  = \sum_{j=1}^q \sum_{k=1}^r A_{ij} B_{jk} C_{kl} \\
  &= \sum_{k=1}^r \sum_{j=1}^q A_{ij} B_{jk} C_{kl}
  = \sum_{k=1}^r \left( \sum_{j=1}^q A_{ij} B_{jk} \right) C_{kl}
  = \sum_{k=1}^r (A \cdot B)_{ik} C_{kl} \\
  &= ((A \cdot B) \cdot C)_{il}.
 \end{align*}
\end{bem}


\begin{bem}
 Wir werden im Folgenden auf die Definition von Folgen und Matrizen als Abbildungen $\Nbb \to X$, bzw. $\{1, \dotsc, m\} \times \{1, \dotsc, n\} \to X$ zurückkommen, um aus Beispielen von Abbildungsräumen (Abbildungen in eine Gruppe, Abbildungen in einen Vektorraum) Aussagen über Folgen und Matrizen zu erhalten.
\end{bem}





\section{Konvergenz und (absolute) Summierbarkeit von Folgen}
Da wir im Folgenden auch Räume konventer Folgen und Reihen betrachten wollen, erinnern wir hier an die entsprechenden grundlegenden Definitionen und Aussagen über konvergente Folgen und Reihen. In diesem Abschnitt sei $\Kbb \in \{\Qbb, \Rbb, \Cbb\}$.


\begin{defi}
 Es sei $(x_n)_{n \in \Nbb} \in \ell(\Kbb)$, also $(x_n)_{n \in \Nbb}$ eine Folge mit Werten in $\Kbb$, und $x \in \Kbb$. Wir sagen, dass die Folge $(x_n)_{n \in \Nbb}$ gegen $x$ \emph{konvergiert}, falls es für jedes $\varepsilon > 0$ ein $N \in \Nbb$ gibt, so dass $|x_n - x| < \varepsilon$ für alle $n \geq N$. Die Folge $(x_n)_{n \in \Nbb}$ heißt dann \emph{konvergent} und $x$ heißt \emph{Grenzwert} der Folge $(x_n)_{n \in \Nbb}$. Wir schreiben abkürzend, dass $x_n \to x$ für $n \to \infty$ falls die Folge $(x_n)_{n \in \Nbb}$ gegen $x$ konvergiert.
\end{defi}


\begin{bem}
 Ist $(x_n)_{n \in \Nbb}$ eine konvergente $\Kbb$-wertige Folge, so ist der Grenzwert eindeutig: Angenommen, es gibt $x,y \in \Kbb$ mit $x_n \to x$ für $n \to \infty$ und $x_n \to y$ für $n \to \infty$, aber $x \neq y$. Dann ist $|x-y| > 0$ und somit auch $\varepsilon \coloneqq |x-y|/3 > 0$. Da $x_n \to x$ für $n \to \infty$ gibt es $N_x \in \Nbb$ mit $|x-x_n| < \varepsilon$ für alle $n \geq N_x$, und da $x_n \to y$ für $n \to \infty$ gibt es $N_y \in \Nbb$ mit $|y-x_n| < \varepsilon$ für alle $n \geq N_y$. Für $n \coloneqq \max \{N_1, N_2\}$ ist daher
 \[
  |x-y|
  = |x-x_n+x_n-y|
  \leq |x - x_n| + |y - x_n|
  < \varepsilon + \varepsilon
  = \frac{2}{3}|x-y|,
 \]
 was $|x-y| > 0$ widerspricht. Also muss bereits $x = y$.
 
 Konvergiert eine $\Kbb$-wertige Folge $(x_n)_{n \in \Nbb}$ gegen ein Element $x \in \Kbb$, so schreiben wir auch $\lim_{n \to \infty} x_n \coloneqq x$.
\end{bem}


\begin{bem}\label{bem: properties of convergent sequences}.
 \begin{enumerate}[leftmargin=*]
  \item
   Ist $(x_n)_{n \in \Nbb} \in \ell(\Kbb)$ eine konstante Folge, d.h.\ es gibt $c \in \Kbb$ mit $x_n = c$ für alle $n \in \Nbb$, so konvergiert $(x_n)_{n \in \Nbb}$ und $\lim_{n \to \infty} x_n = c$. Für beliebige $\varepsilon > 0$ ist nämlich $|c-x_n| = 0 < \varepsilon$ für alle $n \geq 0$.
  \item
   Sind $(x_n)_{n \in \Nbb}, (y_n)_{n \in \Nbb} \in \ell(\Kbb)$ konvergente Folgen, so konvergiert auch die Folge $(x_n + y_n)_{n \in \Nbb}$ und es gilt
   \[
    \lim_{n \to \infty} (x_n + y_n)
    = \left( \lim_{n \to \infty} x_n \right) + \left( \lim_{n \to \infty} y_n \right).
   \]
   Ist nämlich $x \coloneqq \lim_{n \to \infty} x_n$, $y \coloneqq \lim_{n \to \infty} y_n$ und $\varepsilon > 0$, so gibt es $N_x, N_y \in \Nbb$ mit $|x-x_n| < \varepsilon/2$ für alle $n \geq N_x$ und $|y-y_n| < \varepsilon/2$ für alle $n \geq N_y$, weshalb für alle $n \geq N \coloneqq \max\{N_x, N_y\}$ auch
   \[
    |(x+y)-(x_n+y_n)|
    = |(x-x_n) + (y-y_n)|
    \leq |x - x_n| + |y - y_n|
    < \frac{\varepsilon}{2} + \frac{\varepsilon}{2}
    = \varepsilon.
   \]
  \item
   Ist $(x_n)_{n \in \Nbb} \in \ell(\Kbb)$ konvergent und $\lambda \in \Kbb$, so ist auch die Folge $(\lambda x_n)_{n \in \Nbb}$ konvergent und $\lim_{n \to \infty} (\lambda x_n) = \lambda \lim_{n \to \infty} x_n$: Es sei $x \coloneqq \lim_{n \to \infty} x_n$. Ist $\lambda = 0$, so ist $(\lambda x_n)_{n \in \Nbb}$ die konstante Nullfolge und somit
   \[
    \lim_{n \to \infty} \lambda x_n
    = \lim_{n \to \infty} 0
    = 0
    = 0 \cdot \lim_{n \to \infty} x_n.
   \]
   Ist $\lambda \neq 0$ und $\varepsilon > 0$, so gibt es $N \in \Nbb$ mit $|x-x_n| < \varepsilon/|\lambda|$ für alle $n \geq N$. Für alle $n \geq N$ ist daher auch
   \[
    |\lambda x - \lambda x_n|
    = |\lambda| |x-x_n|
    < |\lambda| \frac{\varepsilon}{|\lambda|}
    = \varepsilon.
   \]
 \end{enumerate}
\end{bem}


\begin{defi}
 Eine Folge $(x_n)_{n \in \Nbb} \in \ell(\Kbb)$ heißt \emph{Cauchy-Folge}, falls es für jedes $\varepsilon > 0$ ein $N \in \Nbb$ gibt, so dass $|x_n - x_m| < \varepsilon$ für alle $n,m \geq N$.
\end{defi}


\begin{bem}
 \begin{enumerate}[leftmargin=*]
  \item
   Ist $(x_n)_{n \in \Nbb} \in \ell(\Kbb)$ konvergent, so ist $(x_n)_{n \in \Nbb}$ auch eine Cauchy-Folge: Ist nämlich $\varepsilon > 0$ und $x \coloneqq \lim_{n \to \infty} x_n$, so gibt es $N \in \Nbb$ mit $|x - x_n| < \varepsilon/2$ für alle $n \geq N$, weshalb für alle $n,m \geq N$ auch
   \[
    |x_n - x_m|
    = |x_n - x + x - x_m|
    \leq |x_n - x| + |x - x_m|
    \leq \frac{\varepsilon}{2} + \frac{\varepsilon}{2}
    = \varepsilon.
   \]
  \item
   Für die reellen Zahlen $\Rbb$ und komplexen Zahlen $\Cbb$ gilt auch die Umkehrung: Ist $(x_n)_{n \in \Nbb}$ eine Cauchy-Folge reeller oder komplexer Zahlen, also $(x_n)_{n \in \Nbb} \in \ell(\Rbb)$ oder $(x_n)_{n \in \Nbb} \in \ell(\Cbb)$, so ist $(x_n)_{n \in \Nbb}$ auch konvergent. Man sagt, dass $\Rbb$ und $\Cbb$ \emph{vollständig} sind. (Wir werden dies hier nicht zeigen. Für die reellen Zahlen ist der Beweis abhängig davon, wie diese konstruiert werden. Die Vollständigkeit der komplexen Zahlen lässt sich dann aus der Vollständigkeit der reellen Zahlen folgern.)
  \item
   Für die rationalen Zahlen $\Qbb$ gilt diese Aussage nicht: Es gibt Cauchy-Folgen von rationalen Zahlen, die in $\Qbb$ keinen Grenzwert besitzen. $\Qbb$ ist also nicht vollständig. (Auch diese Aussage werden wir hier nicht zeigen.)
 \end{enumerate}
\end{bem}


\begin{defi}
 Für eine Folge $(a_n)_{n \in \Nbb} \in \ell(\Kbb)$ und $m \in \Nbb$ ist die $m$-te \emph{Partialsumme} der Folge $(a_n)_{n \in \Nbb}$ als $\sum_{n=0}^m a_n$ definiert. Die Folge $(a_n)_{n \in \Nbb}$ heißt \emph{summierbar}, falls die Folge der Partialsummen $(\sum_{n=0}^m a_n)_{m \in \Nbb}$ konvergiert. Es ist dann $\sum_{n=0}^\infty a_n \coloneqq \lim_{m \to \infty} \sum_{n=0}^m a_n$.
\end{defi}


\begin{bem}
 \begin{enumerate}[leftmargin=*]
  \item
   Die konstante Nullfolge $(0)_{n \in \Nbb}$ ist summierbar mit $\sum_{n=0}^\infty a_n = 0$. Für alle $m \in \Nbb$ ist nämlich $\sum_{n=0}^m 0 = 0$, also $(\sum_{n=0}^m a_n)_{m \in \Nbb} = 0$, die Folge der Partialsummen also ebenfalls die Nullfolge. Daher konvergiert die Folge der Partialsummen gegen $0$, also $\lim_{m \to \infty} \sum_{n=0}^m 0 = 0$. Dies bedeutet gerade, dass die Nullfolge $(0)_{n \in \Nbb}$ summierbar ist, und dass $\sum_{n=0}^\infty 0 = 0$.
  \item
   Sind $(a_n)_{n \in \Nbb}, (b_n)_{n \in \Nbb} \in \ell(\Kbb)$ summierbare Folgen, so ist auch die Folge $(a_n + b_n)_{n \in \Nbb}$ summierbar und $\sum_{n=0}^\infty (a_n+b_n) = (\sum_{n=0}^\infty a_n) + (\sum_{n=0}^\infty b_n)$.
   
   Ist nämlich $S_m \coloneqq \sum_{n=0}^m a_n$ und $T_n \coloneqq \sum_{n=0}^m b_n$ für alle $m \in \Nbb$, so bedeutet die Summierbarkeit von $(a_n)_{n \in \Nbb}$ und $(b_n)_{n \in \Nbb}$, dass die Folgen $(S_m)_{m \in \Nbb}$ und $(T_m)_{m \in \Nbb}$ konvergieren, und dass $S \coloneqq \lim_{m \to \infty} S_m = \sum_{n=0}^\infty a_n$ und $T \coloneqq \lim_{m \to \infty} T_m = \sum_{n=0}^\infty b_n$. Da $(S_m)_{m \in \Nbb}$ und $(T_m)_{m \in \Nbb}$ konvergieren, konvergiert auch die Folge $(S_m + T_m)_{m \in \Nbb}$ und $\lim_{m \to \infty} (S_m + T_m) = S + T$.
   
   Für alle $m \in \Nbb$ ist nun
   \begin{gather*}
    S_m + T_m
    = \left( \sum_{n=0}^m a_n \right) + \left( \sum_{n=0}^m b_n \right)
    = \sum_{n=0}^m (a_n + b_n)
   \shortintertext{und es ist}
    \left( \sum_{n=0}^\infty a_n \right) + \left( \sum_{n=0}^\infty b_n \right)
    = S + T
    = \lim_{m \to \infty} (S_m + T_m)
    = \lim_{m \to \infty} \sum_{n=0}^m (a_n + b_n).
   \end{gather*}
   Das zeigt, dass die Folge der Partialsummen $(\sum_{n=0}^m (a_n+b_n))_{m \in \Nbb}$ konvergiert, und dass $\lim_{m \to \infty} \sum_{n=0}^m (a_n+b_n) = (\sum_{n=0}^\infty a_n) + (\sum_{n=0}^\infty b_n)$. Dies bedeutet gerade, dass die Folge $(a_n + b_n)_{n \in \Nbb}$ summierbar ist, und dass $\sum_{n=0}^\infty (a_n+b_n) = (\sum_{n=0}^\infty a_n) + (\sum_{n=0}^\infty b_n)$.
  \item
   Ist $(a_n)_{n \in \Nbb} \in \ell(\Kbb)$ summierbar und $\lambda \in \Kbb$, so ist auch die Folge $(\lambda a_n)_{n \in \Nbb}$ summierbar und $\sum_{n=0}^\infty (\lambda a_n) = \lambda \sum_{n=0}^\infty a_n$.
   
   Ist $S_m \coloneqq \sum_{n=0}^m a_n$ die $m$-te Partialsumme der Folge $(a_n)_{n \in \Nbb}$, so bedeutet die Summierbarkeit der Folge $(a_n)_{n \in \Nbb}$, dass die Folge der Partialsummen $(S_m)_{m \in \Nbb}$ konvergiert, und dass $S \coloneqq \lim_{m \to \infty} S_m = \sum_{n=0}^\infty a_n$. Da die Folge $(S_m)_{m \in \Nbb}$ konvergiert, konvergiert auch die Folge $(\lambda S_m)_{m \in \Nbb}$ und es gilt $\lim_{m \to \infty} \lambda S_m = \lambda S$.
   
   Für alle $m \in \Nbb$ ist nun
   \begin{gather*}
    \lambda S_m
    = \lambda \sum_{n=0}^m a_n
    = \sum_{n=0}^m (\lambda a_n),
   \shortintertext{und es gilt}
    \lambda \sum_{n=0}^\infty a_n
    = \lambda \lim_{m \to \infty} S_m
    = \lim_{m \to \infty} \lambda S_m
    = \lim_{m \to \infty} \lambda \sum_{n=0}^m a_n
    = \lim_{m \to \infty} \sum_{n=0}^m (\lambda a_n).
   \end{gather*}
   Die Konvergenz der Folge $(\lambda S_m)_{m \in \Nbb}$ bedeutet also genau die Summierbarkeit der Folge $(\lambda a_n)_{n \in \Nbb}$, und es ist $\sum_{n=0}^\infty (\lambda a_n) = \lim_{m \to \infty} \sum_{n=0}^m (\lambda a_n) = \lambda \sum_{n=0}^\infty a_n$.
 \end{enumerate}
\end{bem}



\begin{defi}
 Eine Folge $(a_n)_{n \in \Nbb} \in \ell(\Kbb)$ heißt \emph{absolut summierbar}, falls die Folge $(|a_n|)_{n \in \Nbb}$ summierbar ist.
\end{defi}


\begin{bem}
 Ist $(a_n)_{n \in \Nbb}$ eine absolut summierbare Folge reeller oder komplexer Zahlen, so ist $(a_n)_{n \in \Nbb}$ auch summierbar: Dass die Folge $(a_n)_{n \in \Nbb}$ absolut summierbar ist, bedeutet, dass die Folge $(|a_n|)_{n \in \Nbb}$ summierbar ist. Das bedeutet, dass die Folge der Partialsummen $(\sum_{n=0}^m |a_n|)_{m \in \Nbb}$ konvergiert. Also ist die Folge der Partialsummen $(\sum_{n=0}^m |a_n|)_{m \in \Nbb}$ eine Cauchy-Folge.
 
 Es sei $\varepsilon > 0$. Da die Folge der Partialsummen $(\sum_{n=0}^m |a_n|)_{m \in \Nbb}$ eine Cauchy-Folge ist, gibt es $N \in \Nbb$, so dass für alle $l \geq m \geq N$
 \[
  \sum_{n=m+1}^l |a_n|
  = \left| \sum_{n=0}^l |a_n| - \sum_{n=0}^m |a_n| \right|
  < \varepsilon.
 \]
 Für alle $l \geq m \leq N$ ist deshalb auch
 \[
  \left| \sum_{n=0}^l a_n - \sum_{n=0}^m a_n \right|
  = \left| \sum_{n=m+1}^l a_n \right|
  \leq \sum_{n=m+1}^l |a_n|
  < \varepsilon.
 \]
 Deshalb ist die Folge der Partialsummen $(\sum_{n=0}^m a_n)_{m \in \Nbb}$ eine Cauchy-Folge. Da $\Rbb$ und $\Cbb$ vollständig sind, ist die Folge der Partialsummen $(\sum_{n=0}^m a_n)_{m \in \Nbb}$ deshalb bereits konvergent. Dies bedeutet, dass die Folge $(a_n)_{n \in \Nbb}$ summierbar ist.
\end{bem}


\begin{bem}
 Ist die Folge $(a_n)_{n \in \Nbb} \in \ell(\Kbb)$ summierbar, so ist $(a_n)_{n \in \Nbb}$ eine Nullfolge, d.h.\ $(a_n)_{n \in \Nbb}$ ist konvergent und $\lim_{n \to \infty} a_n = 0$: Für alle $m \in \Nbb$ sei $S_m \coloneqq \sum_{n=0}^m a_n$ die $m$-te Partialsumme von $(a_n)_{n \in \Nbb}$. Dass $(a_n)_{n \in \Nbb}$ summierbar ist, bedeutet, dass die Folge $(S_m)_{m \in \Nbb}$ konvergiert. Es konvergiert daher die Folge $(S_{m+1})_{m \in \Nbb}$ mit $\lim_{m \to \infty} S_{m+1} = \lim_{m \to \infty} S_m$. Daher konvergiert auch die Folge $(S_{m+1} - S_m)_{m \in \Nbb}$ mit
 \[
  \lim_{m \to \infty} (S_{m+1} - S_m)
  = \lim_{m \to \infty} S_{m+1} - \lim_{m \to \infty} S_m
  = \lim_{m \to \infty} S_m - \lim_{m \to \infty} S_m
  = 0.
 \]
 Für alle $m \in \Nbb$ ist jedoch
 \[
  S_{m+1} - S_m
  = \sum_{n=0}^{m+1} a_n - \sum_{n=0}^m a_n
  = a_m,
 \]
 also ist $0 = \lim_{m \to \infty} (S_{m+1} - S_m) = \lim_{m \to \infty} a_m$.
\end{bem}




























