\section{Beispiele für Körper}





\subsection{Definition und Notationen}


\begin{defi}
 Ein Körper ist ein Tupel $(K, +, \cdot)$ bestehend aus einer Menge $K$ und zwei binären Verknüpfungen $+ \colon K \times K \to K$, $(x,y) \mapsto x + y$ und $\cdot \colon K \to K$, $(x,y) \mapsto xy$, so dass die folgenden Bedingungen erfüllt sind:
 \begin{enumerate}[label=\roman*)]
  \item
   $(K,+)$ ist eine abelsche Gruppe.
  \item
   $(K \setminus \{0\},\cdot)$ ist eine abelsche Gruppe, wobei $0$ das neutrale Element der Addition bezeichnet. (Insbesondere ist für alle $x,y \in K\setminus\{0\}$ auch $x \cdot y \in K\setminus\{0\}$).
  \item
   Für alle $x,y,z \in K$ ist
   \[
    x \cdot (y+z) = (x \cdot y) + (x \cdot z)
    \quad\text{und}\quad
    (x+y) \cdot z = (x \cdot z) + (y \cdot z).
   \]
 \end{enumerate}
\end{defi}


\begin{bem}
 \begin{enumerate}[leftmargin=*]
  \item
   Man bezeichnet einen Körper meistens nur mit $K$ statt mit $(K,+,\cdot)$, nennt also die Addition und Multiplikation nicht explizit. Das neutrale Element bezüglich der Addition wird mit $0$ bezeichnet, das neutrale Element der Multiplikation auf $K\setminus\{0\}$ also $1$.
  \item
   Man bezeichnet die Gruppe $(K,+)$ als die additive Gruppe von $K$. Man schreibt $K^\times \coloneqq K\setminus\{0\}$ und bezeichnet die Gruppe $(K^\times,\cdot)$ als die multiplikative Gruppe von $K$.
  \item
   In der Definition eines Körpers lässt sich der Begriff einer Gruppe umgehen, indem man die entsprechenden Aussagen alle explizit angibt: Ein Körper ist dann definiert als ein Tupel $(K,+,\cdot)$ bestehend aus einer Menge $K$ und zwei binären Verknüpfungen $+$ und $\cdot$ auf $K$, so dass die folgenden Bedingungen erfüllt sind:
   \begin{enumerate}[label=\roman*)]
    \item
     $+$ ist assoziativ, d.h.\ $x+(y+z) = (x+y)+z$ für alle $x,y,z \in K$.
    \item
     $+$ ist kommutativ, d.h.\ $x+y = y+x$ für alle $x,y \in K$.
    \item
     Es gibt ein additiv neutrales Element $0 \in K$ mit $x + 0 = x$ für alle $x \in K$.
    \item
     Es gibt für jedes $x \in K$ ein additiv inverses Element $-x \in K$ mit $x+(-x) = 0$.
    \item
     $\cdot$ ist assoziativ, d.h.\ $x \cdot (y \cdot z) = (x \cdot y) \cdot z$ für alle $x,y,z \in K$.
    \item
     $\cdot$ ist kommutativ, d.h.\ für alle $x,y \in K$ ist $x \cdot y = y \cdot x$.
    \item
     Es gibt ein multiplikatives neutrales Element $1 \in K\setminus\{0\}$ mit $x \cdot 1 = x$ für alle $x \in K$
    \item
     Für jedes $x \in K\setminus\{0\}$ gibt es ein multiplikativ inverses Element $x^{-1} \in K\setminus\{0\}$ mit $x \cdot x^{-1} = 1$.
    \item
     $+$ ist distributiv bezüglich $\cdot$, d.h.\ $x \cdot (y+z) = (x \cdot y) + (x \cdot z)$ für alle $x,y,z \in K$.
   \end{enumerate}
  \item
   Man rechnet „Punkt vor Strich“, d.h.\ für alle $x,y,z \in K$ ist
   \[
    x + y \cdot z \coloneqq x + (y \cdot z).
   \]
  \item
   In einem Körper $K$ gelten die gewöhnlichen Rechenregeln. So ist beispiesweise für alle $x,y,z \in K$ mit $z \neq 0$
   \begin{enumerate}[label=\alph*)]
    \item $0 \cdot x = 0$,
    \item $1 \cdot x = x$,
    \item $-x = (-1) \cdot x$,
    \item $(-z)^{-1} = -z^{-1}$
    \item $(x+y)^2 = x^2 + 2xy + y^2$ und $(x+y)(x-y) = x^2 - y^2$.
   \end{enumerate}
  \item
   Da $1 \in K^\times = K \setminus \{0\}$ ist $1 \neq 0$.
 \end{enumerate}
\end{bem}


\begin{defi}
 Es sei $L$ ein Körper. Eine Teilmenge $K \subseteq L$ heißt \emph{Unterkörper}, falls die folgenden Bedingungen erfüllt sind:
 \begin{enumerate}[label=\roman*)]
  \item
   $K$ ist eine Untergruppe der additiven Gruppe von $L$.
  \item
   $K\setminus\{0\}$ ist eine Untergruppe der multiplikativen Gruppe von $L$, d.h.\ von $L^\times$.
 \end{enumerate}
\end{defi}


\begin{bem}
 Ist $L$ ein Körper und $K \subseteq L$ ein Unterkörper, so ist $K$ zusammen mit der Addition und Multiplikation aus $L$ selber ebenfalls ein Körper.
\end{bem}





\subsection{Einige kleine Beispiele}
\begin{enumerate}[leftmargin=*]
 \item
  Die rationalen Zahlen $\Qbb$ und die reellen Zahlen $\Rbb$ bilden zusammen mit der üblichen Addition und Multiplikation jeweils einen Körper. $\Qbb$ ist ein Unterkörper von $\Rbb$.
 \item
  Die ganzen Zahlen $\Zbb$ bilden zusammen mit der üblichen Addition und Multiplikation \emph{keinen} Körper: Ansonsten wäre $1$ das neutrale Element der Multiplikation und es gebe ein Element $x \in \Zbb$ mit $2x = 1$. Ein solches Element gibt es aber nicht.
 \item
  Für einen Körper $K$ und $n \in \Nbb$, $n \geq 2$ bilden die quadratischen Matrizen $\Mat(n \times n, K)$ zusammen mit der gewöhnlichen Addition und Multiplikation von Matrizen keinen Körper: Zum einen ist die Multiplikation in diesem Fall nicht kommutativ, da
  \[
   \begin{pmatrix}
    1 & 0 \\
    0 & 0
   \end{pmatrix}
   \begin{pmatrix}
    0 & 1 \\
    0 & 0
   \end{pmatrix}
   =
   \begin{pmatrix}
    0 & 1 \\
    0 & 0
   \end{pmatrix}
   \quad\text{aber}\quad
   \begin{pmatrix}
    0 & 1 \\
    0 & 0
   \end{pmatrix}
   \begin{pmatrix}
    1 & 0 \\
    0 & 0
   \end{pmatrix}
   =
   \begin{pmatrix}
    0 & 0 \\
    0 & 0
   \end{pmatrix}.
  \]
  Zum anderen ist nicht jede Matrix $A \in \Mat(n \times n, K) \setminus \{0\}$ bezüglich der Matrixmultiplikation invertierbar, da etwa
  \[
   \begin{pmatrix}
    1 & 0 \\
    0 & 0
   \end{pmatrix}
   \begin{pmatrix}
    a & b \\
    c & d
   \end{pmatrix}
   =
   \begin{pmatrix}
    a & b \\
    0 & 0
   \end{pmatrix}
   \neq
   \begin{pmatrix}
    1 & 0 \\
    0 & 1
   \end{pmatrix}
  \]
  für alle $a,b,c,d \in K$.
\end{enumerate}





\subsection{Die komplexen Zahlen \texorpdfstring{$\Cbb$}{C}}
Die komplexen Zahlen $\Cbb$ sind ein Körper, der die reellen Zahlen $\Rbb$ enthält, und in dem es ein Element $i \in \Cbb$ gibt, für das $i^2 = -1$. Wir geben an dieser Stelle zwei mögliche Konstruktionen der komplexen Zahlen an:



\subsubsection{Die komplexen Zahlen als besserer \texorpdfstring{$\Rbb^2$}{R2}}
Wir beginnen mit der abelschen Gruppe $\Cbb \coloneqq \Rbb^2 = \{(a,b) \mid a,b \in \Rbb\}$. Die Addition ist gegeben durch
\[
 (a_1, b_1) + (a_2, b_2) = (a_1 + a_2, b_1 + b_2)
 \quad\text{für alle $(a_1, b_1), (a_2, b_2) \in \Cbb$}.
\]
Wir definieren zusätzlich eine Multiplikation $\cdot$ auf $\Cbb$ durch
\[
 (a_1, b_1) \cdot (a_2, b_2)
 = (a_1 a_2 - b_1 b_2, a_1 b_2 + a_2 b_1)
 \quad\text{für alle $(a_1, b_1), (a_2, b_2) \in \Cbb$}.
\]

\begin{beh}
 $\Cbb$ ist zusammen mit der obigen Addition und Multiplikation ein Körper.
\end{beh}
\begin{proof}
 Dass $\Cbb$ zusammen mit der Addition eine abelsche Gruppe ist, ist bekannt. Das additiv neutrale Element ist $(0,0)$ und das additiv Inverse zu $(a,b) \in \Cbb$ ist $(-a,-b)$ Für alle $(a_1, b_1), (a_2, b_2), (a_3, b_3) \in \Cbb$ ist
 \begin{align*}
   &\, (a_1, b_1) \cdot ((a_2, b_2) \cdot (a_3, b_3))
  = (a_1, b_1) \cdot (a_2 a_3 - b_2 b_3, a_2 b_3 + a_3 b_2) \\
  =&\, (a_1 a_2 a_3 - a_1 b_2 b_3 - a_2 b_1 b_3 - a_3 b_1 b_2, a_2 a_3 b_1 - b_1 b_2 b_3 + a_1 a_2 b_3 + a_1 a_3 b_2) \\
  =&\, (a_1 a_2 - b_1 b_2, a_1 b_2 + a_2 b_1) \cdot (a_3, b_3)
  = ((a_1, b_1) \cdot (a_2, b_2)) \cdot (a_3, b_3),
 \end{align*}
 also ist die Multiplikation assoziativ. Für alle $(a_1, b_1), (a_2, b_2) \in \Cbb$ ist
 \begin{align*}
  (a_1, b_1) \cdot (a_2, b_2)
  &= (a_1 a_2 - b_1 b_2, a_1 b_2 + a_2 b_1) \\
  &= (a_2 a_1 - b_2 b_1, a_2 b_1 + a_1 b_2)
  = (a_2, b_2) \cdot (a_1, b_1),
 \end{align*}
 also ist die Multiplikation kommutativ. Für alle $(a,b) \in \Cbb$ ist
 \[
  (1,0) \cdot (a,b) = (1 \cdot a - 0 \cdot b, 0 \cdot a + 1 \cdot b) = (a,b),
 \]
 also ist $(1,0)$ neutral bezüglich der Multiplikation. Ist $(a,b) \in \Cbb$ mit $(a,b) \neq (0,0)$, so ist $a \neq 0$ oder $b \neq 0$, also $a^2+b^2 \neq 0$. Daher ist
 \[
  (a,b) \cdot \left(\frac{a}{a^2+b^2}, -\frac{b}{a^2+b^2}\right)
  = \left( a \frac{a}{a^2+b^2} + b \frac{b}{a^2+b^2}, b \frac{a}{a^2+b^2} - a \frac{b}{a^2+b^2} \right)
  = (1,0).
 \]
 Also ist $(a/(a^2+b^2), -b/(a^2+b^2))$ multiplikativ invers zu $(a,b)$. Sind $z_1, z_2 \in \Cbb$ mit $z_1, z_2 \neq 0$, so sind $z_1$ und $z_2$ invertierbar. Daher ist auch $z_1 \cdot z_2$ invertierbar und somit $z_1 \cdot z_2 \neq 0$. Also ist $\Cbb\setminus\{0\}$ unter der Multiplikation abgeschlossen.
 
 Ingesamt zeigt dies, dass $+$ und $\cdot$ assoziativ und kommutativ ist, und dass $(\Cbb,+)$ und $(\Cbb\setminus\{(0,0)\}, \cdot)$ abelsche Gruppen sind. Es muss nur noch die Distributivität gezeigt werden: Dies gilt, da für alle $(a_1, b_1), (a_2, b_2), (a_3, b_3) \in \Cbb$
 \begin{align*}
   &\, (a_1, b_1) \cdot ((a_2, b_2) + (a_3, b_3)) \\
  =&\,    (a_1, b_1) \cdot (a_2 + a_3, b_2 + b_3) \\
  =&\, (a_1 (a_2 + a_3) - b_1 (b_2 + b_3), a_1 (b_2 + b_3) + (a_2 + a_3) b_1) \\
  =&\, (a_1 a_2 + a_1 a_3 - b_1 b_2 - b_1 b_3, a_1 b_2 + a_1 b_3 + a_2 b_1 + a_3 b_1) \\
  =&\, (a_1 a_2 - b_1 b_2, a_1 b_2 + a_2 b_1) + (a_1 a_3 - b_1 b_3, a_1 b_3 + a_3 b_1) \\
  =&\, (a_1, b_1) \cdot (a_2, b_2) + (a_1, b_1) \cdot (a_3, b_3).
 \end{align*}
 Ingesamt zeigt dies, dass $\Cbb$ zusammen mit der angegebenen Addition und Multiplikation ein Körper ist.
\end{proof}

Wir wollen noch etwas Notation einführen, um das Rechnen mit $\Cbb$ einfacher zu machen: Wir schreiben $0 = (0,0)$ und $1 = (1,0)$. Wir bemerken zunächst, dass für alle $a, a' \in \Rbb$
\[
 (a,0) + (a',0) = (a + a', 0)
 \quad\text{und}\quad
 (a,0) \cdot (a',0) = (a a' - 0 \cdot 0, a \cdot 0 + 0 \cdot a') = (a a', 0)
\]
und dass $0, 1 \in \{(a,0) \mid a \in \Rbb\}$. Also ist $\{(a,0) \mid a \in \Rbb\} \subseteq \Cbb$ ein Unterkörper und für die Abbildung $\phi \colon \Rbb \to \{(a,0) \mid a \in \Rbb\}$, $x \mapsto (x,0)$ gilt:
\begin{enumerate}[label=\roman*)]
 \item
  $\phi$ ist bijektiv.
 \item
  Für alle $a,a' \in \Rbb$ ist $\phi(a+a') = \phi(a)+\phi(a')$ und $\phi(aa') = \phi(a)\phi(a')$.
 \item
  Es ist $\phi(1) = 1$.
\end{enumerate}
Dies erlaubt es uns, $\Rbb$ mit dem Unterkörper $\{(a,0) \mid a \in \Rbb\} \subseteq \Cbb$ zu identifizieren. Wir unterscheiden also im Folgenden nicht mehr zwischen $a \in \Rbb$ und $(a,0) \in \Cbb$.

Wir schreiben $i \coloneqq (0,1) \in \Cbb$. Für jedes $b \in \Rbb$ gilt
\[
 b \cdot i = (b,0) \cdot (0,1) = (b \cdot 0 - 0 \cdot 1, b \cdot 1 + 0 \cdot 0) = (0,b).
\]
Ein beliebiges Element $z = (a,b) \in \Cbb$ lässt sich daher als
\[
 z = (a,b) = (a,0) + (0,b) = a + bi
\]
schreiben. Die beiden reellen Zahlen $a,b \in \Rbb$ mit $z = a + bi$ sind eindeutig, da $z = (a,b)$.

Bemerke, dass $i^2 = (0,1) \cdot (0,1) = (0 \cdot 0 - 1 \cdot 1, 0 \cdot 1 + 1 \cdot 0) = (-1, 0) = -1$. Die Addition lässt sich nun auch so darstellen, dass für alle $a_1, a_2, b_1, b_2 \in \Rbb$
\[
 (a_1 + b_1 i) + (a_2 + b_2 i)
 = a_1 + a_2 + b_1 i + b_2 i
 = (a_1 + a_2) + (b_1 + b_2) i,
\]
und die Multiplikation so, dass
\begin{align*}
 (a_1 + b_1 i)(a_2 + b_2 i)
 &= a_1 a_2 + a_1 b_2 i + a_2 b_1 i + b_1 b_2 i^2 \\
 &= a_1 a_2 - b_1 b_2 + a_1 b_2 i + a_2 b_1 i \\
 &= (a_1 a_2 - b_1 b_2) + (a_1 b_2 + a_2 b_1) i.
\end{align*}



\subsubsection{Die komplexen Zahlen als reelle (\texorpdfstring{$2 \times 2$}{2x2})-Matrizen}
Unabhängig von der ersten Methode geben wir noch eine weitere Konstruktion der komplexen Zahlen an: Es sei
\[
 C \coloneqq
 \left\{
  \begin{pmatrix}
   a &           -b \\
   b & \phantom{-}a
  \end{pmatrix}
  \,\middle|\,
  a,b \in \Rbb
 \right\}
 \subseteq \Mat(2 \times 2, \Rbb).
\]
Wir zeigen, dass $C$ zusammen mit der gewöhnlichen Matrixaddition und -multiplikation einen Körper bildet:

Für alle $a, b \in \Rbb$ schreiben wir abkürzend
\[
 Z(a,b)
 \coloneqq
 \begin{pmatrix}
  a &           -b \\
  b & \phantom{-}a
 \end{pmatrix}.
\]
Für alle $a_1, a_2, b_1, b_2 \in \Rbb$ ist
\begin{gather*}
 \begin{aligned}
  Z(a_1, b_1) + Z(a_2, b_2)
  &=
  \begin{pmatrix}
   a_1 &           -b_1 \\
   b_1 & \phantom{-}a_1
  \end{pmatrix}
  +
  \begin{pmatrix}
   a_2 &           -b_2 \\
   b_2 & \phantom{-}a_2
  \end{pmatrix} \\
  &=
  \begin{pmatrix}
   a_1+a_2 &           -b_1+b_2 \\
   b_1+b_2 & \phantom{-}a_1+a_2
  \end{pmatrix}
  =
  Z(a_1+a_2, b_1+b_2)
 \end{aligned}
\shortintertext{und außerdem}
 \begin{aligned}
  Z(a_1,b_1) \cdot Z(a_2,b_2)
  &=
  \begin{pmatrix}
   a_1 &           -b_1 \\
   b_1 & \phantom{-}a_1
  \end{pmatrix}
  \begin{pmatrix}
   a_2 &           -b_2 \\
   b_2 & \phantom{-}a_2
  \end{pmatrix} \\
  &=
  \begin{pmatrix}
   a_1 a_2 - b_1 b_2 &           -a_1 b_2 - a_2 b_1 \\
   a_1 b_2 + a_2 b_1 & \phantom{-}a_1 a_2 - b_1 b_2
  \end{pmatrix}
  = Z(a_1 a_2 - b_1 b_2, a_1 b_2 + a_2 b_1).
 \end{aligned}
\end{gather*}
Um zu zeigen, dass $C$ bezüglich der Addition eine abelsche Gruppe bildet genügt es zu zeigen, dass $C$ eine Untergruppe der additiven Gruppe von $\Mat(2 \times 2, \Rbb)$ bildet. Es ist $0 = Z(0,0) \in C$ und für alle $a_1, a_2, b_1, b_2 \in \Rbb$ ist
\[
 Z(a_1, b_1) + Z(a_2, b_2) = Z(a_1 + a_2, b_1 + b_2) \in C.
\]
Für alle $a,b \in \Rbb$ ist außerdem
\[
 -Z(a,b)
 =
 -\begin{pmatrix}
  a &           -b \\
  b & \phantom{-}a
 \end{pmatrix}
 =
 \begin{pmatrix}
  -a & \phantom{-}b \\
  -b &           -a
 \end{pmatrix}
 = Z(-a,-b) \in C.
\]
Also ist $C$ eine abelsche Untergruppe der additiven Gruppe von $\Mat(2 \times 2, \Rbb)$.

Die Assoziativität der Multiplikation ist bekannt. Die Matrixmultiplikation ist auf $C$ kommutativ, da für alle $a_1, a_2, b_1, b_2 \in \Rbb$
\begin{align*}
 Z(a_1, b_1) \cdot Z(a_2, b_2)
 &= Z(a_1 a_2 - b_1 b_2, a_1 b_2 + a_2 b_1) \\
 &= Z(a_2 a_1 - b_2 b_1, a_2 b_1 + a_1 b_2)
 = Z(a_2, b_2) \cdot Z(a_1, b_1).
\end{align*}
Es muss also nur noch gezeigt werden, dass $C\setminus\{0\}$ zusammen mit der Matrixmultiplikation eine Gruppe bildet. Es ist $I_2 = Z(1,0) \in C$ das neutrale Element. Für $a,b \in \Rbb$ mit $Z(a,b) \neq 0$ ist $a \neq 0$ oder $b \neq 0$ und somit $a^2+b^2 \neq 0$. Da
\begin{align*}
  &\, Z(a,b) Z\left(\frac{a}{a^2+b^2}, -\frac{b}{a^2+b^2}\right) \\
 =&\, Z\left( a \frac{a}{a^2+b^2} + b\frac{b}{a^2+b^2}, b\frac{a}{a^2+b^2} - a\frac{b}{a^2+b^2} \right) \\
 =&\, Z(1, 0)
 = I_2
\end{align*}
ist $Z(a/(a^2+b^2), -b/(a^2+b^2))$ das multiplikativ Inverse zu $Z(a,b)$ (da die Matrixmultiplikation auf $C$ kommutativ ist, zeigt die obige Rechnung, dass $Z(a/(a^2+b^2), -b/(a^2+b^2))$ bereits beidseitig Invers zu $Z(a,b)$ ist).











